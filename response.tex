\documentclass[12pt]{article}
\listfiles
\usepackage[T1]{fontenc}
\usepackage[utf8]{inputenc}
\usepackage{amsmath,color}
\usepackage[margin=20mm]{geometry}
\usepackage{amsfonts}
\usepackage{amsthm}
\usepackage{graphicx}
\usepackage{enumerate}
\usepackage{amsfonts}
\usepackage{amsthm,mathtools}
\usepackage{paralist}
\usepackage{stmaryrd}

\setlength{\parindent}{0ex}
\setlength{\parskip}{1ex}


\renewcommand{\ge}{\geqslant}
\renewcommand{\le}{\leqslant}
\renewcommand{\geq}{\geqslant}
\renewcommand{\leq}{\leqslant}

\begin{document}

\subsection*{Graph Product Structure for Non-Minor-Closed Classes\\
Vida Dujmovi\'c, Pat Morin, David R. Wood\\
Journal of Combinatorial Theory Series B (JCTB9758)\\
Response to Referees}

\subsection*{Reviewer \#2}

 The Graph Product Structure Theorem of Dujmović, Joret,
Micek, Morin, Ueckerdt, and Wood (FOCS '19) asserts that every planar
graph is a subgraph of the strong product of a graph of treewidth at
most 8 and a path.  This has been the key tool in resolving several
longstanding open problems on planar graphs (and graphs in
minor-closed classes), including problems on queue layouts,
non-repetitive colouring, p-centered colouring, and universal graphs.
This paper extends the Graph Product Structure Theorem beyond
minor-closed classes.  The authors prove product structure theorems
for k-planar graphs, map graphs, string graphs, graph powers, and
nearest neighbour graphs.   The authors also present several
applications of their results.  These are essentially the same
applications as for the original Graph Product Structure Theorem; the
point being that the applications typically follow quite easily once a
product structure theorem is established.

This paper is a significant step in answering the (ambitious) question
of which graph classes admit product structure theorems. It is
extremely well-written (with very few typos) and was a pleasure to
read.  It is certainly well within the scope of JCTB and I strongly
recommend it for acceptance.

Below are some typos and suggestions for the authors.

Page 3.  The sentence 'For each $vw \in E(G')$ there is a path $P$ in $G$
between v and we of length at most k+1'' is correct but should be made
more precise.  For example, there could be a short path between v and
w which does not involve any new dummy vertices.  You mean to say that
the 'subdivided path' between v and w has length at most k+1.

Page 3.  You should add the assumption that no three edges cross at
the same point (which you can assume by perturbing the embedding).
Otherwise, you will not get a (k+1, 2)-shortcut system.

Page 3. P is used twice in the statement of Theorem 3.  It is probably
safer (and necessary?) to use a different variable for the second
occurrence.

Page 6. Regarding (T1), technically a subtree of a rooted tree can be
rooted at any vertex, so it is a bit imprecise to say that T[x] is
rooted at x.  I suggest adding that a subtree of a tree rooted at r,
is always considered to be rooted at the vertex closest to r.

Page 6. Replace 'that satisfies (T2)' by 'that satisfies (T1)'.

Page 6. Replace 'parent in $T'$ by 'parent in $T_0$'.

Page 6. The term 'hierarchical decomposition' is not defined.

Page 7. Although it is clear for me, 'separation' has not been
defined.  Perhaps a blanket note saying all undefined terms are in
Diestel's textbook should be added.

Page 7.  In the very last sentence, I think $B_{x_i}$ should be $B_{z_i}$
(twice).  Moreover, the way the proof is written, it seems to suggest
that we only get that $a \in B_{z_i}$ for each $i \in \{1, \dots, r\}$.
	The conclusion that $a \in B_{z_i}$ for each $i \in \{0, \dots, r\}$ is
		true though.  In a normalized tree decomposition, if $xy$ is an edge and
		$x$ is a $T$-ancestor of $y$, then $x$ must be in $B_y$ (since the subtrees $T_x$
		and $T_y$ intersect).
		
		Page 8.  Can you add a brief explanation why $\mathcal{S}$ is a
		partition of V(G)?  It is clear that the sets in $\mathcal{S}$ are
		disjoint, but why do they cover $V(G)$?
		
		Page 11. Replace 'belong to $F_\delta$' by 'belongs to $F_\delta$'.
		
		Page 11. In footnote 11, what the authors call a 'closed curve' I
		would call just a 'curve'.  To me, a closed curve satisfies the
		additional condition that f(0)=f(1).
		
		Page 13. Could it be that the subgraph induced by the vertices of a
		kite is a $K_4$ with some parallel edges?  This seems relevant later in
		the proof.
		
		Page 13. Perhaps I am misunderstanding something, but I do not see why
		'none of the edges vx, xw, wy, or yv are crossed by any other edges of
		G.'  For example, see the attached PDF for a picture where vw, xw, wy,
		and yv are all crossed by other edges of G.
		
		Page 13. The attached PDF also shows that the definition of 'kite
		face' may not be well-defined.  For example, I do not see why there
		cannot be some edges of G 'inside' a kite face.  Even if this is not
		the case, I think it is more precise to say that a kite face has 'two
		and a half edges' and two vertices of G on its boundary' rather than
		'three edges and two vertices of G on its boundary'.
		
		Page 13.  Regarding the proof of Lemma 4, I think it is better to
		include all the details from reference [14].  As far as I know, JCTB
		does not have a page limit, so I do not see an issue with just
		reproducing the entire proof and telling the reader that they can skip
		all the details if they wish.  This is just my personal opinion
		though, so the authors can ignore this request if they choose.
		
		Page 15. Regarding the proof of Theorem 11, if it really is the same
		proof as the proof of Theorem 2, why not just prove Theorem 11 (from
		which Theorem 2 follows as a special case)?
		
		Page 16.  At the end of Page 16, I do not see why 'every pair of edges
		cross at most once.'  Without further assumptions, it seems as if two
		paths in the path system could intersect several times.  Therefore, by
		drawing 'each edge vw of G alongside $P_vw$ in $G_0$', there may be two
		edges which cross several times.  It seems this can be fixed by
		choosing the path system carefully by 'uncrossing' paths which
		intersect at more than two internal vertices.
		
		Page 18. In point 1 in the proof of Lemma 9, I think '$\phi(w)$' should
		be '$\alpha(w)$'.
		
		Page 18.  Say $H'$ is connected since $X$ is connected.
		
		Page 18.  Replace '$|i'-i''| < p$' by  '$0< |i'-i''| < p$'.
		
		Page 20.  Where does the bound $d(d-3)/2$ come from?  The naive bound I
		compute is $d(d-1)/2$.
		
		Page 21.  Replace the comma at the end of the statement of Theorem 14
		with a period.
		
		Page 23. Replace 'oberve' by 'observe'.
		
		\textcolor{blue}{Done}
		
		Page 24. Petr Hliněný (together with his student) has announced that
		the answer to the open question is yes, with $C=3$ and $\ell=O(k^2)$.
		


\end{document}