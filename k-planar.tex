\documentclass{patmorin}
\usepackage[T1]{fontenc}
\usepackage[utf8]{inputenc}
\usepackage{amsmath}
\usepackage{amsfonts}
\usepackage{amsthm}
\usepackage{graphicx}
\usepackage{enumerate}
\usepackage{pat}
\usepackage{paralist}

\usepackage{tipa}
\usepackage{upgreek}
\usepackage{rotating}
\newcommand{\V}{\rotatebox[origin=c]{180}{$V$}}
\newcommand{\Y}{\rotatebox[origin=c]{180}{$Y$}}
%\usepackage{hyperref}

\setlength{\parskip}{1ex}

\title{\MakeUppercase{Notes on Layered $H$-Partitions of $k$-Planar Graphs}%
    \thanks{This work was partly funded by NSERC and MRI.}}

\author{Vida Dujmovi\'c%
        \thanks{School of Computer Science and Electrical Engineering,
                University of Ottawa},\,\,
        Pat Morin%
        \thanks{School of Computer Science, Carleton University},\,\, and
        David R. Wood%
        \thanks{School of Mathematical Sciences, Monash University}}

\newcommand{\dual}[1]{{#1}^\star}
\newcommand{\note}[2]{{\color{red}[#1:~#2]}}

\DeclareMathOperator{\dist}{dist}
\DeclareMathOperator{\depth}{depth}
\DeclareMathOperator{\ff}{f}
\newcommand{\f}{{\color{red}\ff}}

\newcommand{\fk}{{\color{red}f(k)}}

\begin{document}
\maketitle


\begin{abstract}
  TBD
  % Layered $H$-partitions of graphs (with small layered width, in which $H$ has small treewidth) are a recently-introduced tool that have been used to solve longstanding problems on queue layouts, non-repetitive colouring, and 3-d graph drawing.  Such partitions are known to exist for planar graphs and, more generally, bounded genus graphs.
  % In the current paper, we prove that every $k$-planar graph has a layered $H$-partition of layered width $O(k^3)$ in which $H$ is planar and has treewidth 3. This implies that $k$-planar graphs have a queue number upper-bounded by a polynomial in $k$ and have non-repetitive chromatic number upper-bounded by an exponential function of $k$.
\end{abstract}

\section{Introduction}



\section{Another Way of Looking at It}

\subsection{PS-Tree}

A PS-Partition $(Y_x:x\in V(K))$ for a triangulation $\Delta$ and spanning-tree $T$ of $\Delta$ is a partition of $V(\Delta)$ whose elements are indexed by nodes of a rooted 3-ary tree $K$.  Each bag $Y_x$ of a PS-partition is a tripod in $\Delta$, with respect to $T$.  This means that $T[Y_x]$ has at most 3 components, each of which is a vertical path whose lower endpoints form a clique $\tau_x$ in $\Delta$.  Associated with each node $x\in V(K)$ is a cycle $F_x$ in $\Delta$ and a near-triangulation $N_x$ consisting of the part of $\Delta$ contained in $F_x$.  Note that $Y_x\subseteq V(N_x)\setminus V(F_x)$ and $V(N_x)\subseteq V(N_a)$ for every ancestor $a$ of $x$.

It is also useful to have a name for the \emph{Mercedes graph} $M_x$ that consists of the edges and vertices of $F_x$, the clique $\tau_x$, and the paths in $T$ from the vertices of $\tau_x$ to $F_x$.

In short, $K$ is the 3-ary tree of tripods we get from our proof.  It's the tree we use for our tree-decomposition of $H$.  We've just added some decorations to it. Eventually I should write this as a lemma with bullet points for the relevant properties.

\subsection{$k$-Planar Graphs}

Let $G$ be a $k$-plane graph and let $G^+$ be the graph obtained by making $G$ planar with dummy vertices and then triangulating the resulting graph. Let $T$ be a BFS spanning tree of $G^+$ where the root $r$ of $T$ has degree 3 in $G^+$ (add an additional vertex to $G^+$ if necessary). For each integer $i\ge 0$, let $V_i=\{v\in V(G^+): \dist_{G^+}(r,v)=i\}$.

Let $K$ be the PS-partition for $G^+$ and $T$.  We say that a vertex $v\in V(G)$ \emph{appears} at a node $x\in V(K)$ if $u\in Y_x$ or if there exists $vw\in E(G)$ such that $v,w\in V(N_x)\setminus V(F_x)$ and $vw$ crosses $Y_x$ (i.e,. $vw$ contains a dummy vertex of $V(Y_x)$ in its interior).

Let $S_x$ (the separator at $x$) denote the set of all vertices $v\in V(G)$ such that $v$ appears at $x$ but does not appear at an ancestor of $x$.  We claim that $(S_x:x\in V(K))$ is a partition of $V(G)$.  Indeed, any $v\in V(G)$ can only appear in a node $x$ such that $v\in V(N_x)\setminus V(F_x)$.  These nodes form a path from the root of $K$ to the unique node $x'$ such that $v\in Y_{x'}$.  It is worth noting that, if $vw\in E(G)$, and $v,w\in S_x$, then the embedding of $vw$ is contained in the interior of $F_x$ (otherwise $v$ and $w$ would have appeared in $S_a$ for some ancestor $a$ of $x$.)

We use the partition $(S_x: x\in V(K))$ which generates a graph $H$ with vertex set $V(H)=V(K)$ the edge $xy\in E(H)$ if there exists an edge $vw\in E(H)$ with $v\in S_x$ and $w\in S_y$.

First we argue that this partition has small width, with respect to the layering defined by $T$ (note that this is a layering of $G^+$, not a layering of $G$).

\begin{clm}
  For any $x\in V(H)$, and any integer $i\ge 0$, $|S_x\cap V_i|\le 3(k+1)$.
\end{clm}

\begin{proof}
  Done already, several times.
\end{proof}

Next we describe a tree decomposition $(B_x:x\in V(K))$ of $H$.  (Don't be put off here by the fact that $V(H)=V(K)$).  We do this by definining, for each $x\in V(H)$, the subtree $K[x]=K[\{y:x\in B_y\}]$.  The subtree $K[x]$ contains $x$.  Additionally, for every edge $xz\in E(H)$, $K[x]$ includes the path from $x$ to $z$ in $K$.  This clearly produces a tree decomposition of $H$, what remains is to upper bound the width of this decomposition.

We claim that $K[x]$ is rooted at $x$, i.e., $x$ is the lowest common ancestor of all nodes in $B_x$.  To see why this is so, suppose $K[x]$ has root $a\neq x$.  In short: This happens because some $v\in V(N_x)$ is incident on an edge $vw\in E(G)$ that crosses $Y_a$.  Longer: This implies that $a$ has two children $b\neq c$ where $x$ is a descendant of $b$, some node $y$ is a descendant of $c$, and there is an edge $vw\in E(G)$ such that $v\in V(N_x)\setminus V(F_x)\subseteq V(N_b)\setminus V(F_a)$
and $w\in S_y\subset V(N_y)\subseteq V(N_b)\setminus V(F_a)$.  (Note, we allow $x=b$ and/or $y=c$ here.)  Therefore, $v$ and $w$ are in the interior of two different faces of the Mercedes graph $M_a$.  Therefore the edge $vw$ intersects $M_a$.  If it intersects $M_a$ at some vertex in $Y_a$, then $v$ would be in $S_a$, not $S_x$, a contradiction.  If it intersects $M_a$ at some vertex in $F_a$, then $v$ would be in $S_{a'}$ for some ancestor $a'$ of $a$, also a contradiction.

Now, fix some node $x\in V(K)$ and consider the contents of $B_x$.  As we have just argued, $B_x$ contains only ancestors of $x$ (including $x$ itself). If $B_x$ contains some ancestor $a$ of $x$, it is because there is an edge $vw\in E(G)$ with $v\in V(N_x)\setminus V(F(x))$ and $w\in S_a$.  Now, $w\in S_a$ because $w\in Y_a$ or there is an edge $wz\in E(G)$ that contains a point of $Y_a$.  This implies that there is a curve $C$ from $v$ to $z$ that contains at most $2k$ dummy  vertices of $G^+$.

Now, consider the cubic graph $A$ homeomorphic to the union of $F_{a'}$ for all ancestors $a'$ of $x$. See \figref{A}.  It is helpful to think of this graph as follows:  We begin with $F_{a_0}$ where $a_0$ is the root of $K$.  We then cut $F_{a_0}$ into two pieces $F^{\bar{x}}_{a_0}$ and $F^x_{a_0}$ where $F^x_{a_0}$ contains $F_x$.  This process continues by partitioning $F_{a_1}:=F^x_{a_0}$ recursively.  The process terminates after some number $r$ of iterations, when $F^x_{a_r}=F_x$.

\begin{figure}
  \begin{center}
    \includegraphics{figs/A}
  \end{center}
  \caption{The graph $A$.}
  \figlabel{A}
\end{figure}

We claim that the curve $C$ begins in $F_x\subseteq F^x_a\subseteq F_a$, remains in the interior of $F_a$, and eventually reaches the boundary $F^{\bar{x}}_a\cap F^x_a$.  The fact that $C$ remains in $F_a$ is due to the fact that, if it crossed the boundary of $F_a$, then one of $vw$ or $wz$ would interects $Y_{a'}$ for some ancestor $a'$ of $a$.  But this would contradict the fact that $vw$ (or $wz$) are in $S_a$. The fact that $C$ eventually reaches $F^{\bar{x}}_a\cap F^x_a$ follows from the fact that $w\in S_a$ (so either $w\in Y_a$ or $wz$ crosses $Y_a$).

Therefore, to upper-bound $|B_x|$, it suffices to upper-bound the number of faces $F^{\bar{x}}_a$ of $A$ that can be reached with curves that begin in $F_x$, remain in $F_a$, and terminate in $F^{\bar{x}}_a$ without crossing more than $2k$ edges in $A$.  To avoid triple subscripts we say that a curve visits $a_i$ if it visits $F^{\bar{x}}_{a_i}$.  We can further restrict ourselves to studying curves that cross the fewest number of edges of $A$ since if a curve exists that crosses at most $k$ edges then the curve that crosses the fewest number of possible edges also crosses at most $k$ edges.  We will call such a curve a lightest curve from $F_x$ to $F_a$.

Again, we let $a_0,\ldots,a_r$ denote the vertices of the path, in $K$ from the root $a_0$ to $x=a_r$.  We claim that, there exists a lightest curve from $a_r$ to $a_0$ that is monotonic in the sense that the curve visits $a_{i_0},\ldots,a_{i_t}$ in this order, where $i_0=r > i_1>\cdots>i_t=0$ the Indeed, this follows from the fact that, for any node $a_i$, $i>0$, the (three) faces of $A$ that share a boundary with $F_{a_i}$ also share a boundary with $F^{\bar{x}}_{a_i}$.  Therefore it is never advantageous for a path to go from $a_i$ to $a_j$ for some $j>i$ since the path will eventually have to return to one of these three faces.

At this point we are essentially done.  Any monotonic curve that visits $a_i$ has at most 3 choices for the next face $a_j$ $j<i$ to visit.  Therefore, the number of faces of $A$ that can be reached from $F_x$ by admissible curves that cross $2k$ edges of $A$ is at most $3^{2k}$. Therefore,
\[
   |B_x| \le \sum_{j=0}^2k 9^j = (9^{2k+1}-1)/8 \enspace .
\]

This completes the proof

\begin{thm}
    Every $k$-planar graph $G$ has a $(k9^{2k+1}/8,3k)$-partition.
\end{thm}






\bibliographystyle{plain}
\bibliography{warmup}

\end{document}
