\documentclass{patmorin}
\usepackage[T1]{fontenc}
\usepackage[utf8]{inputenc}
\usepackage{amsmath}
\usepackage{amsfonts}
\usepackage{amsthm}
\usepackage{graphicx}
\usepackage{enumerate}
\usepackage{pat}
\usepackage{paralist}

\usepackage[longnamesfirst,numbers,sort&compress]{natbib}

% \usepackage{tipa}
% \usepackage{upgreek}
% \usepackage{rotating}
% \newcommand{\V}{\rotatebox[origin=c]{180}{$V$}}
% \newcommand{\Y}{\rotatebox[origin=c]{180}{$Y$}}
%\usepackage{hyperref}

\setlength{\parskip}{1ex}

\title{\MakeUppercase{Layered $H$-Partitions of $k$-Planar Graphs}%
    \thanks{This work was partly funded by NSERC and MRI.}}

\author{Vida Dujmovi\'c%
        \thanks{School of Computer Science and Electrical Engineering,
                University of Ottawa},\,\,
        Pat Morin%
        \thanks{School of Computer Science, Carleton University},\,\, and
        David R. Wood%
        \thanks{School of Mathematical Sciences, Monash University}}

\newcommand{\dual}[1]{{#1}^\star}
\newcommand{\note}[2]{{\color{red}[#1:~#2]}}

\DeclareMathOperator{\dist}{dist}
\DeclareMathOperator{\depth}{depth}
\DeclareMathOperator{\ff}{f}

\renewcommand{\proplabel}[1]{\label{prop:#1}}
\renewcommand{\propref}[1]{(PR\ref{prop:#1})}

\begin{document}
\maketitle


\begin{abstract}
  Layered $H$-partitions of graphs (with small layered width, in which $H$ has small treewidth) are a recently-introduced tool that have been used to solve longstanding problems on queue layouts, non-repetitive colouring, and 3-d graph drawing.  Such partitions are known to exist for planar graphs and, more generally, bounded genus graphs.  In the current paper, we prove that every $k$-planar graph has a layered $H$-partition of layered width $O(k)$ in which $H$ has treewidth $O(c^k)$, for some constant $c$. This is the first result of this type of a non-minor closed class of graphs and implies that $k$-planar graphs have both queue number and non-repetitive chromatic number upper-bounded by a function of $k$.  The former result was previously shown using a combination of $H$-partitions for planar graphs with \textit{ad hoc} methods. The latter result is new, for all $k\ge 3$.
\end{abstract}

\section{Introduction}

A \emph{layering} of a graph $G$ is a sequence $\mathcal{L}=\langle V_0,V_1,\ldots,V_h\rangle$ such that $\{V_0,V_1,\ldots,V_h\}$ is a partition of $V(G)$ and there is no edge $vw\in E(G)$ such that $v\in V_i$, $w\in V_j$ and $|j-i|>1$.  For any partition $\mathcal{P}=\{S_1,\ldots,S_p$ of $V(G)\}$, a \emph{quotient graph} $H=G/\mathcal{P}$ has a $p$-element vertex set $V(H)=\{x_1,\ldots,x_p\}$ and $x_ix_j\in E(H)$ if and only there exists an edge $vw\in E(G)$ such that $v\in S_i$ and $w\in S_j$. To highlight the importance of the quotient graph $H$, we call $\mathcal{P}$ an $H$-partition and write this concisely as $\mathcal{P}=\{S_x : x\in V(H)\}$ so that each element of $\mathcal{P}$ is indexed by the vertex it creates in $H$.\footnote{An alternative definition of a layering of $G$ is a layered $H$-partition of $G$ where $H$ is a path.}  A layered $H$-partition $(\mathcal{L},\mathcal{P})$ of a graph $G$ consists of a layering $\mathcal{L}$ of $G$ and an $H$-partition $\mathcal{P}$ of $G$. The \emph{layered width} of $(\mathcal{H},\mathcal{P})$ is $\max\{|L\cap P|: L\in\mathcal{L},\, P\in\mathcal{P}\}$.

Layered $H$-partitions of small layered width in which $H$ has some additional property are a useful tool for a number of graph problems. \citet{dujmovic.joret.ea:planar} show that, if each graph $G$ in some family $\mathcal{G}$ of graphs has a layered $H$-partition for which the layered-width of the partition and the treewidth of $H$ are each upper-bounded by some constant $c_\mathcal{G}$, then each of the following quantities are upper bounded by a constant, for every $G\in\mathcal{G}$: queue number \cite{X}, track number \cite{X}, etc \cite{X}, etc \cite{X}, etc \cite{X}. \citet{dujmovic.esperet.ea:planar} add non-repetitive chromatic number to this list.  \citet{dujmovic.joret.ea:planar} show that two common families of graphs have such $H$-partitions: planar graphs and (more generally) bounded-genus graphs.

In this paper we show that another family of graphs admits layered $H$-partitions of constant layered-width in which $H$ has constant treewidth: $k$-planar graphs, which we now define. An \emph{embedded graph} $G$ is a graph with $V(G)\subset\R^2$ in which each edge $vw\in E(G)$ is a closed curve\footnote{A closed curve is a continuous function $f:[0,1]\to \R^2$. The points $f(0)$ and $f(1)$ are called the \emph{endpoints} of the curve.  When there is no danger of misunderstanding we treat a curve $f$ as a point set $\{f(t):0\le t\le 1\}$.} with endpoints $v$ and $w$ and not containing any vertex of $G$ in its interior.  A \emph{crossing} in an embedded graph $G$ is a triple $(p,vw,xy)$ with $p\in\R^2$, $vw,xy\in E(G)$ and such that $p\in (vw\cap xy)\setminus\{v,w,x,y\}$. An embedded graph $G$ is \emph{$k$-plane} if each edge of $G$ takes part in at most $k$ crossings.  A (not necessarily embedded) graph $G'$ is \emph{$k$-planar} if there exists $k$-plane graph $G$ isomorphic to $G'$.  

Under these definitions $0$-planar graphs are exactly planar graphs and $0$-plane graphs are plane graphs. Thus, like genus-$g$ graphs, $k$-planar graphs are a generalization of planar graphs (which are genus-$0$ graphs).  Unlike genus-$g$ graphs, however, the family of $k$-planar graphs is not minor closed. A graph $G'$ obtained from a $k$-planar graph $G$ by edge deletions and edge contractions may or may not be $k$-planar.

We prove the following result:
\begin{thm}\thmlabel{k-planar}
  Every $k$-planar graph has a layered $H$-partition of layered-width at most $24k^2 + 36k + 12$ in which $H$ has treewidth at most $(3^{k+2}-3)/2$.
\end{thm}

In the special case $k=1$ we obtain better constants and an additional property (planarity) of $H$:

\begin{thm}\thmlabel{1-planar}
  Every 1-planar graph has a layered $H$-partition of layered width at most 30 where $H$ is planar and has treewidth at most 3.
\end{thm}

These results imply that, for any constant $k$, $k$-planar graphs have all the graph parameters mentioned above upper-bounded by some constant $c_k$.

The remainder of the paper is organized as follows: In \secref{1-planar} we prove \thmref{1-planar}.  In \secref{k-planar} we prove \thmref{k-planar}.  In \secref{consequences}, we discuss the implications of Theorems~\ref{thm:k-planar} and \ref{thm:1-planar} for other graph parameters of $k$-planar graphs.


\section{$k$-Planar Graphs}
\seclabel{k-planar}

The purpose of this section is to prove \thmref{k-planar}.  In this section, all graphs are finite, simple, an undirected.  For any graph $G$ and any set $S$ (typically $S\subseteq V(G)$) we use $G[S]$ to denote the graph with vertex set $V(G)\cap S$ and edge set $\{uv\in E(G) : u,v\in S\}$.  We use $G-S$ as a shorthand for $G[V(G)\setminus S]$ and for a graph $G'$, we use $G-G'$ as a shorthand for $G-V(G')$.

We begin by introducing a structural tool for planar graphs, called a PS-tree, that is implicit in previous work on $H$-partitions of planar graphs \cite{dujmovic.joret.ea:planar}.  We then show how the PS-tree of a planarized version of a $k$-planar graph $G$ can be used to find a layered $H$-partition of $G$ of small layered width in which $H$ has small treewidth. 

\subsection{PS-Trees}

Let $T$ be a rooted tree rooted at $r\in V(T)$. The \emph{$T$-depth} of a node $x\in V(T)$ is equal to the length of the path from $x$ to $r$ in $T$. A $P$ path in $T$ is \emph{vertical} if, for each integer $i$, $P$ contains at most one vertex of $T$-depth $i$.  The deepest vertex in a vertical path $P$ is called  $P$'s \emph{lower endpoint} and the other (shallowest) vertex in $P$ is called $P$'s \emph{upper endpoint}. For any node $v\in T$, we call each node on the path, in $T$, from $v$ to $r$ a \emph{$T$-ancestor} of $v$.

A \emph{near-triangulation} $N$ is a plane graph whose outer face, $F$, is bounded by a simple cycle in $N$ and each of whose inner faces is bounded by a 3-cycle in $N$.  A \emph{triangulation} is a near-triangulation whose outer face is bounded by a 3-cycle.  For any cycle $F$ in a near-triangulation $N$ we use $N\circledast F$ to denote the near-triangulation consisting of all the edges and vertices of $G$ that are contained in $F$ or its interior.

For the remainder of this section, $\Delta$ is a triangulation, and $T$ is a BFS spanning tree of $\Delta$ rooted at some degree-3 vertex $r$ on the outer face of $\Delta$.  

Fix some cycle $F$ in $\Delta$ and let $N=\Delta\circledast F$.
Let $\tau=v_1v_2v_3$ be an inner face of $N$ and consider the three minimal vertical paths $Q_1$, $Q_2$, and $Q_3$ in $T$ such that, for each $i\in\{1,2,3\}$ the lower endpoint of $Q_i$ is $v_i$ and the upper endpoint of $Q_i$ is in $V(F)$.  If $Q_1$, $Q_2$, and $Q_3$ are vertex disjoint, then the graph $\overline{Y}=\tau\cup Q_1\cup Q_2\cup Q_3$ is called the \emph{closed tripod} in $N$ determined by $\tau$.  We call $Q_1$, $Q_2$, and $Q_3$ the \emph{legs} of $\overline{Y}$ and we call $\tau$ the \emph{center} of $\overline{Y}$.  We call $Y=\overline{Y}-F$ the \emph{(open) tripod} of $N$ determined by $\tau$.  The graph $M=\overline{Y}\cup F$ is called a \emph{Mercedes graph} and the (at most 3) inner faces of $M$ other than $\tau$ are called the \emph{wedges} of $M$.

A \emph{PS-tree}  $K=K(\Delta, T, r)$ is a 3-ary tree, such that each node $x\in V(K)$ is associated with several object (refer to \figref{K-node}):

\begin{figure}
  \begin{center}
    \begin{tabular}{cc}
      \includegraphics{figs/K-node-1} &
      \includegraphics{figs/K-node-2}
    \end{tabular}
  \end{center}
  \caption{The elements associated with a node $x\in V(K)$.}
  \figlabel{K-node}
\end{figure}

\begin{enumerate}
  \item A cycle $F_x$ in $\Delta$ and the near-triangulation $N_x=\Delta\circledast F_x$.
  
  \item An inner face $\tau_x$ of $N_x$ that determines a closed tripod $\overline{Y}_x$ and an open tripod $Y_x$ in $N_x$.
  
  \item The \emph{Mercedes graph} $M_x=F_x\cup \overline{Y}_x$
\end{enumerate}

The nodes of the PS-tree $K$ satisfy several conditions:
\begin{enumerate}[(PR1)]
  \item For the root $r_K$ of $K$, $F_{r_K}$ is the 3-cycle in $\Delta$ formed by the three vertices adjacent to $r$.
  
  \item \proplabel{children} For each node $x\in V(K)$ such that $M_x$ has $f\le 3$ wedges $F_{x,1},\ldots,F_{x,f}$, the node $x$ has exactly $f$ children $y_1,\ldots,y_f$ in $K$ and $F_{y_i} = F_{x,i}$ for each $i\in\{1,\ldots,f\}$.
  
  \item \proplabel{ancestor-boundary} For each non-root node $x\in V(T)$, there exists a partition of $V(F_x)$ into $g\le 3$ sets $P_1,\ldots,P_g$ such that, for each $i\in\{1,\ldots,g\}$, $F[P_i]$ is a path that is contained in $Y_a$ for some $K$-ancestor $a$ of $x$ \emph{or} $F[P_i]$ is a path or cycle in $F_{r_K}$.

  \item \proplabel{partition} $\{Y_x : x\in V(K)\}$ is a partition of $V(\Delta)\setminus\{r\}\setminus V(F_{r_K})$.
\end{enumerate}

For any triangulation $\Delta$ in which the outer face has a vertex $r$ of degree 3 and for any BFS spanning-tree of $\Delta$ rooted at $r$, there exists a PS-tree $K=K(\Delta,T,r)$.  Indeed, this tree is constructed (albeit implicitly) by \citet[Proof of Lemma~14]{dujmovic.joret.ea:planar}.

% The following property of a PS-Tree $K$ is an immediate consequence of \propref{children} that will be useful for us: \note{PM}{Do we still use this?}
% \begin{compactenum}[(PR5)]
%   \item \proplabel{ancestors} If $vw\in E(\Delta)$, $v\in V(N_x-F_x)$ and $w\in V(N_y-N_y)$ for some nodes $x,y\in V(K)$, then one of $x$ or $y$ is a $K$-ancestor of the other.
% \end{compactenum}

\subsection{$k$-Planar Graphs}

Let $G$ be a $k$-plane graph.  We will assume, for ease of exposition, that any point $p\in\R^2$ is involved in at most one crossing $(p,vw,xy)$ of $G$. This assumption is not critical since it can be enforced by a slight deformation of the edges of $G$. 

Let $G^+$ be the graph obtained as follows:
\begin{compactenum}
  \item Add a dummy vertex at each crossing in $G$ to obtain a plane graph and then add edges to this graph to obtain a triangulation $G^+$.
  \item Add a complete graph $K_4$ that contains $G^{+}$ in one of its inner faces, $F_0$, and then add edges to the resulting graph to obtain a triangulation $G^{++}$.
\end{compactenum}
Let $r$ be the vertex of the $K_4$ not incident on $F_0$ and let $T$ be a BFS spanning tree of $G^+$ rooted at $r$. Observe that, for any edge $vw\in E(G)$, $\dist_{G^+}(v,w) \le k+1$ and therefore, for any two vertices $v,w\in V(G)$,
\begin{equation} 
  \dist_{G^+}(v,w) \le (k+1)\cdot \dist_G(v,w) \eqlabel{distance-preserving}
\end{equation}
For each integer $i\ge 0$, let $V_i=\{v\in V(G^+): \dist_{T}(r,v)=i\}$ and observe that $\{V_0,V_1,\ldots,V_{h}\}$ is a layering of $G^+$ (but not of $G$).

Let $K=K(G^{++},T,r)$ be a PS-tree and observe that the root $r_K$ of $K$ has $F_{r_K}=F_0$.  This implies:
\begin{compactenum}[(PR1)]\setcounter{enumi}{4}
  \item \proplabel{all-in}  $V(G^+)\subseteq V(N_{r_K}-F_{r_K})$.
\end{compactenum}

For each vertex $v\in V(G^+)$, let $P_v=\{a\in V(K): v\in V(N_a-F_a)\}$.  
By \propref{all-in}, $P_{v}$ contains the root $r_K$ of $K$.  By \propref{children}, $K[P_v]$ is a vertical path (whose lower endpoint is the unique node $x\in V(K)$ such that $v\in Y_x$). Therefore $K[P_v]$ is a vertical path in $K$ that contains $r_K$.  We are about to repeatedly use the fact that the common intersection of two or more vertical paths is a vertical path. 

For each edge $vw\in E(G)$, consider the walk $W_{vw}$ in $G^+$ that corresponds to the sequence of vertices in $G^+$ that we encounter while walking along the curve $vw$ from $v$ to $w$. We define $A_{vw}=\bigcap_{z\in W_{vw}} P_z$.  Since $W_{vw}\subseteq V(G^+)$, $A_{vw}$ is well-defined.  For each $z\in W_{vw}$, $K[P_{z}]$ is a vertical path in $K$ that contains $r_K$.   Therefore $A_{vw}$ is a non-empty vertical path in $K$.  For any vertex $v\in V(G)$ define $A_v=\bigcap_{vw\in E(G)} A_{vw}$.  Again, $K[A_v]$ is a non-empty vertical path in $K$ that contains $r_K$.  We define $a(v)$ to be the lower endpoint of $K[A_v]$.  For every node $x\in V(K)$, define the \emph{separator}
\[
   S_x = \{v\in V(G): a(v)=x \} \enspace .
\]
Clearly $\mathcal{S}=\{S_x:x\in V(K)\}$ is a partition of $V(G)$.  We will use $\mathcal{S}$ as the partition that defines our quotient graph $H=G/\mathcal{S}$ with vertex set $V(H)=V(K)$ and the edge $xy\in E(H)$ if and only if there exists an edge $vw\in E(G)$ with $v\in S_x$ and $w\in S_y$.  First we argue that this partition has small width with respect to the layering defined by $T$ (recall that this is a layering of $G^+$, not a layering of $G$).

\begin{clm}\clmlabel{chargeback-to-y}
   For any $x\in V(K)$ and any vertex $v\in S_x$, there exists a vertex $v'\in Y_x$ such that $|\dist_T(r,v')-\dist_T(r,v)|\le k+1$.  
\end{clm}

\begin{proof}
  Since $v\in S_x$, the lower endpoint of $K[A_{vw}]$ is $x$ for some edge $vw\in E(G)$.  This implies that $V(W_{vw})\subset V(N_x-F_x)$ but $V(W_{vw}) \not\subseteq V(N_{x'}-F_{x'})$ for any any child $x'$ of $x$ in $K$.  This implies that $W_{vw}$ contains a vertex $v'\in Y_x$.  Since $G$ is $k$-planar and $v'\in V_{vw}$ this implies that $\dist_{G^+}(v,v') \le k+1$.
  By the triangle inequality,     
  \[
    \dist_{G^+}(r,v') \le \dist_{G^+}(r,v) + \dist_{G^+}(v,v') \le \dist_{G^+}(r,v) + k + 1 \enspace .
  \]
  Similarly, $\dist_{G^+}(r,v) \le \dist_{G^+}(r, v') + k+1$.  Therefore
  $|\dist_{G^+}(r,v)-\dist_{G^+}(r,v')| \le k+1$ which establishes the claim, since $T$ is BFS tree, so $\dist_{T}(r,z)=\dist_{G^+}(r,z)$ for all $z\in V(G^+)$.
\end{proof}

\begin{clm}\clmlabel{width-of-g-plus}
  For any $x\in V(H)$, and any integer $i\ge 0$, $|S_x\cap V_i|\le 24k+12$.
\end{clm}

\begin{proof}
  Since the vertices of $Y_x$ come from at most 3 vertical paths in $T$, $|V(Y_x)\cap V_i|\le 3$.  \clmref{chargeback-to-y} shows that, for any $v\in S_x\cap V_i$, there exists $v'\in Y_x$ with $\dist_T(v,v')\le k+1$. Therefore, $v'\in V(Y_x)\cap V_{j}$ for some $j\in\{i-k-1,\ldots,i+k+1\}$.  If $v'=v$ (so $v'\in V_i$) then $v'$ contributes only one vertex to $S_x$.  If $v'\neq v$ then $v'\in V(G^+-G)$ is a point where two edges of $G$ cross and $v'$ contributes at most the $4$ endpoints of these edges to $S_x$. In either case every $v'\in V(Y_x)\cap \bigcup_{j=i-k-1}^{i+k+1} V_j$ contributes at most $4$ vertices to $S_x\cap V_i$, so $|S_x\cap V_i|\le 3\times 4\times (2k+1) = 24k+12$.
\end{proof}

Next we describe a tree decomposition $(B_x:x\in V(K))$ of $H$.  We deliberately use the same node set $V(H)=V(K)$ here, because the subtree $K[x]:=K[\{z:x\in B_z\}]$ will be rooted at $x$.  

\begin{clm}\clmlabel{xy-ancestor}
   For every edge $xy\in V(H)$, one of $x$ or $y$ is a $K$-ancestor of the other.
 \end{clm}
 
 \begin{proof}
   Since $xy\in E(H)$, there exists an edge $vw\in E(G)$ such that $v\in S_x$ and $w\in S_y$.  By definition, $x\in A_v\subseteq A_{vw}$ and $y\in A_w\subseteq A_{vw}$.  Since $K[A_{vw}]$ is a vertical path in $K$, it follows that one of $x$ or $y$ is a $K$-ancestor of the other. 
\end{proof}

The tree decomposition $(B_x:x\in V(K))$ of $H$ is defined as follows: For each edge $xy\in E(H)$ where $y$ is a $K$-ancestor of $x$, we add $y$ to $B_{x'}$ for every node $x'$ on the path from $x$ to $y$ in $K$.  That this produces a tree-decomposition is due to \clmref{xy-ancestor}.  What remains is to upper bound the width of this decomposition.

\begin{clm}\clmlabel{bag-size}
  The tree decomposition $\{B_x: x\in V(K)\}$ of $H$ contains no bag larger than $(3^{k+2}-3)/2$.
\end{clm}

\begin{proof}
  Fix some node $x\in V(K)$ and consider the contents of $B_x$.  By \clmref{xy-ancestor}, $B_x$ contains only $K$-ancestors of $x$ (including $x$ itself). 
  
  Consider the graph $A$ that contains the edges and vertices of $F_{a}$ for all $K$-ancestors $a$ of $x$. See \figref{A}.  It is helpful to think of this graph as follows:  We begin with $F_{a_0}$ where $a_0=r_K$ is the root of $K$.  We then cut $F_{a_0}$ into two pieces $F^{\bar{x}}_{a_0}$ and $F^x_{a_0}$ where $F^x_{a_0}$ is the piece that contains $F_x$. (The path that makes this cut contains two legs $Q_i$ and $Q_j$ of a tripod $Y$ and one edge $v_iv_j$ of its central triangle $\tau$.)  This process continues by partitioning $F_{a_1}:=F^x_{a_0}$ recursively.  The process terminates after some number $d$ of iterations, when $F^x_{a_{d-1}}=F_x=F_{d}$.

  \begin{figure}
    \begin{center}
        \includegraphics{figs/A-1}
    \end{center}
    \caption{The graph $A$.}
    \figlabel{A}
  \end{figure}

  If $B_x$ contains some $K$-ancestor $a_i$, $i\in\{0,\ldots,d-1\}$, of $x$, it is because there is an edge $vw\in E(G)$ with $v\in S_x$ and $w\in S_{a_i}$.  The edge $vw$ corresponds to a walk $W_{vw}$ in $G^+$. The walk $W_{vw}$ is contained in $N_x-F_x$, otherwise $x\not\in A_{vw}$ so $x\not\in A_v$, and $a(v)\neq x$, contradicting the fact that $v\in S_x$.
  
  However, $w\in S_a$, so there exists an edge $ww'\in E(G)$ such that   $W_{ww'}$ is contained in $V(W_{ww'})\subseteq V(N_{a_i}-F_{a_i})$ but $V(W_{ww'}\not\subseteq V(N_{a_{i+1}}-F_{a_{i+1}})$.  Therefore $W_{ww'}$ must contain a first vertex $w^*\in V(F_{a_{i+1}}-F_{a_{i}})$.

  Summarizing, we have a walk $w'=v_0,\ldots,v_\ell=w^*$ in $G^+$ such that
  \begin{compactenum}[(i)]
    \item $v_0\in V(N_x-F_x)$;
    
    \item $v_0,\ldots,v_{\ell-1}\in V(N_{a_{i+1}}-F_{a_{i+1}})$;
    
    \item $v_\ell\in V(F_{a_{i+1}}-F_{a_i})$; and
    
    \item $\ell\le k+1$.
  \end{compactenum}
  We call any walk with properties (i)--(iv) an \emph{$i$-walk}. Therefore, to upper-bound $|B_x|$, it suffices to upper-bound the number of integers $i\in\{0,\ldots,d-1\}$ such that there exists an $i$-walk.
  
  For each $v'\in V(G^+)$, 
  \[  
      d_{v'}=\max\{j\in\{0,\ldots,d\} : v'\in V(N_{a_j}-F_{a_j})
  \]
  (Intuitively, $d_{v'}$ is the $K$-depth of the lower endpoint of $K[P_{v'}]$ or the $K$-depth ($d$) of $x$, whichever is shallower.)
     
  Let $v_0,\ldots,v_\ell$ be an $i$-walk and and consider the subsequence $d_0,\ldots,d_{\ell'}$ of $d_{v_0},\ldots,d_{v_\ell}$ consisting of only $d_{v_0}$ and those (record-breaking) $d_{v_i}$ such that $d_{v_i}<\min\{d_{v_0},\ldots,d_{v_{i-1}}\}$.  Observe that $d_0=d$ and $d_{\ell'}=i$.
  
  By \propref{ancestor-boundary}, for every $j\in\{1,\ldots,d\}$, there is a set $X_{j}$, of at most 3 integers $\alpha < j$ such that such that every vertex $q\in V(F_{a_j})$ is in $V(Y_{a_{\alpha}})$ for some $\alpha\in X_j$.
  Assign a distinct colour $c(j,\alpha)\in\{1,2,3\}$ for each $\alpha\in X_j$ and do this assignment for each $j\in\{1,\ldots,d\}$.  By definition, for any $j\in\{1,\ldots,d\}$ and any colour $\gamma\in\{1,2,3\}$, there is at most one $\alpha=\alpha(j,\gamma)$ such that $c(j,\alpha)=\gamma$.
  
  Finally, observe that, for every $j\in\{1,\ldots,\ell'\}$,  $c(d_j,d_{j+1})$ is defined.  Refer to \figref{colouring}. Therefore the \emph{colour sequence} $c_0,\ldots,c_{\ell'-1}$ where $c_j=c(d_j,d_{j+1})$ is well defined.  Since $d_0=d$ and $d_{\ell'-1}=i$, from the preceding discussion, we have $\alpha(\cdots (\alpha(\alpha(\alpha(d,c_0),c_1),c_2)\cdots),c_{\ell'-1}) = i$.

  \begin{figure}
    \begin{center}
        \includegraphics{figs/A-2}
    \end{center}
    \caption{The colour sequences defined by two $i$-curves.}
    \figlabel{colouring}
  \end{figure}

  The proof is now almost complete.  Any $i$-walk defines a colour sequence $c_0,\ldots,c_{\ell'}$.  If some $i'$-walk defines the same colour sequence $c_0,\ldots,c_{\ell'}$ then
  \[
        i = (\alpha(\alpha(\alpha(d,c_0),c_1),c_2)\cdots),c_{\ell'-1}) = i' \enspace ,
  \]
  so $i=i'$.  Since there are at most $3^{\ell'}$ colour sequences of length $\ell'$ and $\ell'\le \ell \le k+1$, the number of $i\in\{0,\ldots,d-1\}$ such that there exists an $i$-walk is at most
  \[
     |B_x| \le \sum_{\ell=1}^{k+1} 3^\ell = (3^{k+2}-3)/2 \eqlabel{bag-size} \enspace . \qedhere
  \]
\end{proof}

We now have all the parts needed for the proof of \thmref{k-planar}.

\begin{proof}
  We use the layered $H$-partition $(\mathcal{L}, \mathcal{S})$, where $\mathcal{S}=\{S_x:x\in V(K)\}$ and $\mathcal{L}=\{V_i': i=0,1,2\ldots\}$ where $V_i' = V_{(k+1)i}\cup\cdots\cup V_{(k+1)(i+1)-1}$.  \Eqref{distance-preserving} shows that $\mathcal{L}$ is indeed a layering of $G$.

  By \clmref{bag-size}, $\mathcal{S}$ produces a graph $H=G/\mathcal{S}$ of treewidth $(3^{k+2}-1)/2$.  By \clmref{width-of-g-plus}, for each $x\in V(H)$ and each integer $i$, $|B_x\cap V_i|\le 24k+12$, so $|B_x\cap V_i'|\le (k+1)(24k + 12) = 24k^2+36k+12$. 
\end{proof}

\section{$1$-Planar Graphs}
\seclabel{1-planar}

Let $G$ be an edge-maximal 1-plane multigraph.  Here, edge-maximal should be taken to mean that, if any two vertices $u$ and $v$ appear on a common face\footnote{The \emph{faces} of an embedded graph $G$ are the connected components of $\R^2\setminus \bigcup_{vw\in E(G)} vw$.  We say that a vertex $v\in V(G)$ appears on a face $F$ if $v$ is contained in the closure of $F$.} $F$, then there is an edge $uv\in E(G)$ that is contained in the boundary of $F$.

A \emph{kite} in $G$ is the subgraph $K=G[\{v,w,x,y\}]$ induced by the endpoints of a pair of crossing edges $vw,xy\in E(G)$.  It follows from edge-maximality that every kite is isomorphic to the complete graph $K_4$.
The edges $vw$ and $xy$ are called \emph{spars} of $K$.  The cycle $vxwy$ is called the \emph{sail} of $K$.  It follows from edge-maximality that none of the edges $vx$, $xw$, $wy$, or $yv$ are crossed by any other edges of $G$. Thus any edge that is a spar of a kite $K$ is not part of a sail of any kite $K'$. Observe that any spar of $K$ is incident on exactly four \emph{kite faces} of $G$, each of which has three edges and two vertices of $G$ on its boundary.

The 1-plane graph $G$ has a plane triangulation $G'$ as a subgraph that can be obtained by removing one spar from each kite in $G$.  Observe that, for any spar $vw\in E(G)\setminus E(G')$ that crosses $xy\in E(G')$, $G'$ contains the path $vxw$ (and $vyw$).  It follows that $\dist_{G'}(x,y)\le 2\cdot\dist_G(x,y)$ for any pair of vertices $x,y\in V(G)$.

Our proof of \thmref{1-planar} follows quickly from the following technical lemma, which is an extension of Lemma~X in \citet{dujmovic.joret.ea:planar}.
\begin{lem}\lemlabel{induction} The setup:
  \begin{compactenum}
    \item Let $G$ and $G'$ be defined as above.
    \item Let $T$ be a BFS tree of $G'$ rooted at some vertex $r$.
    \item For each integer $i\ge 0$, let $V_i=\{v\in V(G):\dist_T(r,v)=i\}$. 
    \item Let $C$ be a cycle in $G'$ with $r$ in the exterior of $C$ and such that
    \begin{compactenum} 
      \item No edge of $C$ is crossed by any edge of $G$; and
      \item $V(C)$ can be partitioned into $P_1,\ldots,P_k$, $k\le 3$ such that for each $i\in\{1,\ldots,k\}$,
      \begin{compactenum}
        \item $C[P_i]$ is a path; and
        \item $|P_i\cap V_j| \le 15$ for all integers $j\ge 0$.
      \end{compactenum}
    \end{compactenum}
    \item Let $N$ and $N'$ be the subgraphs of $G$ and $G'$ consisting only of those edges and vertices contained in $C$ or the interior of $C$.
  \end{compactenum}
  Then $N$ has an $H$-partition $(B_x : x\in V(H))$ such that
  \begin{compactenum}
    \item $H$ is planar;
    \item for all integers $j\ge 0$, and all $x\in V(H)$, $|B_x\cap V_j|\le 15$; 
    \item for each $i\in\{1,\ldots,k\}$, there exists some $x_i\in V(H)$ such that $P_i=B_{x_i}$; and
    \item $H$ has a tree decomposition whose largest bag has size at most 4 and such that some bag contains $x_1,\ldots,x_k$.
  \end{compactenum}
\end{lem}

\begin{proof}
  The proof is by induction on the number of vertices of $N$.
  First note that $N'$ is a near-triangulation.  If $k=3$, set $R_i=P_i$ for each $i\in\{1,2,3\}$.  Otherwise, as before, split $P_1,\ldots,P_k$ to partition $V(C)$ into three sets $R_1$, $R_2$, and $R_3$ such that each $C[R_i]$ is a path and each $R_i$ contains vertices from at most one of $P_1,\ldots,P_k$. 
  
  For each $i\in\{1,2,3\}$ and $v\in R_i$, assign the colour $i$ to $v$.
  For each $v\in V(N')$, consider the path $P_v$ that consists of the subpath of the $v$ to $r$ path in $T$ that stops at the first vertex $v'\in V(C)$. Assign $v$ to have the same colour as $v'$.
  
  Sperner's Lemma ensures that $G'$ contains a triangular face $\tau=v_1v_2v_3$ whose three vertices are assigned different colours. For each $i\in\{1,2,3\}$, let $Q_i=P_{v_i}$ be the path in $T$ form $v_i$ to $v_i'$.  Let $Y$ denote the subgraph of $N'$ consisting of vertices and edges $Q_1$, $Q_2$, $Q_3$, and $\tau$.  Let $Y^+$ denote the subgraph of $N$ consisting of the vertices and edges of $Y$ plus the vertices and edges of every kite formed by a crossing between an edge of $G$ and an edge of $Y$.
  
  We claim that, for each integer $i\ge 0$, $|V(Y^+)\cap V_i|\le 15$.  To see this, first observe that $Y$ contains at most 3 vertices of $V_i$. If a vertex $x\in V(Y^+)\setminus V(Y)$ is contained in $V_i$, then this is because $Y$ contains an edge $vw$ with $v\in V_{i'}$ and $w\in\{V_{i'+1}$ for some $i'\in\{i-1,i\}$ and $G$ contains an edge $xy$ that crosses $vw$. (Recall that $vx,xw\in E(G')$, so $\dist_{G'}(w,r)-1\le\dist_{G'}(x,r)\le\dist_{G'}(v,r)+1$.)  The graph $Y$ contains at most 6 such edges, each of which accounts for at most 2 additional vertices of $V_i$.  Therefore, in total, $|Y^+\cap V_i|\le 15$.

  
  Finally, let $S$ and $S^+$ denote the subgraph of $G$ containing the edges and vertices of $Y$, respectively $Y^+$, and the edges and vertices of $C$.  The graph $S^+$ has some number of bounded faces, all contained in the interior of $C$. Some of the bounded faces of $S$ are kite faces. Call the non-kite bounded faces $F_1,\ldots,F_m$ and let $C_1,\ldots,C_m$ denote their boundaries.  We claim that, for each $i\in\{1,\ldots,m\}$, if some portion of $C_i$ is contained in an edge $vw\in E(G)$ then $vw$ is not crossed by any edge of $G$.  To see this, there are three cases to consider:
  \begin{enumerate}
    \item $vw\in E(C)$. By assumption, $vw$ is not crossed by any edge of $G$.
    \item $vw\in E(Y)$. In this case, the kite containing $vw$ is in $Y^+$, so $vw$ is only incident to kite faces.
    \item $vw\in E(Y^+)\setminus E(Y)$. In this case, either $vw$ is a sail edge in which case it is not crossed by definition, or $vw$ is a spar that was added to $Y^+$ because $vw$ crosses some edge $xy\in E(Y)$.  In this latter case, $vw$ is only incident to kite faces.
  \end{enumerate}
  Therefore each $C_i$ is a cycle in $G^+$ consisting entirely of uncrossed edges. The vertices of $C_i$ can be partitioned into at most three sets $P_1'$, $P_2'$, and $P_3'$ where $P_1'\subset V(Y^+)$, $P_2'\subseteq P_a$ and $P_3'\subseteq P_b$ for some $a,b\in\{1,2,3\}$. Furthermore $C_i[P_j']$ is a path for each $j\in\{1,\ldots,3\}$. Finally, the subgraph $N_i$ of $G$ consisting of the edges and vertices of $G$ contained in $C_i$ or its interior does not contain one of the three vertices of $\tau$. Therefore, we can apply induction using the cycle $C_i$ and the partition $P_1',P_2',P_3'$ of $V(C_i)$ to obtain the desired $H$-partition and tree decomposition of $N_i$.
  
  The remainder of the proof finishes in the same way as the proof of Lemma~X in \cite{dujmovic.joret.ea:planar}.  $P_1,\ldots,P_k$, and $V(Y^+)\setminus V(C)$ become elements of the $H$-partition.  Elements in each of the $H$-partitions of $N_1,\ldots,N_3$ that intersect $P_1,\ldots,P_k$, or $V(Y^+)\setminus V(C)$ are discarded and all the resulting sets are combined to obtain an $H$-partition of $G$.  The desired tree decomposition of $G$ is obtained in exactly the same way as in the proof of Lemma~X in blah.
  
  The planarity of $H$ comes from the fact that if two edges $vw$ and $xy$ cross, then they end up in the same bag of the $H$-partition.  This means that $H$ is actually obtained by contracting connected sets of vertices in the planar graph $G'$.  It is well know that any graph obtained by contracting connected sets of vertices in a planar graph is also planar.
\end{proof}


\begin{proof}[Proof of \thmref{1-plane}]
  The same as before except that the layered width 15 from \lemref{induction} becomes 30 because \lemref{induction} uses a layering of $G'$ and distances in $G'$ can be a factor of 2 larger than in $G$.
\end{proof}

\section{Consequences}
\seclabel{consequences}

\bibliographystyle{plainnat}
\bibliography{k-planar}

\end{document}
