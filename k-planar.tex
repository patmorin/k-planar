\documentclass{patmorin}
\usepackage[T1]{fontenc}
\usepackage[utf8]{inputenc}
\usepackage{amsmath}
\usepackage{amsfonts}
\usepackage{amsthm}
\usepackage{graphicx}
\usepackage{enumerate}
\usepackage{pat}
\usepackage{paralist}
\usepackage{stmaryrd}

\usepackage[longnamesfirst,numbers,sort&compress]{natbib}

\setlength{\parskip}{1ex}

\newcommand{\note}[2]{{\color{red}[#1:~#2]}}

\DeclareMathOperator{\dist}{dist}
\DeclareMathOperator{\depth}{depth}
\DeclareMathOperator{\ff}{f}
\DeclareMathOperator{\tw}{tw}

\renewcommand{\proplabel}[1]{\label{prop:#1}}
\renewcommand{\propref}[1]{(PR\ref{prop:#1})}
\newcommand{\jlabel}[1]{\label{j:#1}}
\newcommand{\jref}[1]{(J\ref{j:#1})}
\newcommand{\tlabel}[1]{\label{t:#1}}
\newcommand{\tref}[1]{(T\ref{t:#1})}
\newcommand{\ylabel}[1]{\label{y:#1}}
\newcommand{\yref}[1]{(Y\ref{y:#1})}

\newcommand{\Bagsize}{\ensuremath{\frac{k^3}{6} + \frac{3k^2}{2} + \frac{13k}{3} + 4}}
\newcommand{\bagsize}{\ensuremath{k^3/6 + 3k^2/2 + 13k/3 + 4}}
\newcommand{\treewidth}{\ensuremath{k^3/6 + 3k^2/2 + 13k/3 + 3}}

\title{\MakeUppercase{The Structure of $k$-Planar Graphs}%
    \thanks{This work was partly funded by NSERC and MRI.}}

\author{Vida Dujmović%
        \thanks{School of Computer Science and Electrical Engineering,
                University of Ottawa},\,\,
        Pat Morin%
        \thanks{School of Computer Science, Carleton University},\,\, and
        David R. Wood%
        \thanks{School of Mathematics, Monash University}}

%\renewcommand{\ge}{\geqslant}
%\renewcommand{\le}{\leqslant}

\begin{document}
\begin{titlepage}
\maketitle


\begin{abstract}
  Layered partitions of graphs (with small layered width, in which the quotient has small treewidth) are a recently-introduced tool that have been used to solve longstanding problems on queue layouts, non-repetitive colouring, and 3-d graph drawing.  Such partitions are known to exist for planar graphs and, more generally, apex-minor-free graphs.  In the current paper, we prove that every $k$-planar graph has a layered partition of layered width $O(k^2)$ in  which the quotient graph has treewidth $O(k^3)$. This is the first result of this type for a non-minor-closed class of graphs and implies that $k$-planar graphs have both queue number and non-repetitive chromatic number upper-bounded by a function of $k$. The former result was previously shown by combining layered partitions of planar graphs with \textit{ad hoc} methods. The latter result is new, for all $k\ge 1$.  These results extend to $(g,k)$-planar graphs, the natural generalization of $k$-planar graphs to to genus-$g$ surfaces (rather than the genus-$0$ plane).
\end{abstract}
\end{titlepage}

\tableofcontents
\newpage
\section{Introduction}

In 1965, \citet{ringel:sechsfarbenproblem} introduced the notion of $1$-planar graphs, a generalization of planar graphs that allows edges to cross provided that no edge is involved in more than one crossing. This simple generalization has since become a rich source of problems and results.  Indeed, the annotated bibliography on 1-planar graphs by \cite{kobourov.liotta.ea:annotated} contains more than 140 entries.

The natural generalization of $1$-planar graphs, $k$-planar graphs (that allow edge to be involved in up to $k$ crossings), has been the topic of important recent work on graph drawing. Though planar graphs are, by now, quite well-understood, the situation is much less clear for $k$-planar graphs; even the maximum edge-density of $k$-planar graphs is not completely settled.  For $k\le 4$, the maximum number of edges in an $n$-vertex $k$-planar graph is $(k+3)(n-2)$, which is tight for $k=1,2$ \cite{pach.toth:graphs}.  A near-tight bound of $11(n-2)/2$ is known for $k=3$ \cite{pach.radoicic.ea:improving} but for $k>4$, the current best upper bound is $4.108\sqrt{k}n$ \cite{pach.toth:graphs}.

% In the current paper we investigate the structure of $k$-planar graphs.  Since planarity is closed under taking minors, planar graphs can be characterized by a finite set of forbidden minors.  Indeed, Wagner's Theorem states that this set consists of the complete 5-vertex graph $K_5$ and the complete bipartite graph $K_{3,3}$.  Unfortunately, $k$-planarity is not closed under taking minors, ruling out this kind of clean structural result.  The purpose of the current paper is to extend a new structural result for planar graphs to $k$-planar graphs.

The purpose of the current paper is to show that new structural results for planar graphs can be extended to $k$-planar graphs.  These results can be used to give a rough characterization of $k$-planar graphs of powers of bounded-degree planar graphs
% Rather, we present a rough structural characterization of $k$-planar graphs that, despite its roughness, provides several new results for $k$-planar and related graph families.  
All of these results are based on layered $H$-partitions, which we now define.
{\color{red}TODO: Help.}

A \emph{layering} of a graph $G$ is a sequence $\mathcal{L}=\langle V_0,V_1,\ldots\rangle$ such that $\{V_0,V_1,\ldots\}$ is a partition of $V(G)$ and there is no edge $vw\in E(G)$ such that $v\in V_i$, $w\in V_j$ and $|j-i|>1$.  For any partition $\mathcal{P}=\{S_1,\ldots,S_p\}$ of $V(G)$, a \emph{quotient graph} $H=G/\mathcal{P}$ has a $p$-element vertex set $V(H)=\{x_1,\ldots,x_p\}$ and $x_ix_j\in E(H)$ if and only if there exists an edge $vw\in E(G)$ such that $v\in S_i$ and $w\in S_j$. To highlight the importance of the quotient graph $H$, we call $\mathcal{P}$ an \emph{$H$-partition} and write this concisely as $\mathcal{P}=\{S_x : x\in V(H)\}$ so that each element of $\mathcal{P}$ is indexed by the vertex it creates in $H$.\footnote{An alternative definition of a layering of $G$ is an $H$-partition of $G$ where $H$ is a path.}  A layered $H$-partition $(\mathcal{L},\mathcal{P})$ of a graph $G$ consists of a layering $\mathcal{L}$ of $G$ and an $H$-partition $\mathcal{P}$ of $G$. The \emph{layered width} of $(\mathcal{L},\mathcal{P})$ is $\max\{|L\cap P|: L\in\mathcal{L},\, P\in\mathcal{P}\}$.

Layered $H$-partitions of small layered width in which $H$ has some additional property are a useful tool for a number of graph problems.  One such useful property is small \emph{treewidth}, which we now define. A \emph{tree-decomposition} $\mathcal{K}$ of a graph $G$ consists of a tree $K$ and a collection $\mathcal{K}=(B_x:x\in V(K))$ of subsets of $V(G)$ indexed by nodes of $K$ such that (i)~for every $vw\in E(G)$, there exists some $x\in V(K)$ with $v,w\in B_x$; and (ii)~for every $v\in V(G)$, the induced subgraph $K[x] := K[\{y: x\in B_y\}]$ is connected.  The \emph{width} of the tree-decomposition $\mathcal{K}$ is $\max\{|B_x|:x\in V(K)\}-1$.  The \emph{treewidth} $\tw(G)$ of a graph $G$ is the minimum width of a tree-decomposition of $G$.  

 \citet{dujmovic.joret.ea:planar} show that, if each graph $G$ in some family $\mathcal{G}$ of graphs has a layered $H$-partition for which the layered width of the partition and the treewidth of $H$ are each upper-bounded by some constant $c_\mathcal{G}$, then each of the following quantities are upper bounded by a constant, for every $G\in\mathcal{G}$: queue number, track number, and layered treewidth. \citet{dujmovic.esperet.ea:planar} add non-repetitive chromatic number to this list.  \citet{dujmovic.joret.ea:planar} show that two common families of graphs have such $H$-partitions: planar graphs, graphs of bounded genus, and (more generally) apex-minor-free graphs.\footnote{A graph $M$ is a \textit{minor} of a graph $G$ if a graph isomorphic to $M$ can be obtained from a subgraph of $G$ by contracting edges. A class $\mathcal{G}$ of graphs is \emph{minor-closed} if for every graph $G\in\mathcal{G}$, every minor of $G$ is in $\mathcal{G}$. A minor-closed class is \emph{proper} if it is not the class of all graphs. For example, for fixed $g\geq 0$, the class of graphs with Euler genus at most $g$ is a proper minor-closed class.
 
 A graph $G$ is $t$-apex if it contains a set $A$ of at most $t$ vertices such that $G-A$ is planar. A 1-apex graph is \emph{apex}.  A minor-closed class $\mathcal{G}$ is apex-minor-free if some apex graph is not in $\mathcal{G}$.}

% In this paper we show that another family of graphs admits layered $H$-partitions of bounded layered width in which $H$ has bounded treewidth:
In the current paper we show that $k$-planar graphs also admit layered $H$-partitions in which $H$ has bounded treewidth. First, we formally define $k$-planar graphs.  An \emph{embedded graph} $G$ is a graph with $V(G)\subset\R^2$ in which each edge $vw\in E(G)$ is a closed curve\footnote{A closed curve in a surface $\Sigma$ is a continuous function $f:[0,1]\to \Sigma$. The points $f(0)$ and $f(1)$ are called the \emph{endpoints} of the curve.  When there is no danger of misunderstanding we treat a curve $f$ as the point set $\{f(t):0\le t\le 1\}$.} in $\R^2$ with endpoints $v$ and $w$ and not containing any vertex of $G$ in its interior.  A \emph{crossing} in an embedded graph $G$ is a triple $(p,vw,xy)$ with $p\in\R^2$, $vw,xy\in E(G)$ and such that $p\in (vw\cap xy)\setminus\{v,w,x,y\}$. An embedded graph $G$ is \emph{$k$-plane} if each edge of $G$ takes part in at most $k$ crossings.  A (not necessarily embedded) graph $G'$ is \emph{$k$-planar} if there exists $k$-plane graph $G$ isomorphic to $G'$.  

Under these definitions, $0$-planar graphs are exactly planar graphs and $0$-plane graphs are exactly plane graphs. Thus, like genus-$g$ graphs,\footnote{The \textit{Euler genus} of the orientable surface with $h$ handles is $2h$. The \textit{Euler genus} of the non-orientable surface with $c$ cross-caps is $c$. The \textit{Euler genus} of a graph $G$ is the minimum integer $g$ such that $G$ embeds in a surface of Euler genus $g$. Of course, a graph is planar if and only if it has Euler genus 0; see \citep{mohar.thomassen:graphs} for more about graph embeddings in surfaces.} $k$-planar graphs are a generalization of planar graphs (which are genus-$0$ graphs).  Unlike genus-$g$ graphs, however, the family of $k$-planar graphs is not minor-closed.  A graph $G'$ obtained from a $k$-planar graph $G$ by edge deletions and edge contractions may or may not be $k$-planar. Indeed, \citet{dujmovic.eppstein.ea:structure} construct 1-planar graphs that contain arbitrarily large complete graph minors.

We prove the following result:
\begin{thm}\thmlabel{k-planar}
  Every $k$-planar graph has a layered $H$-partition of layered width at most $24k^2 + 60k + 36$ in which $H$ has treewidth at most $\treewidth$.
\end{thm}

In the special case $k=1$ we obtain better constants and an additional property (planarity) of $H$:

\begin{thm}\thmlabel{1-planar}
  Every 1-planar graph has a layered $H$-partition of layered width at most 30 where $H$ is planar and has treewidth at most 3.
\end{thm}

These results imply that $k$-planar graphs have all the graph parameters mentioned above upper-bounded by some constant $c_k$.

Prior to this work, the strongest structural characterization of $k$-planar graphs was in terms of layered treewidth, which we now define.  A \emph{layered tree-decomposition} $(\mathcal{L},\mathcal{K})$ consists of a layering $\mathcal{L}$ and a tree-decomposition $\mathcal{K}$ of $G$. The layered width of $(\mathcal{L},\mathcal{K})$ is $\max\{|L\cap B|: L\in \mathcal{L},\, B\in \mathcal{K}\}$.  The \emph{layered treewidth} of $G$ is the minimum layered width of any layered tree-decomposition of $G$.

\citet{dujmovic.eppstein.ea:structure} show that $k$-planar graphs have layered treewidth $O(k+1)$.  This already implies that several of the graph parameters mentioned above, including queue-number and non-repetitive chromatic number are upper-bounded by $O(k\log n)$ for $n$-vertex $k$-planar graphs \cite{dujmovic.morin.ea:layered}. \thmref{k-planar} is a significant strengthening of this characterization. It implies that $k$-planar graphs have layered-treewidth bounded by a function of $k$ and gives upper bounds on queue-number and non-repetitive chromatic number of $n$-vertex $k$-planar graphs that are independent of $n$.

There is a natural generalization of $k$-planar to genus-$g$ surfaces, that we now describe.  A graph $G$ embedded on a surface $S$ is $(S,k)$-plane if no edge of $G$ is involved in more than $k$ crossings.  A (not necessarily embedded) graph $G$ is $(g,k)$-planar if it is isomorphic to some $(S,k)$-plane graph $G'$ where $S$ is a surface with Euler genus $g$.

\begin{thm}\thmlabel{g-k-planar}
  Every $(g,k)$-planar graph has a layered $H$-partition of layered width at most $\max\{8gk^2 + 12gk + 4g, 24k^2 + 36k + 12\}$ in which $H$ has treewidth at most $\treewidth$.
  
  Every $(g,1)$-planar graph has a layered $H$-partition of layered width at most $\max\{48g, 30\}$ in which $H$ has treewidth at most $4$.  
\end{thm}


\note{PM}{TODO: Write here about applications. non-repetitive colouring, $p$-centered colouring, $(g,k)$-planar graph, $(g,d)$-map graphs, $k$-NN graphs, David's rough characterization.}


The remainder of the paper is organized as follows: In \secref{k-planar} we prove \thmref{k-planar}.  In \secref{1-planar} we prove \thmref{1-planar}.  
In \secref{g-k-planar} we prove \thmref{g-k-planar}. In \secref{consequences} we point out some consequences of these results for some of the other graph parameters mentioned above.


\section{$k$-Planar Graphs}
\seclabel{k-planar}

The purpose of this section is to prove \thmref{k-planar}.  In this paper, all graphs are finite and undirected. Unless specifically mentioned otherwise, all graphs are also simple.  For any graph $G$ and any set $S$ (typically $S\subseteq V(G)$) we use $G[S]$ to denote the graph with vertex set $V(G)\cap S$ and edge set $\{uv\in E(G) : u,v\in S\}$.  We use $G-S$ as a shorthand for $G[V(G)\setminus S]$ and for a graph $G'$, we use $G-G'$ as a shorthand for $G-V(G')$.  We use $G'\subseteq G$ to denote subgraph containment; that is, $V(G')\subseteq V(G)$ and $E(G')\subseteq E(G)$.

We begin by introducing a structural tool for planar graphs, called a tripod decomposition, that is implicit in previous work on $H$-partitions of planar graphs \cite{dujmovic.joret.ea:planar}.  We then show how the tripod decomposition of a planarized version of a $k$-planar graph $G$ can be used to find a layered $H$-partition of $G$ of small layered width in which $H$ has small treewidth. 

\subsection{PS-Trees}

Let $T$ be a tree rooted at some node $r\in V(T)$. The \emph{$T$-depth} of a node $x\in V(T)$ is equal to the length of the path from $x$ to $r$ in $T$. A path $P$ in $T$ is \emph{vertical} if, for each integer $i$, $P$ contains at most one vertex of $T$-depth $i$.  The deepest vertex in a vertical path $P$ is called  the \emph{lower endpoint} of $P$ and the shallowest vertex in $P$ is called the \emph{upper endpoint} of $P$. For each node $v\in V(T)$, we call each node on the path, in $T$, from $v$ to $r$ (including $v$ itself) a \emph{$T$-ancestor} of $v$.  If $v'$ is a $T$-ancestor of $v$, then $v$ is a \emph{$T$-descendant} of $v'$.

A \emph{near-triangulation} $N$ is a plane graph whose outer face, $F$, is bounded by a simple cycle in $N$ and each of whose inner faces is bounded by a 3-cycle in $N$.  A \emph{triangulation} is a near-triangulation whose outer face is bounded by a 3-cycle.  For any cycle $F$ in a near-triangulation $N$ we use $N\llparenthesis F\rrparenthesis$ to denote the near-triangulation consisting of all the edges and vertices of $N$ that are contained in $F$ or its interior.

For a connected graph $G$ and two vertices $v,w\in V(G)$, we use $\dist_{G}(v,w)$ to denote the length of a shortest path, in $G$, with endpoints $v$ and $w$.  For a vertex $r\in V(G)$, the \emph{BFS layering} of $G$ from $r$ is the layering $\langle V_0,V_1,\ldots\rangle$ of $G$ where $V_i=\{v\in V(G): \dist_{G}(r,v)=i\}$ for each integer $i\ge 0$.  A \emph{BFS spanning tree} $T$ of $G$ rooted at $r$ is a spanning tree of $G$ rooted at $r$ where, for every $v\in V(G)$, $\dist_T(v,r)=\dist_G(v,r)$.

For the remainder of this section, $\Delta$ is a triangulation, $r$ is a degree-3 vertex on the outer face of $\Delta$, and $T$ is a BFS spanning tree of $\Delta$ rooted at $r$.

Fix some cycle $F$ in $\Delta$ and let $N=\Delta\llparenthesis F\rrparenthesis$.
Let $\tau=v_1v_2v_3$ be an inner face of $N$ and consider the three minimal vertical paths $Q_1$, $Q_2$, and $Q_3$ in $T$ such that, for each $i\in\{1,2,3\}$ the lower endpoint of $Q_i$ is $v_i$ and the upper endpoint $v_i'$ of $Q_i$ is in $V(F)$.  If $Q_1$, $Q_2$, and $Q_3$ are vertex-disjoint, then the graph $\overline{Y}:=\tau\cup Q_1\cup Q_2\cup Q_3$ is called the \emph{closed tripod} in $N$ determined by $\tau$.  We call $Q_1$, $Q_2$, and $Q_3$ the \emph{legs}, $v_1'$, $v_2'$ and $v_3'$ the \emph{feet}, and $\tau$ the \emph{crotch} of $\overline{Y}$.  We call $Y:=\overline{Y}-F$ the \emph{(open) tripod} of $N$ determined by $\tau$.  The graph $M:=\overline{Y}\cup F$ is called a \emph{Mercedes graph} and the (at most three) inner faces of $M$ other than $\tau$ are called the \emph{wedges} of $M$.

Let $r$ be a degree-3 node on the outer face of a triangulation $\Delta$.  A \emph{tripod decomposition}  $K=K(\Delta, T, r)$ is a rooted tree in which each non-root node $x\in V(K)$ is associated with several objects (refer to \figref{K-node}):

\begin{figure}
  \begin{center}
    \begin{tabular}{cc}
      \includegraphics{figs/K-node-1} &
      \includegraphics{figs/K-node-2}
    \end{tabular}
  \end{center}
  \caption{The elements associated with a node $x\in V(K)$.}
  \figlabel{K-node}
\end{figure}

\begin{enumerate}
  \item A cycle $F_x$ in $\Delta$ and the near-triangulation $N_x:=\Delta\llparenthesis F_x\rrparenthesis$.
  
  \item An inner face $\tau_x$ of $N_x$ that determines a closed tripod $\overline{Y}_x$ and an open tripod $Y_x:=\overline{Y}_x-F_x$ in $N_x$, both with crotch $\tau_x$.
  
  \item The \emph{Mercedes graph} $M_x:=F_x\cup \overline{Y}_x$
\end{enumerate}

The root node $r^0_K$ of $K$ is exceptional.  We define $F_{r^0_K}$ to be the single-vertex graph containing $r$, we define $\overline{Y}_{r^0_K}$ to be the star formed by $r$ and its three neighbours in $\Delta$, and (as usual) $Y_{r^0_K}:=\overline{Y}_{r^0_K}-F_{r^0_K}$.  The root node $r^0_K$ has exactly one child $r_K$, called the \emph{sub-root} of $K$, for which $F_{r_K}$ is the 3-cycle formed by the neighbours of $r$ in $\Delta$.

The non-root nodes of the tripod decomposition $K$ (including the sub-root $r_K$) satisfy the following conditions:
\begin{enumerate}[(PR1)]
  \item \proplabel{children} For each non-root node $x\in V(K)$ such that $M_x$ has $f\le 3$ wedges $F_{x,1},\ldots,F_{x,f}$, the node $x$ has exactly $f$ children $y_1,\ldots,y_f$ in $K$ and $F_{y_i} = F_{x,i}$ for each $i\in\{1,\ldots,f\}$.
  
  \item \proplabel{ancestor-boundary} For each non-root node $x\in V(K)$, there exists a partition of $V(F_x)$ into $g\le 3$ non-empty sets $P_1,\ldots,P_g$ such that, for each $i\in\{1,\ldots,g\}$, $F[P_i]$ is a path that is contained in $Y_a$ for some $K$-ancestor $a$ of $x$. 
  
  \item \proplabel{tripod-feet} For each $i\in\{1,\ldots,g\}$, at least one vertex in $P_i$ is a foot of the closed tripod $\overline{Y}_x$.

  \item \proplabel{partition} $\{Y_x : x\in V(K)\}$ is a partition of $V(\Delta)\setminus\{r\}$.
\end{enumerate}

For any triangulation $\Delta$ in which the outer face has a vertex $r$ of degree 3 and for any BFS spanning tree of $\Delta$ rooted at $r$, there exists a tripod decomposition $K=K(\Delta,T,r)$.  Indeed, this tree is constructed (albeit implicitly) by \citet[Proof of Lemma~14]{dujmovic.joret.ea:planar}.\footnote{\citet{dujmovic.joret.ea:planar} do not require that $T$ be a BFS spanning tree, only that its root, $r$ has degree 3 in $\Delta$ and in $T$.}


% The following claim is not used until later in the paper. However, it's proof is a helpful exercise in understanding the properties of a PS-Tree.
% 
% \begin{clm}\clmlabel{strong-ancestor-boundary}
%   Let $K$ be a PS-tree and
%   let $y\in V(K)$ be a non-root node with parent $p$ such that $V(F_y\cap Y_a)\neq\emptyset$ for some $K$-ancestor $a\neq p$ of $p$. Then $V(F_p\cap Y_a)\neq\emptyset$.
% \end{clm}
% 
% \begin{proof}
%   By \propref{children}, $F_y$ is a wedge in the Mercedes graph $M_p$, so $V(F_y)$ can be partitioned into  $P_1'=V(F_y\cap Y_p)$ and $P_{2,3}'=V(F_y\cap F_p)$.  Let $P_1,\ldots,P_g$ denote the partition of $V(F_p)$ given by \propref{ancestor-boundary}.  
% 
%   We claim that \propref{tripod-feet} ensures that $P_{2,3}'\subset P_i\cup P_j$ for some $i,j\in \{1,\ldots,g\}$.  If $g=2$, this claim is immediate, so assume $g=3$. Then, $F_y\cap \overline{Y}_p$ is the path formed by two legs $Q_2$ and $Q_3$ of $\overline{Y}_p$ and one edge $v_2v_3$ of $\tau_p$.  
%   By \propref{tripod-feet} and the assumption that $q=3$, $Q_2$ has a foot $v_2'\in P_i$ and $Q_3$ has a foot $v_3'\in P_j$, for some $i\neq j$.  The third foot $v_1'$ of $\overline{Y}_p$ is therefore in $P_\ell$ for the unique $\ell \in\{1,2,3\}\setminus \{i,j\}$.  It follows that $V(P_\ell\cap F_y)=\emptyset$.
% 
%   Setting $P_2'=V(P_i\cup F_y)$ and $P_3'=V(P_j\cup F_y)$ yields a partition $\{P_1',P_2',P_3'\}$ of $V(F_y)$ such that $P_1'\subset V(Y_{p})$,   
%   $P_2'\subset V(Y_{a_2})$, and $P_3'\subset V(Y_{a_3})$  for some $K$-ancestors $a_2$ and $a_3$ of $a$.  Therefore $a_2$ and $a_3$ are the only $K$-ancestors $a\neq p$ of $p$, such that $V(F_y\cap Y_a)\neq\emptyset$.
%   Now, each foot $v_2',v_3'\in V(F_p)$.  Furthermore, $v_2'\in P_i\subseteq V(Y_{a_2})$ and $v_3'\in P_j\subseteq V(Y_{a_3})$. Therefore, $V(F_p\cap Y_{a_2})\supseteq\{v_2'\}\neq\emptyset$ and $V(F_p\cap Y_{a_3})\supseteq\{v_3'\}\neq\emptyset$, as claimed.
% \end{proof}
% 
% Contained in the proof of \clmref{strong-ancestor-boundary} is the following addendum to \propref{ancestor-boundary}:
% 
% \begin{enumerate}[(PR1')]\setcounter{enumi}{1}
%   \item \proplabel{ancestor-boundary-ii} Furthermore, if $g=3$, then  $P_i=V(F_x\cap Y_p)$ where $p$ is the parent of $x$ in $K$, for one $i\in\{1,2,3\}$.
% \end{enumerate}

% \subsubsection{Layered $H$-Partitions from PS-Trees}
% \seclabel{k-to-h}
% 
% Before continuing, it is worth pointing out how an $H$-partition of $\Delta-\{r\}$ can be obtained from a PS-tree $K=K(\Delta,T,r)$. (Note: There is nothing new in this section.  This is just an alternative presentation of material from \citet{dujmovic.joret.ea:planar}.) Given $K=K(\Delta,T,r)$, we obtain a layered $H$-partition $(\mathcal{L},\mathcal{P})$ as follows. 
% 
% We use the layering $\mathcal{L}=\{V_0,V_1,\ldots,V_h\}$ where $V_i=\{ v\in V(\Delta)\setminus\{r\}: \dist_{\Delta}(r,v)=i\}$ and $h=\max\{\dist_\Delta(r,v):v\in V(\Delta)\}$.  The fact that $T$ is a BFS tree ensures that $\mathcal{L}$ is, indeed, a layering of $\Delta-\{r\}$.  We use the vertex partition $\mathcal{P}=\{ S_x:=V(Y_x) : x\in V(K)\}$.  \propref{partition} ensures that $\mathcal{P}$ is, indeed a partition of $V(\Delta)-\{r\}$.
% 
% The resulting layered $H$-partition $(\mathcal{L},\mathcal{P})$ has layered width at most 3 because, for each $x\in V(K)$, $S_x$ is made up of vertices from at most 3 vertical paths in $T$.  These are either the three vertices of $Y_{r_k}$ (if $x=r_K$) or the vertices of the open tripod $Y_x$ whose vertices are included in three vertical paths $Q_1,Q_2,Q_3$ that form the legs of the closed tripod $\overline{Y}_x$.  Therefore $|S_x\cap V_i|\le 3$.
% 
% To show that the quotient graph $H=\Delta/\mathcal{P}$ has small treewidth, we produce a tree-decomposition $\mathcal{K}=\{B_x: x\in V(K)\}$ of $H$ using the tree $K$.  For each $x\in V(K)$, the subtree $K[x]=K[\{y:x\in B_y\}]$ contains
% $x$ as well as every node $y$ such that $V(Y_x)\cap V(F_y) \neq\emptyset$.
% It is straightforward to verify that $K[x]$ is connected for each $x\in V(K)$. The fact that $F_x$ is a cycle in $\Delta$ and \propref{ancestor-boundary} ensures that, if $xy\in E(H)$, then one of $x$ or $y$ is a $K$-ancestor of the other. \propref{ancestor-boundary} also implies that, for any node $y\in V(K)$, $|B_y|\le 4$, since $B_y$ contains $y$ and at most 3 $K$-ancestors of $y$.
% 
% Therefore $(\mathcal{L},\mathcal{P})$ is a layered $H$-partition of $\Delta-\{r\}$ with layered width 3 in which $H$ has treewidth 3.  Furthermore, for each $P\in\mathcal{P}$, $\Delta[P]$ is connected.  Since $\Delta$ is planar, this implies that $H$ is planar.\footnote{It is well-known and easy to see that any graph $\Delta'$ obtained by contracting an edge in a planar graph $\Delta$ is also planar.  Repeatedly applying this fact implies that $\Delta/\mathcal{P}$ is planar provided that $\Delta[P]$ is connected for every $P\in\mathcal{P}$.}

\subsection{$k$-Planar Graphs}

Let $G$ be a $k$-plane graph.  We will assume, for ease of exposition, that any point $p\in\R^2$ is involved in at most one crossing $(p,vw,xy)$ of $G$. This assumption is not critical since it can be enforced by a slight deformation of the edges of $G$ and the resulting (deformed) graph is also $k$-plane.   Consider the following plane graphs (see \figref{wrapper}(a)):
\begin{compactenum}
  \item Add a dummy vertex at each crossing in $G$ to obtain a plane graph and then add edges to this graph to obtain a triangulation $G^+$.
  
  \item Add a complete graph $K_4$ that contains $G^{+}$ in one of its inner faces, $F_0$, and then add edges to the resulting graph to obtain a triangulation $G^{++}$.
\end{compactenum}
Observe that, for any edge $vw\in E(G)$, 
\begin{equation}  
  \dist_{G^+}(v,w) \le k+1 \enspace ,  \eqlabel{distance-preserving}
\end{equation}
since (the curve) $vw$ contains a sequence $W_{vw}$ of at most $k+2$ vertices of $G^+$ that form a walk from $v$ to $w$ in $G^+$.

\begin{figure}
  \begin{center}
    \begin{tabular}{c@{\hspace{2cm}}c}
      \includegraphics{figs/wrapper-1} &
      %\includegraphics{figs/wrapper-3} &
      \includegraphics{figs/wrapper-3} \\
      (a) & (b) 
    \end{tabular}
  \end{center}  
  \caption{Wrapping a $K_4$ around $G^+$: (a)~the graph $G^{++}$
  %, (b)~the spanning tree $T$ of $G^{++}$, 
  and (b)~the modified version of $G^{++}$ used in the proof of \corref{k-planar}.}
  \figlabel{wrapper}
\end{figure}

Let $r$ be the vertex of the $K_4$ not incident on $F_0$ and let $T$ be a BFS spanning tree of $G^{++}$ rooted at $r$. 
For each integer $i\ge 0$, let $V_i:=\{v\in V(G^+): \dist_{T}(r,v)=i\}$ and observe that $\{V_2,V_3,\ldots,V_{h}\}$ is a layering of $G^+$ (but not of $G$).

Let $K=K(G^{++},T,r)$ be a tripod decomposition and observe that the sub-root $r_K$ of $K$ has $F_{r_K}=F_0$.  This implies:
\begin{compactenum}[(PR1)]\setcounter{enumi}{4}
  \item \proplabel{all-in}  $G^+\subseteq N_{r_K}-F_{r_K}$.
\end{compactenum}

For each vertex $z\in V(G^+)$, let $P_z=\{a\in V(K): z\in V(N_a-F_a)\}$.  By \propref{all-in}, $P_{z}$ contains the sub-root $r_K$ of $K$.  By \propref{children}, any node $x\in P_z$ has at most one of its children in $P_z$ and any node $x'\not\in P_z$ has none of its children in $P_z$.  Therefore $K[P_z]$ is a vertical path in $K$ that contains $r_K$ (and whose lower endpoint is the unique node $x\in V(K)$ such that $z\in Y_x$). We are about to repeatedly use the fact that the common intersection of two or more vertical paths is a vertical path. 

For each edge $vw\in E(G)$, consider the walk $W_{vw}$ in $G^+$ that corresponds to the sequence of vertices in $G^+$ encountered while walking along the curve $vw$ from $v$ to $w$. We define $A_{vw}:=\bigcap_{z\in W_{vw}} P_z$.  Since $V(W_{vw})\subseteq V(G^+)$, $A_{vw}$ is well-defined.  For each $z\in V(W_{vw})$, $K[P_{z}]$ is a vertical path in $K$ that contains $r_K$.   Therefore $K[A_{vw}]$ is a vertical path in $K$ that contains $r_K$.  For each vertex $v\in V(G)$ define $A_v:=\bigcap_{vw\in E(G)} A_{vw}$.  Again, $K[A_v]$ is a vertical path in $K$ that contains $r_K$.  We define $a(v)$ to be the lower endpoint of $K[A_v]$.  In words, $a(v)$ is the lowest node in $K$ such that the interior of $F_{a(v)}$ contains all the edges of $G$ incident to $v$.

For each node $x\in V(K)$, define the \emph{separator}
\[
   S_x := \{v\in V(G): a(v)=x \} \enspace .
\]
Clearly $\mathcal{S}:=\{S_x:x\in V(K)\}$ is a partition of $V(G)$.  We will use $\mathcal{S}$ as the partition that defines our quotient graph $H:=G/\mathcal{S}$ with vertex set $V(H):=V(K)$ and the edge $xy\in E(H)$ if and only if there exists an edge $vw\in E(G)$ with $v\in S_x$ and $w\in S_y$.  First we argue that this partition has small layered width with respect to the layering $\{V_2,V_3,\ldots\}$ of $G^+$ defined by $T$.
 % (recall that this is a layering of $G^+$, not a layering of $G$).

\begin{clm}\clmlabel{chargeback-to-y}
   For any $x\in V(K)$ and any vertex $v\in S_x$, there exists an edge $vw\in E(G)$ and a vertex $v'\in Y_x\cap V(W_{vw})$ such that $|\dist_T(r,v')-\dist_T(r,v)|\le k+1$.
\end{clm}

\begin{proof}
  Since $v\in S_x$, the lower endpoint of $K[A_{vw}]$ is $x$ for some edge $vw\in E(G)$.  This implies that $W_{vw}\subseteq N_x-F_x$ but $W_{vw} \not\subseteq N_{x'}-F_{x'}$ for every child $x'$ of $x$ in $K$. Therefore, \propref{children} and \propref{ancestor-boundary} imply that $W_{vw}$ contains a vertex $v'\in Y_x$.  Since $G$ is $k$-planar, the length of $W_{vw}$ is at most $k+1$.  Therefore, $\dist_{G^{++}}(v,v') \le k+1$.
  By the triangle inequality,     
  \[
    \dist_{G^{++}}(r,v') \le \dist_{G^{++}}(r,v) + \dist_{G^{++}}(v,v') 
       \le \dist_{G^{++}}(r,v) + k + 1 \enspace .
  \]
  Similarly, $\dist_{G^{++}}(r,v) \le \dist_{G^{++}}(r, v') + k+1$.  Therefore
  $|\dist_{G^{++}}(r,v)-\dist_{G^{++}}(r,v')| \le k+1$ which establishes the claim, since $T$ is BFS spanning tree of $G^{++}$ rooted at $r$, so $\dist_{T}(r,z)=\dist_{G^{++}}(r,z)$ for all $z\in V(G^+)$.
\end{proof}

\begin{clm}\clmlabel{width-of-g-plus}
  For any $x\in V(H)$, and any integer $i\ge 0$, $|S_x\cap V_i|\le 24k+36$.
\end{clm}

\begin{proof}
  Since the vertices of $Y_x$ come from at most three vertical paths in $T$, $|V(Y_x)\cap V_i|\le 3$.  \clmref{chargeback-to-y} shows that, for any $v\in S_x\cap V_i$, there exists $vw\in E(G)$ and $v'\in Y_x\cap V(W_{vw})$ with $\dist_T(v,v')\le k+1$. Therefore, $v'\in V(Y_x)\cap V_{j}$ for some $j\in\{i-k-1,\ldots,i+k+1\}$.  If $v'=v$ (so $v'\in V_i$) then $v'$ contributes only one vertex to $S_x$.  If $v'\neq v$ then $v'\in V(G^+-G)$ is a point where two edges of $G$ cross and $v'$ contributes at most the four endpoints of these edges to $S_x$. In either case, each $v'\in V(Y_x)\cap \bigcup_{j=i-k-1}^{i+k+1} V_j$ contributes at most four vertices to $S_x\cap V_i$, so $|S_x\cap V_i|\le 3\times 4\times (2k+3) = 24k+36$.
\end{proof}

Next we describe a tree decomposition $(B_x:x\in V(K))$ of $H$.  We deliberately use the same node set $V(H)=V(K)$ here, because the subtree $K[x]:=K[\{z:x\in B_z\}]$ will be rooted at $x$.  

\begin{clm}\clmlabel{xy-ancestor}
   For every edge $xy\in V(H)$, one of $x$ or $y$ is a $K$-ancestor of the other.
 \end{clm}
 
 \begin{proof}
   Since $xy\in E(H)$, there exists an edge $vw\in E(G)$ such that $v\in S_x$ and $w\in S_y$.  By definition, $x\in A_v\subseteq A_{vw}$ and $y\in A_w\subseteq A_{vw}$.  Since $K[A_{vw}]$ is a vertical path in $K$, it follows that one of $x$ or $y$ is a $K$-ancestor of the other. 
\end{proof}

The tree decomposition $(B_x:x\in V(K))$ of $H$ is defined as follows: For each edge $xy\in E(H)$ where $y$ is a $K$-ancestor of $x$, we add $y$ to $B_{x'}$ for every node $x'$ on the path from $x$ to $y$ in $K$.  That this produces a tree-decomposition is due to \clmref{xy-ancestor}.  What remains is to upper bound the width of this decomposition.

\begin{clm}\clmlabel{bag-size}
  The tree decomposition $(B_x: x\in V(K))$ of $H$ contains no bag larger than $\bagsize$.
\end{clm}

\begin{proof}
  Fix some node $x\in V(K)$ and consider the contents of $B_x$.  By \clmref{xy-ancestor}, $B_x$ contains only $K$-ancestors of $x$ (including $x$ itself). 
  
  Consider the graph $A$ that contains the edges and vertices of $F_{a}$ for all $K$-ancestors $a$ of $x$. See \figref{A}(a).  It is helpful to think of this graph as follows:  We begin with $F_{a_0}$ where $a_0:=r_K$ is the root of $K$.  We then cut $F_{a_0}$ into two pieces $F^{\bar{x}}_{a_0}$ and $F^x_{a_0}$ where $F^x_{a_0}$ is the piece that contains $F_x$. (The path that makes this cut contains two legs $Q_i$ and $Q_j$ of a tripod $Y$ and one edge $v_iv_j$ of its crotch $\tau=v_1v_2v_3$.)  This process continues by partitioning $F_{a_1}:=F^x_{a_0}$ recursively.  The process terminates after some number $d$ of iterations, when $F^x_{a_{d-1}}=F_x=F_{a_d}$.

  \begin{figure}
    \begin{center}
      \begin{tabular}{cc}
        \includegraphics{figs/A-1} &
        \includegraphics{figs/A-3} \\ (a) & (b)
      \end{tabular}
    \end{center}
    \caption{(a)~the graph $A$ and (b)~the walks $W_{vw}$ and $W_{ww'}$ that contain the walk $w=v_0,v_1\ldots,v_{\ell-1},v_\ell=w^*$.}
    \figlabel{A}
  \end{figure}

  Refer to \figref{A}(b).  If $B_x$ contains some $K$-ancestor $a_\delta$, $\delta\in\{0,\ldots,d-1\}$, of $x$, it is because there is an edge $vw\in E(G)$ with $w\in S_{a_\delta}$ and $v\in S_{x'}$ for some $K$-descendant $x'$ of $x$.  The edge $vw$ corresponds to a walk $W_{vw}$ in $G^+$. The walk $W_{vw}$ is contained in $N_x-F_x$, otherwise $x\not\in A_{vw}$ which would imply that $x'\not\in A_{vw}$, $x'\not\in A_v$, and $a(v)\neq x'$, contradicting the fact that $v\in S_{x'}$.
  
  However, $w\in S_{a_\delta}$, so there exists an edge $ww'\in E(G)$ such that $W_{ww'}\subseteq N_{a_\delta}-F_{a_\delta}$ but $W_{ww'}\not\subseteq N_{a_{\delta+1}}-F_{a_{\delta+1}}$.  Therefore $W_{ww'}$ must contain a first vertex $w^*\in V(F_{a_{\delta+1}}-F_{a_\delta})\subseteq Y_{a_\delta}$.

  Summarizing, we have a walk $w=v_0,v_1\ldots,v_{\ell-1},v_\ell=w^*$ in $G^+$ such that
  \begin{compactenum}[(i)]
    \item $v_0\in V(N_x-F_x)$;
    
    \item $v_0,\ldots,v_{\ell-1}\in V(N_{a_{\delta+1}}-F_{a_{\delta+1}})$;
    
    \item $v_\ell\in V(F_{a_{\delta+1}}-F_{a_\delta})$; and
    
    \item $\ell\le k+1$.
  \end{compactenum}
  We call any walk with properties (i)--(iv) a \emph{$\delta$-walk}. Therefore, to upper-bound $|B_x|$, it suffices to upper-bound the number of $\delta\in\{0,\ldots,d\}$ such that there exists a $\delta$-walk.\footnote{To account for the fact that $B_x$ contains $x$ we count $\delta=d$.  A $d$-walk corresponds, for example, to a length-0 walk that begins and ends at any vertex $v_0\in V(N_x-F_x)$.}  This is our goal for the remainder of this proof. 
  
  In order to avoid excessive levels of subscripts, we introduce the following abbreviated notation: For $i\in\{0,\ldots,d\}$, we use $F_i$, $N_i$, $Y_i$, and $M_i$ as shorthands for $F_{a_i}$, $N_{a_i}$, $Y_{a_i}$, and $M_{a_i}$, respectively.
  
  Let $d_0:=d$, and $d_1:=\min\{i: V(F_d\cap Y_i)\neq\emptyset \}$.  By \propref{ancestor-boundary}, any vertex $z\in V(F_d)$ is contained in 
  $Y_j$ for some $j\in \{d_1,d_1+1,\ldots,d-1\}$.  Therefore, $F_d$ is contained in the interior of  $F_{d_1}$, that is, $N_{d_0}\subseteq N_{d_1}-F_{d_1}$.   
  
  Refer to \figref{d0d1}.  We will now prove that 
  \begin{equation}
     |\{\delta\in\{d_1+1,\ldots,d_0\}:  \mbox{there exists a $\delta$-walk} \}| \le \binom{k+3}{2} \enspace . \eqlabel{one-round}
  \end{equation}
  \begin{figure}
    \begin{center}
      \includegraphics{figs/d0d1}
    \end{center}
    \caption{$F_{d_1},\ldots,F_{d_0}$ and the resulting graph $I$.} 
    \figlabel{d0d1}
  \end{figure}
  
  To prove \eqref{one-round}, consider the graph $I$, illustrated in \figref{d0d1}, with vertex set $V(I)=\{d_1+1,\ldots,d_0\}$ and $ij\in E(I)$ if and only $i=j$ or $|F_i\cap Y_j|\neq\emptyset$.  Any $\delta$-walk $v_0,\ldots,v_\ell$ with $\delta\in V(I)$ corresponds to a walk $\rho_0,\ldots,\rho_{\ell}$ in $I$, where $\rho_i:=\max\{j\in\{0,\ldots,d\} : v_i\in N_{j}-F_j\}$.  To see why this is true, consider $\rho_{t}= i$ and $\rho_{t+1}=j$ for some $t\in\{0,\ldots,\ell-1\}$:
  \begin{compactenum}
    \item If $i=j$, then $ij\in E(I)$ since $ii\in E(I)$ for each $i\in V(I)$.
    
    \item If $j\neq i$ then assume, without loss of generality, that $j < i$ so that $a_j$ is a $K$-ancestor of $a_i$.  Now $v_t\in V(N_i-F_i)$, but $v_{t+1}\not\in V(N_i-F_i)$ otherwise we would have $j\ge i$. Therefore, $v_{t+1}\in V(F_i)$.  By \propref{ancestor-boundary}, $v_{t+1}\in V(Y_{j'})$ for some $j'< i$ and therefore $ij'\in E(I)$.  By \propref{children}, $v_{t+1}\not\in V(N_{j''}-F_{j''})$ for any $K$-descedant $j''>j'$ of $j'$.  Therefore $j'=\max\{j : v_{t+1}\in V(N_j-F_j)\}$. By definition $j'=\rho(v_{t+1})=j$, so the edge $ij=ij'\in E(I)$. 
  \end{compactenum}

  Properties~(i)--(iv) of $\delta$-walks imply that any walk $\rho_0,\ldots,\rho_\ell$ obtained this way 
  \begin{compactenum}[(i')]
    \item begins at $\rho_0=d_0$ 
    \item is contained in $I[\{\delta,\ldots,d_0\}]$, 
    \item ends at $\rho_\ell=\delta$
    \item has length $\ell \le k+1$.
  \end{compactenum}
  We will call any walk in $I$ that satisfies (i')--(iv') a \emph{$\delta$-walk}.
  
  The edges in $E(I)$ can be partitioned into two classes:  A set $B(I):=\{ij\in E(I): |i-j|\le 1\}$ of \emph{boring edges} and a set $J(I):=\{ij\in E(I): |i-j|>1\}$ of \emph{jump edges}.  For a jump edge $ij$ with $i>j$, $j$ is called a \emph{landing} and $i$ is called a \emph{takeoff}.  We will show that the jump edges of $I$ satisfy the following properties:
  \begin{enumerate}[(J1)]
    \item \jlabel{single} Every $i\in\{d_1+1,\ldots,d_0\}$ is a takeoff for at most one jump edge.
    
    \item \jlabel{nested} For every jump edge $ij\in E(I)$ with takeoff $i$ and landing $j$, and for every $q\in\{j+2,\ldots,i\}$, if $q$ is a takeoff for some jump edge, then this edge is $qj$.
  \end{enumerate}
  Property \jref{nested} can be interpreted informally as saying that jump edges can cross, but barely:  If $i_1j_1$ and $i_2j_2$ are jump edges with $i_1> i_2>j_1>j_2$, then $j_1=i_2-1$.

  We now prove that jump edges satisfy \jref{single}.
  First we claim that, for each $i\in\{d_1+1,\ldots,d_0\}$, $F_{i}$ contains a vertex of $Y_{d_1}$.  For $i=d_0$, this is true by definition so $F_{d_0}$ contains some vertex $v\in V(Y_{d_1})$.  For $i>0$, we have $F_{d_0}-F_{i}\subseteq N_i-F_i$, but $N_i-F_i$ contains no vertex of $Y_{d_1}$. It must therefore be the case that $v\in V(F_i)$, otherwise $F_{d_0}-F_i$ would contain $v$. 
  
  Now, consider the parent $a_{i-1}$ of $a_i$ in $K$.  By \propref{children}, $F_i$ is a wedge of the Mercedes graph $M_{i-1}$.  Let $P_1,\ldots,P_g$ be the partition of $V(F_{i-1})$ guaranteed by \propref{ancestor-boundary}.
  By \propref{tripod-feet}, $V(F_i) = V(F_i)\cap(P_\alpha \cup P_j\cup V(Y_{i-1}))$ for some $\alpha,j\in \{1,\ldots,g\}$.  Since $F_i$ contains $v\in V(Y_{d_1})$, at least one of $\alpha$ or $j$, say $\alpha$, is equal $d_1$. Since $d_1\not\in V(I)$ and the (possible) edge $(i,i+1)$ is boring, the only (possible) jump edge with $i$ as takeoff is $ij$. This establishes \jref{single}. 
  
  To establish that jump edges satisfy \jref{nested} we can use the same argument used to establish that each $F_i$ contains a vertex of $Y_{d_1}$ to establish that each $F_q$ contains a vertex of $Y_j$ and therefore $qj\in E(I)$.  Since $q \ge j+2$, $qj$ is a jump edge and, by \jref{single}, it is the only jump edge with takeoff $q$. 

  It will be helpful to consider an \emph{edge maximal} version, $I^+$, of $I$ that we now define.  Let $j_0=d$ and let $j_1>j_2>\cdots> j_p$ denote the sequence of landings, in decreasing order.  Then \jref{single} and \jref{nested} imply that
  \[
      J(I) \subseteq J^+(I) :=
         \bigcup_{t=1}^p \{ ij_t : i\in\{ j_t,\ldots,j_{t-1}+1 \}\} \enspace . 
  \]
  and clearly $B(I)\subseteq B^+(I) := \{(i,j): i\in\{d_1+2,\ldots,d_0-1\},\, j\in\{i-1,i,i+1\}\}$.
  Any walk in $I$ is also a walk in the supergraph $I^+$ with $V(I^+)=V(I)$ and $E(I^+)=B^+(I)\cup J^+(I)$, so we can focus our attention on $\delta$-walks in $I^+$.
  
  We are almost done proving \eqref{one-round}. Refer to \figref{one-round}.  For any landing $j_t$, $t>0$, $I^+[\{j_t,\ldots,d_0\}]$ contains the path $d_0,j_1,\ldots,j_t$ of length $t$.  A few moments consideration will also convince the reader that $d_0,j_1,\ldots,j_t$ is a shortest path, in $I^+[\{j_{t+1}-1,\ldots,d_0\}]$, from $d_0$ to $j_t$.  Therefore, by (iv') no $\delta$-walk contains $j_t$ for any $t> k+1$.
  
  \begin{figure}
    \begin{center}
      \includegraphics{figs/one-round}
    \end{center}
    \caption{The graph $I^+[\{j_{4}+1,\ldots,d_0\}]$.  The shortest path $d_0,j_1,\ldots,j_3$ is highlighted.}
    \figlabel{one-round}
  \end{figure}
  
  Next, observe that every path in $I^+[\{j_{t+1}+1,\ldots,d_0\}]$ from $d_0$ to some vertex  $j_{t}+i$, $i\ge 0$, contains $j_t$.  Furthermore, the only edges of $I^+[\{j_{t+1}+1,\ldots,d_0\}]$ incident to vertices $j_t+i$, $i>0$, are boring edges.  Therefore, for any $i\ge 0$, the shortest path from $d_0$ to $j_t+i$ in $I^+[\{j_{t+1}+1,\ldots,d_0\}]$ has length $t+i$.  It follows that there are at most $k+2-t$ values $\delta\in\{j_{t+1},\ldots,j_t\}$ such that there exists a $\delta$-walk.
  
  Therefore, the total number of $\delta\in\{d_1+1,\ldots,d_0\}$ such that there exists a $\delta$-walk is at most
  \[
     \sum_{t=0}^{k+1} (k+2-t) = \sum_{t=1}^{k+2} t = \binom{k+3}{2}
  \]
  This establishes \eqref{one-round}.  An illustration of this analysis, showing the case $k=1$ is shown in \figref{one-round-example}.
  \begin{figure}[htbp]
    \begin{center}
      \includegraphics{figs/one-round-example}
    \end{center}
    \caption{The set of $6=\binom{k+3}{2}$ vertices in $I^+$ reachable by paths of walks of length at most $k+1$ (for $k=1$).}
    \figlabel{one-round-example}
  \end{figure}
  
  
  Refer to \figref{nested}. For $s>1$, let $d_s:=\min\{i: V(F_{d_s}\cap Y_i)\neq\emptyset \}$. Consider some $\delta$-walk $v_0,\ldots,v_\ell$ for $\delta\in \{ d_{s+1},\ldots,d_{s}-1\}$.  By (i) and (ii), $v_0\in V(N_{d_0}-F_{d_0})\subseteq V(N_{d_s}-F_{d_s})$ and $v_\ell\in F_{\delta+1}-F_\delta$.  Therefore, there exists some maximum value $i$ such that $v_i\in V(N_{d_{s}}-F_{d_{s}})$.  (In words, $i$ is the last time the walk is in the interior of $F_{d_s}$ before it leaves forever.)
  Observe that $i\ge s$ because $v_0,\ldots,v_i$ contains $v_0\in N_{d_0}-F_{d_0}$ as well as a vertex from each of $F_{d_0},\ldots,F_{d_{s-1}}$.  

  \begin{figure}
    \begin{center}
      \includegraphics{figs/nested}
    \end{center}
    \caption{The nested cycles $F_{d_0}, F_{d_1},F_{d_2}\ldots$.}
    \figlabel{nested}
  \end{figure}
  
  Now, consider the subwalk $v_i,\ldots,v_\ell$.  This subwalk obeys Properties~(i)--(iv) with respect to $F_{d_s}$.  Specifically:
  \begin{compactenum}[(i)]
    \item $v_i\in V(N_{d_s}-F_{d_s})$;
    
    \item $v_i,\ldots,v_{\ell-1}\in V(N_{a_{\delta+1}}-F_{a_{\delta+1}})$;
    
    \item $v_\ell\in V(F_{a_{\delta+1}}-F_{a_\delta})$; and
    
    \item $\ell\le k+1$, so $\ell-i \le \ell-s \le k+1-s$.
  \end{compactenum}
  Exactly the same analysis used to establish \eqref{one-round} shows that
  \begin{equation}
     |\{\delta\in\{d_{s+1}+1,\ldots,d_s\}:  \mbox{there exists a $\delta$-walk} \}| \le \binom{k+3-s}{2} \enspace . \eqlabel{one-round-s}
  \end{equation}

  We conclude the proof with 
  \begin{align*}
    |B_x| & \le |\{\delta\in\{0,\ldots,d\}:  \mbox{$\exists$ a $\delta$-walk} \}| \\
      & = \sum_{s=0}^{k+1}|\{\delta\in\{d_{s+1}+1,\ldots,d_s\}:  \mbox{$\exists$ a $\delta$-walk} \}| \\
      & \le\sum_{s=0}^{k+1} \binom{k+3-s}{2} \\
      & = \sum_{s=2}^{k+3} \binom{s}{2} \\
      & = \frac{1}{6}(k+2)(k+3)(k+4) \\
      & = \frac{k^3}{6} + \frac{3k^2}{2} + \frac{13k}{3} + 4  \enspace . \qedhere
  \end{align*}
\end{proof}

We now have all the pieces needed to complete the proof of \thmref{k-planar}.

\begin{proof}[Proof of \thmref{k-planar}]
  We use the layered $H$-partition $(\mathcal{L}, \mathcal{S})$, where $\mathcal{S}=\{S_x:x\in V(K)\}$ and $\mathcal{L}=\langle V_i': i=0,1,2\ldots\rangle$ where $V_i' = V_{(k+1)i}\cup\cdots\cup V_{(k+1)(i+1)-1}$.  \Eqref{distance-preserving} shows that $\mathcal{L}$ is indeed a layering of $G$.

  By \clmref{bag-size}, $\mathcal{S}$ produces a quotient graph $H=G/\mathcal{S}$ of treewidth at most $\treewidth$.  By \clmref{width-of-g-plus}, for each $x\in V(H)$ and each integer $i$, $|B_x\cap V_i|\le 24k+36$, so $|B_x\cap V_i'|\le (k+1)(24k + 36) = 24k^2+60k+36$. 
\end{proof}

We conclude this section with a small extension of \thmref{k-planar} that is useful when studying $(g,k)$-planar graphs in \secref{g-k-planar}.

\begin{cor}\corlabel{k-planar}
  Let $G$ be a $k$-plane graph, let $r_0$ be an arbitrary vertex of $G$, let $G^+$ be a graph obtained by adding a dummy vertex at each crossing in $G$ and then arbitrarily adding non-crossing edges to obtain a triangulation.  Then $G$ has a layered $H$-partition $(\mathcal{L}=\langle V_0',V_1',\ldots\rangle,\mathcal{P})$ satisfying the conditions of \thmref{k-planar} and such that $V_i' := \{v\in V(G): \floor{\dist_{G^+}(r_0,v)/(k+1)}=i\}$ for each integer $i\ge 0$.
\end{cor}

\begin{proof}
  Without loss of generality, assume that $r_0$ is on the outer face of $G^+$ and let $v_1$ and $v_2$ be the other two vertices of the outer face of $G^+$.
  The proof of \thmref{k-planar} begins with the graph $G^+$ and then creates a graph $G^{++}$ consisting of a $K_4$ containing $G^+$ in one of its inner faces, $F_0$, and six edges required to complete $G^{++}$ to a triangulation.  Instead, modify $G^{++}$ by subdividing the edges joining $F_0$ to $v_1$ and $v_2$, as shown in \figref{wrapper}(b).  In this modified version of $G^{++}$, we have $\dist_{G^++}(r,v)=\dist_{T}(r,v)=\dist_{G^+}(r_0,v)+2$
  which leads to the layering $\langle V_0,V_1,\ldots\rangle$ of $G^+$ where $V_i=\{v\in V(G^+): \dist_{T}(r,v) = i\}$.
  
  The rest of proof remains unchanged until the point where we derive the layering $\mathcal{L}=\langle V_0',V_1',\ldots\rangle$ of $G$.  At this point, we set
  \[  
    V_i' := \{v\in V(G): \floor{\dist_{T}(r,v)-2}=i\} 
         = \{v\in V(G): \floor{\dist_{G^+}(r_0,v)/(k+1)}=i\} \enspace ,
  \]
  for all integers $i\ge 0$.  This layering satisfies the conditions of the corollary.
\end{proof}

\newpage
\section{A Generalization}

We now generalize \thmref{k-planar} to a more abstract setting.  For two sets $A$ and $B$, we use $A\ominus B := (A\cup B)\setminus(A\cap B)$ to denote the symmetric difference between $A$ and $B$.  For a rooted tree $T$ and any node $x\in V(T)$, we define $T_x := T[\{y\in V(T):\mbox{$x$ is a $T$-ancestor of $y$}\}]$ to be the maximal subtree of $T$ rooted at $x$.
We begin with a fairly standard lemma about normalized tree decompositions:

\begin{lem}\lemlabel{nice-decomposition}
  For any graph $H$ of treewidth $t$, there is a rooted tree $T$ with $V(T)=V(H)$ and a $T$-decomposition $(B_x:x\in V(T))$ of $H$ that has following additional properties:  
  \begin{compactenum}[(T1)]
    % \item\tlabel{rooted}\tlabel{first} $T$ is rooted at some node $x_0\in V(T)$ with $B_{x_0}=\emptyset$.
    % \item\tlabel{node-set} $V(T)= V(H)$.
    % \item\tlabel{diff} For each edge $xy\in E(T)$, $|B_x\ominus B_y|\le 1$.
    \item\tlabel{subtree-root} For each $x\in V(H)$, the subtree $T[x]:=T[\{y\in V(T):x\in B_y\}]$ is rooted at $x$.
    \item\tlabel{ancestor-edge}\tlabel{last} For each edge $xy\in E(H)$, one of $x$ or $y$ is a $T$-ancestor of the other.
  \end{compactenum}
\end{lem}

\begin{proof}
  Begin with any width-$t$ tree decomposition $(B_x:x\in V(T_0))$ of $H$ that uses some rooted tree $T_0$ that is rooted at some node $x_0$ with $C_{B_0}=\emptyset$.  Let $f:V(H)\to V(T)$ be function that maps each $x\in V(H)$ onto the root of the subtree $T_0[x]:=T_0[\{y\in V(T_0): x\in B_y]$.  If $f$ is one-to-one, then proceed to the next paragraph. Otherwise, as long as there exists some distinct pair $x,y\in V(H)$ with $a:=f(x)=f(y)$ we subdivide the edge between $a$ and its parent by introducing a new node $a'$ with $B_{a'}=B_{a}\setminus\{x\}$.  This modification reduces the number of distinct pairs $x,y\in V(H)$ with $f(x)=f(y)$, so repeatedly performing this modification will eventually produce a tree-decomposition $(B_x:x\in V(T_0))$ of $H$ in which $f$ is one-to-one.
  
  Now, consider any node $a\in V(T_0)$ such that there is no vertex $x\in V(H)$ with $f(x)=a$.  In this case, $B_{a}\subseteq B_{a'}$ where $a'$ is the parent of $a$ since any $x\in B_a\setminus B_{a'}$ would have $f(x)=a$.  In this case, contract the edge $aa'$ in $T_0$, eliminating the node $a$.  Repeating this operation will eventually produce a width-$t$ tree-decomposition of $(B_x:x\in V(T_0))$ where $f$ is a bijection between $V(H)$ and $V(T_0)$.  Renaming each node $a\in T_0$ as $f^{-1}(a)$ gives a tree-decomposition $(B_x:x\in V(T))$ with $V(T)=V(H)$.  
  

  Now, for each vertex $x\in V(H)$, rename the node $f(x)$ in $T_0$ as $x$. The resulting tree decomposition uses a tree $T_0$ with $V(H)\subseteq V(T_0)$. 
  
     (or introduce a new node)   
  Begin with a width-$t$ tree-decomposition $(C_x:x\in V(T_0))$ of $H$ using a tree $T_0$ with the property that, for any edge $xy\in E(T_0)$, $|C_x\ominus C_y|\le 1$. That is, $C_x$ contains exactly one vertex not in $C_y$ or vice-versa. It is well-known that such a tree-decomposition exists.  Indeed, any tree-decomposition of $H$ can be made to have this property adding $|C_x\ominus C_y|-1$ subdivision vertices along each edge $xy\in E(T_0)$ and assigning approriate bags to the newly introduced subdivision vertices.
  
  Next, select an arbitrary node $x\in V(T_0)$ and let $\{v_1,\ldots,v_{t'}\}:=C_x$ denote the contents of $C_x$. Create a new tree $T$ from $T_0$ by adding a path of nodes $x_0,x_1,x_2,\ldots,x_{t'-1},x_{t'}=x$. Let $C_{x_0}:=\emptyset$ and, for each $i\in\{1,\ldots,t'\}$, $C_{x_i}:=\{v_1,\ldots,v_i\}$.  Root $T_0$ at $x_0$ and note that $(C_x: x\in V(T))$ is a width-$t$ tree-decomposition of $H$ that satisfies Properties~\tref{rooted}, and \tref{diff}.  For the remaining properties some renaming is required, which we now describe.
  
  For each $v\in V(H)$, let $v'\in V(T)$ be the unique node such that $v\in C_{v'}$ and $v'$ has minimum $T$-depth.  The uniqueness of $v'$ follows from the fact that $T[v]:=T[\{x\in V(T):v\in B_x\}]$ is connected, so $v'$ is the root of $T[v]$.  Furthermore, for any pair of distinct vertices $v,w\in V(H)$, \tref{rooted} and \tref{diff} ensure that $v'\neq w'$: Otherwise, $C_{v'}\ominus C_{x'}\supseteq \{v',w'\}$ where $x'$ is the parent of $v'=w'$.  Property~\tref{rooted} ensures that $x'$ exists but then Property~\tref{diff} contradicts $C_{v'}\ominus C_{x'}\supseteq \{v',w'\}$.
  
  For each $x\in V(T)$, let $B_x=\{v' : v\in C_x\}$.  Now define the graph $H'$ to have vertex set $V(H')=\{v':v\in V(H)\}$ and edge set $E(H')=\{v'w': vw\in E(H)\}$.  Obviously, $H'$ isomorphic to $H$, $V(H')\subseteq V(T)$ and $(B_x: x\in V(T))$ is a width-$t$ tree-decomposition of $H'$.  The decomposition $(B_x: x\in V(T))$ inherits Properties~\tref{rooted} and \tref{diff} from $(C_x:x\in V(T))$ and satisfies \tref{node-set} and \tref{subtree-root} by definition.
  
  To see that $(B_x:x\in V(T))$ satisifies \tref{ancestor-edge}, observe that, if $vw\in E(H)$, then at least one of $v'$ or $w'$ is contained in $B_x$ for every node $x$ on the path from $v'$ to $w'$ in $T$.  If neither $v'$ nor $w'$ is an ancestor of the other, then some node $x$ on this path has $T$-depth less than that of $v'$ and $w'$.  If $v'\in B_x$ this contradicts the fact that $v'$ is the root of $T[v']$.  If $w'\in B_x$ this contradicts the fact that $w'$ is the root of $T[w']$.
\end{proof}

\subsection{Generalized Tripod Partitions}

For the remainder of this section, $G$ is a graph and $(\mathcal{L}:=\langle V_1,\ldots,V_h\rangle,\mathcal{Q}:=(Q_x: x\in V(H)))$ is a layered partition of $G$ of layered width $\ell$ where the treewidth of the quotient graph $H=G/\mathcal{Q}$ is at most $t$.  The following lemma shows how to use $(\mathcal{L},\mathcal{Q})$ to define a hierarchical decomposition of $G$.

\begin{lem}\thmlabel{generalized-tripod}
  There exists a tree $T$ rooted at a node $x_0$ where each node $x\in V(T)$ is associated with a subset $Y_x\subseteq V(G)$ that defines several related sets $V_x = \bigcup_{y\in V(T_x)} Y_y$, $F_x=\{w\in V(G): vw\in E(G), v\in V_x,\, w\not\in V_x\}$, and $N_x=V_x\cup F_x$ that satisfy the following properties:
  \begin{compactenum}[(Y1)]
    \item\ylabel{thickness} $\mathcal{Y}:=(Y_x: x\in V(T))$ is a partition of $V(G)$ of layered width (with respect to $\mathcal{L}$) at most $\ell$.
    \item\ylabel{separator} There is no edge $vw\in E(G)$ with $v\in V_x$ and $w\in V(G)\setminus N_x$. 
    \item\ylabel{ancestor-edge} $x$ has $t'\le t$ strict $T$-ancestors $a_1,\ldots,a_{t'}$ such that $F_x \subseteq \bigcup_{i=1}^{t'} Y_{a_i}$.
  \end{compactenum}
\end{lem}

Before proving \thmref{generalized-tripod} we point out more properties that are immediately implied by it:

\begin{compactenum}[(Y1)]\setcounter{enumi}{3}
  \item\ylabel{y-subsets} $Y_x\subseteq V_x$ for every $x\in V(T)$. 
  \item\ylabel{containment-i} $V_x\subseteq V_a$ for every $T$-ancestor $a$ of $x$. 
  \item\ylabel{containment-ii}$N_x\subseteq N_a$ for every $T$-ancestor $a$ of $x$.
\end{compactenum}

Property~\yref{y-subsets} follows from the fact that $V_x$ is the union of several sets, one of which is $Y_x$.  Property~\yref{containment-i} follows from the definition of $V_x$ and the fact that $V(T_x)\subseteq V(T_a)$. To show Property~\yref{containment-ii} first note that, by \yref{containment-i} it suffices to consider vertices in $w\in F_x=N_x\setminus V_x$. By definition, every vertex $w\in F_x$ is adjacent, in $G$, to a vertex $v\in V_x$.  By \yref{containment-i}, $v\in V_a$, so $w$ is either in $V_a$ or $w$ satisfies the condition $vw\in E(G)$, $v\in V_a$, and $w\not\in V_a$, so $w\in F_a$.  In either case $w\in N_a=V_a\cup F_a$.  Note that none of \yref{y-subsets}--\yref{containment-ii} depends on \yref{ancestor-edge} (which is important, since \yref{containment-i} is used to establish \yref{ancestor-edge} in the following proof).

\begin{proof}[Proof of \thmref{generalized-tripod}]
  Apply \lemref{nice-decomposition} to $H$ to obtain a tree-decomposition $(B_x: x\in V(T))$ of $H$ where $T$ is rooted at some node $x_0$ and that satisifies Properties~\tref{first}--\tref{last}. (We skip the introduction of the isomorphic graph $H'$ and assume from the outset that $V(H)\subseteq V(T))$.)

  For each $v\in V(G)$, let $q(v)$ denote the unique node $q\in V(H)$ such that $v\in Q_q$.  For each $x\in V(T)$, define
  \[
      Y_x = \begin{cases} Q_x & \text{if $x\in V(H)$} \\
                          \emptyset & \text{otherwise} .
            \end{cases}
  \]
   % 
   % 
   % and let $b(v)$ denote the root of the subtree $T[v]:=T[\{x\in V(T):q(v)\in B_x\}]$. For each $x\in V(T)$, define $Y_x=\{v\in V(G):c(v)=x\}$.  
  Clearly, $\mathcal{Y}:=(Y_x:x\in V(T))$ is a partition of $V(G)$ and, for any $i\in\{1,\ldots,h\}$ $|Y_x\cap V_i|\le \ell$.
  so $\mathcal{Y}$ has layered width at most $t\ell$ with respect to $\mathcal{L}$.  This establishes \yref{thickness}. 

  Property \yref{separator} is immediate from the definitions of $F_x$ and $N_x$.  In particular, $(N_x,V(G)\setminus V_x)$ is a separation of $G$ with $F_x=N_x\cap(V(G)\setminus V_x)$. 

  To establish Property~\yref{ancestor-edge}, consider some vertex $w\in F_x$.  Since $w\in F_x$, there exists an edge $vw\in E(G)$ with $v\in V_x$ and $w\not\in V_x$.  Since $v\in V_x$, $v\in Y_{x'}=Q_{x'}$ for some $T$-descendant $x'$ of $x$ (possibly $x=x'$). Since $\mathcal{Y}$ is a partition, $w\in Y_{a}=Q_a$ for some $a\not\in V(T_x)$.  Since $vw\in E(G)$, $x'a\in E(H)$.  By \tref{ancestor-edge}, one of $a$ or $x'$ is a $T$-ancestor of the other. Since $w\in Y_a\subseteq V_a$ and $w\not\in V_x\supseteq V_{x'}$, \yref{containment-i} rules out the possibility that $x'$ is a $T$-ancestor of $a$. Therefore, $a$ is a $T$-ancestor of $x$ which is a $T$-ancestor of $x'$.  Let $z_0,\ldots,z_r$ be the path, in $T$ from $z_0:=x'$ to $z_r:=a$.  For each $i\in\{0,\ldots,r\}$, at least one of $a$ or $x'$ is in $B_{z_i}$.  However, by \tref{subtree-root} $x'$ is not contained in $B_{x_i}$ for any $i\in\{1,\ldots,r\}$.  Therefore $a\in B_{x_i}$ for each $i\in\{0,\ldots,r\}$.  In particular, $a$ is contained in $B_x$.
  Property~\yref{ancestor-edge} now follows from the fact that $|B_x|\le t+1$ and $C_x$ contains $x$.
\end{proof}

\subsection{Shortcut Systems}

% For non-negative integers $k$ and $d$, a graph $G$ and distinct vertices $v,w\in V(G)$, a \emph{$(k,d)$ $vw$-shortcut} in $G$ is a path in $G$ with endpoints $v$ and $w$ of length at most $k$ and whose internal vertices each have degree at most $d$.  The \emph{$(k,d)^\text{th}$ power of $G$}, denoted $G^{k,d}$, is the graph with vertex $V(G^{k,d})=V(G)$ in which, for each distinct pair $v,w\in V(G)$, $vw\in E(G^{k,d})$ if and only if $G$ contains a $(k,d)$ $vw$-shortcut.

A collection $\mathcal{P}$ of paths in a graph $G$ is a \emph{$(k,d)$-shortcut system} (for $G$) if every path in $\mathcal{P}$ has length at most $k$ and if, for every $v\in V(G)$, the number of paths in $\mathcal{P}$ that use $v$ as an internal vertex is at most $d$. Each path in $P\in\mathcal{P}$ is called a \emph{shortcut}; if $P$ has endpoints $v$ and $w$ then it is a \emph{$vw$-shortcut}.  A $vw$-shortcut is also referred to as \emph{$v$-shortcut} (and a \emph{$w$-shortcut}). Given a graph $G$ and a $(k,d)$-shortcut system $\mathcal{P}$ for $G$, we use $G^{\mathcal{P}}$ to denote the supergraph of $G$ obtained by adding the edge $vw$ for each $vw$-shortcut in $\mathcal{P}$.

In this section we will show how, given a $(k,d)$-shortcut system $\mathcal{P}$, a layered $H$-partition of $G$ can be used to obtain a layered $I$-partition of $G^{\mathcal{P}}$ where the layered width of the partition does not increase dramatically and the treewidth of $I$ is not much more than the treewidth of $H$.  

For convenience, it will be helpful to assume that $\mathcal{P}$ contains a
length-1 $vw$-shortcut for every edge $vw\in E(G)$.  This assumption has no effect on the resulting graph $G^{\mathcal{P}}$ but avoids special cases in some of our proofs.

As in the rest of this section, $(\mathcal{L},\mathcal{Q})$ denotes a layered $H$-partition of $G$; $T$ is a tree used in a tree-decomposition $(B_x:x\in V(T))$ of $H:=G/\mathcal{Q}$ satisfying the conditions of \lemref{nice-decomposition}; and, for each $x\in V(T)$, $Y_x$, $V_x$, $F_x$, and $N_x$ are obtained obtained by applying \thmref{generalized-tripod} to $G$ $(\mathcal{L},\mathcal{Q})$, and $H$.

For a node $x\in V(T)$, we say that a shortcut $P\in\mathcal{P}$ \emph{crosses} $x$ if $Y_x$ contains an internal vertex of $P$, that is, $P=(v_0,\ldots,v_r)$ and $\{v_1,\ldots,v_{r-1}\}\cap Y_x\neq\emptyset$.  We say that a vertex $v\in V(G)$ \emph{participates} in $x$ if $v\in Y_x$ or $\mathcal{P}$ contains a shortcut $P$ with $v\in V(P)$ and $P$ crosses $x$. We let $X_v$ denote the set of nodes $x\in V(T)$ such that $v$ participates in $x$.

\begin{clm}
  For any $v\in V(G)$ there exists a (unique) node $a(v)\in X_v$ such that
  $a(v)$ is a $T$-ancestor of every node in $X_v$.
\end{clm}

\begin{proof}
  Let 
  \[
     Z := \{v\} \cup \{\{v_1,\ldots,v_{r-1}\}:\mbox{$(v_0,\ldots,v_r)\in\mathcal{P}$, $v\in \{v_0,\ldots,v_r\}$}\}
  \]
  Clearly $G[Z]$ is connected because $Z$ is the union of (vertex sets of) paths in $G$, each of which contains $v$. 
  
  We claim that $v$ participates in a node $x\in V(T)$ if and only if $Z\cap Y_x\neq\emptyset$.  If $v$ participates in $x$ then either $v\in Y_x$, so $Z\cap Y_x\supseteq\{v\}$; or $v\in \{v_0,\ldots,v_r\}$ for some shortcut $(v_0,\ldots,v_r)\in\mathcal{P}$ that crosses $x$, so $Z\cap Y_x\supseteq \{v_i\}$ for some $i\in\{1,\ldots,r-1\}$.  In the other direction, if $Z\cap Y_x\neq\emptyset$, then either $Z\cap Y_x\supseteq \{v\}$, so $v\in Y_x$; or $Z\cap Y_x\supseteq \{v_i\}$ where $i\in\{1,\ldots,r\}$ and $v\in\{v_0,\ldots,v_r\}$, so $v\in V(P)$ for a path $P=(v_0,\ldots,v_r)\in\mathcal{P}$ that crosses $x$. 

  Let $Z_H:=\{x\in V(H): Z\cap Y_x\neq\emptyset\}$.  The connectivity of $G[Z]$ implies that $H[X_H]$ is connected.  
  %Property~\tref{ancestor-edge} implies that, for every $xy\in E(H)$, one of $x$ or $y$ is a $T$-ancestor of the other. 
  Choose $a(v)\in X_H$ to be any member of $X_H$ that does not have a strict $T$-ancestor in $X_H$.  Transitivity of the $T$-ancestor relationship, \tref{ancestor-edge}, and connectivity of $H[X_H]$ implies that $a(v)$ is a $T$-ancestor of every node $x\in X_H$, as required.
\end{proof}

For each $x\in V(T)$, we define the \emph{separator}
\[  
     S_x := \{v\in V(G): a(v)= x\} \enspace .
\]
Observe that $\mathcal{S}:=(S_x : x\in V(T))$ is a partition of $V(G)$.  We let $I:=G^\mathcal{P}/\mathcal{S}$ denote the resulting quotient graph and we let $V(I)\subseteq V(T)$ in the obvious way, so that each $x\in V(I)$ is the vertex obtained by contracting $S_x$ in $G^{k,d}$. (Nodes $x\in V(T)$ with $S_x=\emptyset$ do not contribute a vertex to $I$.)

\begin{clm}\clmlabel{s-subset}
  For every $x\in V(T)$, $S_x\subseteq V_x$.
\end{clm}

\begin{proof}
  For the sake of contradiction, assume otherwise, so there exists some $v\in S_x\setminus V_x$.
  
  By \yref{y-subsets}, $Y_x\subseteq V_x$, so $v\not\in Y_x$.  Therefore, $\mathcal{P}$ contains a path $P$ that crosses $x$.  The path $P$ contains a subpath $v_0,v_1,\ldots,v_{r}$ such that $v=v_0$ and $v_r\in Y_x$. Since $v\not\in V_x$ and $v_r\in Y_x\subseteq V_x$, \yref{separator}, implies that $v_i\in F_x$ for some $i\in\{0,\ldots,r-1\}$. Now \yref{ancestor-edge} implies $v_i\in Y_a$ for some strict $T$-ancestor $a$ of $x$.  Therefore, either $v\in Y_a$ or $P$ crosses $a$. But this implies that $a(v)$ is an ancestor of $a$, contradicting the assumption that $v\in S_x$.
\end{proof}

Next we show that $\mathcal{S}$ has small layered width with respect to $\mathcal{L}$:

\begin{clm}\clmlabel{general-width}
  For each $i\in\{1,\ldots,h\}$ and each $x\in V(I)$, $|S_x\cap V_i|\le d\ell(k^2+k+4)/2$.  
\end{clm}

\begin{proof}
  We will count the number of vertices in $S_x\cap V_i$ by upper-bounding the number of vertices contributed to $S_x\cap V_i$ by each vertex $w\in Y_x$.
  If $w\in Y_x\cap V_i$ and no path in $\mathcal{P}$ includes $w$ as an internal vertex then $w$ contributes only one vertex, itself, to $S_x\cap V_i$.
  
  Otherwise, consider some path $P\in\mathcal{P}$ that contains $w$ as an internal vertex.  If $w\in V_{i}$, then $P$ contributes at most $k+1$ vertices to $S_x\cap V_i$.  If $w\in V_{i-1}\cup V_{i+1}$, then $P$ contributes at most $k$ vertices to $S_x\cap V_i$. If $w\in V_{i-j}\cup V_{i+j}$ for $j\ge 2$, then $P$ contributes at most $k-j$ vertices to $S_x\cap V_i$.
  
  For any $j$, the number of vertices $w\in V_{i+j}\cap Y_x$ is at most $\ell$. Each such vertex $w$ is an internal vertex of at most $d$ paths in $\mathcal{P}$. Therefore, 
  \[  |S_x\cap V_i|\le d\ell\cdot\left(k+1 + k + \sum_{j=2}^k (k-j)\right) %= d(2k+1) + \sum_{i=1}^{k-2} i
      = d\ell(k^2 +k + 4)/2 \enspace . \qedhere
  \]
\end{proof}

\begin{clm}\clmlabel{i-ancestor}
  For each edge $xy\in E(I)$, one of $x$ or $y$ is a $T$-ancestor of the other.
\end{clm}

\begin{proof}
  Suppose, for the sake of contradiction, that neither $x$ nor $y$ is a $T$-ancestor of the other.  Since $xy\in E(I)$, $G^\mathcal{P}$ contains an edge $vw$ with $v\in S_x$ and $w\in S_y$.  Since $xy\in E(G^{\mathcal{P}})$,  $\mathcal{P}$ contains a $vw$-shortcut $P$.  By \clmref{s-subset}, $v\in V_x$ and $w\in V_y$.  By \yref{containment-i}, if neither $x$ nor $y$ is a $T$-ancestor of the other, then $V_x$ and $V_y$ are disjoint.  By \yref{ancestor-edge} $N_x$ and $V_y$ are also disjoint.  By \yref{separator} $P$ contains an internal vertex $v'\in F_x$.  By \yref{ancestor-edge}, $v'\in Y_a$ for some strict $T$-ancestor $a$ of $x$.  But this implies that $v\in S_{a'}$ for some $T$-ancestor $a'$ of $a$, contradicting the assumption that $v\in S_x$.
\end{proof}

\begin{clm}\clmlabel{general-bag-size}
  The graph $I$ has a tree-decomposition in which every bag has size at most $\binom{k+t}{t}$.
\end{clm}

\begin{proof}
  For the tree-decomposition $(C_x:x\in V(T))$ of $I$ we use the same tree $T$ used in the tree-decomposition $(B_x:x\in V(T))$ of $H$. For each node $x$ of $T$, $C_x$ contains $x$ as well as every $T$-ancestor $a$ of $x$ such that $I$ contains an edge $ax'$ where $x$ is a $T$-ancestor of $x'$ (including the possibility that $x=x'$).
  \clmref{i-ancestor} ensures that, for every edge $ax'\in E(I)$, $a,x'\in C_{x'}$ and the connectivity of $T[a]:=T[\{x\in V(T):a\in C_x]$ is obvious.  Therefore $(C_x:x\in V(T))$ is indeed a tree-decomposition of $I$.  What remains is to bound the size, $\max\{|C_x|:x\in V(T)\}$, of the largest bag.
  
  Consider an arbitrary node $x\in V(T)$ whose ancestors are $x_0,\ldots,x_{r-1},x_r$ where $x_0$ is the root of $T$ and $x_r:=x$.  To avoid triple-subscripts in what follows, we abuse notation slightly by using $V_i$, $F_i$, and $N_i$,  as shorthands for $V_{x_i}$, $F_{x_i}$ and $N_{x_i}$, respectively.   

  If $x_\delta\in C_x$, it is because $x_\delta x'\in E(I)$ for some $T$-descendant $x'$ of $x$.  This implies $G^{\mathcal{P}}$ contains an edge $vw$ with $v\in S_{x'}$ and $w\in S_{x_\delta}=S_\delta$.  This implies that $\mathcal{P}$ a $vw$-shortcut $P_{vw}$.  Let $v'$ be the second-last vertex of $P_{vw}$ (so $v'w\in E(G)$).  
  
  Since $w\in S_{\delta}$, at least one of the following is true:
  \begin{enumerate}
    \item $\mathcal{P}$ contains a $ww'$-shortcut $P_{ww'}$ that has an internal vertex in $Y_{\delta}$; or
    \item $w\in Y_\delta$.  In this case, we define $P_{ww'}$ to be the path of length 0 that contains only $w=w'$. 
  \end{enumerate}
  Let $w''$ denote the first vertex of $P_{ww'}$ contained in $Y_{\delta}$.
  
  Let $w_0,w_1,\ldots,w_p$ be the path that begins $w_0:=v'$ and then follows the subpath of $P_{ww'}$ that begins at $w_1:=w$ and ends at $w_p:=w''$.  For each $i\in\{0,\ldots,p\}$, let $s_i=\max\{j\in\{1,\ldots,r\}: \{w_0,\ldots,w_i\}\subseteq V_{j}\}\}$, and let $a_i=x_{s_i}$.  Note that $s_0,\ldots,s_p$ is a non-increasing sequence and $a_0,\ldots,a_p$ is a sequence of nodes of $T$ whose distance from the root is non-increasing.

  By definition, $a_0=x_r$.  We claim that $a_p=x_\delta$, i.e., $s_p=\delta$. 
  To see this, first observe that, for each $i\in\{1,\ldots,p\}$, $w_i\in V_{\delta}$ since, otherwise, an internal vertex of $P_{ww'}$ belong to $F_\delta$, which would imply (by \yref{ancestor-edge}) that $w\in S_{\delta'}$ for some $\delta' < \delta$, contradicting the assumption that $w\in S_\delta$.  Therefore $s_p\ge\delta$.  To see that $s_p<\delta-1$,
  observe that either $w=w''\in Y_\delta$ or $P_{ww'}$ contains an internal vertex $w''$ in $Y_\delta$.  By \yref{thickness} and the defnition of $V_x$, $V_{\delta-1}$ does not contain $w''$, so $s_p<\delta-1$.
  
  Let $H^+$ denote the supergraph of $H$ with vertex set $V(T)$ and in which $xy\in E(H^+)$ if and only there exists some $z\in V(T)$ such that $x,y\in B_z$. 
  % Note that, by \tref{ancestor-edge}, $xy\in E(H^+)$ if and only if $x\in B_y$ or $y\in B_x$.
  We claim that $a_0,\ldots,a_p$ is a lazy walk\footnote{A \emph{lazy walk} in a graph $H$ is a walk in the non-simple graph $H'$ obtained by adding a self loop to each vertex of $H$.} in $H^+$.  Indeed, if $a_i\neq a_{i+1}$ for some $i\in\{0,\ldots,p-1\}$ then this is precisely because $w_i\in V_{a_i}$ but $w_{i+1}\not\in V_{a_i}$.  By definition, $w_i\in Y_{a_i'}$ for some $T$-descendant $a_i'$ of $a_i$.
  By \yref{separator}, $w_{i+1}\in F_{a_i}$ so by \yref{ancestor-edge} $w_{i+1}\in Y_{a_i''}$ for some strict $T$-ancestor $a_i''$ of $a_i$.  Since $w_iw_{i+1}\in E(G)$, $a_i'a_i''\in E(H)$.  By \tref{subtree-root}, $a_i''\in B_{a_i''}$ and $a_i''\in B_{a_i'}$.  Since $a_i$ is on the path from $a_i'$ to $a_i''$ in $T$ this implies that $a_i''\in B_{a_i}$.  Therefore $a_ia_i''\in E(H^+)$ as claimed.
  
  Thus, $a_0,\ldots,a_p$ is a lazy walk in $H^+$ of length $p\le k$ where the distance $s_i$ between $a_i$ and the root $x_0$ of $T$ is non-decreasing.  By removing repeated vertices this gives a path in the directed graph $\overrightarrow{H}^+$ obtained by directing each edge $xy\in E(H^+)$ from its $T$-descendant $x$ towards its $T$-ancestor $y$. 
  Finally, we are in a position to appeal to \cite[Lemma~24]{pilipczuk.siebertz:polynomial-arxiv} which states that the number of nodes in $\overrightarrow{H}^+$ that can be reached from any node $x$ by a directed path of length at most $k$ is at most $\binom{k+t}{t}$.
\end{proof}

% Together, \clmref{general-width} and \clmref{general-bag-size} imply the following generalization of \thmref{k-planar}:

\begin{thm}\thmlabel{modern-major-general}
  Let $G$ be a graph having a layered $H$-partition of layered width $\ell$ in which $H$ has treewidth at most $t$ and let $\mathcal{P}$ be a $(k,d)$-shortcut system for $G$.  Then $G^{\mathcal{P}}$ has a layered $I$-partition of layered width at most $d\ell(k^3+k^2+4k)/2$ where the treewidth of $H$ is at most $\binom{k+t}{t}-1$.
\end{thm}

\begin{proof}
  The proof is almost immediate, except that the layering $\mathcal{L}$ of $G$ may not be a valid layering of $G^{\mathcal{P}}$.  In particular, $G^{\mathcal{P}}$ may contain an edge $vw$ with $v\in V_i$ and $w\in V_{i+j}$ for any $j\in\{0,\ldots,k\}$.  To resolve this, we use a new layering $\mathcal{L}':=\langle V_0',\ldots,V_h'\rangle$ in which $V_i'=\bigcup_{j=ki}^{ki+k-1} V_i$.  This increases the thickness given by \clmref{general-width} from $d\ell(k^2+k+4)/2$ to $d\ell(k^3+k^2+4k)/2$.
\end{proof}

% It is natural to ask if \thmref{modern-major-general} is tight.  We do not answer this question, but we do point out that, even when $t=1$ and $d=3$, an exponential dependence on $k$ is necessary.  If $G$ is a complete binary tree on $2^{k/2}$ leaves, then $G^{k,3}$ is a clique of size  $s=2^{k/2+1}-1$.  Any partition of $G^{k,3}$ into $r$ parts will give an quotient graph $I$ that is an $r$-clique, which has treewidth $r-1$. On the other hand, any layering of $V(G^{k,3})$ has at most two layers, so the layered width of any partition into $r$ parts is at least $s/(2r)$. Therefore any layered $I$-partition of $G^{k,d}$ of layered width $\ell'$ in which the treewidth of $I$ is at most $t'-1$ must satisfy
\[
    \ell'\times t' \ge 2^{k/2}-1/2 \enspace .
\]   
\newpage
  
\section{$1$-Planar Graphs}
\seclabel{1-planar}

Let $G$ be an edge-maximal 1-plane multigraph with no two parallel edges on the boundary of a single face.  Here, edge-maximal should be taken to mean that, if any two vertices $v$ and $w$ appear on a common face\footnote{The \emph{faces} of an embedded graph $G$ are the connected components of $\R^2\setminus \bigcup_{vw\in E(G)} vw$.  We say that a vertex $v\in V(G)$ appears on a face $F$ if $v$ is contained in the closure of $F$.} $F$, then there is an edge $vw\in E(G)$ that is contained in the boundary of $F$.  We assume that no two edges incident to a common vertex cross each other since, in a 1-plane graph, such a crossing can always be removed by a local modification to obtain an isomorphic 1-plane graph in which the two edges do not cross.\footnote{While this is true for 1-plane graphs it is not true for $k$-plane graphs with $k\ge 3$; the uncrossing operation can increase the number of crossing on a particular edge from $k$ to $2(k-1)$.}

A \emph{kite} in $G$ is the subgraph $K=G[\{v,w,x,y\}]$ induced by the endpoints of a pair of crossing edges $vw,xy\in E(G)$.  It follows from edge-maximality that every kite is isomorphic to the complete graph $K_4$.
The edges $vw$ and $xy$ are called \emph{spars} of $K$.  The cycle $vxwy$ is called the \emph{sail} of $K$.  It follows from edge-maximality that none of the edges $vx$, $xw$, $wy$, or $yv$ are crossed by any other edges of $G$. Thus any edge that is a spar of a kite $K$ is not part of a sail of any kite $K'$. Observe that any spar of $K$ is incident on exactly four \emph{kite faces} of $G$, each of which has three edges and two vertices of $G$ on its boundary.

The 1-plane graph $G$ has a plane triangulation $G'$ as a subgraph that can be obtained by removing one spar from each kite in $G$.  Observe that, for any spar $xy\in E(G)\setminus E(G')$ that crosses $vw\in E(G')$, $G'$ contains the path $vxw$ (and $vyw$).  It follows that 
\begin{equation}
  \dist_{G'}(v,w)\le 2 \enspace . \eqlabel{spar-distance}
\end{equation}

Our proof of \thmref{1-planar} follows quickly from the following technical lemma, which is an extension of the analagous result for plane graphs \cite{dujmovic.joret.ea:planar}.
\begin{lem}\lemlabel{induction} The setup:
  \begin{compactenum}
    \item Let $G$ and $G'$ be defined as above.
    \item Let $T$ be a BFS spanning tree of $G'$ rooted at some vertex $r$.
    \item For each integer $j\ge 0$, let $V_j=\{v\in V(G):\dist_T(r,v)=j\}$. 
    \item Let $F$ be a cycle in $G'$ with $r$ in the exterior of $F$ and such that
    \begin{compactenum} 
      \item No edge of $F$ is crossed by any edge of $G$; and
      \item $V(F)$ can be partitioned into $P_1,\ldots,P_k$, for some $k\in\{1,2,3\}$ such that for each $i\in\{1,\ldots,k\}$,
      \begin{compactenum}
        \item $F[P_i]$ is a path; and
        \item $|V(P_i)\cap V_j| \le 15$ for all integers $j\ge 0$.
      \end{compactenum}
    \end{compactenum}
    \item Let $N$ and $N'$ be the subgraphs of $G$ and $G'$ consisting only of those edges and vertices contained in $F$ or the interior of $F$.
  \end{compactenum}
  Then $N$ has an $H$-partition $\mathcal{P}=\{S_x : x\in V(H)\}$ such that
  \begin{compactenum}
    \item $H$ is planar;
    \item for all integers $j\ge 0$, and all $x\in V(H)$, $|S_x\cap V_j|\le 15$; 
    \item for each $i\in\{1,\ldots,k\}$, there exists some $x_i\in V(H)$ such that $P_i=S_{x_i}$; and
    \item $H$ has a tree decomposition whose largest bag has size at most 4 and such that some bag contains $x_1,\ldots,x_k$.
  \end{compactenum}
\end{lem}

\begin{proof}
  This proof is very similar to the proof of Lemma~14 in \citet{dujmovic.joret.ea:planar}. Rather than duplicate every detail of that proof here, we focus on the differences and refer the reader to the original proof for the remaining details.
  
  The proof is by induction on the number of vertices of $N$.
  First note that $N'$ is a near-triangulation.  If $k=3$, set $R_i=P_i$ for each $i\in\{1,2,3\}$.  Otherwise, as in \citet{dujmovic.joret.ea:planar}, split $P_1,\ldots,P_k$ to partition $V(F)$ into three sets $R_1$, $R_2$, and $R_3$ such that each $F[R_i]$ is a non-empty path and each $R_i$ contains vertices from exactly one of $P_1,\ldots,P_k$. 
  
  Next, as in \citet{dujmovic.joret.ea:planar}, use Sperner's Lemma to find an inner face $\tau=v_1v_2v_3$ of $N'$ such that, $T$ contains disjoint vertical paths $Q_1,Q_2,Q_3$ such that each $Q_i$ begins at $v_i$, ends at some vertex in $R_i$, and whose internal vertices (if any) are contained in $N'-V(F)$.
  
  Let $\overline{Y}$ denote the subgraph of $N'$ consisting of vertices and edges of $Q_1$, $Q_2$, $Q_3$, and $\tau$.  Let $\overline{Y}^+$ denote the subgraph of $N$ consisting of the vertices and edges of $\overline{Y}$ plus the vertices and edges of every kite formed by a crossing between an edge of $G$ and an edge of $\overline{Y}$.
  
  We claim that, for each integer $i\ge 0$, $|V(\overline{Y}^+)\cap V_i|\le 15$.  First observe that, since $Q_1,Q_2,Q_3$ are each vertical paths in $T$,  $\overline{Y}$ contains at most three vertices of $V_i$, each incident on at most two edges of $\overline{Y}$.  By \eqref{spar-distance}, any vertex $x\in V(\overline{Y}^+)\setminus V(\overline{Y})\cap V_i$, is incident to an edge $xy\in E(G)$ that crosses one of the at most six edges in $\overline{Y}$ having an endpoint in $V_i$.  These at most six edges have at most 12 endpoints.  Therefore $|V(\overline{Y}^+)\setminus V(\overline{Y})\cap V_i|\le 6\times 2=12$, so $|V(\overline{Y}^+)\cap V_i|\le 12+3=15$.
  
  Let $M$ and $M^+$ denote the subgraph of $G$ containing the edges and vertices of $\overline{Y}$, respectively $\overline{Y}^+$, and the edges and vertices of $F$.  The graph $M^+$ has some number of bounded faces, all contained in the interior of $F$. Some of the bounded faces of $M^+$ are kite faces. Let $F_1,\ldots,F_m$ be the non-kite bounded faces of $M^+$.  
  
  We claim that, for each $i\in\{1,\ldots,m\}$, the boundary of $F_i$ is a cycle in $G'$ that contains no spars. Otherwise, some edge $vw$ contributes to the boundary of $F_i$ but is crossed by an edge $xy\in E(G)$. Then, $vw\not\in E(F)$ since no edge of $F$ is crossed by any edge of $G$. Therefore $vw\in E(\overline{Y}^+)$ so $xy\in E(Y^+)$. But then the only faces of $M^+$ incident to $vw$ are kite faces.  In particular $vw$ cannot be incident to the non-kite face $F_i$.
  
  Note that $M$ is a Mercedes graph and that each of the faces $F_1,\ldots,F_m$ is contained in a single wedge of $M$.   Let $Y^+ := \overline{Y}^+-F$. Therefore, $V(F_i)$ can be partitioned into at most three sets $P_1'$, $P_2'$, and $P_3'$ where $P_1'\subset V(Y^+)$, $P_2'\subseteq P_a$, $P_3'\subseteq P_b$ for some $a,b\in\{1,2,3\}$, and $F_i[P_j']$ is a path, for each $j\in\{1,2,3\}$. 

  Finally, the subgraph $N_i$ of $G$ consisting of the edges and vertices of $G$ contained in $F_i$ or its interior does not contain one of the three vertices of $\tau$. Therefore, we can apply induction using the cycle $F_i$ and the partition $P_1',P_2',P_3'$ of $V(C_i)$ to obtain the desired $H$-partition and tree decomposition of $N_i$.
  
  The remainder of the proof finishes in the same way as the proof of Lemma~14 in \cite{dujmovic.joret.ea:planar}.  The paths $P_1,\ldots,P_k$, and $S=V(Y^+)$ become elements of the $H$-partition.  Elements in each of the $H$-partitions of $N_1,\ldots,N_3$ that intersect $P_1,\ldots,P_k$, or $V(\overline{Y}^+-F)$ are discarded and all the resulting sets are combined to obtain an $H$-partition of $G$.  The desired tree decomposition of $G$ is obtained in exactly the same way as in the proof of Lemma~14 in \cite{dujmovic.joret.ea:planar}, except that now each node $x$ has a child for each face $F_i$ of $M^+_x$ that contains a vertex of $G$ in its interior.
  
  The planarity of $H$ comes from two properties:
  \begin{enumerate}
    \item $G/\mathcal{P}$ and $G^+/\mathcal{P^+}$ are isomorphic, where $G^+$ is the triangulation obtained by adding dummy vertices at each crossing in $G$ and $\mathcal{P}^+$ is the partition we obtain by adding a dummy vertex $z$ to $\overline{Y}^+$ if $\overline{Y}^+$ contains an edge $vw$ that contains $z$ in its interior.  
    
    \item $G^+[\overline{Y}^+-F]$ is connected. To see why this is so, first observe that $\overline{Y}-F$ is connected, and then observe that every vertex of $\overline{Y}^+$ is either a vertex of $\overline{Y}$ or adjacent to a vertex of $\overline{Y}$.
  \end{enumerate}
  Since $G^+$ is planar, the second point implies that $H=G^+/\mathcal{P}$ is planar.
  % \footnote{It is well-known and easy to see that any graph $\Delta'$ obtained by contracting an edge in a planar graph $\Delta$ is also planar.  Repeatedly applying this fact implies that $\Delta/\mathcal{P}$ is planar provided that $\Delta[P]$ is connected for every $P\in\mathcal{P}$.} The first point then ensures that $G/\mathcal{P}$ is also planar.
\end{proof}

Using \lemref{induction}, the proof of \thmref{1-planar} is now straightforward. 

\begin{proof}[Proof of \thmref{1-planar}]
Given a 1-plane graph $G$, add edges to make it edge maximal so that it has an outer face $F=v_1v_2v_3$. Next, add a vertex $r$ adjacent to $v_1$, $v_2$, and $v_3$ to obtain an edge-maximal 1-plane graph $\overline{G}$ with one vertex $r$ of degree 3 on its outer face. 
  
  Let $G'$ be the plane graph obtained by removing one spar from each kite of $\overline{G}$ and let $T$ be a BFS tree of $G'$ rooted at $r$.  Now apply \lemref{induction} with $G=\overline{G}$, $G'$, $F$, and $P_i=\{v_i\}$ for each $i\in\{1,2,3\}$.  This gives an $H$-partition $\{S_x:x\in V(H)\}$ of $\overline{G}-\{r\}\supseteq G$ in which $H$ is planar and has treewidth at most 3.
  
  Use the layering $\mathcal{L}=\langle V_0',V_1'\ldots\rangle$ where $V_i'=V_{2i}\cup V_{2i+1}$ for each integer $i\ge 0$. That this is a layering of $G$ follows from \eqref{spar-distance}.  Since $|V_j\cap S_x|\le 15$ for every integer $i\ge 0$, $|V_i'\cap S_x|\le 30$ for every integer $i\ge 0$ and every $x\in V(H)$.
\end{proof}

Exactly the same argument used to establish \corref{k-planar} as a consequence of \thmref{k-planar} establishes the following result as a consequence of \thmref{1-planar}:

\begin{cor}\corlabel{1-planar}
  Let $G$ be a $1$-plane graph, let $r_0$ be an arbitrary vertex of $G$, let $G^+$ be a graph obtained by adding a dummy vertex at each crossing in $G$ and then arbitrarily adding non-crossing edges to obtain a triangulation.  Then $G$ has a layered $H$-partition $(\mathcal{L}=\langle V_0',V_1',\ldots\rangle,\mathcal{P})$ satisfying the conditions of \thmref{1-planar} and such that $V_i' = \{v\in V(G): \floor{\dist_{G^+}(r_0,v)/2}=i\}$ for each integer $i\ge 0$.
\end{cor}


\section{$(g,k)$-Planar Graphs}
\seclabel{g-k-planar}

Next we prove \thmref{g-k-planar}, which generalizes our results to $(g,k)$-plane graphs.  We make use of the following lemma due to \citet[Lemma~21]{dujmovic.joret.ea:planar}:


\begin{lem} \lemlabel{make-planar}
Let $G^+$ be a $(\Sigma,0)$-plane graph where $\Sigma$ is a surface with Euler genus $g$. 
For each $r\in V(G^+)$ there is a vertex set $Z\subseteq V(G^+)$ such that $G^+-V(Z)$ is planar and there is a connected planar graph $G^*$ containing  $G^+-Z$ as a subgraph with a vertex $r^*\in V(G^*)$ such that, for every integer $i\ge 0$,
\[  \{v\in V(G^*): \dist_{G^*}(r^*,v) = i\} \supseteq \{v\in V(G^+): \dist_{G^+}(r,v) = i\} \]
and
\[ |\{v\in Z: \dist_{G^+}(r,v) = i\}| \le 2g \enspace . \]

\end{lem}

\begin{proof}[Proof of \thmref{g-k-planar}]
  Let $G$ be an $(\Sigma,k)$-plane graph where $\Sigma$ is a surface of Euler genus $g$.  Let $G^+$ be the $(\Sigma,0)$-plane graph obtained by adding a dummy vertex at each crossing in $G$.  Let $r_0$ be any vertex in $V(G)$, let $T$ be BFS spanning tree of $G^+$ rooted at $r_0$ and let $\langle V_0,V_1,\ldots\rangle$ be the resulting layering of $G^+$.
    
  Apply \lemref{make-planar} to $G^{+}$ to obtain the set $Z\subseteq V(G^{+})$ and the planar graph $G^*$ with special vertex $r^*$.  For each edge $vw\in E(G)$, let $W_{vw}$ be the corresponding walk in $G^+$. The set $Z$ contains vertices in $V(G)$ as well as dummy vertices in $V(G^+)-V(G)$. Create a new set $Z'\subseteq V(G)$ from $Z$ by replacing each dummy vertex $z\in Z\cap (V(G^+)\setminus V(G))$ with one endpoint from each of the edges that cross $vw,xy\in E(G)$ that cross at $Z$.  In this way, each vertex $z\in Z\cap V(G^+)\setminus V(G)$ can be charged for at most two vertices in $v,x\in Z'$ and this happens only if $(z,vw,xy)$ is a crossing in $G$.
  Since $|Z\cap V_i| \le 2g$, the same argument used in the proof of \clmref{width-of-g-plus} shows that $|Z'\cap V_i|\le 2g\times 2\times (2k+1) = 4g(2k+1)$ for each integer $i\ge 0$.
  
  % For every edge $vw\in E(G)$, $W_{vw}$ has length at most $k+1$.  Therefore $\dist_{G^+}(v,w)\le k+1$ for each $v\in E(G)$.  Since $W_{vw}\subseteq G^+-Z$ for each $vw\in E(G-Z')$,  $\dist_{G^+-Z}(v,w)\le k+1$ for each $vw\in E(G-Z')$.  
  
  Next, observe that any plane embedding of $G^*\supseteq G^+-Z$ gives a $k$-plane embedding of $G-Z'$. Indeed, for every edge $vw\in E(G-Z')$, $G^+-Z$ contains a walk $W_{vw}$ that contains at most $k$ dummy vertices of $G^+-Z$.  Therefore, replacing edge $vw\in E(G-Z')$ with the corresponding walk in $G^*$ gives a $k$-plane embedding of $G-Z'$.  Call the resulting $k$-plane graph $G^*_0$.
  
  Let $G^*_1$ be the supergraph of $G^*_0$ obtained by adding every vertex and edge of $G^*$ that is not contained in some edge of $G^*_0$.  Finally, let $G^{*+}_1$ be the triangulation obtained by adding edges to $G^*$ so that $\dist_{G^*}(r^*,v)=\dist_{G^{*+}}(r^*,v)$ for every $v\in V(G^*)$.  The fact that $G^*$ can be completed to a triangulation $G^{*+}$ while preserving distances to $r^*$ follows from the fact that any face $F$ of $G^*$ with four or more vertices contains two non-adjacent vertices $x$ and $y$ such that $|\dist_{G^*}(r^*,x)-\dist_{G^*}(r^*,y)|\le 1$.  Indeed, if $z$ is a vertex of $F$ that minimizes $\dist_{G^*}(r^*,z)$, then two neighbours $x$ and $y$ of $z$ have this property. Now, observe that $G^*_1$ is a $k$-plane graph and $G^{*+}_1$ is obtained by adding dummy vertices at crossings in $G^*_1$ and triangulating the resulting graph.    
  
  Apply \corref{k-planar} or (in the case $k=1$) \corref{1-planar} to $G^*_1$ using $G^{*+}_1$ and the distinguished vertex $r^*$ to obtain a layered $H$-partition of $G^*_1$.  Since $G^*_1\supseteq G^*_0$ and $G^*_0$ is isomorphic to $G-Z'$, this gives a layered $H$-partition $(\mathcal{L}=\langle V_0',V_1',\ldots\rangle,\mathcal{P})$ of $G-Z'$ that satisfies the conditions of \corref{k-planar} or \corref{1-planar}, as appropriate.  In particular, for every integer $i\ge 0$,
  \begin{align*} 
    V_i'& = \{v\in V(G-Z'):\floor{\dist_{G^*}(r^*,v)/(k+1)} = i\} \\
        & = \{v\in V(G-Z'):\floor{\dist_{G^+}(r_0,v)/(k+1)} = i\} \enspace .
  \end{align*}
  It follows that $|V_i'\cap Z'|\le 4g(2k+1)(k+1) = 8gk^2+12gk+4g$.  
  
  We now turn $\mathcal{P}$, the $H$-partition of $G-Z'$, into a partition $\mathcal{P}'=\mathcal{P}\cup \{Z'\}$ of $G$ by placing all vertices of $Z'$ into one new common part.  Let $H'=G/ \mathcal{P}'$.  Then the treewidth of $H'$ is at most one more than the treewidth of $H$.
  
  % In the worst-case, the vertex of $H'=G/ \mathcal{P}'$ corresponding to $V(G)\cap Z'$ is adjacent to every vertex of $H=(G-Z')/\mathcal{P}$, so the treewidth of $H'$ is at most one more than the treewidth of $H$.
   
  Summarizing, we obtain an $H'$-partition $(\mathcal{L},\mathcal{P}')$ of $G$ where $H'$ has treewidth 
  \[
      \tw(H') 
        \le \begin{cases}
              4 & \text{if $k=1$} \\
              k^3/6 + 3k^2/2 + 13k/3+5 & \text{otherwise}
            \end{cases} 
  \]
  and, for each $P\in\mathcal{P}'$,
  \[ 
    |V_i' \cap P| \le \begin{cases} 8gk^2+12gk+4g & \text{if $P=Z'$} \\
                                    30 & \text{if $P\neq Z'$ and $k=1$} \\
                                    24k^2+60k+36 & \text{otherwise.} 
                      \end{cases} \qedhere
  \]
\end{proof}

\section{Consequences}
\seclabel{consequences}

Here we discuss some of the consequences of Theorems~\ref{thm:k-planar}--\ref{thm:g-k-planar} for other graph parameters that can be bounded using layered partitions and other graph classes that are $k$-planar.

\subsection{Applications to Colouring}

For a graph $G$ and a set $C$ of size $c$, any function $\phi\colon V(G)\to C$  is called a \emph{$c$-colouring} of $G$.  

\subsubsection{Non-Repetitive Colouring}

A $c$-colouring $\phi$ of $G$ is \emph{non-repetitive} if, for every path $v_1,\ldots,v_{2h}$ in $G$, there exists some $i\in\{1,\ldots,k\}$ such that $\phi(v_i)\neq\phi(v_{i+h})$.  The \emph{non-repetitive chromatic number} $\pi(G)$ of $G$ is the minimum value $c$ such that $G$ has a non-repetitive $c$-colouring.

\citet[Corollaries~9 and 10]{dujmovic.esperet.ea:planar} prove the following result:

\begin{lem}\lemlabel{non-repetitive}
  If a graph $G$ has a layered $H$-partition of layered width at most $\ell$ in which $H$ has treewidth at most $t$, then $\pi(G)\le \ell 4^{t+1}$.
\end{lem}

Combining \lemref{non-repetitive} with Theorems~\ref{thm:k-planar}--\ref{thm:g-k-planar} we immediately obtain the following corollaries:

\begin{cor}\corlabel{first}
  For every $k$-planar graph $G$, $\pi(G)\le 12(2k+3)(k+1)\times 4^{(k+2)(k+3)(k+4)/6}$.
\end{cor}

\begin{cor}
  For every $1$-planar graph $G$, $\pi(G)\le 30\times 4^4=7680$. 
\end{cor}

\begin{cor}
  For every $(g,k)$-planar graph $G$, 
  \[ 
    \pi(G)\le \max\{4g(3k+1),12(2k+3)(k+1)\}\times 4^{(k+2)(k+3)(k+4)/6} 
    \enspace .
  \]
\end{cor}

\begin{cor}
  For every $g\ge 1$ and every $(g,1)$-planar graph $G$, $\pi(G)\le 48g\times 256= 12288g$.
\end{cor}

Prior to the current work, the strongest upper bound on the non-repetitive chromatic number of $n$-vertex planar $k$-planar graphs was $O(k\log n)$ \cite{dujmovic.morin.ea:layered}.

\subsubsection{Centered Colourings}

A $c$-colouring $\phi$ of $G$ is \emph{$p$-centered} if, for every connected subgraph $X\subseteq G$, $|\{\phi(v):v\in V(X)\}| > p$ or there exists some $v\in V(X)$ such that $\phi(v)\neq \phi(w)$ for every $w\in V(X)\setminus\{v\}$.  In words, either $X$ receives more than $p$ colours or some vertex in $X$ receives a unique colour.  Let $\chi_p(G)$ denote the smallest integer $c$ such that $G$ has a $p$-centered $c$-colouring.

We make use of the following lemma due to \citet{pilipczuk.siebertz:polynomial-arxiv,pilipczuk.siebertz:polynomial-soda}
\begin{lem}\lemlabel{p-centered-treewidth}
  Any graph $H$ of treewidth at most $t$ has $\chi_p(H)\in O(p^t)$.
\end{lem}

The following lemma is due to Micek \cite{micek:personal}. We include the proof here for completeness.

\begin{lem}\lemlabel{p-centered}
  Any graph $G$ that has a layered $H$-partition of layered width at most $\ell$ in which $H$ has treewidth at most $t$, has $\chi_p(G)\le \ell\times (p+1)\times O(p^t)$.
\end{lem}

\begin{proof}
  Let $(\mathcal{L}=\langle V_0,V_1,\ldots\rangle, \mathcal{P}=(B_x:x\in V(H))$ be an $H$-partition of $G$ having layered width at most $\ell$ where $H$ as treewidth at most $t$.
  We use a product colouring $\phi:V(G)\to \{1,\ldots,\ell\}\times\{0,\ldots,p\}\times\{1,\ldots,O(p^t)\}$.  For each integer $i\ge 0$ and each $x\in V(H)$, the colour $\phi(v)=(\alpha(v),\beta(v),\gamma(v))$ of a vertex $v\in V_i\cap B_x$ is assigned to satisfy the following conditions:
  \begin{enumerate}
    \item $\alpha(v)$ is unique among $\{\phi(w): w\in V_i\cap B_x\}$.
    
    \item $\beta(v)= i\bmod (p+1)$.
    
    \item $\gamma(v)=\gamma(x)$ where $\gamma:V(H)\to\{1,\ldots,O(p^t)\}$ is a $p$-centered colouring of $H$.
  \end{enumerate}
  Condition 1 is easy to satisfy since $|V_i\cap B_x|\le \ell$.  Condition~2 is straightforward.  Condition~3 is possible to satisfy thanks to \lemref{p-centered-treewidth}.  To see why this is a $p$-centered colouring, consider some connected subgraph $X\subseteq G$.
  \begin{enumerate}
    \item If there exists $v,w\in V(X)$ with $v\in V_i$ and $w\in V_j$ with $j-i\ge p$ then, since $G[X]$ is connected, $G[X]$ contains a path from $v$ to $w$.  By the definition of layering, this path contains at least one vertex from $V_{i'}$ for each $i'\in\{i,i+1,\ldots,j\}$. Therefore, $|\{\beta(v'):v'\in X\}|\ge j-i+1 > p$, so $X$ receives more than $p$ distinct colours.
    
    \item Otherwise, $X\subseteq V_{i},\ldots,V_{i+s}$ for some $s<p$.  Let $H'=H[\{x\in V(H):B_x\cap X\neq\emptyset]$.  If $|\{\gamma(x):x\in V(H')\}| > p$ then $|\{\gamma(v):v\in X\}|> p$ so $|\{\phi(v):v\in X\}|> p$ and we are done.  Otherwise, since $\gamma$ is a $p$-centered colouring of $H$, there must exist some $x\in V(H')$ such that $\gamma(x)\neq\gamma(y)$ for every $y\in V(H')\setminus\{x\}$.
    For any $v,w\in B_x$ with $v\neq w$, either $v,w\in V_{i'}$ for some $i'\in\{i,i+1,\ldots,i+s\}$ in which case $\alpha(v)\neq\alpha(w)$; or $v\in V_{i'}$ and $w\in V_{i''}$ with $|i'-i''|< p$, in which case $\beta(v)\neq\beta(w)$. Therefore every vertex $v\in B_x$ receives a colour $\phi(v)$ distinct from every colour in $\{\phi(z):z\in X\setminus\{x\}\}$. Therefore, every vertex in $B_x$ receives colour distinct from every other vertex in $X$.
  \end{enumerate}
\end{proof}

Combining \lemref{p-centered} with Theorems~\ref{thm:k-planar}--\ref{thm:g-k-planar} immediately gives the following results, for every $p\ge 2$:

\begin{cor}
  For every $k$-planar graph $G$, $\chi_p(G)\le p^{O(k^3)}$.
\end{cor}

\begin{cor}
  For every $1$-planar graph $G$, $\chi_p(G)\le O(p^4)$. 
\end{cor}

\begin{cor}
  For every $(g,k)$-planar graph $G$, $\chi_p(G)\le O(gk^2)\times p^{O(k^3)}$. 
\end{cor}

\begin{cor}\corlabel{last}
  For every $(g,1)$-planar graph $G$, $g\ge 1$, $\chi_p(G)\le O(gp^4)$.
\end{cor}

Prior to the current work, the strongest upper bounds on the number of colours required for $p$-centered colourings of $(g,k)$-planar graphs used the fact that $(g,k)$-planar graphs are have layered treewidth $O(gk)$ \cite{dujmovic.eppstein.ea:structure} and therefore have bounded (indeed, linear) expansion \cite{dujmovic.morin.ea:layered}.  Using a result of \citet{zhu:colouring} this leads to bounds on $\chi_p(G)$ that are doubly-exponential in $gk$.

\note{PM}{Please check the preceding paragraph.}

% \note{PM}{I didn't follow the discussion on the previous best-known bounds for $p$-centered colourings.  Can someone fill me in?}

\subsection{Other Graph Classes}

We conclude by pointing out three other graph classes, namely $(g,d)$-map graphs, $(g,\delta)$-string graphs, and $k$-nearest-neighbour graphs that are $(g,k)$-planar for $k=O(d^2)$ and $k=\delta^2$, respectively.  This implies that Theorems~\ref{thm:k-planar}--\ref{thm:g-k-planar} and Corollaries~\ref{cor:first}--\ref{cor:last} apply to $(g,d)$-map graphs, $(g,\delta)$-string graphs, and $k$-nearest-neighbour graphs.


\subsubsection{Map Graphs}

Map graphs are defined as follows. Start with a graph $G_0$ embedded in a surface of Euler genus $g$, with each face labelled a `nation' or a `lake', where each vertex of $G_0$ is incident with at most $d$ nations. Let $G$ be the graph whose vertices are the nations of $G_0$, where two vertices are adjacent in $G$ if the corresponding faces in $G_0$ share a vertex. Then $G$ is called a \emph{$(g,d)$-map graph}. 
% A $(0,d)$-map graph is called a (plane) \emph{$d$-map graph}; such graphs have been extensively studied \citep{FLS-SODA12,DFHT05,CGP02,Chen07,Chen01}. The $(g,3)$-map graphs are precisely the graphs of Euler genus at most $g$ (see \citep{DEW17}). So $(g,d)$-map graphs provide a natural generalisation of graphs embedded in a surface.

The following lemma comes from the natural drawing of a map graph obtained by positioning each vertex of $G$ inside the corresponding nation and each edge of $G$ as a curve passing through the corresponding vertex of $G_0$. See \citet[Page~4]{dujmovic.eppstein.ea:structure}.

\begin{lem}
  Every $(g, d)$-map graph is $(g, \floor{\frac{d-2}{2}}\ceil{\frac{d-2}{2}})$-planar.
\end{lem}

\subsubsection{String Graphs}

For a surface $S$ and a graph $G$, an \emph{$S$-string representation} $f$ of $G$ is obtained by representing each vertex of $v\in V(G)$ as a curve $f(v)$ in $S$ so that, for any pair of vertices $v,w\in V(G)$, the curves $f(v)$ and $f(w)$ intersect if and only if $vw\in E(G)$.  If $G$ has maximum degree $\delta$, then an $S$-representation of $G$ is also called an $(S,\delta)$-string representation.  A graph $G$ is a \emph{$(g,\delta)$-string graph} if it has an $(S,\delta)$-string representation where $S$ is a surface of Euler genus $g$.

\begin{lem}
  Every $(g,\delta)$-string graph is $(g,2\delta^2)$-planar.
\end{lem}

\begin{proof}
  Consider some $(S,\delta)$-string representation, $f$, of $G$.  We will describe an embedding of $G$ into $S$ that is an $(S,k)$-plane graph.  Each vertex $v\in v(G)$ is embedded at $f(v)(0)$. For each edge $vw\in E(G)$ there is an intersection point $p$ between $f(v)$ and $f(w)$, i.e., $f(v)(t_0)=f(w)(t_1)=p$ for some $t_0,t_1\in[0,1]$.  Embed the edge $vw$ very close to the curve whose image is $\{f(v)(t): 0\le t\le t_1\}\cup f(w)(t): 0 \le t\le t_2\}$. The embedding of any edge $vw\in E(G)$ is contained in $f(v)$ and $f(w)$.  Therefore the embedding of $vw$ crosses at most $2\delta$ other curves $f(x_1),\ldots,f(x_d)$.  Each curve $f(x_i)$ is used to embed at most $\delta$ edges incident to $x_i$.  Therefore the embedding of any edge $vw\in E(G)$ crosses at most $2\delta^2$ other edges.  
\end{proof}

\section{$k$-Nearest-Neighbour Graphs}

In this section, we show that $k$-Nearest neighbour graphs of point in the plane are $O(k^2)$-planar.  For two points $x,y\in\R^2$, let $d_2(x,y)$ denote the Euclidean distance between $x$ and $y$. The $k$-nearest-neighbour graph of a point set $P\subset\R^2$ is the geometric graph $G$ with vertex set $V(G)=P$ and edge set
\[  
  E= \{vw : v,w\in V(G),\, |\{x\in P\setminus\{v\}: d_2(v,x)\le d_2(v,w)\}|\le k\} \enspace .
\] 
(The edges of $G$ are straight-line segments joining their endpoints.)

The following result, which is immediate from \citet[Corollary~4.2.6]{abrego.munroy.ea:on} states that $k$-nearest-neighbour graphs have bounded maximum degree:
\begin{lem}\lemlabel{k-nn-max-degree}
  The degree of any vertex in a $k$-nearest-neighbour graph is at most $6k$.
\end{lem}

We make use of the following well-known observation (see e.g., \citet[Lemma~2]{bose.morin.ea:routing}):
\begin{obs}\obslabel{convex}
  If $v_0,\ldots,v_3$ are the vertices of a convex quadrilateral in counterclockwise order then there exists at least one $i\in\{0,\ldots,3\}$ such that $\max\{d_2(v_i,v_{i-1}), d_2(v_i,v_{i+1})\} < d_2(v_{i-1},v_{i+1})$, where subscripts are taken modulo 4.
\end{obs}

\begin{lem}\lemlabel{nearest-neighbour}
  Every $k$-nearest-neighbour graph is $O(k^2)$ planar.
\end{lem}

\begin{proof}
  Let $G$ be a $k$-nearest-neighbour graph and consider any edge $vw\in E(G)$.
  Let $xy\in E(G)$ be an edge that crosses $vw$.  Note that $vxwy$ are the vertices of a convex quadrilateral in (without loss of generality) counterclockwise order. Then we say that 
  \begin{compactenum}
    \item $xy$ is of Type~$v$ if $\max\{d_2(v,x), d_2(v,y)\}< d_2(x,y)$;
    \item $xy$ is of Type~$w$ if $\max\{d_2(w,x), d_2(w,y)\}< d_2(x,y)$; or
    \item $xy$ is of Type~C otherwise.
  \end{compactenum}
  If $xy$ is of Type~C, then \obsref{convex} implies that (without loss of generality) $\max\{d_2(x,v),d_2(x,w)\} < d_2(v,w)$.  In this case, we call $x$ a Type~C vertex.  We claim that $V(G)$ contains at most $k-1$ Type~C vertices.  Indeed, more than $k-1$ Type~C vertices would contradict the fact that $vw\in E(G)$ since every Type~C vertex is closer to both $v$ and $w$ than $d_2(v,w)$.
  
  Next oberve that, if $xy$ is of Type~$v$, then at least one of $xv$ or $yv$ is in $E(G)$ in which case we call $x$ (respectively $y$) a Type~$v$ vertex.  By \lemref{k-nn-max-degree}, there are at most $6k$ Type~$v$ vertices.  Similarly, there are at most $6k$ Type~$w$ vertices.
  
  Thus, in all, there are at most $13k-1$ Type~$v$, Type~$w$, and Type~C vertices in total. By \lemref{k-nn-max-degree}, each of these vertices is incident on at most $6k$ edges that cross $vw$. Therefore, there are at most $78k^2-6k$ edges of $G$ that cross $vw$.  Since this is true for every edge $vw\in E(G)$, $G$ is $(78k^2-6k)$-planar.
\end{proof}

We note that \lemref{nearest-neighbour} is tight up to the leading constant:  Every $n$-vertex nearest neighbour graph has at least $kn/2$ at at most $kn$ edges.  For $k\ge 8$, the Crossing Lemma \cite{ajtai.chvatal.ea:crossing-free,leighton:complexity} implies that the total number of crossings is therefore $\Omega(k^3n)$ so that the average number of crossings per edge is at $\Omega(k^2)$.

% \subsection{More Stuff}
% 
% \begin{enumerate}
%   % \item clustered colouring;
%   \item connection with queue layouts;
%   % \item $d$-map graphs, which are $d^2$-planar;
%   \item connection with low-treewidth/treedepth colourings;
%   \item An important open question is whether $k$-planar graphs have layered $H$-partitions whose layered width is upper bounded by a function of $k$ but in which the treewidth of $H$ is bounded by an absolute constant, independent of $k$.
%   \item Are our current $H$-partitions $K_{3,s}$-free for some $s=s(k)$?  If so, this is good news for clustered colouring.
% \end{enumerate}
% 
% \noindent
% \framebox{
% \begin{minipage}{\textwidth}
%   \note{DW}{%
%     % I think we need to define layered treewidth, and
%     % give the result of Dujmovic-Eppstein-Wood that (g,k)-planar graphs
%     % have O(gk)  layered treewidth. This is the best previous structural
%     % description of k-planar graphs. State that layered partitions are
%     % richer than  layered treewidth, and lead to O(1) bounds on non-rep
%     % chromatic number and queue-number, instead of O(log n) bounds via
%     % layered treewidth.
%     I also think we should give explicit upper bounds on non-rep chromatic
%     number for k-planar graphs, and also for queue-number (especially if
%     they improve the bound in the FOCS / J.ACM paper).
% 
%     We should add corollaries for d-map graphs, since they are $d^2$-planar.
%     }
% 
%     \note{PM}{
%     A couple of other things to look into: 
%     \begin{compactenum}
%       \item $p$-centered colourings
%       \item low treewidth colourings
%     \end{compactenum}
%     }
% \end{minipage}
% }

\section{Conclusions}

We have shown that $k$-planar graphs and several related graph families have layered $H$-partitions of layered width $O(k^2)$ in which $H$ has treewidth $O(k^3)$.  We conclude with the following question:  Does there exist a function $f:\N\to\N$ and a universal constant $C$ such that every $k$-planar graph has a layered $H$-partition of layered width at most $f(k)$ in which $H$ has treewidth at most $C$?  Perhaps $C=3$.

\bibliographystyle{plainnat}
\bibliography{k-planar}

\end{document}
