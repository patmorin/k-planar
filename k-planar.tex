\documentclass{patmorin}
\listfiles
\usepackage[T1]{fontenc}
\usepackage[utf8]{inputenc}
\usepackage{amsmath}
\usepackage{amsfonts}
\usepackage{amsthm}
\usepackage{graphicx}
\usepackage{enumerate}
\usepackage{amsfonts}
\usepackage{amsthm,mathtools}
\usepackage{paralist}
\usepackage{stmaryrd}
\usepackage{thm-restate}
\usepackage[usenames,dvipsnames,svgnames,table]{xcolor}
\usepackage{todonotes}

\usepackage[longnamesfirst,numbers,sort&compress]{natbib}

\usepackage[noabbrev,capitalise]{cleveref}
\crefname{thm}{Theorem}{Theorems}
\crefname{lem}{Lemma}{Lemmas}
\crefname{cor}{Corollary}{Corollaries}
\crefname{prop}{Proposition}{Propositions}
\crefname{conj}{Conjecture}{Conjectures}
\crefname{open}{Open Problem}{Open Problems}
\crefname{claim}{Claim}{Claims}
\crefname{enumi}{Item}{Items}
\crefname{obs}{Observation}{Observations}
\crefname{clm}{Claim}{Claims}
\crefname{openproblem}{Open Problem}{Open Problems}
\crefformat{equation}{(#2#1#3)}
\Crefformat{equation}{Equation #2(#1)#3}

\theoremstyle{plain}
\newtheorem{thm}{Theorem}
\newtheorem{lem}[thm]{Lemma}
\newtheorem{cor}[thm]{Corollary}
\newtheorem{obs}[thm]{Observation}
\newtheorem{prop}[thm]{Proposition}
\newtheorem{clm}{Claim}
\theoremstyle{definition}
\newtheorem{open}[thm]{Open Problem}
\newtheorem{conj}[thm]{Conjecture}
\newtheorem{ques}[thm]{Question}


\definecolor{brew2}{rgb}{1.0, 1.0, 0.701960784314}
\definecolor{brew8}{rgb}{0.988235294118, 0.803921568627, 0.898039215686}

% \allowdisplaybreaks
%\sloppy
%%%%%%%%%%
 \makeatletter
 \def\NAT@spacechar{~}
 \makeatother

\newcommand{\snote}[1]{{\color{red}#1}}

\newcommand{\defin}[1]{\textcolor{Maroon}{\emph{#1}}}

\newcommand{\note}[2]{\noindent{\color{red}[#1:~#2]}}
\newcommand{\notex}[2]{}
\newcommand{\referee}[2]{\noindent\textcolor{blue}{\framebox{\begin{minipage}{\textwidth} Ref \#{#1}: #2\end{minipage}}}}

\DeclareMathOperator{\dist}{dist}
\DeclareMathOperator{\depth}{depth}
\DeclareMathOperator{\ff}{f}
\DeclareMathOperator{\tw}{tw}
\DeclareMathOperator{\pw}{pw}
\DeclareMathOperator{\qn}{qn}

\DeclarePairedDelimiter{\ceil}{\lceil}{\rceil}
\DeclarePairedDelimiter{\floor}{\lfloor}{\rfloor}
\DeclarePairedDelimiter{\blah}{\langle}{\rangle}

\newcommand{\PRlabel}[1]{\label{PR:#1}}
\newcommand{\PRref}[1]{(PR\ref{PR:#1})}

\newcommand{\jlabel}[1]{\label{j:#1}}
\newcommand{\jref}[1]{(J\ref{j:#1})}

\newcommand{\tlabel}[1]{\label{t:#1}}
\newcommand{\tref}[1]{(T\ref{t:#1})}
\newcommand{\ylabel}[1]{\label{y:#1}}
\newcommand{\yref}[1]{(Y\ref{y:#1})}

\newcommand{\TT}{\mathcal{T}}
\newcommand{\PP}{\mathcal{P}}
\renewcommand{\SS}{\mathcal{S}}

\renewcommand{\ge}{\geqslant}
\renewcommand{\le}{\leqslant}
\renewcommand{\geq}{\geqslant}
\renewcommand{\leq}{\leqslant}

\newcommand{\R}{\mathbb{R}}
\newcommand{\N}{\mathbb{N}}
\newcommand{\Z}{\mathbb{Z}}

\title{\MakeUppercase{Graph Product Structure for Non-Minor-Closed Classes}}

\author{%
Vida Dujmovi\'c%
\thanks{School of Computer Science and Electrical Engineering, University of Ottawa, Ottawa, Canada (\texttt{vida.dujmovic@uottawa.ca}). Research supported by NSERC and the Ontario Ministry of Research and Innovation.},\,\,
Pat Morin%
\thanks{School of Computer Science, Carleton University, Ottawa, Canada (\texttt{morin@scs.carleton.ca}). Research  supported by NSERC and the Ontario Ministry of Research and Innovation.},\,\, and
David R. Wood%
\thanks{School of Mathematics, Monash University, Melbourne, Australia (\texttt{david.wood@monash.edu}). Research supported by the Australian Research Council.}
}

\begin{document}
\begin{titlepage}
\maketitle

\begin{abstract}
Dujmovi\'c~et~al.~[\emph{J.~ACM}~'20] recently proved that every planar graph is isomorphic to a subgraph of the strong product of a bounded treewidth graph and a path. Analogous results were obtained for graphs of bounded Euler genus or apex-minor-free graphs. These tools have been used to solve longstanding problems on queue layouts, non-repetitive colouring, $p$-centered colouring, and adjacency labelling. This paper proves analogous product structure theorems for various non-minor-closed classes. One noteable example is $k$-planar graphs (those with a drawing in the plane in which each edge is involved in at most $k$ crossings). We prove that every $k$-planar graph is isomorphic to a subgraph of the strong product of a graph of treewidth $O(k^5)$ and a path. This is the first result of this type for a non-minor-closed class of graphs. It implies, amongst other results, that $k$-planar graphs have non-repetitive chromatic number upper-bounded by a function of $k$. All these results generalise for drawings of graphs on arbitrary surfaces. In fact, we work in a more general setting based on so-called shortcut systems, which are of independent interest. This leads to analogous results for certain types of map graphs, string graphs, graph powers, and nearest neighbour graphs.
\end{abstract}
\end{titlepage}
\pagenumbering{roman}
\tableofcontents
\newpage

\pagenumbering{arabic}
% \setcounter{\page}{1}
\section{Introduction}
\label{Introduction}

% \note{DW}{I suggest we delete ``We use $G'\subseteq G$ to denote subgraph containment; that is, $V(G')\subseteq V(G)$ and $E(G')\subseteq E(G)$. '' This is absolutely standard.}

The starting point for this work is the following `product structure theorem' for planar graphs\footnote{In this paper, all graphs are finite and undirected. Unless mentioned otherwise, all graphs are also simple. For any graph $G$ and any set $S$ (typically $S\subseteq V(G)$), let $G[S]$  denote the graph with vertex set $V(G)\cap S$ and edge set $\{uv\in E(G) : u,v\in S\}$.  We use $G-S$ as a shorthand for $G[V(G)\setminus S]$. Undefined terms are in \citep{Diestel5}.} by \citet{DJMMUW20} (with an improvement by \citet{UWY}). A graph $G$ is \defin{contained} in a graph $X$ if $G$ is isomorphic to a subgraph of $X$.

\begin{thm}[\citep{DJMMUW20,UWY}]
\label{PlanarProduct}
Every planar graph is contained in:
\begin{compactenum}[(a)]
	\item $H\boxtimes P$ for some graph $H$ of treewidth at most $6$ and for some path $P$,
	\item $H\boxtimes P \boxtimes K_3$ for some graph $H$ of treewidth at most $3$ and for some path $P$.
\end{compactenum}
\end{thm}

Here $\boxtimes$ is the strong product,\!\footnote{The \defin{strong product} of graphs $A$ and $B$, denoted by $A\boxtimes B$, is the graph with vertex set $V(A)\times V(B)$, where distinct vertices $(v,x),(w,y)\in V(A)\times V(B)$ are adjacent if
	$v=w$ and $xy\in E(B)$, or
	$x=y$ and $vw\in E(A)$, or
	$vw\in E(A)$ and $xy\in E(B)$.}
and treewidth\footnote{For a tree $T$, a \defin{$T$-decomposition} of a graph $G$ is a collection $\mathcal{T}=(B_x:x\in V(T))$ of subsets of $V(G)$ indexed by the nodes of $T$ such that
(i) for every $vw\in E(G)$, there exists some node $x\in V(T)$ with $v,w\in B_x$; and
(ii) for every $v\in V(G)$, the induced subgraph $T[v] := T[\{x: v\in B_x\}]$ is connected. The \defin{width} of $\mathcal{T}$ is $\max\{|B_x|:x\in V(T)\}-1$.  A \defin{tree-decomposition} is a $T$-decomposition for any tree $T$. The \defin{treewidth} $\tw(G)$ of a graph $G$ is the minimum width of a tree-decomposition of $G$.  Treewidth is the standard measure of how similar a graph is to a tree. Indeed, a connected graph has treewidth 1 if and only if it is a tree. Treewidth is of fundamental importance in structural and algorithmic graph theory; see \citep{Reed03,HW17,Bodlaender-TCS98} for surveys.} is an invariant that measures how `tree-like' a given graph is; see \cref{ProductExample} for an example. Loosely speaking, \cref{PlanarProduct} says that every planar graph is contained in the product of a tree-like graph and a path. This enables combinatorial results for graphs of bounded treewidth to be generalised for planar graphs (with different constants).

\begin{figure}[!h]
\centering
\includegraphics{ProductExample}
\caption{Example of a strong product.
\label{ProductExample}}
\end{figure}

\noindent\cref{PlanarProduct} has been the key tool in solving the following well-known open problems:
\begin{compactitem}
\item \citet{DJMMUW20} use it to prove that planar graphs have bounded queue-number (resolving a conjecture of \citet{HLR92}).
\item  \citet{dujmovic.esperet.ea:planar} use it to prove that planar graphs have bounded non-repetitive chromatic number (resolving a conjecture of \citet{AGHR-RSA02}).
\item \citet{DFMS21} use it to prove that planar graphs have $p$-centered chromatic number $O(p^2\log p)$ and give a matching lower bound.
\item \citet{DEJGMM21} use it to find asymptotically optimal adjacency labellings of planar graphs (resolving a problem of \citet{kannan.naor.ea:implicit}).
\item \citet{EJM} use it to show the existence of a `universal graph' with $n^{1+o(1)}$ vertices and edges that contains every $n$-vertex planar graph as an induced subgraph (resolving a problem of \citet{babai.chung.ea:on}).
\end{compactitem}
In addition, \cref{PlanarProduct} has been used to resolve or make substantial progress on a number of other problems on planar graphs, including $\ell$-vertex ranking~\citep{BDJM}, twin-width~\citep{BKW}, and odd colourings~\citep{DMO}.

% \item \citet{DFMS21} use it to make dramatic improvements to the best known bounds for $p$-centered colourings of planar graphs.
% \item \citet{bonamy.gavoille.ea:shorter} use it to find shorter adjacency labellings of planar graphs (improving on a sequence of results starting with the work of \citet{kannan.naor.ea:implicit}).
% \item \citet{BDJM} use it to show that every $n$-vertex planar graph has local vertex ranking using $O(\log n/\log\log\log n)$ colours.
% % \item The result of \citet{DEJGMM21} implies that, for every integer
% $n>0$, there is a `induced-universal' graph $U_n$ with $n^{1+o(1)}$ vertices such that every $n$-vertex planar graph is an induced
% subgraph of $U_n$. This has recently been extended by \citet{EJM}, who show the existence of a `subgraph-universal' graph with $(1+\epsilon)n$ vertices that contains every $n$-vertex planar graph as a subgraph as well as the existence of an `induced universal' graph with $n^{1+o(1)}$ vertices \emph{and} edges.  The former result is the first progress on subgraph-universal graphs for planar graphs since the $O(n^{3/2})$-vertex construction of \citet{babai.chung.ea:on}.
% aking the first progress on this problem since the work of \citet{babai.xx} in 1984.
% \end{compactitem}

% \note{DW}{Mention other application such as vertex ranking~\citep{BDJM}, twin-width~\citep{BKW}, odd colourings~\citep{DMO}? We write a lot about adjacency labellings and universal induced subgraphs. Do we need so much detail here? We also mention this work in \cref{WrappingUp}.}

% \note{PM}{Agreed. I cut this down to a list of famous open problem. We don't need to sell this so hard anymore.}

All of these results hold for any graph class that has a product structure theorem analogous to \cref{PlanarProduct}; that is, for any graph class  $\mathcal{G}$ where every graph in $\mathcal{G}$ is contained in $H\boxtimes P\boxtimes K_\ell$ where $H$ has bounded treewidth, $P$ is a path, and $\ell$ is bounded.\footnote{It is easily seen that $\tw(H\boxtimes K_\ell) \leq (\tw(H)+1)\ell-1$, so we may assume that $\ell=1$ in this statement.} These applications motivate finding product structure theorems for other graph classes. \citet{DJMMUW20} prove product structure theorems for graphs of bounded Euler genus\footnote{The \textit{Euler genus} of a surface with $h$ handles and $c$ crosscaps is $2h+c$. The \textit{Euler genus} of a graph $G$ is the minimum integer $g$ such that $G$ embeds in a surface of Euler genus $g$. Of course, a graph is planar if and only if it has Euler genus 0; see \citep{mohar.thomassen:graphs} for more about graph embeddings in surfaces.} and for apex-minor-free graphs,\footnote{A graph $M$ is a \textit{minor} of a graph $G$ if a graph isomorphic to $M$ can be obtained from a subgraph of $G$ by contracting edges. A class $\mathcal{G}$ of graphs is \defin{minor-closed} if for every graph $G\in\mathcal{G}$, every minor of $G$ is in $\mathcal{G}$. A minor-closed class is \defin{proper} if it is not the class of all graphs. For example, for fixed $g\geq 0$, the class of graphs with Euler genus at most $g$ is a proper minor-closed class. A graph $G$ is $t$-apex if it contains a set $A$ of at most $t$ vertices such that $G-A$ is planar. A 1-apex graph is \defin{apex}.  A minor-closed class $\mathcal{G}$ is apex-minor-free if some apex graph is not in $\mathcal{G}$.} and \citet{DEMWW22} do so for bounded-degree graphs in any minor-closed class. See
% \cref{Generalisations} for more precise statements and see
\citep{DHJLW21} for a survey on this topic.

The main purpose of this paper is to prove product structure theorems for several non-minor-closed classes of interest. Our results are the first of this type for non-minor-closed classes.

\subsection{$k$-Planar Graphs}
% \label{k_planar_section}

We start with the example of $k$-planar graphs. A graph is \defin{$k$-planar} if it has a drawing in the plane in which each edge is involved in at most $k$ crossings, where no three edges cross at a single point (see \cref{k_planar_section} for a formal definition). Such graphs provide a natural generalisation of planar graphs, and are important in graph drawing research; see the recent bibliography on 1-planar graphs and the 140 references therein \citep{kobourov.liotta.ea:annotated}. It is well-known that the family of $k$-planar graphs is not minor-closed.  Indeed, 1-planar graphs may contain arbitrarily large complete graph minors~\citep{dujmovic.eppstein.ea:structure}. Hence the above results are not applicable for  $k$-planar graphs. We extend \cref{PlanarProduct} as follows.

\begin{thm}
\label{kPlanarProduct}
Every $k$-planar graph is contained in $H\boxtimes P\boxtimes K_{18k^2+48k+30}$, for some graph $H$ of treewidth $\binom{k+4}{3}-1$ and for some path $P$.
\end{thm}

%\note{DW}{Do we want to mention ``every $1$-planar graph is contained in $H\boxtimes P\boxtimes K_7$ for some planar graph $H$ of treewidth $3$ and for some path $P$'' here?}
%\note{PM}{Yes, I think so.  Two theorems, like this, or one?}
%\note{DW}{two theorems}

For the important special case of $1$-planar graphs, we obtain the following result with $H$ planar and a best possible bound on the treewidth of $H$.

\begin{thm}\label{1_planar_product}
Every $1$-planar graph is contained in $H\boxtimes P\boxtimes K_{7}$, for some planar graph $H$ of treewidth $3$ and for some path $P$.
\end{thm}

% A planar graph of treewidth at most $3$ is said to have \emph{simple treewidth} $3$.

For at least two applications, the planarity of $H$ in \cref{1_planar_product} is critical in obtaining asymptotically tight bounds \cite{BDJM,DFMS21}. \cref{1_planar_product} is proven in the more general setting of $d$-framed multigraphs, which also implies a product structure theorem for $d$-map graphs in which $H$ is a planar with treewidth 3. 

% These theorems have applications in diverse areas, which we explore in \cref{Applications}.

% This theorem has applications in diverse areas, including queue layouts  \citep{DJMMUW20}, non-repetitive colouring  \citep{dujmovic.esperet.ea:planar}, $p$-centered colouring  \citep{DFMS21}, and adjacency labelling \citep{DEJGMM21}, which we explore in \cref{Applications}. For example, we prove that $k$-planar graphs have bounded non-repetitive chromatic number (for fixed $k$). Prior to the recent work of \citet{dujmovic.esperet.ea:planar}, it was even open whether planar graphs have bounded non-repetitive chromatic number.
%
% \referee{1}{3. pages 2 – 4: The paragraph following Theorem 2 and Section 1.3
% (except for the fact that the apex assumption is necessary) is repetition
% of things discussed earlier on in the introduction.}
%
% \note{PM}{Kind of, but I suggest we keep that paragraph and section anyway. Justification: They're both short and they describe the exact bounds for bounded genus graphs, which you don't get from the one-paragraph description in Section~1.0.}
%
% \note{DW}{I suggest we write, ``This theorem has applications in diverse areas, which we explore in \cref{Applications}....''}

\subsection{Shortcut Systems}

% \note{DW}{I'm uncomfortable about saying that $k$-planar graphs are our most high-profile target. Why are they more important than map graphs or nearest neighbour graphs? I suggest ``
% The above result for $k$-planar graphs (\cref{kPlanarProduct}) is in fact a special case of a more general result that relies on the following definition. A non-empty set $\SS$  ....''}

The above result for $k$-planar graphs (\cref{kPlanarProduct}) is in fact a special case of a more general result that relies on the following definition. A non-empty set $\SS$ of non-trivial paths in a graph $G$ is a \defin{$(k,d)$-shortcut system} (for $G$) if:

\begin{compactitem}
\item every path in $\SS$ has length at most $k$,\footnote{A path of length $k$ consists of $k$ edges and $k+1$ vertices.  A path is \defin{trivial} if it has length 0 and \defin{non-trivial} otherwise.} and
\item for every $v\in V(G)$, the number of paths in $\SS$ that include $v$ as an internal vertex is at most $d$.
\end{compactitem}
Each path $P\in\SS$ is called a \defin{shortcut}; if $P$ has endpoints $v$ and $w$ then it is a \defin{$vw$-shortcut}. Given a graph $G$ and a $(k,d)$-shortcut system $\SS$ for $G$, let $G^{\SS}$ denote the supergraph of $G$ obtained by adding the edge $vw$ for each $vw$-shortcut in $\SS$.

This definition is related to $k$-planarity by the following observation:

\begin{obs}
\label{AddDummy}
Every $k$-planar graph is contained in $G^\SS$ for some planar graph $G$ and some $(k+1,2)$-shortcut system $\SS$ for $G$.
\end{obs}

The proof of \cref{AddDummy} is trivial: Given a $k$-plane embedding of a graph $G'$, create a planar graph $G$ by adding a dummy vertex at each crossing point. For each edge $vw\in E(G')$ there is a path $P$ in $G$ between $v$ and $w$ of length at most $k+1$ in which every internal vertex is a dummy vertex. Let $\SS$ be the set of such paths $P$. For each vertex $v$ of $G$, at most two paths in $\SS$ use $v$ as an internal vertex (since no original vertex of $G'$ is an internal vertex of a path in $\SS$). Thus $\SS$ is a $(k+1,2)$-shortcut system for $G$, such that $G'\subseteq G^\SS$. This idea can be pushed further to obtain a rough characterisation of $k$-planar graphs, which is interesting in its own right, and is useful for showing that various classes of graphs are $k$-planar (see \cref{Characterisation}).


The following theorem is one of the main contributions of the paper. It says that if a graph class $\mathcal{G}$ has a product structure theorem like \cref{PlanarProduct}(b), then so too does the class of graphs obtained by applying shortcut systems to graphs in $\mathcal{G}$.

\begin{thm}
\label{ShortcutProduct}
Let $G$ be a graph contained in $H\boxtimes P \boxtimes K_\ell$, for some graph $H$ of treewidth at most $t$ and for some path $P$. Let $\SS$ be a $(k,d)$-shortcut system for $G$. Then $G^\SS$ is contained in $J\boxtimes P\boxtimes K_{d\ell(k^3+3k)}$ for some graph $J$ of treewidth at most $\binom{k+t}{t}-1$ and some path $P$.
\end{thm}

% \referee{2}{Page 3. $P$ is used twice in the statement of Theorem 3.  It is probably
% safer (and necessary?) to use a different variable for the second
% occurrence.}

% \note{DW}{Use $\SS$ for shortcut system, okay?}
% \note{PM}{Ok.  I've implemented this change.  I was careful, but will still reread.}

Theorem~\ref{PlanarProduct}(b), \cref{ShortcutProduct}, and \cref{AddDummy} imply \cref{kPlanarProduct} with $K_{6(k^3+3k)}$ instead of $K_{18k^2+48k+30}$. Some further observations presented in \cref{k_planar_section} lead to the improved result.  \cref{ShortcutProduct} is applicable for many graph classes in addition to $k$-planar graphs. Some examples are explored in \cref{examples}.

% \note{DW}{I suggest we replace ``\cref{first_example,mid_example,last_example}''by ``\cref{examples}''.}

% Here is one example. The \defin{$k$-th power} of a graph $G$ is the graph $G^k$ with vertex set $V(G^k):=V(G)$, where $vw\in E(G^k)$ if and only if $\dist_G(v,w)\leq k$.\footnote{For a graph $G$ and two vertices $v,w\in V(G)$, $\dist_G(v,w)$  denotes the length of a shortest path, in $G$, with endpoints $v$ and $w$.  We define $\dist_G(v,w):=\infty$ if $v$ and $w$ are in different connected components of $G$.} If $G$ has maximum degree $\Delta$, then $G^k = G^\SS$ for some $(k,2k\Delta^{k})$-shortcut system $\SS$; see \cref{PowerShortcut}. Theorems~\ref{PlanarProduct}(b) and \ref{ShortcutProduct} then imply:

% \begin{thm}
% \label{kPowerBasic}
% For every planar graph $G$ with maximum degree $\Delta$ and for every integer $k\geq 1$, $G^k$ is contained in $H\boxtimes P\boxtimes K_{6k^2(k^2+3)\Delta^{k}}$ for some graph $H$ of treewidth at most $\binom{k+3}{3}-1$ and some path $P$.
% \end{thm}

% \cref{Examples} presents further examples of graph classes that can be constructed using shortcut systems, including certain types of map graphs, string graphs, and $k$-nearest neighbour graphs. \cref{ShortcutProduct} implies product structure theorems for each of these classes. All of the above-mentioned applications also hold for these examples.


\subsection{Overview and Outline}

\cref{summary_table} summarizes existing results on product structure theorems for minor-closed graph classes and new results for non-minor-closed graph classes.

% \note{DW}{I suggest the table headers are not in bold}

\begin{table}[h]
  \centering{
    \begin{tabular}{lccl}
    	\hline
      {Graph class} & $\tw(H)$ & $\ell$ & {Reference} \\ \hline
      planar & $3^*$ & $3$ & \cite{DJMMUW20} \\
      planar & $4^*$ & $2$ & \cite{bose.morin.ea:optimal} \\
      planar & $6^*$ & $1$ & \cite{UWY} \\
      genus $g$ & $3^*$ & $\max\{2g,3\}$ & \cite{DHHW} \\
      genus $g$ & $2g+6$ & 1 & \cite{UWY} \\
      apex-minor-free & $O(1)$ & $O(1)$ & \cite{DJMMUW20} \\
      % B-minor-free & $O(1)$ & $O(1)$ & \cite{DJMMUW20} \\
      $k$-planar & $\binom{k+4}{3}-1$ & $18k^2 + 48k + 30$ & \cref{kPlanarProduct} \\
      $(g,k)$-planar & $\binom{k+4}{3}-1$ & $\max\{2g,3\}(6k^2+16k+10)$ & \cref{gkPlanarProduct} \\
      $(g,\delta)$-string & $\binom{\delta+4}{3}-1$ & $\max\{2g,3\}(\delta^4 + 4\delta^3 + 9\delta^2 + 10\delta+4)$ & \cref{StringProduct} \\
      $k$-nearest-neighbour & $O(k^6)$ & $O(k^4)$ & \cref{nn_product_structure} \\
      $d$-framed & $3^*$ & $d+ 3\floor{d/2} - 3$ & \cref{d_framed_product_stucture} \\
      $1$-planar & $3^*$ & $7$ & \cref{1_planar_product} \\
      $d$-map & $3^*$ & $d+ 3\floor{d/2} - 3$ & \cref{dMapProduct} \\
      $(g,d)$-map & $9$ & $\max\{2g,3\}\,(7d^2 -21d)$ & \cref{gdMapProduct} \\ \hline
      \multicolumn{4}{l}{$^*$ these bounds also apply to the simple treewidth of $H$.}
    \end{tabular}
  }
  \caption{Product structure theorems (of the form $G\subseteq H\boxtimes P\boxtimes K_\ell$).}
  \label{summary_table}
\end{table}

% \note{DW}{why is the $2g+6$ commented out from the table?}

The remainder of this paper is organized as follows:
\begin{compactitem}
  \item In \cref{Structure} we prove our main result for shortcut systems (\cref{ShortcutProduct}), and its optimisations for $k$-planar graphs and their genus-$g$ generalisation (\cref{kPlanarProduct,gkPlanarProduct}).
  \item \Cref{examples} uses \cref{ShortcutProduct,kPlanarProduct,gkPlanarProduct} to derive product structure theorems for string graphs, powers of bounded-degree graphs, map graphs, and $k$-nearest neighbour graphs.
  \item \Cref{FramedSection} proves a product structure theorem for $d$-framed multigraphs in which the graph $H$ has treewidth $3$ and is planar.  Using this result, we derive improved product structure theorems for $1$-planar graphs and planar map graphs, again with a planar graph $H$.
  \item \Cref{Applications} quickly surveys applications of product structure theorems and the consequences of the current work.
\end{compactitem}




%%%%%%%%%%%%%%%%%%%%%%%%%%%
\section{\boldmath Shortcut Systems and $k$-Planar Graphs}
\label{Structure}

This section proves \cref{ShortcutProduct} and its specialization to $k$-planar graphs, \cref{kPlanarProduct}. While strong products enable concise statements of the theorems in \cref{Introduction}, to prove such results it is helpful to work with layerings and partitions, which we now introduce.

\subsection{Layered Partitions}

A \defin{layering} of a graph $G$
% is a sequence $\mathcal{L}=\langle L_0,L_1,\ldots\rangle $ such that $\{L_0,L_1,\ldots\}$ is a partition of $V(G)$
% \note{DW}{just write ``
is an ordered partition $\mathcal{L}=\langle L_0,L_1,\ldots\rangle $ such that for every edge
 % ...''}
$vw\in E(G)$, if $v\in L_i$ and $w\in L_j$ then $|j-i|\leq 1$.  For any partition $\PP=\{S_1,\ldots,S_p\}$ of $V(G)$, a \defin{quotient graph} $H=G/\PP$ has a $p$-element vertex set $V(H)=\{x_1,\ldots,x_p\}$ and $x_ix_j\in E(H)$ if and only if there exists an edge $vw\in E(G)$ such that $v\in S_i$ and $w\in S_j$. To highlight the importance of the quotient graph $H$, we call $\PP$ an \defin{$H$-partition} and write this concisely as $\PP=\{S_x : x\in V(H)\}$ so that each element of $\PP$ is indexed by the corresponding vertex in $H$.

For any partition $\PP$ of $V(G)$ and any layering $\mathcal{L}$ of $G$ we define the \defin{layered width} of $\PP$ with respect to $\mathcal{L}$ as $\max\{|L\cap P|: L\in\mathcal{L},\, P\in\PP\}$.  For any partition $\PP$ of $V(G)$, we define the \defin{layered width} of $\PP$ as the minimum, over all layerings $\mathcal{L}$ of $G$, of the layered width of $\PP$ with respect to $\mathcal{L}$.

% \citet{DJMMUW20} introduced the study of partitions with bounded layered width such that the quotient has some additional desirable property, like small treewidth. Dujmovi\'c~et~al.\ define a class $\mathcal{G}$ of graphs to \defin{admit bounded layered partitions} if there exist $t,\ell\in\mathbb{N}$ such that every graph $G\in \mathcal{G}$ has an $H$-partition of layered width at most $\ell$ for some graph $H=H(G)$ of treewidth at most $t$.

These definitions relate to strong products as follows.

\begin{lem}[\citep{DJMMUW20}]
\label{PartitionProduct}
For every graph $H$, a graph $G$ has an $H$-partition of layered width at most $\ell$ if and only if $G$ is contained in $H \boxtimes P \boxtimes K_\ell$ for some path $P$.
\end{lem}

As an example of the use of layered partitions, to prove \cref{PlanarProduct}(a),
\citet{DJMMUW20} build on an earlier result of \citet{PS21} to show that every planar graph has an $H$-partition of layered width $1$ for some planar graph $H$ of treewidth at most $8$ (improved to $6$ in \cite{UWY}). The proof is constructive and gives a simple quadratic-time algorithm for finding the corresponding partition and layering.\footnote{\citet{bose.morin.ea:optimal} have recently given linear time algorithms for computing layered $H$-partitions of planar graphs.}


% \note{PM}{This feels out of place now that this section has been moved out of the introduction.  Delete?
%
% At the core of their work is the elegant proof by \citet{PS21} of the following result:
%
% \begin{thm}[\citep{PS21}]
% \label{ps}
%   Every planar triangulation $G$ has an $H$-partition $\PP$ such that $\tw(H)\leq 8$ and $G[P]$ is a shortest path in $G$ for each $P\in\PP$.
% \end{thm}
%
% Indeed, the above-mentioned result of \citet{DJMMUW20} is a slight strengthening of \cref{ps}, where for each $P\in\PP$ no two vertices of $P$ have the same distance to some fixed root vertex $r$.
% }

By \cref{PartitionProduct}, \cref{ShortcutProduct} is equivalent to the following result, whose proof is the subject of the next section.

\begin{thm}
	\label{ShortcutPartition}
	Let $G$ be a graph having an $H$-partition of layered width $\ell$ in which $H$ has treewidth at most $t$ and let $\SS$ be a $(k,d)$-shortcut system for $G$.  Then $G^\SS$ has a $J$-partition of layered width at most $d\ell(k^3+3k)$ for some graph $J$ of treewidth at most $\binom{k+t}{t}-1$.
\end{thm}


\subsection{Shortcut Systems}

We now prove \cref{ShortcutPartition}.
For convenience, it will be helpful to assume that $\SS$ contains a length-1 $vw$-shortcut for every edge $vw\in E(G)$.  Since $G^\SS$ is defined to be a supergraph of $G$, this assumption has no effect on $G^{\SS}$ but eliminates special cases in some of our proofs.

Let $T$ be a tree rooted at some node $x_0\in V(T)$.  A node $a\in V(T)$ is a \defin{$T$-ancestor} of $x\in V(T)$ (and $x$ is a \defin{$T$-descendant} of $a$) if $a$ is a vertex of the path, in $T$, from $x_0$ to $x$.  Note that each node $x\in V(T)$ is a $T$-ancestor and $T$-descendant of itself.  We say that a $T$-ancestor $a\in V(T)$ of $x\in V(T)$ is a \defin{strict} $T$-ancestor of $x$ if $a\neq x$.
The \defin{$T$-depth} of a node $x\in V(T)$ is the length of the path, in $T$, from $x_0$ to $x$.  The \defin{lowest common $T$-ancestor} of a non-empty set $S\subseteq V(T)$ is the $T$-ancestor of every node in $S$ with maximum $T$-depth. In the following, we treat any subtree $T'$ of $T$ as a rooted tree whose root is the lowest common $T$-ancestor of $V(T')$.



% For each node $x\in V(T)$, define
% \[T_x := T[\{y\in V(T):\mbox{$x$ is a $T$-ancestor of $y$}\}] \]
% to be the maximal subtree of $T$ rooted at $x$.

% \note{PM}{I have significantly simplified this section, which should make the referees happy. I will read it very carefully again tomorrow}

We begin with a standard technique that allows us to work with a normalised tree-decomposition whose tree has the same vertex set as the graph it decomposes:

\begin{lem}\label{nice-decomposition}
  For every graph $H$ of treewidth $t$, there is a rooted tree $T$ with $V(T)=V(H)$ and a width-$t$ $T$-decomposition $(B_x:x\in V(T))$ of $H$ that has following additional properties:
  \begin{compactenum}[(T1)]
    % \item\tlabel{rooted}\tlabel{first} $T$ is rooted at some node $x_0\in V(T)$ with $B_{x_0}=\emptyset$.
    % \item\tlabel{node-set} $V(T)= V(H)$.
    % \item\tlabel{diff} For each edge $xy\in E(T)$, $|B_x\ominus B_y|\le 1$.
    \item\tlabel{subtree-root} for each node $x\in V(H)$, the subtree $T[x]:=T[\{y\in V(T):x\in B_y\}]$ is rooted at $x$; and consequently
    \item\tlabel{ancestor-edge}\tlabel{last} for each edge $xy\in E(H)$, one of $x$ or $y$ is a $T$-ancestor of the other.
  \end{compactenum}
\end{lem}

\begin{proof}
  That \tref{subtree-root} implies \tref{ancestor-edge} is a standard observation: If two subtrees intersect, then one contains the root of the other.  Thus, it suffices to construct a width-$t$ tree-decomposition that satisfies \tref{subtree-root}.

  Begin with any width-$t$ tree-decomposition $(B_x:x\in V(T_0))$ of $H$ that uses some tree $T_0$ and has no empty bags.  Select any node $x_1\in V(T_0)$, add a leaf $x_0$, with $B_{x_0}=\emptyset$, adjacent to $x_1$ and root $T_0$ at $x_0$. (The purpose of $x_0$ is to ensure that every node $x$ for which $B_x$ is non-empty has a parent.)  Let $f:V(H)\to V(T)$ be the function that maps each $x\in V(H)$ onto the root of the subtree $T_0[x]:=T_0[\{y\in V(T_0): x\in B_y\}]$.  If $f$ is not one-to-one, then select some distinct pair $x,y\in V(H)$ with $a:=f(x)=f(y)$.  Subdivide the edge between $a$ and its parent in $T_0$ by introducing a new node $a'$ with $B_{a'}=B_{a}\setminus\{x\}$. Now $f(y)=a'$ and $f(x)=a$, so this modification reduces the number of distinct pairs $x,y\in V(H)$ with $f(x)=f(y)$.  Repeatedly performing this modification will eventually produce a tree-decomposition $(B_x:x\in V(T_0))$ of $H$ in which $f$ is one-to-one.

  Next, remove the node $x_0$ from $T_0$ (so that $x_1$ becomes the new root of $T_0$).  Consider any node $a\in V(T_0)$ such that there is no vertex $x\in V(H)$ with $f(x)=a$.  In this case, $B_{a}\subseteq B_{a'}$ where $a'$ is the parent of $a$ since any $x\in B_a\setminus B_{a'}$ would have $f(x)=a$.  In this case, contract the edge $aa'$ in $T_0$, eliminating the node $a$.  Repeating this operation will eventually produce a width-$t$ tree-decomposition of $(B_x:x\in V(T_0))$ where $f$ is a bijection between $V(H)$ and $V(T_0)$.  Renaming each node $a\in V(T_0)$ as $f^{-1}(a)$ gives a tree-decomposition $(B_x:x\in V(T))$ with $V(T)=V(H)$.  By the definition of $f$, the tree-decomposition $(B_x:x\in V(T))$ satisfies \tref{subtree-root}.
\end{proof}

% \referee{1}{6. page 6, last paragraph of Lemma 2: This isn’t quite right since you have to be careful with $x_0$.}
% \note{PM}{I addressed this by removing $x_0$ from $T_0$ at the beginning of the second paragraph.}

%\subsection{Generalized Tripod Partitions}
%
%
% \begin{figure}[htbp]
%   \begin{center}
%     \includegraphics{figs/tripoddo}
%   \end{center}
%   \caption{The sets $Y_x$, $F_x$, and $V_x$ associated with $x\in V(T)$
%   and the ancestors $a_1,\ldots,a_{t'}$ of $X$ such that $F_x \subseteq \bigcup_{i=1}^{t'} Y_{a_i}$.}
%   \label{fig:generalized-tripod}
% \end{figure}

% The following lemma shows how to interpret an $H$-partition of $G$ and a tree-decomposition of $H$ as a `hierarchical' decomposition of $G$; refer to \cref{fig:generalized-tripod}.

% For a vertex $v$ of $G$, $N_G(v):=\{w\in V(G):vw\in E(G)\}$ and $N_G[v]:=N_G(v)\cup\{v\}$.  For a vertex set $S\subseteq V$, $N_G(S):=\bigcup_{v\in S} N_G(S)\setminus S$ and $N_G[S]:=N_G(S)\cup S$.

% \note{PM}{
% Guide to old notation:
% \begin{compactitem}
%   \item $V_x := \bigcup_{y\in V(T_x)} Y_y = G_x$
%   \item $F_x:=\{w\in V(G): vw\in E(G), v\in V_x,\, w\not\in V_x\}= N_G(G_x)$
%   \item $N_x:=V_x\cup F_x=N_G[G_x]$
% \end{compactitem}
% }
%
% \begin{lem}\label{generalized-tripod}
%   Let $G$ be a graph; let $\mathcal{L}:=\langle L_1,\ldots,L_h\rangle$ be a layering of $G$; let $\mathcal{Y}:Y_x: x\in V(H))$ be an $H$-partition of $G$ of layered width at most $\ell$ with respect to $\mathcal{L}$ where $H$ has treewidth at most $t$; and let $\mathcal{T}:=(B_x:x\in V(T))$ be a tree-decomposition of $H$ satisfying the conditions of \cref{nice-decomposition}.
%   For each $x\in V(T)$, let $H_x:=V(T_x)$ and let $G_x:=\bigcup_{y\in H_x}Y_y$.
%
%   \begin{compactenum}[(Y1)]
%     % \item\ylabel{thickness} $\mathcal{Y}=(Y_x: x\in V(T))$ is a partition of $V(G)$ of layered width at most $\ell$ with respect to $\mathcal{L}$.
%     % \item\ylabel{separator} For each $x\in V(T)$, $N_G(\phi(T_x))$ separates $\phi(T_x)$ from $N
%     % there is no edge $vw\in E(G)$ with $v\in V_x$ and $w\in V(G)\setminus N_x$.
%
%     \item\ylabel{ancestor-edge} For each $x\in V(T)$, there is a set $\{a_1,\ldots,a_{t'}\}$ of $t'\le t$ strict $T$-ancestors of $x$ such that $N_G(G_x) \subseteq \bigcup_{i=1}^{t'} Y_{a_i}$.
%   \end{compactenum}
% \end{lem}
%
% \begin{proof}
%   Let $\{a_1,\ldots,a_{t'}\}:= B_x\setminus\{x\}$.  Then, by the definition of a normalized tree decomposition $N_H(H_x))\subseteq\{a_1,\ldots,a_{t'}\}$.  Therefore $N_G(G_x)\subseteq \bigcup_{i=1}^{t'} Y_{a_i}$.
% \end{proof}


% \referee{1}{1. Lemma 3: This lemma is really about ``normalised'' tree-decompositions, and says that for each $x \in V(T)$, the set $V(T_x)$ has at most $t$ neighbours in $H$. Call this property (T3) and remove (Y1) through (Y5), which should now be obvious.}
%
% \note{PM}{I've added to the \texttt{response.tex} file explaining some of this.  Each of (Y1), (Y3), (Y4), and (Y5) have a one-sentence justification. The justification for (Y2) is just definition unpacking and then using the observation that the referee recommends.}
%
% \note{PM}{I plan to do an experiment with the referee's suggested notation, in a separate git branch to see how it turns out.}
%
% Before proving \cref{generalized-tripod} we point out more properties that are immediately implied by it:
%
% \begin{compactenum}[(Y1)]\setcounter{enumi}{2}
%   \item\ylabel{y-subsets} $Y_x\subseteq V_x$ for every $x\in V(T)$. \note{PM}{$Y_x$, for $x\in V(T)$ is not defined!}
%   \item\ylabel{containment-i} $V_x\subseteq V_a$ for every $T$-ancestor $a$ of $x$.
%   \item\ylabel{containment-ii}$N_x\subseteq N_a$ for every $T$-ancestor $a$ of $x$.
% \end{compactenum}

% Property~\yref{y-subsets} follows from the fact that $V_x$ is the union of several sets, one of which is $Y_x$.  Property~\yref{containment-i} follows from the definition of $V_x$ and the fact that $V(T_x)\subseteq V(T_a)$. To show Property~\yref{containment-ii} first note that, by \yref{containment-i} it suffices to consider vertices $w\in F_x=N_x\setminus V_x$. By definition, every vertex $w\in F_x$ is adjacent, in $G$, to a vertex $v\in V_x$.  By \yref{containment-i}, $v\in V_a$, so $w$ is either in $V_a$ or $w$ satisfies the condition $vw\in E(G)$, $v\in V_a$, and $w\not\in V_a$, so $w\in F_a$.  In either case $w\in N_a=V_a\cup F_a$.  Note that none of \yref{y-subsets}--\yref{containment-ii} depends on \yref{ancestor-edge} (which is important, since \yref{containment-i} is used to establish \yref{ancestor-edge} in the following proof).
%
%
% \begin{proof}[Proof of \cref{generalized-tripod}]
%   % Property~\yref{thickness} follows immediately from the fact that $V(T)=V(H)$  and the fact that $\mathcal{Y}$ has layered width at most $\ell$ with respect to $\mathcal{L}$.
%   Property \yref{separator} is immediate from the definitions of $F_x$ and $N_x$.  In particular, $(N_x,V(G)\setminus V_x)$ is a separation of $G$ with $F_x=N_x\cap(V(G)\setminus V_x)$.
%
%   To establish Property~\yref{ancestor-edge}, consider some vertex $w\in F_x$.  Since $w\in F_x$, there exists an edge $vw\in E(G)$ with $v\in V_x$ and $w\not\in V_x$.  Since $v\in V_x$, $v\in Y_{x'}$ for some $T$-descendant $x'$ of $x$ (possibly $x=x'$). Since $\mathcal{Y}$ is a partition, $w\in Y_{a}$ for some $a\not\in V(T_x)$.  Since $vw\in E(G)$, we have $x'a\in E(H)$.  By \tref{ancestor-edge}, one of $a$ or $x'$ is a $T$-ancestor of the other. Since $w\in Y_a\subseteq V_a$ and $w\not\in V_x\supseteq V_{x'}$, \yref{containment-i} rules out the possibility that $x'$ is a $T$-ancestor of $a$. Therefore, $a$ is a $T$-ancestor of $x$ which is a $T$-ancestor of $x'$.  Let $z_0,\ldots,z_r$ be the path in $T$ from $z_0:=x'$ to $z_r:=a$.  Since $x'a\in E(H)$, at least one of $a$ or $x'$ is in $B_{z_i}$, for each $i\in\{0,\ldots,r\}$ and there is at least one $i$ such that $B_{z_i}$ contains both $x'$ and $a$.  However, by \tref{subtree-root} $x'$ is not contained in $B_{z_i}$ for any $i\in\{1,\ldots,r\}$.  Therefore $a\in B_{z_i}$ for each $i\in\{0,\ldots,r\}$.  In particular, $a$ is contained in $B_x$.
%   Property~\yref{ancestor-edge} now follows from the fact that $|B_x|\le t+1$ and $B_x$ contains $x$.
% \end{proof}

% \referee{2}{Page 7.  In the very last sentence, I think $B_{x_i}$ should be $B_{z_i}$
% (twice).  Moreover, the way the proof is written, it seems to suggest
% that we only get that $a \in B_{z_i}$ for each $i \in \{1, \dots, r\}$.
% The conclusion that $a \in B_{z_i}$ for each $i \in \{0, \dots, r\}$ is
% true though.  In a normalized tree-decomposition, if $xy$ is an edge and
% $x$ is a $T$-ancestor of $y$, then $x$ must be in $B_y$ (since the subtrees $T_x$
% and $T_y$ intersect).}

%\note{PM}{I addressed the preceding referee's.}

% We are now ready to complete the proof.
% , which we restate here for convenience:

% \mmg*

\begin{proof}[Proof of \cref{ShortcutPartition}]
  Let $\mathcal{L}:=\langle L_1,\ldots,L_h\rangle$ be a layering of $G$; let $\mathcal{Y}:=(Y_x: x\in V(H))$ be an $H$-partition of $G$ of layered width at most $\ell$ with respect to $\mathcal{L}$; and let $\mathcal{T}:=(B_x:x\in V(T))$ be a tree-decomposition of $H$ satisfying the conditions of \cref{nice-decomposition}.

For a node $x\in V(T)$, we say that a shortcut $P\in\SS$ \defin{crosses} $x$ if $Y_x$ contains an internal vertex of $P$.  In other words, if $P=v_0,\ldots,v_r$ and $\{v_1,\ldots,v_{r-1}\}\cap Y_x\neq\emptyset$.  We say that a vertex $v\in V(G)$ \defin{participates} in $x$ if $v\in Y_x$ or if $\SS$ contains a shortcut $P$ with $v\in V(P)$ and $P$ crosses $x$.  For each $v\in V(G)$, let $a(v)$ denote the lowest common $T$-ancestor of all the nodes $x\in V(T)$ in which $v$ participates.

For each $x\in V(T)$, define $S_x := \{v\in V(G): a(v)= x\}$. Since $a:V(G)\to V(T)$ is a function with domain $V(G)$ and range $V(T)$, $\PP:=(S_x : x\in V(T))$ is a sequence of pairwise disjoint sets that cover $V(G)$. Let $J:=G^\SS/\PP$ denote the resulting quotient graph. We consider $V(J)\subseteq V(T)$, where each $x\in V(J)$ is the vertex obtained by contracting $S_x$ in $G^{\SS}$. (Any node $x\in V(T)$ with $S_x=\emptyset$ does not contribute a vertex to $J$.)

From this point onward, the plan is to show that:
\begin{compactenum}[(i)]
  \item $\PP$ has small layered width with respect to the layering $\mathcal{L}$ of $G$, and 
  \item $J$ has small treewidth.
\end{compactenum}
Once we have established (i) and (ii), the result follows easily since a layering of $G^\SS$ is easily obtained from $\mathcal{L}$ by `compressing' groups of $k$ consecutive layers.  We begin with Step~(i):

\begin{clm}
	\label{general-width}
  For each $i\in\{1,\ldots,h\}$ and each $x\in V(J)$, $|S_x\cap L_i|\le d\ell(k^2+3)$.
\end{clm}

\begin{proof}
  Recall that $S_x$ is defined by vertices that participate in $x$, and these are vertices that are either in $Y_x$ or in shortcuts that cross $Y_x$.  We say that a vertex $w\in Y_x$ \defin{contributes} a vertex $v\in S_x$ if $v=w$ or if some path in $\SS$ that contains $v$ has $w$ as an internal vertex.
  We upper bound the number of vertices in $S_x\cap L_i$ by upper-bounding the number of vertices contributed to $S_x\cap L_i$ by each $w\in Y_x$.

  Refer to \cref{contribute}.  If $w\in Y_x\cap L_i$ and no path in $\SS$ includes $w$ as an internal vertex then $w$ contributes at most one vertex, itself, to $S_x\cap L_i$.  Otherwise, consider some path $P\in\SS$ that contains $w$ as an internal vertex.  If $w\in L_{i}$, then $P$ contributes at most $k+1$ vertices to $S_x\cap L_i$.  If $w\in L_{i-1}\cup L_{i+1}$, then $P$ contributes at most $k$ vertices to $S_x\cap L_i$. If $w\in L_{i-j}\cup L_{i+j}$ for $j\ge 2$, then $P$ contributes at most $k-j$ vertices to $S_x\cap L_i$.

  \begin{figure}[htbp]
    \begin{center}
      \includegraphics{figs/contribute}
    \end{center}
    \caption{A path $P$ containing an internal vertex $w\in Y_x\cap L_{i-j}$.}
    \label{contribute}
  \end{figure}


  For any $j$, the number of vertices $w\in L_{i+j}\cap Y_x$ is at most $\ell$. Each such vertex $w$ is an internal vertex of at most $d$ paths in $\SS$. Therefore,
  \[  |S_x\cap L_i|\le d\ell  \, \Big(k+1 + 2k + \sum_{j=2}^k 2(k-j)\Big)
  %= d(2k+1) + \sum_{i=1}^{k-2} i
      = d\ell(k^2 +3) \enspace . \qedhere
  \]
\end{proof}

We now proceed with Step~(ii), showing that $J$ has small treewidth. To accomplish this, we construct a small width tree-decomposition $\mathcal{C}:=(C_x:x\in V(T))$ of $J$ using the same tree $T$ used in the tree-decomposition $\mathcal{T}$ of $H$.  The following claim will be useful in showing that the resulting decomposition has small width.

\begin{clm}\label{i-ancestor}
  For each edge $xy\in E(J)$, one of $x$ or $y$ is a $T$-ancestor of the other.
\end{clm}

\begin{proof}
  If $xy\in E(J)$ then there exists a shortcut $P:=v_0,\ldots,v_r\in\SS$ with endpoints $v_0\in S_x$ and $v_r\in S_y$.\footnote{Recall that we have made the assumption that $\SS$ contains a length-$1$ $vw$-shortcut for each edge $vw\in E(G)$.}  For each $i\in\{0,\ldots,r\}$, let $Y_{x_i}$ be the part in $\mathcal{Y}$ that contains $v_i$.  Since $P$ is a connected subgraph of $G$, $H[\{x_0,\ldots,x_r\}]$ is a connected subgraph of $H$.  Since $\mathcal{T}$ is a normalized tree-decomposition of $H$, \tref{ancestor-edge} implies that some node $x_i$ is a $T$-ancestor of all nodes in $x_0,\ldots,x_r$.

  We claim that at least one of $v_0$ or $v_r$ participates in $x_i$.  If $i=0$ then $v_0\in Y_{x_0}=Y_{x_i}$ so $v_0$ participates in $x_i$. Similarly, if $i=r$ then $v_r$ participates in $x_r$.  Otherwise $i\in\{1,\ldots,r-1\}$ so $v_i$ is an internal vertex of the shortcut $P$, so $P$ crosses $x_i$.  Since $v_0,v_r\in V(P)$, this implies that $v_0$ and $v_r$ both participate in $x_i$.

  Suppose, without loss of generality, that $v_0$ participates in $x_i$.  Then $a(v_0)=x$ is a $T$-ancestor of $x_i$ which a $T$-ancestor of $x_r$.  Finally $a(v_r)=y$ is a $T$-ancestor of $x_r$.  Therefore, both $x$ and $y$ are contained in the path from the root of $T$ to $x_r$, so at least one of $x$ or $y$ is a $T$-ancestor of the other.  This completes the proof of \cref{i-ancestor}
\end{proof}

\begin{clm}
\label{general-bag-size}
The graph $J$ has a tree-decomposition in which every bag has size at most $\binom{k+t}{t}$.
\end{clm}

\begin{proof}
  For the tree-decomposition $(C_x:x\in V(T))$ of $J$ we use the same tree $T$ used in the tree-decomposition $(B_x:x\in V(T))$ of $H$. For each node $x$ of $T$, we define $C_x$ as follows: $C_x$ contains $x$ as well as every $T$-ancestor $a$ of $x$ such that $J$ contains an edge $ax'$ where $x$ is a $T$-ancestor of $x'$ (including the possibility that $x=x'$).
  \cref{i-ancestor} ensures that, for every edge $ax'\in E(J)$, both $a$ and $x$ are contained in $C_{x'}$.  The connectivity of $T[a]:=T[\{x\in V(T):a\in C_x\}]$ follows from the fact that, for every node $x'\in V(T[a])$, every node $x$ on the path in $T$ from $a$ to  $x'$ is also a node of $T[a]$.  Therefore $(C_x:x\in V(T))$ is indeed a tree-decomposition of $J$.

  It remains to show that each bag in this tree-decomposition has size at most $\binom{k+t}{t}$.  We will do this by appealing to an elegant result of \citet{PS21} on $k$-reachability in directed skeletons, which we now explain.

  Let $H^+$ denote the supergraph of $H$ that contains the edge $xy$ if and only if some bag $B_z\in\mathcal{T}$ contains both $x$ and $y$ (that is, $H^+$ is the chordal closure of $H$ with respect to $\mathcal{T}$).  Let $\overrightarrow{H}^+$ be the directed graph obtained by directing each edge $xy$ of $H^+$ in the direction $\overrightarrow{xy}$ so that $y$ is a $T$-ancestor of $x$. Observe that any directed path in $\overrightarrow{H}^+$ that begins at $x$ leads to a $T$-ancestor of $x$.  \citet[Lemma~13]{PS21} show that, for any $x\in V(\overrightarrow{H}^+)$, the number of strict $T$-ancestors of $x$ that can be reached by a directed path of length at most $k$ that begins at $x$ is at most $\binom{k+t}{t}-1$.

  Consider an arbitrary node $x\in V(T)$ and suppose that some strict $T$-ancestor $a$ of $x$ is contained in $C_x$.  We will show that $\overrightarrow{H}^+$ contains a directed path from $x$ to $a$ of length at most $k$.  By \cite[Lemma~13]{PS21}, this implies that the number of strict $T$-ancestors of $x$ contained in $C_x$ is at most $\binom{k+t}{t}-1$ and therefore $|C_x|\le\binom{k+t}{t}$.

  Since $a\in C_x$, there exists a $T$-descendant $x'$ of $x$ such that $ax'\in E(J)$.  Therefore, there exists a shortcut $P:=v_0,\ldots,v_r$ in $\SS$ with $v_0\in S_{x'}$ and $v_r\in S_a$.  For each $i\in\{0,\ldots,r\}$, let $Y_{x_i}$ be the part of $\mathcal{Y}$ that contains $v_i$.  For each $i\in\{1,\ldots,r-1\}$, $v_i\in Y_{x_i}$ is an internal vertex of $P$, so $v_r$ participates in $x_i$.  Therefore $a=a(v_r)$ is a $T$-ancestor of $x_i$ for each $i\in\{1,\ldots,r-1\}$.

  Since $v_0,\ldots,v_r$ is a path in $G$, $x_0,\ldots,x_r$ is a lazy walk in $H$.\footnote{A \defin{lazy walk} is a graph $G$ is a walk in the pseudograph obtained by adding a self-loop at each vertex of $G$.}  Let $j$ be the minimum integer such that $x_j$ is a strict $T$-ancestor of $x$.  Since $x'=x_0$ is a $T$-descendant of $x$ and $a=x_r$ is a strict $T$-ancestor of $x$, $j\in\{1,\ldots,r\}$ and $x_{j-1}$ is a $T$-descendant of $x$ (possibly $x_{j-1}=x$).  Since $x_{j-1}x_j\in E(H)$, $x_j \in B_{x_{j-1}}$ and therefore $x_j\in B_x$.  Therefore $xx_j\in E(H^+)$.

  Therefore $W:=y_0,\ldots,y_s:=x,x_j,x_{j+1},\ldots,x_r$ is a lazy walk in $H^+$ of length $r\le k$.  For each $j\in \{0,\ldots,s\}$, let $z_j$ be the lowest common $T$-ancestor of $y_0,\ldots,y_j$.  The same reasoning used to conclude that $xx_j\in E(H^+)$ can be used to conclude that $z_{i-1}=z_i$ or $z_{i-1}z_i\in E(H^+)$ for each $i\in\{1,\ldots,s\}$.  Therefore $z_0,\ldots,z_s$ is a directed lazy walk in $\overrightarrow{H}^+$.  Finally, let $\overrightarrow{W}$ be the path in $\overrightarrow{H}^+$ obtained by removing removing duplicates from $z_0,\ldots,z_s$.  Then $\overrightarrow{W}$ is a directed path in $\overrightarrow{H}^+$ from $x$ to $a$ of length at most $s\le r\le k$.  This completes the proof of \cref{general-bag-size}.
\end{proof}

At this point, the proof of \cref{ShortcutPartition} is almost immediate from \cref{general-width,general-bag-size}, except that the layering $\mathcal{L}$ of $G$ may not be a valid layering of $G^{\SS}$.  In particular, $G^{\SS}$ may contain edges $vw$ with $v\in L_i$ and $w\in L_{i+j}$ for any $j\in\{0,\ldots,k\}$.  To resolve this, we use a new layering $\mathcal{L}':=\langle L_0',\ldots,L_h'\rangle$ in which $L_i'=\bigcup_{j=ki}^{ki+k-1} L_i$.  This increases the layered width given by \cref{general-width} from $d\ell(k^2+3)$ to $d\ell(k^3+3k)$.  Therefore $G$ has an $H$-partition of layered width at most $d\ell(k^3+3k)$ in which $H$ has treewidth at most $\binom{k+t}{t}-1$, completing the proof of \cref{ShortcutPartition}.
\end{proof}

\subsection{\boldmath $k$-Planar Graphs}
\label{k_planar_section}

% \note{DW}{These definitions should appear earlier, certainly before the definition of framed graph, which should be defined as an ``embedded graph''. One option is to put this whole paragraph as a footnote in the introduction where we first mention $k$-planar graphs. The experienced reader can skip this paragraph, which says to me it should be a footnote.}

We first formally define $k$-planar graphs.  For a surface $\Sigma$,  a \defin{$\Sigma$-embedded graph} $G$ is a graph with $V(G)\subset\Sigma$ in which each edge $vw\in E(G)$ is a curve\footnote{A \defin{curve} in a surface $\Sigma$ is a continuous function $f:[0,1]\to \Sigma$. The points $f(0)$ and $f(1)$ are called the \defin{endpoints} of the curve.  When there is no danger of misunderstanding we treat a curve $f$ as the point set $\{f(t):0\le t\le 1\}$.  The \defin{interior} of $f$ is the point set $\{f(t):0<t<1\}$.} in $\Sigma$ with endpoints $v$ and $w$ and not containing any vertex of $G$ in its interior. The \defin{faces} of $G$ are the connected components of $\R^2\setminus \bigcup_{vw\in E(G)} vw$. A \defin{crossing} in a $\Sigma$-embedded (multi)graph $G$ is an unordered pair of distinct edges $\{vw,xy\}\subseteq E(G)$ whose interiors have a point in common.

If an embedded graph $G$  has no crossings then it is \defin{non-crossing} graph.  When $G$ is non-crossing and $F$ is a face of $G$ then $V(F)$ denotes the set of vertices in $G$ on the boundary of $F$.  When $F$ is homeomorphic to a disk, the \defin{facial walk} of $F$ is the cycle consisting of the edges and vertices of $G$ on the boundary of $F$.  If  $\Sigma=\R^2$ is the Euclidean plane and each edge of $G$ is involved in at most $k$ crossings, then $G$ is a $k$-plane graph.  A graph $G$ is \defin{$k$-planar} if is isomorphic to some $k$-plane graph.  A $0$-plane graph is a \defin{plane} graph and a $0$-planar graph is a \defin{planar} graph.

Without any loss of generality we may assume that the interiors of any two edges of a $\Sigma$-embedded graph $G$ have at most one point in common and that no point in $\R^2$ is contained in the interior of three or more edges of $G$.  This assumption can be enforced by local changes that do not change the number of crossings each edge is involved in.  If $p$ is a common point in the interior of two crossing edges $vw$ and $xy$, then we say that $vw$ and $xy$ \defin{cross at} $p$.



% (This assumption is already present in the informal definition of $k$-planar graphs used in \cref{k_planar_section}.)


%
% \note{DW}{We need to assume that at most two edges cross at a single point, so we can add a dummy vertex of degree 4, right?}\note{PM}{No, under the current definition, $r$ edges crossing at a single point $p$ generate $\binom{r}{2}$ crossings, with each edge taking part in $r-1$ crossings, just like they would if we perturbed the edges so that each pair crosses at a distinct point.  I guess we could mention this.  (See the first paragraph in the proof of Theorem~2 below.)} \note{DW}{The downside of this approach is that in the opening section we give an informal definition of $k$-planar, which needs the assumption of having no three edges cross at a point, since we use this assumption in the construction of a $(k+1,2)$-shortcut system for $k$-planar graphs give in the Introduction. I think we should not have different definitions in two parts of the paper.} A (not necessarily embedded) graph $G'$ is \defin{$k$-planar} if there exists a $k$-plane graph $G$ isomorphic to $G'$. Under these definitions, $0$-planar graphs are exactly planar graphs and $0$-plane graphs are exactly plane graphs.

As mentioned in \cref{Introduction}, \cref{PlanarProduct,ShortcutProduct} imply a product structure theorem for $k$-planar graphs. We get improved bounds as follows.

%\begin{thm}
%\label{k-planar}
%Every $k$-planar graph has an $H$-partition of layered width at most $18k^2 + 48k+30$ in which $\tw(H)\leq \binom{k+4}{3}-1$.
%\end{thm}

\begin{proof}[Proof of \cref{kPlanarProduct}]
	Let $G$ be a $k$-plane graph.
    % We assume, for ease of exposition, that any point $p\in\R^2$ is involved in at most one crossing $(p,vw,xy)$ of $G$. This assumption is justified since it can be enforced by a slight deformation of the edges of $G$ and the resulting (deformed) graph is also $k$-plane.
	As in the proof of \cref{AddDummy}, let $G_0$ be the plane graph obtained by adding a dummy vertex at each crossing in $G$. In this way, each edge $vw\in E(G)$ corresponds naturally to a path $P_{vw}$ of length at most $k+1$ in $G_0$.  Let $\SS := \{P_{vw}: vw\in E(G)\}$. Observe that $\SS$ is a $(k+1,2)$-shortcut system for $G_0$ and that $G_0^{\SS}\supseteq G$.  Specifically, $G_0^{\SS}$ contains every edge and vertex of $G$ as well as the dummy vertices in $V(G_0)\setminus V(G)$ and their incident edges.

	Since $G_0$ is planar,  \cref{PlanarProduct}(b) and \cref{PartitionProduct} implies that $G_0$ has an $H$-partition of layered width 3 for some planar graph $H$ of treewidth at most 3.  Applying \cref{ShortcutPartition} to $G_0$ and $\SS$ immediately implies that $G$ (an arbitrary $k$-planar graph) has an $H$-partition of layered width $6((k+1)^3+3(k+1))$ for some graph $H$ of treewidth at most $\binom{k+4}{3}-1$.

	We can reduce the layered width of the $H$-partition of $G$ from $O(k^3)$ to $O(k^2)$ by observing that the dummy vertices in $V(G_0)\setminus V(G)$ do not contribute to the layered width of this partition.  In this setting, the proof of \cref{general-width} is simpler since each vertex $w\in Y_x$ contributes at most two vertices to $L_i\cap Y_x$.  More precisely, each path $P\in\SS$ containing an internal (dummy) vertex $w\in Y_x\cap (L_{i-j}\cup L_{i+j})$ contributes: (i)~at most two vertices to $S_x\cap L_i$ for $j\in\{0,\ldots,\floor{(k+1)/2}\}$; (ii)~at most one vertex to $S_x\cap L_j$ for $j\in\{\floor{(k+1)/2}+1,\ldots,k+1\}$; or (iii)~no vertices to $S_x\cap L_j$ for $j > k+1$.
	Redoing the calculation at the end of the proof of \cref{general-width} then yields
	\begin{align*}
	|S_x\cap Y_i| \le d\ell\left(
	2
	+ 4\left\lfloor\tfrac{k+1}{2}\right\rfloor
	+ 2\left\lceil\tfrac{k+1}{2}\right\rceil
	\right)
	 =
	d\ell\left(
	2 + 2(k+1) + 2\left\lfloor\tfrac{k+1}{2}\right\rfloor
	\right)
	 \le
	d\ell(3k+5)
	= 18k+30 \enspace .
	\end{align*}
	With this change, the layered width of the partition given by \cref{ShortcutPartition} becomes $(18k+30)(k+1)=18k^2+48k+30$.
	The result follows from \cref{PartitionProduct}.
	%This establishes \cref{k-planar}.
	% Note to self-and-others. In our original SODA submission we had 18k^2+30k here because we forgot that the layer compression step has compress k+1 consecutive layers, not just k consecutive layers.
\end{proof}

% \subsection{Examples}


\cref{kPlanarProduct} shows that every $k$-planar graph is contained in $H\boxtimes P \boxtimes K_\ell$ for some graph $H$ with treewidth $O(k^3)$ where $\ell\leq O(k^2)$.  In some applications, the treewidth of $H$ is much more significant than the value of $\ell$, which leads to the following question:

\noindent\textbf{Open Problem:}
  Does there exist a function $\ell:\N\to\N$ and a universal constant $C$ such that every $k$-planar graph is contained in $H\boxtimes P \boxtimes K_{\ell(k)}$ for some graph $H$ with treewidth at most $C$?  Perhaps $C=3$, which is the case for planar graphs (\cref{PlanarProduct}(b)) and for $1$-planar graphs (\cref{1_planar_product}). Note that $C\geq 3$ even for planar graphs \citep{DJMMUW20}.

\subsection{\boldmath $(g,k)$-Planar Graphs}
\label{gk_planar_section}

As mentioned in the introduction, product structure theorems have been established for several minor-closed classes in addition to planar graphs, including the following version of \cref{PlanarProduct} for graphs of bounded Euler genus.

\begin{thm}[\citep{DJMMUW20,UWY,DHHW}]\label{GenusProduct}
  Every graph of Euler genus $g$ is contained in:
  \begin{compactenum}[(a)]
  \item $H  \boxtimes P$ for some graph $H$ of treewidth at most $2g+6$  and some path $P$.
  \item $H \boxtimes P \boxtimes K_{\max\{2g,3\}}$ for some planar graph $H$ of treewidth at most $3$ and for some path $P$.
  \end{compactenum}
\end{thm}

The definition of $k$-planar graphs naturally generalises to genus-$g$ surfaces. A $\Sigma$-embedded graph $G$ is \defin{$(\Sigma,k)$-plane} if every edge of $G$ is involved in at most $k$ crossings.  A graph $G$ is \defin{$(g,k)$-planar} if it is isomorphic to some $(\Sigma,k)$-plane graph, for some surface $\Sigma$ with Euler genus at most $g$. \cref{AddDummy} immediately generalises as follows:

\begin{obs}
\label{gAddDummy}
Every $(g,k)$-planar graph $G$ is contained in $G_0^\SS$ for some graph $G_0$ of Euler genus at most $g$ and some $(k+1,2)$-shortcut system $\SS$ for $G_0$. Moreover, $V(G) \subseteq V(G_0)$ and for every edge $vw \in E(G)$ there is a $vw$-path $P$ in $G_0$ of length at most $k+1$, such that every internal vertex in $P$ has degree at most $4$ in $G_0$.
\end{obs}

Theorems~\ref{ShortcutProduct}  and \ref{GenusProduct}(b) imply a product structure theorem for $(g,k)$-planar graphs. The resulting bounds are improved by the following theorem, which is proved using exactly the same approach used in the proof of \cref{kPlanarProduct} (applying \cref{GenusProduct}(b) instead of \cref{PlanarProduct}(b)). We omit repeating the details.

\begin{thm}
\label{gkPlanarProduct}
Every $(g,k)$-planar graph is contained in $H\boxtimes P \boxtimes K_\ell$ for some graph $H$ with $\tw(H) \leq \binom{k+4}{3}-1$, where $\ell:=\max\{2g,3\}\cdot(6k^2+16k+10)$.
% \note{DW}{I have changed $\tw(H) \leq \binom{k+5}{4}-1$ to $\tw(H) \leq \binom{k+4}{3}-1$, okay?}
%\note{PM}{Ok}
\end{thm}






\subsection{Rough Characterisation}
\label{Characterisation}

\cref{gAddDummy} shows that $(g,k)$-planar graphs can be obtained by a shortcut system applied to a graph of Euler genus $g$, where internal vertices on the paths have bounded degree. This observation and the following converse result together provide a rough characterisation of $(g,k)$-planar graphs, which is interesting in its own right, and is useful for showing that various classes of graphs are $(g,k)$-planar.

\begin{lem}
	\label{DrawG}
	Fix integers $g\geq 0$ and $k,\Delta\geq 2$.
	Let $G_0$ be a graph of Euler genus at most $g$. Let $G$ be
	a graph with $V(G) \subseteq V(G_0)$ such that for every edge $vw \in
	E(G)$ there is a $vw$-path $P_{vw}$ in $G_0$ of length at most $k$, such
	that every internal vertex on $P_{vw}$ has degree at most $\Delta$ in
	$G_0$. Then $G$ is $(g, 2k(k+1)\Delta^{k} )$-planar.
\end{lem}

\begin{proof}
	For a vertex $x$ of $G_0$ with degree at most $\Delta$, and for $i\in\{1,\dots,k-1\}$, say a vertex $v$ is \defin{$i$-close} to $x$ if there is a $vx$-path $P$ in $G_0$ of length at most $i$ such that every internal vertex in $P$  has degree at most $\Delta$ in $G_0$.
	For each edge $vw$ of $G$, say that $vw$ \defin{passes through} each internal vertex on $P_{vw}$.	Say $vw$ passes through $x$. Then $v$ is $i$-close to $x$ and $w$ is $j$-close to $x$ for some $i,j\in\{1,\dots,k-1\}$ with $i+j\leq k$. At most $\Delta^{i}$ vertices are $i$-close to $x$.
	Thus, the number of edges of $G$ that pass through $x$ is at most
	\[
	\sum_{i=1}^{k-1} \sum_{j=1}^{k-i} \Delta^i \Delta^j
	= \sum_{i=1}^{k-1} \Delta^i  \sum_{j=1}^{k-i} \Delta^j
	< \sum_{i=1}^{k-1} \Delta^i  2 \Delta^{k-i}
	= \sum_{i=1}^{k-1} 2\Delta^k
	< 2k \Delta^k \enspace.
	\]
	Fix a $(\Sigma,k)$-plane drawing of $G_0$ in a surface $\Sigma$ of Euler genus at most $g$. Choose $\epsilon>\epsilon'>0$.
	For each vertex $v$ of $G_0$, let $B_v$ be the \defin{ball} $\{x\in \mathbb{R}^2:\dist(x,v)\leq\epsilon\}$. 
	For each edge $vw$ of $G_0$, let $C_{vw}$ be the \defin{channel} $\{x\in \mathbb{R}^2:\dist(x,vw)\leq\epsilon'\}\setminus(B_v\cup B_w)$.
	We may choose $\epsilon>\epsilon'>0$ sufficiently small so that
	(i) $B_v\cap B_w=\emptyset$ for distinct $v,w\in V(G_0)$,
	(ii) $C_{vw}\neq\emptyset$ for $vw\in E(G_0)$,
	(iii) $B_v\cap C_{xy}=\emptyset$ for $v\in V(G_0)$ and $xy\in E(G_0)$, and
	(iv) $C_{vw}\cap C_{xy}=\emptyset$ for distinct $vw,xy\in E(G_0)$.
	Draw each edge $vw$ of $G$ following the sequence of balls and channels defined by the vertices and edges in $P_{vw}$.
	This can be done so that whenever edges $vw$ and $xy$ of $G$ cross, the crossing point is in $B_z$ for some vertex $z$ of $G_0$ that is internal in $P_{vw}$ or in $P_{xy}$, and $vw$ and $xy$ cross at most once in each such $B_z$.

  Thus, for each edge $vw$ of $G$, every edge of $G$ that crosses $vw$ passes through a vertex on $P_{vw}$ (including $v$ and/or $w$ if they too have degree at most $\Delta$).
	Since $P_{vw}$ has at most $k+1$ vertices, and less than $2k\Delta^{k}$ edges of $G$ pass through each vertex on $P_{vw}$, the edge $vw$ is crossed by less than $2k(k+1)\Delta^{k}$ edges in $G$. Hence $G$ is $(g, 2k(k+1)\Delta^{k} )$-planar.
\end{proof}

%\referee{2}{Page 16.  At the end of Page 16, I do not see why 'every pair of edges cross at most once.'  Without further assumptions, it seems as if two paths in the path system could intersect several times.  Therefore, by drawing 'each edge vw of G alongside $P_{vw}$ in $G_0$', there may be two edges which cross several times.  It seems this can be fixed by choosing the path system carefully by 'uncrossing' paths which intersect at more than two internal vertices.}
%
%\note{DW}{I have revised the proof to address this error. I couldn't think of a way to do this without introducing balls and channels (and without vigorous hand-waving). Is it readable?}





\section{Graphs Derived from Shortcut Systems}
\label{examples}

In this section we give four examples of well-known graph families that can be expressed in terms of shortcut systems and which therefore have product structure theorems.


%%%%%%%
\subsection{String Graphs}
\label{first_example}

A \defin{string graph} is the intersection graph of a set of curves in the plane with no three curves meeting at a single point; see  \cite{PachToth-DCG02,FP10,FP14} for example. For an integer $\delta\geq 2$, if each curve is in at most $\delta$ intersections with other curves, then the corresponding string graph is called a \defin{$\delta$-string graph}. A \defin{$(g,\delta)$-string} graph is defined analogously for curves on a surface of Euler genus at most $g$.

\begin{lem}
\label{StringShortcut}
Every $(g,\delta)$-string graph $G$ is contained in $G_0^\SS$ for some graph $G_0$ with Euler genus at most $g$ and some $(\delta+1,\delta+1 )$-shortcut system $\SS$ for $G_0$.
%Moreover, every internal vertex on the paths in $\SS$ has degree 4 in $G_0$.
\end{lem}

\begin{proof}
Let $\mathcal{C}=\{C_v:v\in V(G)\}$ be a set of curves in a surface of Euler genus at most $g$ whose intersection graph is $G$.  Let $G_0$ be the graph obtained by adding a vertex at the intersection point of every pair of curves in $\mathcal{C}$ that intersect,  where two such consecutive vertices on a curve $C_v$ are adjacent in $G_0$. For each vertex $v\in V(G)$, if $C_v$ intersects $k\leq\delta$ other curves, then introduce a new vertex called $v$ on $C_v$ between the
$\floor{\frac{k}{2}}$-th vertex already on $C_v$ and the $\floor{\frac{k}{2}+1}$-th such vertex. For each edge $vw$ of $G$, there is a path $P_{vw}$ of length at most $2\ceil{\frac{\delta}{2}}\leq \delta+1$ in $G_0$ between $v$ and $w$. Let $\SS$ be the set of all such paths $P_{vw}$. Consider a vertex $x$ in $G_0$ that is an internal vertex on some path in $\SS$. Then $x$ is at the intersection of $C_v$ and $C_w$ for some edge $vw\in E(G)$. If some path $P\in \SS$ passes through $x$, then $P=P_{vu}$ for some edge $vu$ incident to $v$, or $P=P_{wu}$ for some edge $wu$ incident to $w$. At most $\ceil{\frac{\delta}{2}}$ paths in $\SS$ corresponding to edges incident to $v$ pass through $x$, and similarly for edges incident to $w$. Thus at most $2\ceil{\frac{\delta}{2}}\leq\delta+1$ paths in $\SS$ use $x$ as an internal vertex. Thus $\SS$ is a $(\delta+1,\delta+1)$-shortcut system for $G_0$, and by construction, $G \subseteq G_0^\SS$.
\end{proof}

% A similar proof to that of \cref{StringShortcut} shows that every $(g,\delta)$-string graph is $(g,2\delta^2)$-planar.

\cref{StringShortcut,ShortcutProduct,GenusProduct} now imply:
 % and \cref{StringShortcut,qn} imply:

\begin{thm}\label{StringProduct}
For integers $g\geq 0$ and $\delta\geq 2$, let $\ell:= \max\{2g,3\} \,(\delta^4 + 4 \delta^3 + 9 \delta^2 + 10 \delta + 4)$ and $t:= \binom{ \delta+4}{3}-1$.
Then  every $(g,\delta)$-string graph is contained in $H\boxtimes P \boxtimes K_{\ell}$ for some path $P$ and for some graph $H$ with treewidth $t$,
\end{thm}


% \begin{thm}
% \label{StringPartition}
% For integers $g\geq 0$ and $\delta\geq 2$, let $\ell:= \max\{2g,3\} \,(\delta^4 + 4 \delta^3 + 9 \delta^2 + 10 \delta + 4)$ and $t:= \binom{ \delta+4}{3}-1$.
% Then  every $(g,\delta)$-string graph:
% \begin{compactitem}
% 	\item is contained in $H\boxtimes P \boxtimes K_{\ell}$ for some path $P$ and for some graph $H$ with treewidth $t$,
% 	\item has queue-number $\qn(G^k) \leq 3 \ell \, 2^t - \ceil{\tfrac{3}{2}\ell}$.
% \end{compactitem}
% \end{thm}

% \note{DW}{\cref{StringPartition} has been updated.}

% Our results also give bounds on the non-repetitive chromatic number and the $p$-centered chromatic number of $(g,\delta)$-string graphs, but the bounds are weak, since such graphs $G$ have maximum degree at most $2\delta$, implying that $\pi(G) \leq (4+o(1))\delta^2$ and $\chi_p(G)\le p(64\delta)^2$ by results of \citet{DJKW16} and \citet{DFMS21}, respectively.

%%%%%%%%%%
\subsection{Powers of Bounded Degree Graphs}
\label{Powers}
\label{mid_example}

Recall that the \defin{$k$-th power} of a graph $G$ is the graph $G^k$ with vertex set $V(G^k):=V(G)$, where $vw\in E(G^k)$ if and only if $\dist_G(v,w)\leq k$. If $G$ is planar with maximum degree $\Delta$, then $G^k$ is $2k(k+1)\Delta^{k}$-planar by \cref{DrawG}.  Thus we can immediately conclude that bounded powers of planar graphs of bounded degree have product structure. However, the bounds we obtain are improved by the following lemma that constructs a shortcut system directly.

\begin{lem}
\label{PowerShortcut}
If a graph $G$ has maximum degree $\Delta$, then $G^k = G^\SS$ for some $(k,2k \Delta^{k})$-shortcut system $\SS$.
\end{lem}

\begin{proof}
For each pair of vertices $x$ and $y$ in $G$ with $\dist_G(x,y)\in\{1,\dots,k\}$, fix an $xy$-path $P_{xy}$ of length
$\dist_G(x,y)$  in $G$. Let $\SS:=\{P_{xy}: \dist_G(x,y)\in\{1,\dots,k\} \}$. Say $P_{xy}$ uses some vertex $v$ as an internal vertex. If $\dist_G(v,x)=i$ and $\dist_G(v,y)=j$, then $i,j\in\{1,\dots,k-1\}$ and $i+j\leq k$. The number of vertices at distance $i$ from $v$ is at most $\Delta^i$. Thus the number of paths in $\SS$ that use $v$ as an internal vertex is at most
\[\sum_{i=1}^{k-1} \sum_{j=1}^{k-i} \Delta^i\Delta^j
= \sum_{i=1}^{k-1} \Delta^i \sum_{j=1}^{k-i} \Delta^j
< \sum_{i=1}^{k-1} \Delta^i ( 2 \Delta^{k-i} )
< 2k \Delta^k\enspace.\]
Hence $\SS$ is a $(k, 2k \Delta^k)$-shortcut system.
\end{proof}

\cref{ShortcutProduct,PowerShortcut} imply:

\begin{thm}
\label{PowerProduct}
Let $G$ be contained in $H\boxtimes P\boxtimes K_\ell$ with maximum degree $\Delta$, for some graph $H$ of treewidth at most $t$ and for some path $P$. Then for every integer $k\geq 1$, the $k$-th power $G^k$ is contained in $J\boxtimes P\boxtimes K_{2k \ell \Delta^{k}(k^3+3k)}$ for some graph $J$ of treewidth at most $\binom{k+t}{t}-1$ and some path $P$.
\end{thm}

% Theorems~\ref{PlanarProduct}(b), \ref{GenusProduct}(b) and \cref{ShortcutProduct} and  \cref{DrawG,qn,p-centered,non-repetitive,PowerShortcut} imply the following result, which with $g=0$ implies  \cref{kPowerBasic} in the introduction.
%
% \begin{thm}
% \label{PowerGenus}
% For integers $g\geq 0$ and $k,\Delta\geq 1$, let $\ell:= \max\{2g,3\} (2k^4+6k^2) \Delta^{k}$ and $t:= \binom{k+3}{3}-1$. Then for every graph $G$ of Euler genus $g$ and maximum degree $\Delta$,
% \begin{compactitem}
% \item $G^k$ is contained in $H\boxtimes P \boxtimes K_{\ell}$ for some path $P$ and for some graph $H$ with treewidth $t$
% \item $G^k$ is $(g, 2k(k+1)\Delta^{k} )$-planar,
% \item $G^k$ has queue-number $\qn(G^k) \leq 3 \ell \cdot 2^t - \ceil{\tfrac{3}{2}\ell}$.
% \item $G^k$ has  $p$-centered chromatic number $\chi_p(G^k) \leq \ell (p+1)\,  \binom{p+t}{t}$,
% \item $G^k$ has  non-repetitive chromatic number $ \pi(G^k) \leq \ell \, 4^{t+1}$.
% \end{compactitem}
% \end{thm}
%
% % \note{DW}{\cref{PowerGenus} has been updated}
%
% This result is the first constant upper bound on the queue-number of bounded powers of graphs with bounded degree and bounded Euler genus.  For every graph $G$, since $G^k$ has maximum degree at most $\Delta^k$, a result of \citet{DJKW16} implies that $\pi(G^k) \leq (1+o(1))\Delta^{2k}$. \cref{PowerGenus} improves upon this bound when $k,g\ll\Delta$.  Similarly, a result of \citet{DFMS21} implies that $\chi_p(G^k)\le 1024p\Delta^{2k}$ and \cref{PowerGenus} improves upon this bound when $p,k,g\ll\Delta$.
%
% \cref{qn,p-centered,non-repetitive,MinorFreeDegree,PowerProduct} imply the following analogous result for powers of graphs in any minor-closed class with bounded maximum degree.
%
% \begin{thm}
% \label{PowerMinor}
% For every graph $X$ there exists an integer $c$ such that for all integers $k,\Delta\geq 1$, if $t:= 2k\Delta^{k}(k^3+3k)\binom{k+c\Delta}{c\Delta}-1$ and $G$ is an $X$-minor-free graph with maximum degree $\Delta$, then:
% \begin{compactitem}
% 	\item $G^k$ is contained in $H\boxtimes P$ for some graph $H$ with treewidth $t$ and for some path $P$.
% 	\item $G^k$ has queue-number at most $3\cdot 2^t-2$,
% 	\item $G^k$ has $p$-centered chromatic number at most $(p+1)\binom{p+t}{t}$.
% \end{compactitem}
% \end{thm}





\subsection{Map Graphs}

% \note{PM}{Check David's commit: a0e20aff83d12c74127331854ce9fb75e002efe3}

Map graphs are defined as follows. Start with a graph $G_0$ embedded in a surface of Euler genus $g$, with each face labelled a `nation' or a `lake', where each vertex of $G_0$ is incident with at most $d$ nations. Let $G$ be the graph whose vertices are the nations of $G_0$, where two vertices are adjacent in $G$ if the corresponding faces in $G_0$ share a vertex. Then $G$ is called a \defin{$(g,d)$-map graph}.  A $(0,d)$-map graph is called a (plane) \defin{$d$-map graph}; see \citep{FLS-SODA12,CGP02} for example. The $(g,3)$-map graphs are precisely the graphs of Euler genus at most $g$; see \citep{dujmovic.eppstein.ea:structure}.


% \note{DW}{give a better version of \cref{MapPartition} with $g=0$. }

%So $(g,d)$-map graphs generalise graphs embedded in a surface, and we now assume that $d\geq 4$ for the remainder of this section.

There is a natural drawing of a map graph obtained by positioning each vertex of $G$ inside the corresponding nation and each edge of $G$ as a curve passing through the corresponding vertex of $G_0$. It is easily seen that each edge is in at most $\floor{\frac{d-2}{2}}\ceil{\frac{d-2}{2}}$ crossings; see \citep{dujmovic.eppstein.ea:structure}. Thus $G$ is $(g,\floor{\frac{d-2}{2}}\ceil{\frac{d-2}{2}})$-planar. Also note that \cref{DrawG} with $k=2$ implies that $G$ is $(g, O(d^{2}) )$-planar. \cref{gkPlanarProduct} then establishes a product structure theorem for map graphs, but we get much better bounds by constructing a shortcut system directly.  The following lemma is reminiscent of the characterisation of $(g,d)$-map graphs in terms of the half-square of bipartite graphs \citep{CGP02,dujmovic.eppstein.ea:structure}.

\begin{lem}
\label{MapShortcut}
Every $(g,d)$-map graph $G$ is contained in $G_1^\SS$ for some graph $G_1$ with Euler genus at most $g$ and some $(2,\tfrac12 d(d-3) )$-shortcut system $\SS$ for $G_1$.
\end{lem}

\begin{proof}
Let $G$ be a $(g,d)$-map graph. So there is a graph $G_0$ embedded in a surface of Euler genus $g$, with each face labelled a `nation' or a `lake', where each vertex of $G_0$ is incident with at most $d$ nations. Let $N$ be the set of nations. Then $V(G)=N$ where two vertices are adjacent in $G$ if the corresponding nation faces of $G_0$ share a vertex. Let $G_1$ be the graph with $V(G_1):=V(G_0) \cup N$, where distinct vertices $v,w\in N$ are adjacent in $G_1$ if the boundaries of the corresponding nations have an edge of $G_0$ in common, and $v\in V(G_0)$ and $w\in N$ are adjacent in $G_1$ if $v$ is on the boundary of the nation corresponding to $w$. Observe that $G_1$ embeds in the same surface as $G_0$ with no crossings, and that each vertex in $V(G_0)$ has degree at most $d$ in $G_1$. Consider an edge $vw\in E(G)$. If the nations corresponding to $v$ and $w$ share an edge of $G_0$, then $vw$ is an edge of $G_1$. Otherwise,  $v$ and $w$ have a common neighbour $x$ in $V(G_0)$. In the latter case, let $P_{vw}$ be the path $(v,x,w)$. Let $\SS$ be the set of all such paths $P_{vw}$. Each vertex $x\in V(G_0)$ is the middle vertex on at most $\tfrac12 d(d-3)$  paths in $\SS$. Thus $\SS$ is a $(2,\tfrac12 d(d-3))$-shortcut system for $G_1$, and by construction, $G \subseteq G_1^\SS$.
\end{proof}


\cref{ShortcutProduct,MapShortcut,GenusProduct} imply:

\begin{thm}
\label{gdMapProduct}
For integers $g\geq 0$ and $d\geq 3$, let $\ell:= \max\{2g,3\}(7d^2-21d)$.
Then every $(g,d)$-map graph is contained in $H\boxtimes P \boxtimes K_{\ell}$ for some path $P$ and for some graph $H$ with treewidth $9$,
\end{thm}


\subsection{$k$-Nearest-Neighbour Graphs}
\label{last_example}

In this section, we show that $k$-nearest neighbour graphs of point sets in the plane are $O(k^2)$-planar.  For two points $x,y\in\R^2$, let $d_2(x,y)$ denote the Euclidean distance between $x$ and $y$. The $k$-nearest-neighbour graph of a point set $P\subset\R^2$ is the geometric graph $G$ with vertex set $V(G)=P$, where the edge set is defined as follows. For each point $v\in P$, let $N_k(v)$ be the set of $k$ points in $P$ closest to $v$. Then $vw\in E(G)$ if and only if $w\in N_k(v)$ or $v\in N_k(w)$. (The edges of $G$ are straight-line segments joining their endpoints.) See \citep{ProximityGraphs} for a survey of results on $k$-nearest neighbour graphs and other related proximity graphs.

The following result, which is immediate from \citet[Corollary~4.2.6]{abrego.munroy.ea:on} states that $k$-nearest-neighbour graphs have bounded maximum degree:
\begin{lem}
\label{k-nn-max-degree}
The degree of every vertex in a $k$-nearest-neighbour graph is at most $6k$.
\end{lem}

We make use of the following well-known observation (see for example, \citet[Lemma~2]{bose.morin.ea:routing}):
\begin{obs}
\label{convex}
If $v_0,\ldots,v_3$ are the vertices of a convex quadrilateral in counterclockwise order then there exists at least one $i\in\{0,\ldots,3\}$ such that $\max\{d_2(v_i,v_{i-1}), d_2(v_i,v_{i+1})\} < d_2(v_{i-1},v_{i+1})$, where subscripts are taken modulo 4.
\end{obs}

\begin{lem}
\label{nearest-neighbour}
  Every $k$-nearest-neighbour graph is $O(k^2)$-planar.
\end{lem}

\begin{proof}
  Let $G$ be a $k$-nearest-neighbour graph and consider any edge $vw\in E(G)$.
  Let $xy\in E(G)$ be an edge that crosses $vw$.  Note that $vxwy$ are the vertices of a convex quadrilateral in (without loss of generality) counter-clockwise order. Then we say that
  \begin{compactenum}
    \item $xy$ is of Type~$v$ if $\max\{d_2(v,x), d_2(v,y)\}< d_2(x,y)$;
    \item $xy$ is of Type~$w$ if $\max\{d_2(w,x), d_2(w,y)\}< d_2(x,y)$; or
    \item $xy$ is of Type~C otherwise.
  \end{compactenum}
  If $xy$ is of Type~C, then \cref{convex} implies that $\max\{d_2(x,v),d_2(x,w)\} < d_2(v,w)$ without loss of generality.
   In this case, we call $x$ a Type~C vertex.  We claim that $V(G)$ contains at most $k-1$ Type~C vertices.  Indeed, more than $k-1$ Type~C vertices would contradict the fact that $vw\in E(G)$ since every Type~C vertex is closer to both $v$ and $w$ than $d_2(v,w)$.

  Next observe that, if $xy$ is of Type~$v$, then at least one of $xv$ or $yv$ is in $E(G)$ in which case we call $x$ (respectively $y$) a Type~$v$ vertex.  By \cref{k-nn-max-degree}, there are at most $6k$ Type~$v$ vertices.  Similarly, there are at most $6k$ Type~$w$ vertices.

  Thus, in total, there are at most $13k-1$ Type~$v$, Type~$w$, and Type~C vertices. By \cref{k-nn-max-degree}, each of these vertices is incident with at most $6k$ edges that cross $vw$. Therefore, there are at most $78k^2-6k$ edges of $G$ that cross $vw$.  Since this is true for every edge $vw\in E(G)$, $G$ is $(78k^2-6k)$-planar.
\end{proof}

Note that \cref{nearest-neighbour} is tight up to the leading constant:  Every $k$-nearest neighbour graph on $n\ge k+1$ vertices has at least $kn/2$ edges and at most $kn$ edges.  For $k\ge 7$, the Crossing Lemma~\citep{ajtai.chvatal.ea:crossing-free,leighton:complexity} implies that the total number of crossings is therefore $\Omega(k^3n)$ so that the average number of crossings per edge is $\Omega(k^2)$.

\cref{nearest-neighbour} and \cref{kPlanarProduct} immediately imply


\begin{thm}\label{nn_product_structure}
  Every $k$-nearest neighbour graph is a subgraph of $H\boxtimes P\boxtimes K_{\ell}$ for some graph $H$ of treewidth $O(k^6)$ and for some $\ell\in O(k^4)$.
\end{thm}

\section{Framed Graphs}
\label{FramedSection}

% \note{DW}{Suggested section name ``Graphs Derived From Framed Graphs'' or just``Framed Graphs''}

% Next we consider
%
% %%%%%%%%%%%%%%%%%%%
% \subsection{Framed Graphs}

% \note{PM}{I don't like this structure anymore.  The proof of \cref{kPlanarProduct} ($k$-planar) refers to the innards of the proof of \cref{ShortcutPartition} ($(k,d)$-shortcuts).  The proofs should be close to each other.
%
% Let's consider something like this:
% \begin{compactenum}
%   \item Introduction
%   \item Shortcut systems and $k$-planar graphs
%   \item All graphs obtained from shortcut systems or $k$-planar graphs
%   \item $d$-framed graphs, $1$-planar graphs, and map graphs
%   \item All applications (colouring, labelling, etc).
% \end{compactenum}
% % (Possibly with the order of 2 and 3 reversed.)
% }

% \note{DW}{I suggest we use the language of ``embedded graph'' here}
For any integer $d\ge 3$, an embedded (multi)graph $G$ is \defin{$d$-framed} if it has a biconnected plane spanning (multi)subgraph $G_0$, whose facial cycles each have length at most $d$,  and such that the interior of each edge in $E(G)\setminus E(G_0)$ is contained in the interior of some face of $G_0$. The embedded (multi)graph $G_0$ is called the \defin{frame} of $G$. This definition (for simple graphs) was introduced by \citet{BDGGMR}.

Every $d$-framed multigraph can be described by a $(2,d(d-3)/2)$-shortcut system applied to a plane multigraph by adding a vertex inside each face $F$ of $G_0$ adjacent to the vertices of $F$ and creating a shortcut between each non-adjacent pair of vertices in $V(F)$. Thus \cref{ShortcutProduct} is applicable. However, for framed graphs we prove the following stronger result, where the treewidth bound on $H$ is best possible.\footnote{A multigraph $G$ is \defin{contained} in a graph $X$ if the simple graph underlying $G$ is contained in $X$.}

\begin{thm}\label{d_framed_product_stucture}
For any integer $d\ge 3$, every $d$-framed multigraph is contained in $H\boxtimes P\boxtimes K_{d+3\floor{d/2}-3}$ for some planar graph $H$ with treewidth at most 3 and for some path $P$.
\end{thm}

\cref{d_framed_product_stucture} is used below to obtain product structure theorems for 1-planar multigraphs and for map graphs.  Note that even for simple 1-planar graphs it is essential that we allow frames with parallel edges.

\cref{d_framed_product_stucture} is a consequence of the following technical lemma, which is an extension of the analogous result for plane graphs~\cite{DJMMUW20}. The following definition is a convenient way to establish the planarity of $H$ in \cref{d_framed_product_stucture}:  A $T$-decomposition of a graph $G$ is \defin{$t$-simple} if each of its bags has size at most $t+1$ and, for each $t$-element subset $\{v_1,\ldots,v_t\}\subseteq V(G)$, at most two bags contain $v_1,\ldots,v_t$.  Finally, a \defin{BFS spanning forest} $T$ of $G$ rooted at a set $V_0\subseteq V(G)$ is a spanning forest of $G$ with the property that for each $v\in V(G)$,  \[\min\{\dist_G(v,w):w\in V_0\}=\min\{\dist_T(v,w):w\in V_0\}.\]

% \note{DW}{I suggest $G'$ be changed to $G_0$ throughout (since $G'$ feels like a graph derived from $G$, which it is not). Similarly, replace $N'$ by $N_0$.}
%
% \note{PM}{I would also like to change $G'$ to $G_0$ and $N'$ to $N_0$.  One issue is holding me back, though: We recurse on cycles $F_1,\ldots,F_m$ that define subgraphs $N_1,\ldots,N_m$ of $G$ and subgraphs $N_1',\ldots,N_m'$ of $G'$.  Will these latter subgraphs becomes $N_{0,1},\ldots,N_{0,m}$?}
%
% \note{DW}{Some suggestions: Replace $G'$ by $F$ (the frame), replace $F_0$ by $C$ (the outer-cycle), replace $F_1,\dots,F_m$ by $C_1,\dots,C_m$. Use $F_1,\dots,F_m$ for the subgraph of $F$ inside $C_i$. Use $G_i$ for the subgraph of $G$ inside $C_i$.}

\begin{lem}
	\label{induction} The setup:
	\begin{compactenum}
		\item Let $G$ be a $d$-framed multigraph and let $G_0$ be a frame of $G$.
		\item Let $F_0$ be the outer face of $G_0$ and let $T$ be a BFS spanning forest of $G_0$ rooted at $V(F_0)$.
		\item For every integer $j\ge 0$, let $L_j=\{v\in V(G):\dist_T(v,V(F_0))=j\}$.
		\item Let $F$ be a cycle in $G_0$ whose vertices are partitioned into $k\le 3$ sets $P_1,\ldots,P_k$ such that for each $i\in\{1,\ldots,k\}$,
		\begin{compactenum}
			\item $F[P_i]$ is connected; and
			\item $P_i$ has a partition $\{X_i,Y_i\}$ where $|X_i|\le d-3$ and $|Y_i\cap L_j| \le 3$ for each $j\ge 0$.
		\end{compactenum}
		\item Let $N$ and $N_0$ be the subgraphs of $G$ and $G_0$ consisting only of those edges and vertices contained in the interior or boundary of $F$.
	\end{compactenum}
	Then $N$ has an $H$-partition $\PP=\{S_x: x\in V(H)\}$ such that:
	\begin{compactenum}[(i)]
		\item for each $i\in\{1,\ldots,k\}$, there exists $x_i\in V(H)$ such that $P_i=S_{x_i}$;
		\item For each $x\in V(H)$, $S_x$ has a partition $\{X_x,Y_x\}$ where $|X_x|\le d-3$ and $|Y_x\cap L_j|\le 3$ for each $j\ge 0$;
		\item $H$ has a $3$-simple tree-decomposition $\mathcal{T}$, and if $k=3$ then $\{x_1,x_2,x_3\}$ is contained in exactly one bag of $\mathcal{T}$.
	\end{compactenum}
%\note{PM}{I've changed my mind about this notation change.  We end up applying the lemma inductively on the faces $F_1,\ldots,F_m$ of $M$ (which are cycles in $G$).  It makes sense that the original cycle in $G$ is called $F$.}
%\note{DW}{okay}
%\begin{lem}
%  \label{induction} The setup:
%  \begin{compactenum}
%    \item Let $G$ be a $d$-framed multigraph and let $G'$ be a frame of $G$.
%    \item Let $F_0$ be the outer face of $G'$ and let $T$ be a BFS spanning forest of $G'$ rooted at $V(F_0)$.
%    \item For every integer $j\ge 0$, let $L_j=\{v\in V(G):\dist_T(v,V(F_0))=j\}$.
%    \item Let $F$ be a cycle in $G'$ whose vertices are partitioned into $k\le 3$ sets $P_1,\ldots,P_k$ such that for each $i\in\{1,\ldots,k\}$,
%    \begin{compactenum}
%      \item $F[P_i]$ is connected; and
%      \item $P_i=Y_i\cup X_i$ where $|X_i|\le d-3$ and $|Y_i\cap L_j| \le 3$ for each $j\ge 0$.
%    \end{compactenum}
%    \item Let $N$ and $N'$ be the subgraphs of $G$ and $G'$ consisting only of those edges and vertices contained in the closure of $F$.
%  \end{compactenum}
%  Then $N$ has an $H$-partition $\PP=\{S_x: x\in V(H)\}$ such that:
%  \begin{compactenum}[(i)]
%    \item for each $i\in\{1,\ldots,k\}$, there exists $x_i\in V(H)$ such that $P_i=S_{x_i}$;
%    \item For each $x\in V(H)$, $S_x=Y_x\cup X_x$ where $|X_x|\le d-3$ and $|Y_x\cap L_j|\le 3$ for each $j\ge 0$;
%    \item $H$ has a $3$-simple tree-decomposition $\mathcal{T}$ and, if $k=3$, then $\{x_1,x_2,x_3\}$ is contained in exactly one bag of $\mathcal{T}$.
%  \end{compactenum}
\end{lem}

\begin{proof}
	This proof is similar to the proof of Lemma~13 by \citet{DJMMUW20}, but is complicated by several factors.  In particular, the reader should keep in mind that $G$ and $G_0$ are multigraphs and that the cycle $F$ may consist of two vertices and two (parallel) edges.

  We proceed by induction on $(x,y)$ where $x$ is the number of inner vertices of $N_0$ and $y$ is the number of inner faces of $N_0$.  The induction base case occurs when $N_0$ has no inner vertices (that is, $V(N_0)=V(F)$). In this case we can simply take
	% \note{DW}{Suggested revision, ``In the base case, $N_0$ has only one inner face, and we can simply take ...}
  % \note{PM}{I debated what to put here since, even when $N'$ has only one face, the rules below work for picking $\tau$.  The \emph{real} base case occurs when $N'$ has zero faces, but that's not something that ever happens since we only ever apply induction on the inner faces of $M$ (except $\tau$).  I settled on this ``base case'' since it's clear that, in this case, the partition $\mathcal{P}$ is already decided.}
  % \note{PM}{The longwinded version is that induction is on $(x,y)$ where $x$ is the number of inner vertices of $N_0$ and $y$ is the number of inner faces of $N_0$.  The base cases are $(0,y)$ for any $y$.  Unfortunately, we can't just induct on $x$ because it doesn't always decrease.} \note{DW}{We need to make it clear what we are doing induction on.
% I suggest we write, ``we proceed by induction on $(x,y)$ where $x$ is the number of inner vertices of $N_0$ and $y$ is the number of inner faces of $N_0$. ''}
	$\PP:=\{P_1,\ldots,P_k\}$ and verify that the preconditions of the lemma (the `setup') ensure that $\PP$ satisfies the requirements (i)--(iii):
	\begin{compactenum}[(i)]
		\item By definition, each of $P_1,\ldots,P_k$ is a part of $\mathcal{P}$.
		\item Assumption 4(b) ensures, for each $x\in V(H)$, the existence of $X_x$ and $Y_x$ such that $|X_x|\le d-3$ and $|Y_x\cap L_j|\le 3$ for each integer $j\ge 0$.
		\item The trivial tree-decomposition of $H$ that has a single bag containing $\{x_1,\ldots,x_k\}$ is $3$-simple and, if $k=3$, has exactly one bag that contains $\{x_1,\ldots,x_3\}$.
	\end{compactenum}

% \note{DW}{Now assume that $N'$ has at least two internal faces.}

We now move onto the case in which $N_0$ contains at least one inner vertex.
Our $H$-partition $\mathcal{P}$ will include $P_1$, $P_2$ , $P_3$ and a (possibly empty) part $S$ that will be determined by three vertices $v_1v_2v_3$ that belong to a common inner face $\tau$ of $N_0$.  We first explain how $v_1$, $v_2$ and $v_3$ are chosen and then explain how these are used to define $S$.  There is one case to consider for each possible value of $k$ (see \cref{boring_figure}):

\begin{compactenum}
		\item If $k=1$ then let $\tau:=v_1,\ldots,v_p$ be an inner face of $N_0$ that contains at least one edge $v_1v_2$ of $F$ on its boundary.

		\item If $k= 2$ then let $\tau:=v_1,\ldots,v_p$ be an inner face of $N_0$ that contains at least one edge $v_1v_2$ of $F$ on its boundary and such that $v_1\in P_1$ and $v_2\in P_2$.

		\item If $k=3$ then assign each vertex $v$ of $N_0$ a colour $\alpha(v)\in\{1,\ldots,3\}$ as follows:  Let $w$ be the first vertex of $V(F)$ encountered on the path in $T$ from $v$ to the appropriate vertex of $F_0$.  (This implies that $w=v$ when $v$ is a vertex of $F$.)  Since $w\in V(F)$, $w\in P_i$ for some $i\in\{1,\ldots,k\}$ and we define $\alpha(v):=i$.  Now, for each $d'\in\{4,\ldots,d\}$ and each $d'$-sided inner face $F$ of $N_0$, add $d'-3$ edges inside of $F$ to split it into $3$-sided faces.  This yields a plane supergraph $N_0'$ of $N_0$ in which each inner face has exactly three edges on its boundary. By Sperner's Lemma (see \citep{Proofs4}) there exists an inner face $\tau':=v_1,v_2,v_3$ of $N_0'$ such that $\alpha(v_i)=i$ for each $i\in\{1,2,3\}$.  Let $\tau:=v_1,\ldots,v_p$ be  the facial cycle of $N_0$ that contains $\tau'$.
	\end{compactenum}

	%  \todo[inline]{%
	%    DW:Does this work for multigraph near-triangulations? \newline
	%    PM: Yes, it only requires that the inner faces have three sides and that the outer face is bounded by a simple cycle.  The proof goes through without change.  Look at the subgraph $X$ of the dual that contains only the edges dual to red/blue edges.  In our setting, the outer face has degree $1$ in $X$, which is odd.  Any bichromatic inner face has degree $0$ or degree $2$ in $X$.  There has to be another vertex of odd degree somewhere, which can only come from a trichromatic inner face.\newline
	%    DW: great. This is basically the same as the proof of the Hex Lemma.
	%  }


\begin{figure}[!b]
	\begin{center}
		\begin{tabular}{c@{}c@{}c}
			\includegraphics{figs/zoomba-1} &
			\includegraphics{figs/zoomba-2} &
			\includegraphics{figs/zoomba-3} \\
			\multicolumn{3}{c}{{\color{brew8}\raisebox{-3pt}{\rule{12pt}{12pt}}} $S$\qquad {\color{brew2}\raisebox{-3pt}{\rule{12pt}{12pt}}} $X$} \\
			$k=1$ & $k=2$ & $k=3$
		\end{tabular}
	\end{center}
	\caption{Three cases when choosing the face $\tau$ of $N_0$.}
	\label{boring_figure}
\end{figure}

	We now define the vertices in $S$.  For each $i\in\{1,2,3\}$, let $Q_i$ be the shortest path, in $T$, from $v_i$ to a vertex in $V(F)$.  Let $\overline{S}$ denote the subgraph of $N_0$ consisting of vertices and edges of $Q_1$, $Q_2$, $Q_3$, and $\tau$.
	Let $S:=V(\overline{S})\setminus V(F)$, let $Y:=(V(Q_1)\cup V(Q_2)\cup V(Q_3))\setminus V(F)$, and let $X:=S\setminus Y$. Observe that $|V(Q_i)\cap L_j|\le 1$ 	for each $i\in\{1,2,3\}$ and each $j\ge 0$.  Therefore $|Y\cap L_j|\le 3$ for each $j\ge 0$. Furthermore, $X\subseteq V(\tau)\setminus\{x_1,x_2,x_3\}$, so $|X|\le p-3\le d-3$. Therefore, using $S:=Y\cup X$ as a part in the partition $\PP$ satisfies condition~(ii).

	% We claim that, for each integer $i\ge 0$, $|V(\overline{Y}^+)\cap L_i|\le 15$.  First observe that, since $Q_1,Q_2,Q_3$ are each vertical paths in $T$,  $\overline{Y}$ contains at most three vertices of $L_i$, each incident on at most two edges of $\overline{Y}$.  Since $\dist_{G'}(v,w)\le 2$ for each $vw\in E(G)$, any vertex $x\in V(\overline{Y}^+)\setminus V(\overline{Y})\cap L_i$, is incident to an edge $xy\in E(G)$ that crosses one of the at most six edges in $\overline{Y}$ having an endpoint in $L_i$.  These at most six edges have at most 12 endpoints.  Therefore $|V(\overline{Y}^+)\setminus V(\overline{Y})\cap L_i|\le 6\times 2=12$, so $|V(\overline{Y}^+)\cap L_i|\le 12+3=15$.


Let $M$ denote the subgraph of $N_0$ containing the edges and vertices of $\overline{S}$ and the edges and vertices of $F$. By our choice of $\tau$, the graph $M$ is $2$-connected.
  % Since $N'\subseteq G'$ contains only edges of $N\subseteq G$ that are not crossed, each component of $N-V(M)$ is contained in a single face of $M$.
	Let $F_1,\ldots,F_m,\tau$ be the inner faces of $M$. By our choice of $v_1v_2v_3$, the vertices of each $F_i$ can be partitioned into at most three sets $P_{i,1}$, $P_{i,2}$, and $P_{i,3}$ where $P_{i,1}\subset Y$, $P_{i,2}\subseteq P_b$, and $P_{i,3}\subseteq P_c$ for some $b,c\in\{1,\ldots,k\}$.

	In the following, in order to avoid introducing even more notation or constantly adding the modifier ``when it exists'', we will assume that $V(F_i)\cap S\neq\emptyset$ and that there are exactly two distinct $b,c\in\{1,\ldots,k\}$ such that $V(F_i)\cap P_b\neq\emptyset$ and $V(F_i)\cap P_c\neq\emptyset$.  When this assumption does not hold, exactly the same reasoning can be used, while omitting one or more of $P_{i,1}$, $P_{i,2}$ or $P_{i,3}$.

	Since $M$ is $2$-connected and $\overline{S}$ is connected, $F_i[P_{i,j}]$ is connected, for each $j\in\{1,2,3\}$.  Let $N_i$ and $N_{0,i}$ be the subgraphs of $N$ and $N_0$ contained in the interior or boundary of $F_i$.  Every inner vertex of $N_{0,i}$ is an inner vertex of $N_{0}$.  Every inner face of $N_{0,i}$ is an inner face of $N_0$ and $\tau$ is not an inner face of $N_{0,i}$.  Therefore $N_{0,i}$ has no more inner vertices than $N_0$ and has fewer inner faces than $N_{0}$.  Therefore, we can apply induction using the cycle $F_i$ and the partition $P_{i,1},P_{i,2},P_{i,3}$ of $V(F_i)$.  The result of the induction is a $H_i$-partition $\mathcal{P}_i$ of $N_i$ satisfying (i)--(iii) above. We now define our partition
	\[
	\mathcal{P}:=\{Y, P_1,\ldots,P_k\} \cup \bigcup_{i=1}^m \{P\in\mathcal{P}_i: P\cap V(M)=\emptyset\} \enspace .
	\]
	It is straightforward to verify that $\mathcal{P}$ is indeed a partition of $V(N)$ and, by definition this is an $H$-partition of $N$ for the graph $H:=N/\mathcal{P}$.  It remains to verify that $\mathcal{P}$ satisfies (i)--(iii).

	By construction $\mathcal{P}$ contains parts $P_1,\ldots,P_k$, so $\mathcal{P}$ satisfies (i).  By Assumption~4(b) each of the parts $P_1,P_2,P_3$ satisfies condition~(ii).  We have already argued that the part $S$ satisfies condition~(ii).  The inductive hypothesis ensures that each of the remaining parts satisfies condition~(ii).   Therefore $\mathcal{P}$ satisfies (ii).

	It remains to construct a tree-decomposition $\mathcal{T}:=(B_y:y\in V(T_0))$ of $H$ that satisfies (iii). Let $x_0$ be the vertex of $H$ obtained from contracting $S$ and, for each $j\in\{1,\ldots,k\}$, let $x_j$ denote the vertex of $H$ obtained from contracting $P_j$.  The tree $T_0$ has a node $y_0$ with bag $B_{y_0}:=\{x_0,\ldots,x_k\}$. The inductive hypothesis ensures that, for each $i\in\{1,\ldots,m\}$,  $H_i:=N_i/\mathcal{P}_i$ has a tree-decomposition $\mathcal{T}_i:=(B_z:z\in V(T_i))$ that satisfies (iii).  Let $a_i,b_i,c_i\in V(H_i)$ be the vertices obtained by contracting $P_{i,1}$, $P_{i,2}$, and $P_{i,3}$, respectively. Since $a_i$, $b_i$, $c_i$ form a clique in $H_i$, there exists a bag $B_{z_i}$ in $\mathcal{T}_i$ that contains $\{a_i,b_i,c_i\}$.

	Recall that $P_{i,1}:=V(N_i)\cap S$, $P_{i,2}:=V(N_i)\cap P_b$ and $P_{i,3}:=V(N_i)\cap P_c$ for some $b,c\in\{1,\ldots,k\}$.  For each bag $B_z$, $z\in V(T_i)$, replace each occurrence of $a_i$ by $x_0$, replace each occurrence of $b_i$ by $x_b$ and replace each occurrence of $c_i$ by $x_c$.  These replacements do not increase the size of any bag and they result in a tree-decomposition of the graph obtained from $H_i$ by renaming the vertices $a_i$, $b_i$, and $c_i$ with the names $x_0$, $x_a$, and $x_b$. Now add an edge that joins $y_0$ to $z_i$ so that $T_i$ becomes a subtree of $T_0$ that is adjacent to $y_0$.  We perform this operation for each $i\in\{1,\ldots,m\}$ to obtain our final tree-decomposition $\mathcal{T}:=(B_y:y\in V(T_0))$.

  It is now straightforward to verify that $\mathcal{T}$ is a tree-decomposition of $H_0:=N_0/\mathcal{P}$ in which each bag has size at most $4$.    To see that $\mathcal{T}$ is also a tree-decomposition of $H:=N/\mathcal{P}$, consider any edge $vw\in E(N)\setminus E(N_0)$ with $v\in P_\alpha$ and $w\in P_\beta$.  We must verify that some bag of $\mathcal{T}$ contains $\alpha$ and $\beta$.  If both $v$ and $w$ are in $V(M)$, then $\alpha=x_a$ and $\beta=x_b$ for some $a,b\in\{0,\ldots,3\}$, in which case $B_{y_0}=\{x_0,\ldots,x_k\}$ contains $\alpha$ and $\beta$.  If neither $v$ nor $w$ are in $V(M)$ then, since no edge of $N\subseteq G$ crosses an edge of $M\subseteq G_0$, both $v$ and $w$ are contained in the interior of the same face $F_i$ of $M$. So, by the inductive hypothesis there exists a bag of $\mathcal{T}_i$ that contains $\alpha$ and $\beta$ and this becomes a bag of $\mathcal{T}$ that contains $\alpha$ and $\beta$.  Finally, if $v\in V(M)$ and $w\not\in V(M)$, then $v\in P_{i,j}$ for some $j\in\{1,2,3\}$ and $w$ is in the interior of some face $F_i$ of $M$. By induction, some bag of $\mathcal{T}_i$ contains $\beta$ and $\alpha'$, where $\alpha'$ is the vertex of $H_i$ obtained by contracting $P_{i,j}$.  However, before $T_i$ is attached to $T_0$ each occurence of $\alpha'$ in $\mathcal{T}_i$ is replaced by $\alpha$, so some bag of $\mathcal{T}$ contains both $\alpha$ and $\beta$. Therefore $\mathcal{T}$ is a tree-decomposition of $H$.

  Next we verify that, when $k=3$ exactly one bag of $\mathcal{T}$ contains $\{x_1,\ldots,x_3\}$.  The bag $B_{y_0}$ contains $x_1,\ldots,x_k$. When $k=3$, each neighbour of $z_i$ of $y_0$ has a bag that contains at most two of $\{x_1,x_2,x_3\}$.  Therefore, when $k=3$, $B_y$ is the unique bag in $\mathcal{T}$ that contains $\{x_1,x_2,x_3\}$.

It remains to show that any triple of vertices of $H$ is contained in at most two bags of $\mathcal{T}$. By the inductive hypothesis, the only triples we need to be concerned about are the $3$-element subsets of  $\{x_0,x_1,\ldots,x_k\}$. The case $k=1$ is trivial since no triples can be formed by $\{x_0,x_1\}$.  When $k=2$, $B_{y_0}=\{x_0,x_1,x_2\}$.  In this case, the fact that $\tau$ contains an edge $v_1v_2$ of $F$ with $v_1\in P_{x_1}$ and $v_2\in P_{x_2}$ ensures that there is at most one face $F_i$ of $M$ that has vertices of $S$, $P_1$, and $P_2$ on its boundary  (see the middle part of \cref{boring_figure}).  By the inductive hypothesis (applied to $H_i$) the bag $B_{z_i}$ is the only bag of $\mathcal{T}_i$ that contains $\{x_0,x_1,x_2\}$.  Therefore there are at most two bags, $B_y$ and $B_{z_i}$ of $\mathcal{T}$ that contain $\{x_0,x_1,x_2\}$.  This establishes (iii) for $k\in\{1,2\}$ so, from this point on we assume that $k=3$.

	We claim that, for distinct $a,b\in \{1,2,3\}$ there is at most one face $F_i$ such that $V(F_i)\cap S_{x_j}\neq\emptyset$ for each $j\in\{0,a,b\}$.  Suppose to the contrary that there are two such faces $F_{i_1}$ and $F_{i_2}$.  Create a graph $M^+$ from $M$ by placing a vertex $v_{i_b}$ inside $F_{i_b}$ and adjacent to each vertex in $V(F_{i_b})$ for each $b\in\{1,2\}$.  Since $M$ is planar and $F_{i_1}$ and $F_{i_2}$ are distinct faces of $M$, $M^+$ is planar.  However, contracting each of $S_{x_0},\ldots,S_{x_3}$ in $M^+$ produces a graph isomorphic to the complete bipartite graph $K_{3,3}$, a contradiction.  (In this contracted graph, $x_0,x_a,x_b$ are on one side of the bipartition and $v_{i_a}$, $v_{i_b}$, and the node in $\{x_1,x_2,x_3\}\setminus\{x_a,x_b\}$ is on the other side.)

	We have already established that, when $k=3$, the only bag that contains $\{x_1,x_2,x_3\}$ is $B_{y_0}$, so we need only consider triples $\{x_0,x_a,x_b\}$ for distinct $a,b\in\{1,2,3\}$. By the preceding claim, there is at most one face $F_i$ with vertices of $S$, $P_a$, and $P_b$ on its boundary.  By the inductive hypothesis, the only other bag in $\mathcal{T}$ that contains $\{x_0,x_a,x_b\}$ is the bag $B_{z_i}$.  This completes the proof.
\end{proof}

% \todo[inline]{Alternative to the final three paragraphs in the preceding proof:
%
% To see that $H$ is planar, consider a plane multigraph $G''$ obtained by adding stars in some of the faces of $G'$, as follows:  When creating the part $S$, add a vertex $v_\tau$ to the interior of $\tau$ that is adjacent to each vertex of $\tau$ and place $v_\tau$ in $S$.  The graph $G''\supseteq G'$ obtained this way is clearly planar.  Furthermore, for each part $P\in\mathcal{P}$, $G''[P]$ is connected.  Since planarity is preserved under edge contraction, it follows that $G''/\mathcal{P}$ is planar.  Now argue that $G/\mathcal{P}=G''/\mathcal{P}$\ldots
% }
%
% \note{DW}{The proof via 3-simplicity is shorter, so just keep it as is.}

Using \cref{induction}, the proof of \cref{d_framed_product_stucture} is now straightforward.

% \note{PM}{Annoying detail:  When we allow framed graphs to have crossings in the outer face, we can't guarantee that $G'$ has any face smaller than $d$. This makes it awkward to start the induction using \cref{induction}. Possible solutions:
% \begin{compactenum}
%   \item Disallow the addition of edges in the outer face of a frame.
%   \item Modify \cref{induction} so that $T$ is a BFS forest rooted at the set of vertices on the outer face of $G'$. Then we can start the induction with $k=1$.
% \end{compactenum}
% I prefer the second option.  Thoughts?
% }

\begin{proof}[Proof of \cref{d_framed_product_stucture}]
	Let $G$ be a $d$-framed multigraph with frame $G_0$ having outer face $F_0$
	% \note{DW}{Replace ``outer face'' by ``outer-cycle'' since a face is a defined to be a portion of the plane. }.
  %\note{PM}{I added some conventions around the definition of $\Sigma$-embedded graph so that $V(F)$ is defined for a face $F$ in a non-crossing embedded graph.}
  Let $T$ be a BFS forest of $G_0$ rooted at $V(F_0)$ and let $\langle L_0,L_1,\ldots\rangle$ be the resulting BFS layering.  Let $P_1:= V(F_0)$, let $Y_1$ be any subset of $\min\{|P_1|,3\}$ vertices in $P_1$ and let $X_1:=P_1\setminus Y_1$.  Now apply \cref{induction} on $G$, $G_0$, $T$, $F=F_0$, $k=1$, and $P_1=Y_1\cup X_1$.
    This gives an $H$-partition $\mathcal{P}:=\{S_x:x\in V(H)\}$ of $G$ in which $H$ has a $3$-simple tree-decomposition.  Since any graph having a $3$-simple tree-decomposition is planar \cite{knauer.ueckerdt:simple,kratochvil.vaner:note,elmallah.colbourn:on}, $H$ is planar.

	Define $\mathcal{L}=\langle L_0',L_1'\ldots\rangle$ where $L_i'=L_{\lfloor d/2\rfloor i}\cup \cdots \cup L_{\lfloor d/2\rfloor(i+1)-1}$ for each integer $i\ge 0$. This is a layering of $G$ since $\dist_{G_0}(v,w)\le \floor{ \frac{d}{2} }$ for each edge $vw\in E(G)$.  Since $|L_j\cap Y_x|\le 3$ for each integer $j\ge 0$, $|L_i'\cap Y_x|\le 3\floor{\frac{d}{2}}$ and therefore $|L_i'\cap S_x|= |L_i'\cap Y_x|+ |L_i'\cap X_x|\le \floor{ \frac{d}{2}} + d -3$ for each integer $i\ge 0$ and each $x\in V(H)$. The result now follows from \cref{PartitionProduct}.
\end{proof}


\subsection{\boldmath $1$-Planar Graphs}
\label{sec-1-planar}

% $1$-planar graphs are of particular interest to the graph drawing community~\citep{kobourov.liotta.ea:annotated}. In this case, we obtain better constants and an additional property (planarity) of $H$.

%(\htmladdnormallink{https://arxiv.org/abs/1703.02261}{arXiv:1703.02261}) has 143 entries, and even this is now four years out of date.
%
%   An important aspect of our result for $1$-planar graphs is that graph $H$ in the product is planar and has treewidth $3$, i.e., it has simple treewidth $3$.  In a more recent application of product structure \citet{BDJM} show that such graphs have $\ell$-vertex rankings using $O(\log n/\log\log\log n)$ colours, and this is tight. This result requires that $H$ have simple treewidth $3$, even treewidth $3$ is not sufficient.
% }



%
% \note{PM}{Use restateable here...}
%
% \begin{thm}
% \label{1-planar}
% Every 1-planar graph is contained in $H\boxtimes P\boxtimes K_7$ for some planar graph $H$ with treewidth at most 3 and for some path $P$.
% \end{thm}

%In \cref{1_planar_is_4_framed}, below, we show that every $1$-planar graph is a subgraph of a $4$-framed multigraph. This makes \cref{1-planar} an immediate consequence of the following theorem, whose proof is the subject of the rest of this section.


\cref{1_planar_product} is an immediate consequence of \cref{d_framed_product_stucture} and the next lemma. Similar results appear in earlier works \cite{CGP06,BDGGMR,Brandenburg19,Brandenburg20} but we state the precise lemma and provide a proof here for the sake of completeness.

\begin{lem}\label{1_planar_is_4_framed}
A multigraph $G$ is $1$-planar if and only if it is contained in a $4$-framed multigraph.
\end{lem}

% \note{DW}{Write  ``A multigraph $G$ is $1$-planar if and only if ...''}

%Before proving \cref{1_planar_is_4_framed}, we remark that the lemma seems not to be true if we replace `multigraph' with `graph'.  We are only aware of two previous results on the relationship between $1$-planar graphs and $d$-framed graphs \cite{BDGGMR}:
%\begin{inparaenum}[(i)]
%  \item Every $3$-connected $1$-planar graph is contained in some $4$-framed graph.
%  \item Every $1$-planar graph is contained in some $8$-framed graph.
%\end{inparaenum}
%Neither of these results allows a proof of \cref{1-planar} without sacrificing generality or accepting a larger constant.

\begin{proof}
Let $G$ be a $1$-plane multigraph. We may assume that no two edges incident to a common vertex of $G$ cross, since such a crossing can be removed by a local modification to obtain an isomorphic 1-plane graph in which the two edges do not cross\footnote{While this is true for 1-plane graphs it is not true for $k$-plane graphs with $k\ge 3$; the uncrossing operation can increase the number of crossings on a particular edge from $k$ to $2(k-1)$.}.
  % We say that a vertex $v\in V(G)$ appears on a face $F$ if $v$ is contained in the closure of $F$.
We make $G$ into an \defin{edge-maximal} $1$-plane multigraph, by repeating the following operation:  If two vertices $v$ and $w$ of $G$ appear on a common face $F$ and there is no edge $vw\in E(G)$ contained in the boundary of $F$, then add the edge $vw$, embedded in the face $F$.  This may introduce parallel edges into $G$, but no two such edges  appear on the boundary of a common face.

\begin{figure}[h]
  \begin{center}
    \begin{tabular}{c@{\hspace{1cm}}c}
        \includegraphics{figs/one_planar_example-3} &
        \includegraphics{figs/one_planar_example-4} \\
        $G$ & $G_0$
     \end{tabular}
    \end{center}
    \caption{The graph $G_0$ obtained by removing pairs of crossing edges from $G$ is a plane multigraph whose faces all have three or four sides.}
    \label{one_planar_example}
\end{figure}

Refer to \cref{one_planar_example}.
To understand the structure of $G$, it is helpful to consider the plane graph $G'$ obtained by replacing each pair of edges $vw$ and $xy$ that cross at $p$ with four edges $vp$, $wp$, $xp$, and $yp$ meeting at a newly added \defin{dummy vertex} $p$. Let $F$ be a face of $G_0$ and let $W:=w_0,\ldots,w_{r-1}$ be the facial walk around $F$. In the following paragraphs, subscripts on the vertices in $W$ are implicitly taken modulo $r$. By edge-maximality, if $W$ contains only non-dummy vertices, then it contains exactly three vertices and $F$ is bounded by three edges of $G$.

If $W$ contains a dummy vertex $w_i$, then neither $w_{i-1}$ nor $w_{i+1}$ is a dummy vertex since if $w_{i-1}$ (respectively $w_{i+1}$) were a dummy vertex then the edge of $G$ that contains $w_{i-1}w_i$ (respectively $w_iw_{i+1}$) would be involved in at least two crossings.  Therefore each dummy vertex $w_i$ of $W$ is flanked by two non-dummy vertices $w_{i-1},w_{i+1}\in V(G)$.  Since $w_i$ has degree $4>1$ in $G'$ and no two edges incident to a common vertex cross each other, $w_{i-1}\neq w_{i+1}$.  Since $G$ is edge-maximal the edge $w_{i-1}w_{i+1}$ is an edge of $G$ on the boundary of $F$.  Therefore, if $F$ has a dummy vertex $w_i$ in its facial walk $W$, then $W=w_{i-1},w_i,w_{i+1}$ and $F$ is bounded by three edges of $G'$, each of which is contained in a different edge of $G$.

% Therefore each face of $G_0$ is a $3$-cycle that contains at most one dummy vertex.  This implies that each face of $G$ is one of two types:
% \begin{inparaenum}[(i)]
%   \item bounded by three edges of $G$, none of which are crossed; or
%   \item bounded by one edge of $G$ that is not crossed plus portions of two edges of $G$ that cross each other.
% \end{inparaenum}

Consider some pair of edges $vw$ and $xy$ that cross at point $p$, which is a degree-4 vertex of $G'$.  Since no pair of edges incident to a common vertex cross each other, $v$, $w$, $x$, and $y$ are all distinct.  There are four faces $F_1,\ldots,F_4$ of $G'$ with $p$ on their boundary and, from the preceding discussion each of $vx$, $xw$, $wy$, and $yv$ is an uncrossed edge of $G$.  Therefore, by removing each pair of crossing edges from $G$ we obtain a plane multigraph $G_0$ each of whose faces is bounded by three or four edges.  Therefore $G$ is a $4$-framed graph with frame $G_0$.
% Since, for any pair of crossing edges $\{vw,xy\}$ that we remove this way, $vxwy$ is a cycle in $G'$, $\dist_{G'}(v,w)\le 2$ and $\dist_{G'}(x,y)\le 2$.
% It follows immediately that, for any pair of vertices $a,b\in V(G)$, $\dist_{G'}(a,b)\le 2\dist_G(a,b)$.
\end{proof}

\subsection{Plane Map Graphs}

% \note{PM}{Check David's commit: a0e20aff83d12c74127331854ce9fb75e002efe3}

% Map graphs are defined as follows. Start with a graph $G_0$ embedded in a surface of Euler genus $g$, with each face labelled a `nation' or a `lake', where each vertex of $G_0$ is incident with at most $d$ nations. Let $G$ be the graph whose vertices are the nations of $G_0$, where two vertices are adjacent in $G$ if the corresponding faces in $G_0$ share a vertex. Then $G$ is called a \defin{$(g,d)$-map graph}.  A $(0,d)$-map graph is called a (plane) \defin{$d$-map graph}; see \citep{FLS-SODA12,CGP02} for example. The $(g,3)$-map graphs are precisely the graphs of Euler genus at most $g$; see \citep{dujmovic.eppstein.ea:structure}.
%
% \note{DW}{This subsection needs a clean-up; define $(g,d)$-framed graph.}

In addition to $1$-planar graphs, framed graphs can be used to obtain the following improved product structure theorem for (plane) $d$-map graphs.

\begin{thm}
\label{dMapProduct}
Every $d$-map graph is contained in $H \boxtimes P \boxtimes K_{ d + 3\floor{d/2} -3 }$ for some planar graph $H$ with $\tw(H) \leq 3$ and for some path $P$.
\end{thm}

\cref{dMapProduct} is an immediate consequence of \cref{d_framed_product_stucture} and the next lemma. Similar results to \cref{MewMapGraphLemma} appear in earlier works \cite{CGP06,BDGGMR,Brandenburg19,Brandenburg20} but we state the precise lemma and provide a proof here for the sake of completeness.


%\note{PM}{Got rid of genus-$g$ in the following. We use shortcuts for $(g,d)$-map graphs and now we don't have to define $(g,d)$-frame.}

\begin{lem}
\label{MewMapGraphLemma}
For every %all
integer %s $g\geq 0$ and
 $d\geq 3$, every $d$-map graph is a spanning subgraph of a $d$-framed multigraph.
\end{lem}


% \note{PM}{Suggested new Proof}
% \note{DW}{I am happy with the new proof, once we address the issues.}

\begin{proof}
  Let $G_0$ be a graph embedded in the plane, with each face labelled a nation or a face, and where each vertex of $G_0$ is incident with at most $d$ nations and let $G$ be the corresponding map graph.  Let $G_0^*$ be the dual graph of $G_0$. So the vertices of $G_0^*$ correspond to faces of $G_0$, and each vertex of $G_0^*$ is a nation vertex or a lake vertex.

  Let $x$ be a vertex of $G_0$, let $F_x$ be the corresponding face of $G_0^*$, and let $v_1,\ldots,v_s$ be the facial cycle of $F_x$.  Let $C:=w_1,\ldots,w_r$ be the subsequence of $v_1,\ldots,v_s$ consisting of only the nation vertices.  Since $x$ is incident to at most $d$ nations, $r\le d$.
  Call $C_x$ the \defin{nation cycle} of $F_x$. 
  
  \note{DW}{What if there are 1 or 2 nations incident to $x$? Is $C_x$ a cycle? Do we need parallel edges here?}
  
  \note{PM}{If $r=1$ then the ``nation cycle'' has no edges, so this doesn't contribute any edges to $G_1$, which is the right thing to do.}

  \note{PM}{If $r=2$ then the ``nation cycle'' has one edge, so this contributes one edge to $G_1$ (which is needed, because the corresponding nations are adjacent, but no other edges are needed.)}
  
  Let $G_1$ be a plane supergraph of $G_0^*$ obtained by adding each nation cycle of each face $F$ of $G_0$ and then triangulating each face with more than $d$ vertices on its boundary.  Each nation cycle is now a face of $G_1$ and no face of $G_1$ has more than $d$ vertices on its boundary. Let $\widehat{G}_1$ be the $d$-framed graph whose frame is $G_1$.

By definition, $V(G)=V(\widehat{G}_1)$. To prove the claim, it suffices to show that $E(G)\subseteq E(\widehat{G}_1)$.  Indeed, if $vw\in E(G)$ then the nation faces corresponding to $v$ and $w$ have a common vertex $x$ on their boundary. The vertex $x$ corresponds to a face $F_x$ in $G_0^*$ and the facial cycle of $F_x$ contains $v$ and $w$.  Therefore, the nation cycle $C_x$ of $F_x$ contains $v$ and $w$. Since $C_x$ bounds a face in $G_1$, $vw\in E(\widehat{G}_1)$.
\end{proof}

%\note{PM}{Old proof}
%
%\begin{proof}
%% \note{PM}{I fixed some issues here that come from the fact that some vertices of $G_0$ do not contribute faces to $G_1$.  This happens when two nations and two lakes meet at a vertex $x$.  It's not an issue but it means we can't write ``the face of $G_1$ that corresponds to $x$.''} \note{DW}{Good}
%Let $G_0$ be a graph embedded in the plane, with each face labelled a nation or a face, and where each vertex of $G_0$ is incident with at most $d$ nations. If two lakes have a common edge in their boundaries, then we may delete the edges and merge the lakes \note{DW}{Delete the previous sentence? This sentence is incompatible with ``in the clockwise or anticlockwise order of faces around $x$, all the faces between $A$ and $B$ are lakes'' below.}. Now assume that no two lakes have a common edge in their boundaries.
%Let $G$ be the corresponding map graph.
%Let $G_1$ be the graph with $V(G_1):=V(G)$, which is the set of nations, and whose edge set we now define.  Consider vertices $v$ and $w$ of $G_1$.
%Let $A$ and $B$ be the nation faces of $G_0$ corresponding to $v$ and $w$.
%Then $vw\in E(G_1)$ if $A$ and $B$ have a common edge in their boundaries,
%or there is a vertex $x$ of $G_0$ in the boundary of $A$ and $B$, and in the clockwise or anticlockwise order of faces around $x$, all the faces between $A$ and $B$ are lakes.
%% Then $G_1$ has Euler genus at most $g$.
%Each face of $G_1$ corresponds to a vertex of $G_0$ or a lake of $G_0$.
%Triangulate each face of $G_1$ corresponding to a lake of $G_0$.
%
%For each face $F$ of $G_1$ that corresponds to a vertex $x$ of $G_0$, the size of $F$ \note{DW}{Is ``size'' defined? Do we mean length of the facial cycle? } equals the number of nations incident to $x$, which is at most $d$. So every face of $G_1$ has size at most $d$.  Let $\widehat{G}_1$ be the $d$-framed graph whose frame is $G_1$.  \note{DW}{I like this notation. Should we use it throughout?} \note{PM}{No. For most things I prefer $G$ and $G_0$ rather than $\widehat{G}_0$ and $G_0$.}
%
%To complete the proof, we now show that $\widehat{G}_1$ contains every edge of $G$.
%If $vw\in E(G)$ then the nation faces of $G_0$ corresponding to $v$ and $w$ have a vertex $x$ of $G_0$ in common.   If some face of $G_1$ corresponds to $x$, then $v$ and $w$ are on a common face of $G_1$ so $vw\in E(\widehat{G}_1)$.  If no face of $G_1$ corresponds to $x$\footnote{This can occur when $x$ is on the boundary of exactly two nation faces and two lake faces.}, then the only two nation faces of $G_0$ with $x$ on their boundary are those corresponding to $v$ and $w$, in which case $vw\in E(G_1)\subseteq E(\widehat{G}_1)$.  Therefore $G\subseteq\widehat{G}_1$.
%\end{proof}

%%%%%%%%%%%%%%%%%%%%%%%%
\section{Applications}
\label{Applications}

As discussed in the introduction, product structure has been used to resolve a number of problems on planar graphs.  In most cases, these results hold for any graph class with product structure. In this section, we survey some of the consequences of this for the graph classes considered in the previous two sections.

% Here we discuss some of the consequences of the above theorems for $k$-planar and $(g,k)$-planar graphs.
%
% \referee{1}{4. Sections 4 and 5: I think as they are currently organized they distract from the purpose of the paper, as stated in the introduction, to ``prove product structure theorems for several non-minor-closed classes of interest.'' Which of Corollaries 1--4 and Theorems 12--17 are most important? I would remove any mention of the applications from the examples section, put that section first, and state in that section a single main theorem with all of the best bounds on the product structure. The applications section is mostly a survey and I think should be focused on the most exciting new corollaries. Theorems 18 and 19 should be in the applications section.}
%
% \note{DW}{I don't want to do a complete restructuring. In our response, let's just say why we structure it the way we do.}
% \note{PM}{Sure, though I think they proposing a chop rather than a restructuring.}

%%%%%%%%%%%%%%%%%%%
\subsection{Queue Layouts}

For an integer $k\geq 0$, a \defin{$k$-queue layout} of a graph $G$ consists of a linear ordering $\preceq$ of $V(G)$ and a partition $\{E_1,E_2,\dots,E_k\}$ of $E(G)$, such that for $i\in\{1,2,\dots,k\}$, no two edges in $E_i$ are nested with respect to $\preceq$. That is, it is not the case that $v\prec x \prec y \prec w$ for edges $vw,xy\in E_i$. The \defin{queue-number} of a graph $G$, denoted by \defin{$\qn(G)$}, is the minimum integer $k$ such that $G$ has a $k$-queue layout. Queue-number was introduced by \citet{HLR92}, who famously conjectured that planar graphs have bounded queue-number. \citet{DJMMUW20} recently proved this conjecture using \cref{PlanarProduct} and the following lemma. Indeed, resolving this question was the motivation for the development of \cref{PlanarProduct}.

\begin{lem}[\citep{DJMMUW20}]
\label{qn}
If $G\subseteq H \boxtimes P \boxtimes K_\ell$ then
$\qn(G) \leq  3 \ell \, \qn(H) + \floor{\tfrac{3}{2}\ell}
\leq 3 \ell \, 2^{\tw(H)}  - \ceil{\tfrac{3}{2}\ell}$.
\end{lem}

Since all the product structure theorems presented thus far upper bound the treewidth of $H$, \cref{qn} immediately implies bounds on the queue number of all graphs in these classes.  An improvement can be made for $1$-planar and $d$-map graphs since then $H$ is planar with treewidth at most $3$. \citet{ABGKP20} proved that every planar graph with treewidth at most $3$ has queue-number at most $5$. Thus the graph $H$ in \cref{1_planar_product,dMapProduct} has queue-number at most $5$.
% , which with \cref{qn} implies:
%
% \begin{cor}
% \label{1PlanarQueue}
% Every 1-planar graph has queue-number at most
% $3 \times 7 \times 5 + \floor{\tfrac{3}{2} \times 7} = 115$.
% \end{cor}
%
% \begin{cor}
% \label{dMapQueue}
% Every $d$-map graph has queue-number at most
% $\floor{ \frac{33}{2} (d+3\floor{d/2}-3) }$.
% \note{DW}{I have updated this bound using the better qn bound for planar 3-trees. Note that with $d=3$ we get the original bound of 49.}
% \end{cor}
%
%
%
% \cref{qn,gkPlanarProduct} imply that $(g,k)$-planar graphs have queue-number at most $g 2^{O(k^3)}$.
% \note{DW}{I have changed $g 2^{O(k^4)}$ to $g 2^{O(k^3)}$, okay?}
% \note{OM}{Ok}
% \note{DW}{I have changed $g>2^{k^3}$ to $g>2^{k^2}$, okay?}.
% \note{PM}{Ok}
The following corollary summarizes all of these results.
\begin{cor}\label{q_cor}
  The following bounds hold for the queue number of any graph from each of the following classes:
  \begin{compactitem}
    \item $k$-planar: $2^{O(k^3)}$
    \item $(g,k)$-planar: $\max\{1,g\}\cdot 2^{O(k^3)}$
    \item $(g,\delta)$-string: $\max\{1,g\}\cdot 2^{O(\delta^3)}$
    \item $1$-planar: $115$
    \item $d$-map: $\floor{ \frac{33}{2} (d+3\floor{\frac{d}{2}}-3) }$
    \item $(g,d)$-map: $\max\{1,g\}\cdot 2^{O(d)}$
    \item $k$-nearest-neighbour: $2^{O(k^6)}$
  \end{compactitem}
\end{cor}
%\note{PM}{For uniformity, I got rid of the two corollaries on $d$-map and $1$-planar graphs and kept them in \cref{q_cor}.  The text is still commented out in the source in case someone feels strongly about keeping them.}


Note that \citet{DJMMUW20} previously proved the bound of
$O(g^{k+2})$ for $k$-planar graphs using \cref{GenusProduct} and an ad-hoc method. Our result provides a better bound when $g>2^{k^2}$.

%%%%%%%%%%%%%%%%%%%
\subsection{Non-Repetitive Colouring}

The next two applications are in the field of graph colouring. For our purposes, a \defin{$c$-colouring} of a graph $G$ is any function $\phi\colon V(G)\to C$, where $C$ is a set of size at most $c$.
A $c$-colouring $\phi$ of $G$ is \defin{non-repetitive} if, for every path $v_1,\ldots,v_{2h}$ in $G$, there exists $i\in\{1,\ldots,h\}$ such that $\phi(v_i)\neq\phi(v_{i+h})$.  The \defin{non-repetitive chromatic number} $\pi(G)$ of $G$ is the minimum integer $c$ such that $G$ has a non-repetitive $c$-colouring. This concept, introduced by \citet{AGHR-RSA02}, has since been widely studied; see \citep{dujmovic.esperet.ea:planar} for more than 40 references. Up until recently the main open problem in the field has been whether planar graphs have bounded non-repetitive chromatic number, first asked by \citet{AGHR-RSA02}. \citet{dujmovic.esperet.ea:planar} recently solved this question using \cref{PlanarProduct} and the following lemma.

\begin{lem}[\citep{dujmovic.esperet.ea:planar}]
\label{non-repetitive}
If $G\subseteq H\boxtimes P \boxtimes K_\ell$ then $\pi(G)\le \ell\, 4^{\tw(H)+1}$.
\end{lem}

% \cref{non-repetitive,kPlanarProduct,1-planar,gkPlanarProduct} imply the following result:
%
% \begin{cor}\quad
% \begin{compactitem}
% \item	For every $1$-planar graph $G$, $\pi(G)\le 7\times 4^4=1792$.
% %\item For every $k$-planar graph $G$,
% %	$\pi(G)\le (18k^2+48k+30) 4^{\binom{k+4}{3}}$.
% \item	For every $(g,k)$-planar graph $G$,
% 	$  \pi(G)\le \max\{2g,3\}\cdot(6k^2+16k+10) 4^{\binom{k+4}{3}}.$
% % \note{DW}{I have changed $4^{\binom{k+5}{4}}$ to $4^{\binom{k+4}{3}}$, okay?}
% % \note{PM}{Ok}
% \end{compactitem}
% \end{cor}

Combining \cref{non-repetitive} with our results on product structure, we obtain the following corollary:
\begin{cor}\label{non-repetitive_cor}
  The following bounds hold for the non-repetitive chromatic number of any graph from each of the following classes:
  \begin{compactitem}
    \item $k$-planar: $(18k^2+48k+30)\cdot 4^{\binom{k+4}{3}}$
    \item $(g,k)$-planar: $\max\{2g,3\}\cdot(6k^2+16k+10) 4^{\binom{k+4}{3}}$
    % \item $(g,\delta)$-string: $g4^{O(\delta^3)}$
    \item $1$-planar: $1792$
    \item $d$-map: $256(d+3\floor{\frac{d}{2}}-3)$
    % \item $(g,d)$-map: $g4^{O(d)}$
    % \item $k$-nearest-neighbour: $4^{O(k^6)}$
    \item $(g,d)$-map graph: $(7d^2-21d)\cdot 4^{10}$
  \end{compactitem}
\end{cor}

Prior to the current work, the strongest upper bound on the non-repetitive chromatic number of $n$-vertex  $k$-planar graphs was $O(k\log n)$ \cite{dujmovic.morin.ea:layered}.  Our results also give bounds on the non-repetitive chromatic number of $(g,\delta)$-string graphs and $k$-nearest-neighbour graphs, but these bounds are weaker than existing results based on maximum degree.  The bounds obtained from product structure are exponential in $\delta$ and $k^2$, respectively.  However, each of these graph classes has degree bounded by $O(\delta)$ and $O(k^2)$, respectively, and therefore have non-repetitive chromatic number $O(\delta^2)$ and $O(k^{4})$, respectively \cite{DJKW16}.

%%%%%%%%%%%%%%%%%%%
\subsection{Centered Colourings}
\label{centered-colourings}

A $c$-colouring $\phi$ of $G$ is \defin{$p$-centered} if, for every connected subgraph $X\subseteq G$, $|\{\phi(v):v\in V(X)\}| > p$ or there exists some $v\in V(X)$ such that $\phi(v)\neq \phi(w)$ for every $w\in V(X)\setminus\{v\}$.  In words, either $X$ receives more than $p$ colours or some vertex in $X$ receives a unique colour.  Let $\chi_p(G)$ be the minimum integer $c$ such that $G$ has a $p$-centered $c$-colouring. Centered colourings are important since they characterise classes of bounded expansion, which is a key concept in the sparsity theory of \citet{Sparsity}.

% \citet{PS21} and
\citet{DFMS21} use \cref{PlanarProduct} and \cref{PlanarProduct}(b), respectively, to show that $\chi_p(G)\in O(p^2\log p)$ when $G$ is planar or of bounded Euler genus.  Upper bounds on $\chi_p$ for graphs of given treewidth~\citep{PS21} and for graph products~\citep{DFMS21} imply the next lemma.

%In the journal version, \citet{DFMS21} proved that $\chi_p(H_1\boxtimes H_2) \leq \chi_p(H_1) \cdot \chi(H_2^p)$ where $H_2^p$ is the $p$-th power of $H_2$. Apply this with $H_2:= P \boxtimes K_\ell$. Then $H_2^p= P^p\boxtimes K_\ell$ and $\chi(H_2^p)=(p+1)\ell$. So $\chi_p(H\boxtimes P \boxtimes K_\ell) \leq \chi_p(H_1) \cdot (p+1)\ell$. }

\begin{lem}[\citep{DFMS21,PS21}]
\label{p-centered}
For every graph $H$ of treewidth at most $t$ and for every path $P$, if $G\subseteq H\boxtimes P \boxtimes K_\ell$ then
\[\chi_p(G)\le \ell (p+1)\, \chi_p(H) \leq \ell (p+1) \tbinom{p+t}{t}.\]
%If $G\subseteq H\boxtimes P \boxtimes K_\ell$ then $\chi_p(G)\le \ell (p+1)\, \chi_p(H)$.
\end{lem}

%\note{DW}{For consistency with \cref{qn}, I suggest we combine \cref{p-centered,p-centered-treewidth}, and just write, ``}

%\begin{proof}
%	By \cref{PartitionProduct}, $G$ has an $H$-partition $(\mathcal{L}=\langle L_0,L_1,\ldots\rangle, \PP=(B_x:x\in V(H))$ with layered width at most $\ell$. Use a product colouring $\phi:V(G)\to \{1,\ldots,\ell\}\times\{0,\ldots,p\}\times\{1,\ldots,\chi_p(H)\}$.  For each integer $i\ge 0$ and each $x\in V(H)$, assign the colour $\phi(v):=(\alpha(v),\beta(v),\gamma(v))$ to each vertex $v\in L_i\cap B_x$ such that:
%   \begin{compactenum}
%     \item $\alpha(v)$ is unique among $\{\alpha(w): w\in L_i\cap B_x\}$, which is possible
%     since $|L_i\cap B_x|\le \ell$,
%     \item $\beta(v)= i\bmod (p+1)$, and
%     \item $\gamma(v)=\gamma(x)$ where $\gamma:V(H)\to\{1,\ldots,\chi_p(H)\}$ is a $p$-centered colouring of $H$.
%   \end{compactenum}
% To show this is a $p$-centered colouring, consider some connected subgraph $X\subseteq G$.
%
% First suppose that there exists $v,w\in V(X)$ with $v\in L_i$ and $w\in L_j$ with $j-i\ge p$. Since $X$ is connected, $X$ contains a path from $v$ to $w$.  By the definition of layering, this path contains at least one vertex from $L_{i'}$ for each $i'\in\{i,i+1,\ldots,j\}$. Therefore, $|\{\beta(v'):v'\in V(X)\}|\ge j-i+1 > p$, so $X$ receives more than $p$ distinct colours.
%
% Otherwise, $V(X)\subseteq L_{i}\cup\cdots\cup L_{i+s}$ for some $s<p$.  Let $H':=H[\{x\in V(H):B_x\cap V(X)\neq\emptyset]$.  If $|\{\gamma(x):x\in V(H')\}| > p$ then $|\{\gamma(v):v\in V(X)\}|> p$ so $|\{\phi(v):v\in V(X)\}|> p$ and we are done.  Otherwise, since $\gamma$ is a $p$-centered colouring of $H$, there must exist some $x\in V(H')$ such that $\gamma(x)\neq\gamma(y)$ for every $y\in V(H')\setminus\{x\}$.
% For any $v,w\in B_x$ with $v\neq w$, either $v,w\in L_{i'}$ for some $i'\in\{i,i+1,\ldots,i+s\}$ in which case $\alpha(v)\neq\alpha(w)$; or $v\in L_{i'}$ and $w\in L_{i''}$ with $|i'-i''|< p$, in which case $\beta(v)\neq\beta(w)$. Therefore every vertex $v\in B_x$ receives a colour $\phi(v)$ distinct from every colour in $\{\phi(z):z\in X\setminus\{x\}\}$. Therefore, every vertex in $B_x$ receives a colour distinct from every other vertex in $X$.
% \end{proof}
%
%\referee{2}{Page 18.  Replace '$|i'-i''| < p$' by  '$0< |i'-i''| < p$'.}

% \cref{p-centered,kPlanarProduct,1-planar,gkPlanarProduct} immediately imply the following results, for every $p\ge 2$:
%
% \begin{cor}\quad
% \begin{compactitem}
% \item For every $1$-planar graph $G$,\; $ \chi_p(G)\le 5 (p+3)(p+2)(p+1)^2$.
% %\item For every $k$-planar graph $G$,\; $\displaystyle \chi_p(G)\le (18k^2+48k+30)(p+1) \binom{p+ \binom{k+4}{3}-1}{ \binom{k+4}{3}-1}$.
% \item For every $(g,k)$-planar graph $G$,\;
% $\displaystyle \chi_p(G) \le \max\{2g,3\}\cdot(6k^2+16k+10) (p+1) \binom{p+\binom{k+4}{3}-1}{\binom{k+4}{3}-1}$.
% % \note{DW}{I have changed $\binom{k+5}{4}$ to $\binom{k+4}{3}$, okay?}
% % \note{PM}{Ok}
%  \end{compactitem}
% \end{cor}

For planar graphs of treewidth at most $3$ a $O(p^2\log p)$ upper bound is known:
% \note{DW}{I prefer ``a $O(p^2\log p)$'', pronounced ``a big-O...''}

\begin{lem}\cite{DFMS21}\label{simple_3tree_centered_colouring}
  For every planar graph $H$ of treewidth at most $3$,
  \[
    \chi_p(H)\le (p+1)(p\ceil{\log(p+1)}+2p+1) \enspace .
  \]
\end{lem}

Combining \cref{p-centered,simple_3tree_centered_colouring} with our results on product structure, we obtain the following corollary:

\begin{cor}\label{p_centered_cor}
  The following bounds hold for the $p$-centered chromatic number of any graph from each of the following classes:
  \begin{compactitem}
    \item $k$-planar: $(18k^2+48k+30)\cdot 4^{\binom{k+4}{3}}$
    \item $(g,k)$-planar: $\max\{2g,3\}\cdot(6k^2+16k+10)(p+1)\binom{p+\binom{k+4}{3}-1}{\binom{k+4}{3}-1}$
    % \item $(g,\delta)$-string: $g4^{O(\delta^3)}$
    \item $1$-planar: $7(p+1)^2(p\ceil{\log(p+1)}+2p+1)$
    \item $d$-map: $(d+3\floor{d/2}-3)\cdot (p+1)^2(p\ceil{\log(p+1)}+2p+1)$
    % \item $(g,d)$-map: $g4^{O(d)}$
    % \item $k$-nearest-neighbour: $4^{O(k^6)}$
    \item $(g,d)$-map: $\max\{2g,3\}\cdot 7d(d-3)(p+1)\cdot \binom{p+9}{9}$
  \end{compactitem}
\end{cor}


Prior to the current work, the strongest known upper bounds on the $p$-centered chromatic number of $(g,k)$-planar graphs $G$ were doubly-exponential in $p$, as we now explain. \citet{dujmovic.eppstein.ea:structure} proved that $G$ has layered treewidth $(4g+6)(k+1)$.
Van den Heuvel and Wood~\citep{vdHW18} showed this implies that $G$  has $r$-strong colouring number at most $(4g + 6)(k + 1)(2r + 1)$. By a result of \citet{zhu:colouring}, $G$ has $r$-weak colouring number at most $( (4g + 6)(k + 1)(2r + 1) )^r$, which by another result of  \citet{zhu:colouring} implies that $G$ has  $p$-centered chromatic number at most $( (4g+6)(k+1)(2^{p-1} + 1) )^{2^{p-2}}$. The above results are substantial improvements, providing bounds on $\chi_p(G)$ that are polynomial in $p$ for fixed $g$ and $k$.

As with non-repetitive chromatic number, our results also imply bounds on the $p$-centered chromatic numer of $(g,\delta)$-string graphs, powers of bounded-degree graphs, and $k$-nearest-neighbour graphs, but these bounds are weaker than the bounds implied by the fact that these graphs have bounded maximum degree \cite{DFMS21}.

\subsection{Universal Graphs}

As mentioned in \cref{Introduction}, \citet{DEJGMM21} and \citet{EJM} prove results about adjacency labelling schemes for planar graphs. Stated in terms of universal graphs, their main theorem is interpreted as follows:

\begin{thm}[\citep{DEJGMM21,EJM}]
	\label{Universal}
	For every fixed integer $t$ and every integer $n>0$ there exists a
	graph $U_n$ with $n^{1+o(1)}$ vertices and edges such that for every graph $H$ of treewidth at most $t$ and path $P$, every $n$-vertex subgraph of $H\boxtimes P$ is isomorphic to an induced subgraph of $U_n$.
\end{thm}

Combining \cref{Universal} with our results on product structure yields the following:
% this with \cref{gkPlanarProduct,k-nn,PowerMinor,StringPartition,MapPartition} yields the following:

\begin{cor}\label{universal_cor}
	\label{UniversalUniversal}
	For every fixed graph $X$ and all fixed integers $d,\delta,\Delta,g,k>0$, and every integer $n>0$, there exists a graph $U_n$ with $n^{1+o(1)}$ vertices and edges such that $U_n$ contains the following graphs as induced subgraphs:
	\begin{compactitem}
		\item every $n$-vertex $(g,k)$-planar graph;
		\item every $n$-vertex $(g,d)$-map graph;
		\item every $n$-vertex $(g,\delta)$-string graph;
		\item every $n$-vertex graph $G^k$ where $G$ is $X$-minor-free and has maximum degree at most $\Delta$;
		\item every $k$-nearest neighbour graph of $n$ points in $\R^2$.
	\end{compactitem}
\end{cor}








% \section{Orphaned Text}
%
% \subsection{Text following \cref{gkPlanarProduct}}
%
% Prior to this work, the strongest structural description of $k$-planar or $(g,k)$-planar graphs (or any of the other classes presented in \cref{examples}) was in terms of layered treewidth, which we now define.  A \defin{layered tree-decomposition} $(\mathcal{L},\mathcal{T})$ consists of a layering $\mathcal{L}$ and a tree-decomposition $\mathcal{T}$ of $G$. The layered width of $(\mathcal{L},\mathcal{T})$ is $\max\{|L\cap B|: L\in \mathcal{L},\, B\in \mathcal{T}\}$.  The \defin{layered treewidth} of $G$ is the minimum layered width of any layered tree-decomposition of $G$. \citet{dujmovic.morin.ea:layered} proved that planar graphs have layered treewidth at most 3, that graphs of Euler genus $g$ have layered treewidth at most $2g+3$, and more generally that a minor-closed class has bounded layered treewidth if and only if it excludes some apex graph. \citet{dujmovic.eppstein.ea:structure} show that every $k$-planar graph has layered treewidth at most $6(k+1)$, and more generally that every $(g,k)$-planar graph has layered treewidth at most $(4g+6)(k+1)$. It follows from this result that $(g,k)$-planar graphs have treewidth $O(\sqrt{(g+1)(k+1)n})$ and thus have balanced separators of the same order, which can also be concluded from the work of \citet{FP08}. In related work, \citet{grigoriev.bodlaender:algorithms} used structural results to obtain approximation algorithms for $(g,k)$-planar graphs, and \citet{PachToth97} determined the maximum number of edges in a $k$-planar graph (up to a constant factor).
%
% If a graph class admits bounded layered partitions, then it also has bounded layered treewidth. In particular, if $\PP=(P_x:x\in V(H))$ is an $H$-partition of $G$ of layered width $\ell$ with respect to some layering $\mathcal{L}$ of $G$ and $(B_x:x\in V(T))$ is a width-$t$ tree-decomposition of $H$, then setting $C_x = \bigcup_{y\in B_x} P_y$ for each $x\in V(T)$ gives a tree-decomposition $(C_x:x\in V(T))$ of $G$ that has layered treewidth $(t+1)\ell$ \cite{DJMMUW20}. Therefore, any property that holds for graphs of bounded layered treewidth also holds for $G$. What sets layered partitions apart from layered treewidth is that they lead to constant upper bounds on the queue-number  and non-repetitive chromatic number, whereas for both these parameters, the best known upper bound obtainable via layered treewidth is $O(\log n)$; see \cref{Applications}.
%
%
%
% %%%%%%%%%%%%%%%%%%%
% \label{sec-k-planar}
%
%
% %\note{DW}{I am pretty sure one can get treewidth 3 for $(g,1)$-planar graphs, combining \cref{d_framed_product_stucture} with the method in \citep{DHHW}. I think the best place for this result is reference \citep{DHHW} since the present paper pre-dates \citep{DHHW} (and this addition will increase the significance of \citep{DHHW}). Okay?}
% %\note{PM}{Okay to this and to all similar questions that come later (which I've now commented out.)}
% %\note{DW}{Robert is working on it.}
%
%
%
%
%
% This section describes several examples of graph classes that can be obtained from a shortcut system typically applied to graphs of bounded Euler genus.
%
% \subsection{Orphaned Theorems}
%
% Theorems~\ref{PlanarProduct}(b),  \ref{ShortcutProduct} and  \ref{GenusProduct}(b) and \cref{MapShortcut,qn,p-centered,non-repetitive} imply the following result.
%
% %\begin{thm}
% %\label{PlaneMapPartition}
% %\note{DW}{For every integer $d\geq 4$, }
% %Every $d$-map graph $G$:
% %	\begin{compactitem}
% %		\item is contained in $H \boxtimes P \boxtimes K_{21d(d-3)}$ for some path $P$ and for some graph $H$ with $\tw(H)\leq 9$,
% %		\item has queue-number $\qn(G) < 32225\, d(d-3)$.
% %		\item has $p$-centered chromatic number $\chi_p(G) \leq 21d(d-3) (p+1)  \binom{p+9}{9}$,
% %		\item has non-repetitive chromatic number $ \pi(G) \leq 21 \cdot 4^{10} d(d-3)$.
% %	\end{compactitem}
% %\end{thm}
%
% \begin{thm}
% \label{MapPartition}
% For integers $g\geq 0$ and $d\geq 4$, if $\ell:=  7d(d-3)\, \max\{2g,3\}$ then every $(g,d)$-map graph $G$:
% \begin{compactitem}
% \item is contained in $H \boxtimes P \boxtimes K_{\ell}$ for some path $P$ and for some graph $H$ with $\tw(H)\leq 9$,
% \item has queue-number $\qn(G) <  1535\, \ell $.
% \item has $p$-centered chromatic number $\chi_p(G) \leq \ell (p+1)\,  \binom{p+9}{9}$,
% \item has non-repetitive chromatic number $ \pi(G) \leq 4^{10}\,\ell $.
% \end{compactitem}
% \end{thm}
%
% These results give the first constant upper bound on the non-repetitive chromatic number of map graphs, the first polynomial bounds on the $p$-centered chromatic number of map graphs, and the best known bounds on the queue-number of map graphs.
%
% % \note{DW}{I am pretty sure one can get treewidth 3 here using \cref{framed} and the method in \citep{DHHW}. I think the best place for this result is reference \citep{DHHW} since the present paper pre-dates \citep{DHHW} (and this addition will increase the significance of \citep{DHHW}). Okay?}
%
%
% % \subsection{Orphaned Section}
% %
% % \cref{nearest-neighbour,kPlanarProduct,qn,p-centered} imply:
% %
% % \begin{cor}
% % \label{k-nn}
% % For every integer $k\geq 1$ there exists integers $t\leq O(k^6)$ and $\ell\leq O(k^4)$ such that every $k$-nearest-neighbour graph:
% % \begin{compactitem}
% % \item is contained in $H\boxtimes P \boxtimes K_\ell$ for some graph $H$ with treewidth $t$ and some path $P$,
% % \item has queue-number at most $2^{O(k^6)}$, and
% % \item has $p$-centered chromatic number at most $\ell (p+1)\binom{p+t}{t}$.
% % \end{compactitem}
% % \end{cor}
% %
% % \cref{non-repetitive,k-nn} also give bounds on the non-repetitive chromatic number of a $k$-nearest neighbour graph $G$. However, the bound is weak, since $G$ has maximum degree at most $6k$, implying that $\pi(G) \leq (36+o(1))k^2$ by a result of \citet{DJKW16}.
%
%
%
%
% \subsection{Generalisations}
% \label{Generalisations}
%
% % As mentioned above, product structure theorems have been established for several minor-closed classes in addition to planar graphs. The first generalises \cref{PlanarProduct} for graphs of bounded Euler genus.
% %
% % \begin{thm}[\citep{DJMMUW20,UWY,DHHW}]
% % \label{GenusProduct}
% % Every graph of Euler genus $g$ is contained in:
% % \begin{compactenum}[(a)]
% % \item $H  \boxtimes P$ for some graph $H$ of treewidth at most $2g+6$  and some path $P$.
% % \item $H \boxtimes P \boxtimes K_{\max\{2g,3\}}$ for some graph $H$ of treewidth at most $3$ and for some path $P$.
% % \end{compactenum}
% % \end{thm}
%
% \citet{DJMMUW20} generalised \cref{GenusProduct} for apex-minor-free graphs as follows.
%
% \begin{thm}[\citep{DJMMUW20}]
% \label{ApexMinorFree}
% For every apex graph $X$, there exists $c\in\mathbb{N}$ such that every $X$-minor-free graph is contained in $H\boxtimes P$ for some graph $H$ with $\tw(H)\leq c$ and some path $P$.
% \end{thm}
%
% The assumption that $X$ is apex is needed in \cref{ApexMinorFree}, since if the class of $X$-minor-free graphs has a product structure theorem analogous to \cref{PlanarProduct}, then $X$ is apex \citep{DJMMUW20}. On the other hand, \citet{DEMWW22} proved a product structure theorem for bounded degree graphs in any minor-closed class.
%
% \begin{thm}[\citep{DEMWW22}]
% 	\label{MinorFreeDegree}
% 	For every graph $X$ there exists $c\in\mathbb{N}$ such that for every $\Delta\in\mathbb{N}$, every $X$-minor-free graph $G$ with maximum degree at most $\Delta$ is contained in $H\boxtimes P$ for some graph $H$ with $\tw(H) \leq c\Delta$ and for some path $P$.
% \end{thm}
%
% %The next three results are immediate corollaries of \cref{GenusProduct,ApexMinorFree,ShortcutProduct,MinorFreeDegree}.
% %
% %\begin{thm}
% %Let $\SS$ be a $(k,d)$-shortcut system for a graph $G$ of Euler genus $g$. Then $G^\SS$ is contained in $H\boxtimes P$ for some graph $H$ of treewidth at most $d(k^3+3k)\binom{k+2g+8}{2g+8}-1$ and for some path $P$.
% %\end{thm}
% %
% %\begin{thm}
% %For every apex graph $X$ and for all integers $k,d\geq 1$, there is an integer $c$ such that for every $X$-minor-free graph $G$ and for every $(k,d)$-shortcut system $\SS$ for $G$, $G^\SS\subseteq H\boxtimes P$ for some graph $H$ with $\tw(H)\leq c$ and for some path $P$.
% %\end{thm}
% %
% %\begin{thm}
% %For every graph $X$ and for all integers $k,\Delta\geq 1$, there is an integer $c$ such that for every $X$-minor-free graph $G$ with maximum degree $\Delta$ and for every $(k,\Delta)$-shortcut system $\SS$ for $G$, $G^\SS \subseteq H \boxtimes P$ for some graph $H$ with $\tw(H)\leq c$ and for some path $P$.
% %\end{thm}
% %
% %\note{DW}{delete the above three theorems, they are never used}
%
%
%
%
% %\subsection*{Note Added in Proof} Subsequent to the initial release of this paper, the treewidth 8 bound in \cref{PlanarProduct} was improved to 6 by \citet{UWY}, which implies that the treewidth $2g+8$ bound in \cref{GenusProduct}(a) can be improved to $2g+6$. Simmilarly, the treewidth 4 bound in \cref{GenusProduct}(b) was improved to 3 by \citet{DHHW}. \note{DW}{Say more about the consequence for the bounds in the rest of the paper}

% %%%  Squashing the bibliography
   \let\oldthebibliography=\thebibliography
   \let\endoldthebibliography=\endthebibliography
   \renewenvironment{thebibliography}[1]{%
     \begin{oldthebibliography}{#1}%
       \setlength{\parskip}{0ex}%
       \setlength{\itemsep}{0ex}%
   }{\end{oldthebibliography}}

\bibliographystyle{DavidNatbibStyle}
\bibliography{k-planar}
\end{document}
