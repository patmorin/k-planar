\documentclass{patmorin}
\usepackage[T1]{fontenc}
\usepackage[utf8]{inputenc}
\usepackage{amsmath}
\usepackage{amsfonts}
\usepackage{amsthm}
\usepackage{graphicx}
\usepackage{enumerate}
\usepackage{pat}
\usepackage{paralist}

\usepackage{tipa}
\usepackage{upgreek}
\usepackage{rotating}
\newcommand{\V}{\rotatebox[origin=c]{180}{$V$}}
\newcommand{\Y}{\rotatebox[origin=c]{180}{$Y$}}
%\usepackage{hyperref}

\setlength{\parskip}{1ex}

\title{\MakeUppercase{Layered $H$-Partitions of $k$-Planar Graphs}%
    \thanks{This work was partly funded by NSERC and MRI.}}

\author{Vida Dujmovi\'c%
        \thanks{School of Computer Science and Electrical Engineering,
                University of Ottawa},\,\,
        Pat Morin%
        \thanks{School of Computer Science, Carleton University},\,\, and
        David R. Wood%
        \thanks{School of Mathematical Sciences, Monash University}}

\newcommand{\dual}[1]{{#1}^\star}
\newcommand{\note}[2]{{\color{red}[#1:~#2]}}

\DeclareMathOperator{\dist}{dist}
\DeclareMathOperator{\depth}{depth}
\DeclareMathOperator{\ff}{f}
\newcommand{\f}{{\color{red}\ff}}

\newcommand{\fk}{{\color{red}f(k)}}

\begin{document}
\maketitle


\begin{abstract}
  Layered $H$-partitions of graphs (with small layered width, in which $H$ has small treewidth) are a recently-introduced tool that have been used to solve longstanding problems on queue layouts, non-repetitive colouring, and 3-d graph drawing.  Such partitions are known to exist for planar graphs and, more generally, bounded genus graphs.  In the current paper, we prove that every $k$-planar graph has a layered $H$-partition of layered width $O(k)$ in which $H$ has treewidth $k9^k$. This implies that $k$-planar graphs have both queue number and non-repetitive chromatic number upper-bounded by a function of $k$.
\end{abstract}

\section{Introduction}


\section{1-Planar Graphs}

Put the 1-Planar stuff back here because the constants are better.

\section{$k$-Planar Graphs}

\subsection{PS-Tree}

Let $\Delta$ be a plane triangulation and let $T$ be a BFS spanning tree of $T$ rooted at some vertex $r$ of degree-3 in $\Delta$.  Then the \emph{PS-tree} for $(\Delta, T, r)$ is a 3-ary tree $K$ that such that each node $x\in V(K)$ is associated with several objects:

\begin{enumerate}
  \item  A cycle $F_x$ in $\Delta$.  There are at most three ancestors $a$, $b$, and $c$, of $x$ in $T$ such that the vertices of $F_x$ are contained in $V(Y_a)\cup V(Y_b)\cup V(Y_c)$. (See below for the definition of $Y_a$ $Y_b$, and $Y_c$.)
  
  \item $A_x=F[Y_a]$, $B_x=F[Y_b]$, $C_x=F[Y_c]$
  
  \item A near-triangulation $N_x$ consisting of the part of $\Delta$ contained in $F_x$.  Note that $Y_x\subseteq V(N_x)\setminus V(F_x)$ and $V(N_x)\subseteq V(N_a)$ for every ancestor $a$ of $x$.

  \item A \emph{tripod} $Y_x$ in $\Delta$, with respect to $T$. This is a clique $\tau_x\in N(x)$ and at most three disjoint vertical paths in $T$ whose lower endpoints are in $\tau_x$.  Each of these paths is contained in $N_x-V(F_x)$ and the upper endpoint of each path is adjacent, in $T$ to a vertex of $F_x$.  Furthermore, for each of $A_x$, $B_x$, and $C_x$, at least one upper endpoint is adjacent, in $T$ to it.  
  
  \item A \emph{Mercedes graph} $M_x$ that consists of the edges and vertices of $F_x$, the clique $\tau_x$, and the paths in $T$ from the vertices of $\tau_x$ to $F_x$.
\end{enumerate}  

{\color{red}Expand the preceding points more fully and point out the relationsships better.  Deal with degenernate cases.}

\subsection{$k$-Planar Graphs}

Let $G$ be a $k$-plane graph and let $G^+$ be the graph obtained by making $G$ planar with dummy vertices and then triangulating the resulting graph. Let $T$ be a BFS spanning tree of $G^+$ where the root $r$ of $T$ has degree 3 in $G^+$ (add an additional vertex to $G^+$ if necessary). For each integer $i\ge 0$, let $V_i=\{v\in V(G^+): \dist_{G^+}(r,v)=i\}$.

Let $K$ be the PS-partition for $G^+$ and $T$.  We say that a vertex $v\in V(G)$ \emph{appears} at a node $x\in V(K)$ if $u\in Y_x$ or if there exists $vw\in E(G)$ such that $v,w\in V(N_x)$ and $vw$ contains a vertex of $Y_x$, either as an endpoint or in its interior.

Let $S_x$ (the separator at $x$) denote the set of all vertices $v\in V(G)$ such that $v$ appears at $x$ but does not appear at an ancestor of $x$.  We claim that $(S_x:x\in V(K))$ is a partition of $V(G)$. Indeed, if $v$ appears at nodes $x$ and $y$ where neither $x$ nor $y$ is an ancestor of the other, then there exists edges $vw,vz\in E(G)$ (possibly $w=z$) such that $vw$ contains a vertex of $Y_x$ (as an endpoint or in its interior) and $vz$ contains a vertex of $Y_y$ (as endpoint or in its interior).  At least one of these two edges contains a vertex $Y_a$ for some common ancestor $a$ of $x$ and $y$.

We use the partition $(S_x: x\in V(K))$ which generates a graph $H$ with vertex set $V(H)=V(K)$ the edge $xy\in E(H)$ if there exists an edge $vw\in E(H)$ with $v\in S_x$ and $w\in S_y$.
First we argue that this partition has small width with respect to the layering defined by $T$ (note that this is a layering of $G^+$, not a layering of $G$).

\begin{clm}
  For any $x\in V(H)$, and any integer $i\ge 0$, $|S_x\cap V_i|\le 3(k+1)$.
\end{clm}

\begin{proof}
  Done already, several times.
\end{proof}

Next we describe a tree decomposition $(B_x:x\in V(K))$ of $H$.  (Don't be put off here by the fact that $V(H)=V(K)$, this is actually helpful because the subtree $K[x]$ will be rooted at $x$).  We claim that, for every edge $xz\in V(H)$, $x$ is an ancestor of $z$ (or vice-versa).  Indeed, this follows from the same argument used to show that $S_x$ is a partition.  If neither $x$ nor $z$ is an ancestor of the other, then any edge $vw$ with $v\in S_x$ and $w\in S_z$ contains a vertex of $Y_a$ for some common ancestor $a$ of $x$ and $z$. But this implies that neither $v$ nor $w$ appears in $S_x$ nor $S_z$.

To describe our tree decomposition we define, for each $x\in V(H)$, the subtree $K[x]=K[\{y:x\in B_y\}]$.  The subtree $K[x]$ contains $x$.  Additionally, for every edge $xz\in E(H)$, where $x$ is an ancestor of $z$, $K[x]$ includes the path from $x$ to $z$ in $K$.  That this produces a tree-decomposition is due to the fact that for every edge $xz$ of $H$, $x$ is an ancestor of $z$ (or vice-versa).  What remains is to upper bound the width of this decomposition.

First observe that the subtree $K[x]$ is rooted at $x$, i.e., $x$ is the lowest common ancestor of all nodes in $B_x$.  Now, fix some node $x\in V(K)$ and consider the contents of $B_x$.  As we have just argued, $B_x$ contains only ancestors of $x$ (including $x$ itself). If $B_x$ contains some ancestor $a$ of $x$, it is because there is an edge $vw\in E(G)$ with $v\in V(N_x)\setminus V(F(x))$ and $w\in S_a$.  Now, $w\in S_a$ because $w\in Y_a$ or there is an edge $wz\in E(G)$ that contains a point of $Y_a$.  This implies that there is a curve $C$ from $v$ to $z$ that contains at most $2k$ dummy  vertices of $G^+$.

Now, consider the cubic graph $A$ homeomorphic to the union of $F_{a'}$ for all ancestors $a'$ of $x$. See \figref{A}.  It is helpful to think of this graph as follows:  We begin with $F_{a_0}$ where $a_0$ is the root of $K$.  We then cut $F_{a_0}$ into two pieces $F^{\bar{x}}_{a_0}$ and $F^x_{a_0}$ where $F^x_{a_0}$ contains $F_x$.  This process continues by partitioning $F_{a_1}:=F^x_{a_0}$ recursively.  The process terminates after some number $r$ of iterations, when $F^x_{a_r}=F_x$.

\begin{figure}
  \begin{center}
    \includegraphics{figs/A}
  \end{center}
  \caption{The graph $A$.}
  \figlabel{A}
\end{figure}

We claim that the curve $C$ begins in $F_x\subseteq F^x_a\subseteq F_a$, remains in the interior of $F_a$, and eventually reaches the boundary $F^{\bar{x}}_a\cap F^x_a$.  The fact that $C$ remains in $F_a$ is due to the fact that, if it crossed the boundary of $F_a$, then one of $vw$ or $wz$ would interects $Y_{a'}$ for some ancestor $a'$ of $a$.  But this would contradict the fact that $vw$ (or $wz$) are in $S_a$. The fact that $C$ eventually reaches $F^{\bar{x}}_a\cap F^x_a$ follows from the fact that $w\in S_a$ (so either $w\in Y_a$ or $wz$ crosses $Y_a$).

Therefore, to upper-bound $|B_x|$, it suffices to upper-bound the number of faces $F^{\bar{x}}_a$ of $A$ that can be reached with curves that begin in $F_x$, remain in $F_a$, and terminate in $F^{\bar{x}}_a$ without crossing more than $2k$ edges in $A$.  To avoid triple subscripts we say that a curve visits $a_i$ if it visits $F^{\bar{x}}_{a_i}$.  We can further restrict ourselves to studying curves that cross the fewest number of edges of $A$ since if a curve exists that crosses at most $k$ edges then the curve that crosses the fewest number of possible edges also crosses at most $k$ edges.  We will call such a curve a lightest curve from $F_x$ to $F_a$.

Again, we let $a_0,\ldots,a_r$ denote the vertices of the path, in $K$ from the root $a_0$ to $x=a_r$.  We claim that, there exists a lightest curve from $a_r$ to $a_0$ that is monotonic in the sense that the curve visits $a_{i_0},\ldots,a_{i_t}$ in this order, where $i_0=r > i_1>\cdots>i_t=0$ the Indeed, this follows from the fact that, for any node $a_i$, $i>0$, the (three) faces of $A$ that share a boundary with $F_{a_i}$ also share a boundary with $F^{\bar{x}}_{a_i}$.  Therefore it is never advantageous for a path to go from $a_i$ to $a_j$ for some $j>i$ since the path will eventually have to return to one of these three faces.

At this point we are essentially done.  Any monotonic curve that visits $a_i$ has at most 3 choices for the next face $a_j$ $j<i$ to visit.  Therefore, the number of faces of $A$ that can be reached from $F_x$ by admissible curves that cross $2k$ edges of $A$ is at most $3^{2k}$. Therefore,
\[
   |B_x| \le \sum_{j=0}^2k 9^j = (9^{2k+1}-1)/8 \enspace .
\]

This completes the proof

\begin{thm}
    Every $k$-planar graph $G$ has a $(k9^{2k+1}/8,3k)$-partition.
\end{thm}

\bibliographystyle{plain}
\bibliography{warmup}

\end{document}
