\documentclass[12pt]{article}
\usepackage[T1]{fontenc}
\usepackage[utf8]{inputenc}
\usepackage[margin=20mm]{geometry}
\usepackage{amsmath,color,amsthm,enumerate,amsfonts,mathtools}
\usepackage{graphicx}
\usepackage{enumerate}
\usepackage{amsfonts}
\usepackage{paralist}
\usepackage{stmaryrd}
\usepackage{amssymb}
\usepackage{todonotes}

\newcommand{\done}{\textcolor{blue}{Done.}}
\newcommand{\changed}{\textcolor{blue}{Revised accordingly.}}
\newcommand{\tba}{\textcolor{red}{To be addresssed.}}

\newenvironment{response}{\color{blue}}{}

\setlength{\parindent}{0ex}
\setlength{\parskip}{1ex}

\begin{document}

\subsection*{Graph Product Structure for Non-Minor-Closed Classes\\
Vida Dujmovi\'c, Pat Morin, David R. Wood\\
Journal of Combinatorial Theory Series B (JCTB9758)\\
Response to Referee Report received 4th October 2022}

Thanks to the referee for thoroughly checking the paper, and for highlighting several places for improvement. Here we respond to the referee's comments.

\hrulefill

1. In the proof of Theorem 7, I see no reason why we should have $a = x_r$
(or $x'=x_0$, for that matter). I think you need to use the fact that $v_r$ is
in $S_a$. (Also, if we did have that $a = x_r$, then $j$ would be $r$, so maybe
there is a different mistake, or the intention is different.) It might be
helpful to define $H^+$ first and then have a separate claim which shows
how to use paths in $H$ to get paths in $H^+$.

\begin{response}
  Thanks to the referee for catching this embarrassing mistake.  In our previous round of simplifications, we oversimplified here.  The existence of the directed path in $\overrightarrow{H}^+$ does not come from the shortcut $v_0,\ldots,v_r$.  Rather, it comes from some shortcut that contains $v_r$ and some vertex in $Y_a$ (which must exist since $a=a(v_r)$ so $v_r\in Y_a$ or some shortcut that contains $v_r$ crosses $a$).  We have rewritten this part accordingly.
\end{response}


2. Are the nations incident to a vertex counted with multiplicity in dmap graphs? The definition does not say so, but the proof of Lemma 27 (as currently stated) assumes not only this, but that the embedded graph under consideration is 2-connected.  It also seems that frames are loopless, so slightly more care is called for when they are constructed.

\begin{response}
  We have addressed this by adding the assumption that the underlying planar graph $G_0$ is edge-maximal. More precisely, it is not possible to add an edge to $G_0$ (and relabel faces as nation or lake) to obtain the same map graph $G$.  We then argue that this implies that $G_0$ is biconnected. 
\end{response}


It may be worth noting explicitly that you allow $d$-framed graphs to
have edges outside the outer face of their frame - I was unsure about
this given the phrase “interior of the face”.

\tba

3. Sperner’s Lemma is usually stated for simple graphs (as is the case
in the source given) so there should be a reduction argument. It is
more than just removing parallel edges because there may be separating
cycles of length two.

\begin{response}
  We have added a footnote here explaining that the proof of Sperner's Lemma is based on a certain graph defined on the faces of the embedding (triangles plus outer-face).  It relies only on the fact that the number of odd-degree vertices in this graph is even, which is true even if this graph is not simple.
\end{response}

Other comments:

1. There are some proofreading things to do - some lines overflow, there
are citations on a new line (instead of using tilde), and there are typos
like missing words (page 3, line 7, should read “H is a planar graph”).

\begin{response}
	We have fixed ``$H$ is a planar graph'', but do not see the point in fixing minor line overflows, since the paper will be reformatted in the journal style anyway.
\end{response}


2. Page 4, second bullet point. It says that you derive product structure
theorems for string graphs, which is clearly not possible. Please use
more precise language. This comment was in the previous review as
well. Some but not all instances have been corrected.

\begin{response}
  We have added the qualifier "bounded-degree" here.
\end{response}

3. page 6, line -6: The elements of P are not ordered so it is not a sequence.

\begin{response}
  We have replaced ``sequence'' with ``collection''.
\end{response}

4. Page 7, line 6: It is confusing to say that Sx “is defined by vertices
that participate in x” because it suggests that Sx is the set of vertices
which participate in x. Moreover, for the proven upper bound to be
enough, you need to show that the set of all nodes which participate in
x induces a subtree of T (and thus that a(v) is in this set).

\begin{response}
  As part of addressing the referee's primary comment, we have introduced an extra claim which shows that, for any $v\in V(G)$, there is a path in $H$ from $v$ to $a(v)$ of length at most $k-1$ in which $v$ participates in every vertex on this path, including $a(v)$.  We have rephrased this confusing sentence and replaced it with a reference to this claim.
\end{response}


5. The paper regularly says things like “let P be an H-partition” without
saying what H is. Typically this is understandable, but it could be
written more clearly. For instance, on page 8, line 2, it says “let Yxi
be the part in Y...” without saying what xi
is (and similarly in other
places). This shouldn’t be a serious hang-up for anyone, but still, it
would be better to acknowledge technical inaccuracies.

\begin{response}
  We have added a note at the beginning of Section 2.2 stating that, rather write ``$x$ is the unique vertex in $V(H)$ such that $v\in B_x$'' we write
  ``$P_x$ is the part in $\mathcal{H}$ that contains $v$.''  (Where the $H$-partition $\mathcal{H}:=\{B_x:\in V(H)\}$ has already been defined.)
\end{response}

6. page 8, Claim 3. The definition of Cx, a subset of V (J), includes things
which may not be vertices of J. This, of course, is not a real problem,
but the proof treats the fact that V (J) may be a proper subset of
V (T) sloppily. Also, the notation T[x] is unfortunately ambiguous - it
is strange to need to define something which has already been defined.

\begin{response}
  We have adjusted the definition of $C_x$ so that it only includes $x$ if $x\in V(J)$.  We agree that the overloading of the notation $T[\cdot]$ is slightly unfortunate, but can live with it since the two uses can be distinguished from the type of the argument.  For a single node $a$, $T[a]$ denote the subtree induced by all nodes whose bags contain $a$. For a set $S$, $T[S]$ is the subgraph of $T$ induced by $S$.
\end{response}

7. page 16, line 6: Say why that is a convenient way to establish the
planarity of H. (It does not actually say here that any such graph G
is planar.)

\begin{response}
	Added the desired sentence.
\end{response}

8. page 16, line 8. I would not call these “BFS spanning forests rooted
at $V_0$” since they need not be BFS spanning forests (unless the same
terminology has already appeared elsewhere?).

\begin{response}
  We now specify that this forest contains one tree rooted at each vertex of $V_0$.
\end{response}

9. page 16, Lemma 24. The lemma is rather long to state and I would
imagine that anyone reading it has already read the proof of the product
structure theorem for planar graphs. So I would add a sentence or two
beforehand about how the set-up relates to that proof/tripods.

\begin{response}
  We have added a couple of sentences here pointing to the exact lemma (Lemma~17) in the JACM paper and discussing the differences between the tripods here and there.
\end{response}

10. page 16, Lemma 24: Add “non-empty” before “sets $P_1,\dots,P_k$”.

\done

11. page 16, line -2. “$P_1, P_2, P_3$” should be $P_1,\dots,P_k$.

\done

12. page 17, top. In the first two cases there may not be a vertex $v_3$ defined
as we could have $p=2$.

\begin{response}
  We have added an explanation, prior to this, of how to eliminate the possibility that an edge of $F$ bounds a face with with only two vertices and two parallel edges.  Simply apply induction on the cycle $F'$ that uses the second of the parallel edges rather than the first.  This is valid since $F'$ has fewer faces in its interior.
\end{response}


13. page 17, right after 3. The end of Qi should be the same as the vertex w
chosen in case 3 because there may be multiple “appropriate” vertices.
It should also really say what “appropriate” means.

\begin{response}
  We have replaced "appropriate" with the "root of tree in $T$ that contains $v$".
\end{response}


14. page 17, line -9. “$\{x_1, x_2, x_3\}$” should be “$\{v_1, v_2, v_3\}$”

\done

15. page 19, second to last paragraph. Add a sentence about why such
a forest exists. The definition of a BFS layering is missing.


\begin{response}
  The more precise definition of ``BFS forest rooted $V_0$'' takes care of this.
\end{response}

A reader may expect that you will just use the assumptions about H and P that are stated, so it would be better to say something like “This gives an H-partition... and conditions (i), (ii), and (iii) of Lemma 24 are satisfied”.

\done


16. page 20, first paragraph of Lemma 25. Well, parallel edges can appear
on the boundary of a common face if they were there before.

\begin{response}
  We rephrased this to clarify that it only applies to newly-introduced edges.
\end{response}


Add a sentence about why the process ends.

\done

17. page 20, second to last paragraph. Figure 4 is mentioned several paragraphs before the graph which it depicts is defined. The graph $G_0$ in
this paragraph should be $G'$, and check for this throughout.

\begin{response}
  We have moved Figure 4 a bit later and, indeed, several occurrences of $G_0$ should have been $G'$.  These are corrected now.
\end{response}

18. page 21, Lemma 27. Should be “labelled a nation or a lake” in the
first paragraph. In the third paragraph, the graph $\hat{G}_1$ should be “edge-maximal” or similar.

\textcolor{orange}{DW: First point fixed. I have suggested words to add. }

\tba
\end{document}
