\appendix
\section{Framed Graphs}

\note{DW}{Work in progress}

\note{DW}{Use the language of framed graphs}

For a plane multigraph $G$, let $\widehat{G}$ be the embedded graph obtained from $G$ as follows: for each face $f$ of $G$ and for each pair of non-consecutive vertices $v$ and $w$ on the boundary of $f$, add the edge $vw$ to $\widehat{G}$ drawn across $f$.

\begin{thm}
\label{CliqueFaceProductStructure}
Let $G$ be a 2-connected plane multigraph, in which every facial cycle has length in $\{3,\dots,d\}$ for some fixed integer $d\geq 3$. Then $\widehat{G}$ is contained in $H \boxtimes P \boxtimes K_{ d +3 \floor{d/2}-3 }$ for some planar graph $H$ with $\tw(H) \leq 3$ and for some path $P$.
\end{thm}

Note that \cref{CliqueFaceProductStructure} with $d=3$ implies \cref{PlanarProduct}(b)  by \citet{DJMMUW20} (since every planar graph is a subgraph of a triangulation). Moreover, the proof of \cref{CliqueFaceProductStructure} is a direct generalisation of the proof of
\cref{PlanarProduct}(b).

For a cycle $C$, we write $C = [ P_1 , \dots , P_k ]$ if $P_1 , \dots, P_k$ are pairwise disjoint (possibly empty) path subgraphs in $C$ such that $V(C)=\bigcup_i V(P_i)$ and $P_1,\dots,P_k$ are ordered around $C$.

%and the endpoints of each path $P_i$ can be labelled $x_i$ and $y_i$ so that $y_ix_{i+1} \in E(C)$ for $i\in\{1,\dots,k\}$, where $x_{k+1}$ means $x_1$. This implies that $\{P_1,\dots,P_k\}$ is a vertex-partition of $C$. \note{DW}{In the proof below, when we apply induction, we need to allow $P_i=\emptyset$, right?}

If $Q$ is a path in a graph, and $P_1, \dots,P_k$ are edge-disjoint non-empty subpaths of $Q$ ordered by $Q$ and partitioning $E(Q)$, then we write $Q=P_1,\dots,P_k$. \note{DW}{Do we need this?}

We employ the following well-known variation of Sperner's Lemma (see~\citep{Proofs4}).

\begin{lem}[Sperner's Lemma]
\label{Sperner}
Let $G$ be a near-triangulation whose vertices are coloured $1,2,3$, with the outer-face $F=[P_1,P_2,P_3]$ where each vertex in $P_i$ is coloured $i$. Then $G$ contains an internal face whose vertices are coloured $1,2,3$. \note{DW}{Does this allow parallel edges: check the proof in \citep{Proofs4}.}
\end{lem}

%If $C$ is a cycle in a graph, and $P_1, \dots,P_k$ are edge-disjoint non-empty subpaths of $C$ ordered around $C$ and partitioning $E(C)$, then we write $C=(P_1,\dots,P_k)$.

Say $G$ is a 2-connected plane multigraph. Let $T$ be a spanning tree of $G$ rooted at a vertex $r$ on the outerface of $G$. A \defin{bipod} in $G$ (with respect to $T$) is any path $P_1SP_2$ (considered to be a subgraph), where $S$ is a facial path in $G$, and each $P_i$ is a vertical path in $T$ whose lower endpoint is in $S$. Here it is possible that $S$ or $P_i$ is empty. So every subpath of a bipod is a bipod. A \defin{tripod} in $G$ (with respect to $T$) is either a bipod in $G$ or a subgraph of $G$ consisting of a facial cycle $F$ of $G$, plus three distinct vertical paths $P_1,P_2,P_3$ in $T$ where the lower endpoint of each $P_i$ is in $F$. These definitions generalise definitions in \citep{DJMMUW20} where $G$ is assumed to be a plane triangulation, in which case $F$ is a 3-cycle. \note{DW}{We actually managed to omit the definition of `bipod' in \citep{DJMMUW20}!}

%In the original proof for planar triangulation, B_i is an edge.


For a cycle $C$ in a plane multigraph $G$, define \defin{$G\blah{C}$} to be the subgraph of $G$ consisting of $C$ and all the edges and vertices of $G$ in the interior of $C$.


\begin{lem}
\label{NewLemma}
Let $G_0$ be a 2-connected plane multigraph. Let $T$ be a spanning tree of $G_0$ rooted at some vertex $r$ on the boundary of the outerface of $G_0$. Let $C = [P_1,\dots,P_k]$ be a cycle in $G_0$ with $r$ in the exterior of $C$, where $k\in\{1,2,3\}$ and $P_1, \dots, P_k$ are bipods in $G_0$. Let $G:=G_0\blah{C}$. Then $G$ has a partition $\PP$ into tripods with respect to $T$ such that:
\begin{compactenum}[(i)]
	\item $P_1,\dots,P_k\in\PP$,
	\item $G / \PP$ has a tree-decomposition $\mathcal{T}$ of width at most 3,
	\item $G/\PP=\widehat{G}/\PP$.
%	 such that for every facial cycle $F$ of $G$ there is a bag $B$ of $\mathcal{T}$ such that every part of $\PP$ that intersects $F$ is in $B$.
\end{compactenum}
\end{lem}

\begin{proof}
We may assume that each of $P_1,\dots,P_k$ is non-empty (since if some $P_i$ is empty, then we just consider $C=[P_1,\dots,P_{i-1},P_{i+1},\dots,P_k]$).

The proof is by induction on the number of faces of $G$. Suppose $G$ has just one internal face (which is bounded by $C$). Let $\PP=\{P_1,\ldots,P_k\}$.
Then $G/\PP=\widehat{G}/\PP = K_k$. Then desired tree-decomposition of $G/\PP$ consists of a single bag $\{P_1,\ldots,P_k\}$ indexed by a 1-node tree. Now assume that $G$ has at least two internal faces.

First suppose that $C$ is a 2-cycle. Thus $k\leq 2$. Let $e$ be one of the edges in $C$. Let $F$ be the internal facial cycle of $G$ that includes $e$. Since each facial cycle of $G_0$ has length at least 3, $F$ consists of a non-empty path $P$ plus the edge $e$. Let $C'$ be obtained by replacing $e$ in $C$ by $P$. Let $G':=G\langle{C'}\rangle$. Thus $G'$ has one fewer face than $G$. Note that $C'=[P_1,\dots,P_k,P]$ (\note{DW}{or it can be written that way?}).  By induction, $G'$ has a partition $\PP$ into tripods with respect to $T$ such that $P_1,\dots,P_k,P\in\PP$, and $G / \PP$ has a tree-decomposition $\mathcal{T}$ of width at most 3 such that for every facial cycle $F'$ of $G$ there is a bag $B$ of $\mathcal{T}$ such that every part of $\PP$ that intersects $F'$ is in $B$. The bag of $\mathcal{T}$ that contains $P_1,\dots,P_k,P$ satisfies this property for $F$. Thus $\PP$ and $\TT$ satisfy the desired properties for $G$. Now assume that $|V(C)|\geq 3$.

%\note{DW}{One option is to do induction on $|V(G)\setminus V(C)|-k$ and reduce to the $k=3$ case (similar to the $|V(C)|\leq 2$ case). When $k<3$ we can grab another face $F$ on the inside of $C$ and make the boundary of $F$ a new path. Then the main proof will only deal with the $k=3$ case. Is this any easier?}

%Now suppose that $k=1$. Let $v$ and $w$ be the first and last vertex in $P_1$. So $C$ consists of $P_1$ plus the edge $vw$. Let $F$ be the internal facial cycle of $G$ containing $vw$.......... Now suppose that $k=2$ .....

Partition the vertices of $C$ into paths $R_1$, $R_2$, and $R_3$ as follows:
\begin{compactenum}
	\item If $k=1$ then, $P_1$ has at least three vertices (since $|V(C)|\geq 3$), so $P_1=[v, P_1', w]$ for two distinct vertices $v$ and $w$. We set $R_1:=v$, $R_2:=P_1'$ and $R_3:=w$.

	\item If $k=2$ then we may assume without loss of generality that $P_1$ has at least two vertices so $P_1=[v,P_1']$. We set $R_1:=v$, $R_2:=P_1'$ and $R_3:=P_2$.

	\item If $k=3$ then let $R_i:=P_i$ for each $i\in\{1,2,3\}$.
\end{compactenum}

Note that $R_i \subseteq P_{\alpha(i)}$ for some $\alpha(i)\in \{1,\dots,k\}$.

Let $G'$ be a near-triangulation obtained from $G$ by triangulating every internal face. For $i\in\{1,2,3\}$, colour each vertex in $R_i$ by $i$. Now, for each remaining vertex $v$ in $G$, consider the path $P_v$ from $v$ to the root of $T$. Since $r$ is on the outer-face of $G_0$, $P_v$ contains at least one vertex of $C$. If the first vertex of $P_v$ that belongs to $F$ is in $R_i$, then assign the colour $i$ to $v$. We obtain a 3-colouring of the vertices of $G'$ that satisfies the conditions of Sperner's Lemma. By Sperner's Lemma there exists a triangular face $\tau=v_1v_2v_3$ of $G'$ whose vertices are coloured $1,2,3$ respectively. Let $F$ be the cycle bounding the face of $G$ containing $\tau$.

For each $i\in\{1,2,3\}$, let $Q_i$ be the path in $T$ from $v_i$ to the first ancestor $v_i'$ of $v_i$ in $T$ that is contained in $C$. Observe that $Q_1$, $Q_2$, and $Q_3$ are disjoint since $Q_i$ consists only of vertices coloured $i$. Let $Q_i'$ be $Q_i$ minus its final vertex $v_i'$. Note that $Q_i$ may consist of the single vertex $v_i=v_i'$, in which case $Q_i'$ is empty. Imagine for a moment that the cycle $C$ is oriented clockwise, which defines an orientation of $R_1$, $R_2$ and $R_3$. Let $R_i^-$ be the subpath of $R_i$ that contains $v'_i$ and all vertices that precede it, and let $R_i^+$ be the subpath of $R_i$ that contains $v'_i$ and all vertices that succeed it.

Let $Y$ be the tripod $F\cup Q_1'\cup Q_2' \cup Q_3'$. Say $F=v_1 S_1 v_2 S_2 v_3 S_3$, where $S_i$ is a facial path in $F$ clockwise from $v_i$ to $v_{i+1}$ (where $v_4=v_1$). Let $L_i$ be the path $Q_i'S_iQ_{i+1}'$ for $i\in\{1,2,3\}$ (where $Q_4=Q_1$ and $R_4=R_1$). So $L_i$ is a bipod. \note{DW}{Note that $S_i\neq\emptyset$ since $v_i\neq v_{i+1}$. Is this useful.}

%Note that $C_i$ may be \defin{degenerate} in the sense that $[Q_i',R_i^+,R_{i+1}^-,Q_{i+1}']$ may consist only of a single edge $v_iv_{i+1}$.

% Consider any non-degenerate $C_i=[Q_i',R_i^+,R_{i+1}^-,Q_{i+1}']$. Note that these four paths  are pairwise disjoint, and thus $C_i$ is a cycle. If  $Q_i'$ and $Q_{i+1}'$ are non-empty, then each is a vertical path in $T$.  Furthermore, each of $R_i^-$ and $R_{i+1}^+$ consists of at most two vertical paths in $T$.  Thus, $C_i$ is the concatenation of at most six vertical paths in $T$ \note{DW}{re-do}.  Let $G_i$ be the subgraph of $G_0$ consisting of all the edges and vertices of $G_0$ contained in $C_i$ and the interior of $C_i$.  Observe that $G_i$ contains $v_i$ and $v_{i+1}$ but not the third vertex of $\tau$.

Let $Z$ be the plane graph $F\cup C\cup Q_1\cup Q_2\cup Q_3$. Then $Z$ is 2-connected, each face of $Z$ is bounded by a cycle, and the outerface of $Z$ is $C$. Let $Z_1,\dots,Z_m$ be the internal facial cycles of $Z$ except $F$.


Consider $j\in\{1,\dots,m\}$. Let $G_j$ be the subgraph of $G$ consisting of $Z_j$ and all the edges and vertices of $G$ in the interior of $Z_j$. The outerface of $G_j$ is $Z_j$. Observe that for some $i\in\{1,2,3\}$, we have
$Z_j=[R_{i,j}^+,R_{i+1,j}^-,S_{i,j}]$, where
$R_{i,j}^+$ is some subpath of $R_i^+$,
$R_{i+1,j}^-$ is some subpath of $R_{i+1}^-$, and
$L_{i,j}$ is the path $Z_j-V(C)$ (which is contained in $Y$). \note{DW}{Add a figure of a good case and a weird case.}
Since $R_i^+$, $R_{i+1}^-$ and $L_i$ are bipods,
$R_{i,j}^+$, $R_{i+1,j}^-$ and $L_{i,j}$ are bipods.
By construction, $v_{i-1}$ is not in $G_j$ (where $v_0=v_3$).
Thus $V(G_j)\setminus V(Z_j) \subsetneq V(G)\setminus V(C)$, and we may apply induction to $G_j$.  Thus $G_j$ has a partition $\PP_j$ into tripods with respect to $T$ such that
\begin{compactenum}[(i)]
	\item
	$R_{i,j}^+, R_{i+1,j}^-, L_{i,j}\in\PP_j$,
	\item $G_j / \PP_j$ has a tree-decomposition $\mathcal{T}_j=\{B_x:x\in V(D_j)\}$ of width at most 3,
	\item $G_j / \PP=\widehat{G_j}/\PP$.
	%	 such that for every facial cycle $F$ of $G$ there is a bag $B$ of $\mathcal{T}$ such that every part of $\PP$ that intersects $F$ is in $B$.
\end{compactenum}


%This graph has up to three \note{DW}{can be more} internal faces $C_1,C_2,C_3$ where each $C_i=[Q_i',R_i^+,R_{i+1}^-,Q_{i+1}']$ and $R_{i}^+$ and $R_{i}^-$ are the same portions of $R_i$ as defined in \cref{NearTriang} \note{DW}{move in here}    Observe that  $C_i=[ R_i^+ , R_{i+1}^-, I_i ]$, where $R_i^+$ and $R_{i+1}^-$ are bipods, and $I_i$ is the bipod $Q_i' L_i Q_{i+1}'$. Let $G_i$ be the subgraph of $G$ whose vertices and edges are in $C_i$ or its interior.

%For $i\in\{1,2,3\}$, if $C_i$ is non-empty, then $G_i$ and $C_i=[ R_i^+ , R_{i+1}^-, I_i ]$ satisfy the conditions of the lemma, and we may apply induction to $G_i$.  (Note that one or two of $R_i^+$, $R_{i+1}^-$ and $I_i$ may be empty, in which case we apply the inductive hypothesis with $k=2$ or $k=1$, respectively.)\  This gives a partition $\PP_i$ of $G_i$ such that $H_i:= G_i / \PP_i$ satisfies the conclusions of the lemma. Let $(B^i_x:x\in V(J_i))$ be a tree-decomposition of $H_i$, in which every bag has size at most 4, and some bag $B^i_{u_i}$ contains the vertices of $H_i$ corresponding to  $R_i^+$, $R_{i+1}^-$ and $I_i$ (if they are non-empty).  We do this for each non-degenerate $C_i$.

We now construct the desired partition $\PP$ of $G$.
Let $\PP$ be obtained from $\bigcup_{j\in\{1,\dots,m\}} \PP_j$ by
replacing $R_{i,j}^+\in\PP_j$ by $P_{\alpha(i)}$,
replacing $R_{i+1,j}^-\in\PP_j$ by $P_{\alpha(i+1)}$, and
replacing $L_{i,j}\in\PP_j$ by $Y$.
By construction, $\PP$ partitions $V(G)$ into tripods with respect to $T$ and $P_1,\ldots,P_k\in \PP$.

%Since $P_1,
%Since $C$ is a facial cycle of $G_j$, there is a bag $B_{u_j}$ of $G_j$ such that $P_1,\dots,P_k\in B_{u_j}$.


We now construct the desired tree-decomposition of $G/\PP$. Let $D$ be the tree obtained from the disjoint union of $D_1,\dots,D_m$ by adding one new node $u$ adjacent to $u_1,\dots,u_m$. Define $B_u:=\{P_1,\dots,P_k,Y\}$. Apply the above replacements within the bags of $\mathcal{T}_1,\dots,\mathcal{T}_m$. We claim that $\mathcal{T}:=\{B_x:x\in V(D)\}$ is the desired tree-decomposition of $G/\PP$. By construction, every bag of $\mathcal{T}$ has size at most 4.

First we show that for each part $X\in\PP$, the set $\{x\in V(D) : X\in B_x\}$ induces a subtree of $D$. If $X\not\in \{P_1,P_2,P_3,Y\}$ then $X$ is only in bags in some $D_j$, implying $\{x\in V(D) : X\in B_x\}$ induces a subtree of $D$. If $X\in \{P_1,P_2,P_3,Y\}$ then $X$ is in $B_u$, and if some part $X'$ of some $\PP_j$ was replaced by $X$ in the above construction, then $X'$ is in $B_{u_j}$ and $\{x\in V(D_j):X'\in B_x\}$ induces a subtree of $D_j$ (since $\mathcal{T}_j$ is a tree-decomposition of $G_j$). Thus $\{x\in V(D) : X\in B_x\}$ induces a subtree of $D$ (since $uu_j\in E(D)$).

We now show that, for every edge $XW$ of $G/\PP$, there is a node $x\in V(D)$ such that $X,W\in B_x$. If $X,W\in \{P_1,P_2,P_3,Y\}$ then $X,W\in B_u$ as desired. Otherwise, without loss of generality, $X\in \PP_j\setminus \{P_1,P_2,P_3,Y\}$ for some $j\in\{1,\dots,m\}$.
Thus $X\subseteq V(G_j)\setminus V(Z_j)$. Since $Z_j$ separates $V(G_j)\setminus V(Z_j)$ and $V(G)\setminus V(G_j)$ and $XW\in E(G/\PP)$, it must be that $W$ is also in $\PP_j$. Say $W$ replaced $W'$ in $\PP_j$ (where possibly $W=W'$). Since $\mathcal{T}_j$ is a tree-decomposition of $G_j$, $XW'\in B_x$ for some node $x\in V(D_j)$, implying $XW\in B_x$. This concludes the proof that $(B_x:x\in V(D))$ is the desired tree-decomposition of $G/\PP$.

For every facial cycle $F_0$ of $G$, either $F_0=F$ or $F_0=C$ or $F_0$ is a facial cycle of some $G_j$. If $F=F_0$ then the only part of $\PP$ that intersects $F$ is $Y$, so $B_u$ is a bag of $\mathcal{T}$ such that every part of $\PP$ that intersects $F$ is in $B$. If $F_0=C$ then $P_1,\dots,P_k$ are the only parts of $\PP$ that intersect $F_0$, so $B_u$ is a bag of $\mathcal{T}$ such that every part of $\PP$ that intersects $F$ is in $B$.

In the first case, there is a bag $B_x$ of $\mathcal{T}_j$ such that $V(F)\subseteq \bigcup_{X\in B_x} X$. This property is maintained by the above replacements since
$R_{i,j}^+ \subseteq P_{\alpha(i)}$ and
$R_{i+1,j}^- \subseteq P_{\alpha(i+1)}$ and
$L_{i,j}\subseteq Y$.

%For each non-degenerate $F_i$, and for each node $x\in V(J_i)$, initialise $B_x:= B^i_x$.
%Recall that vertices of $H_i$ correspond to contracted paths in  $\PP_i$.
%Each internal path in $\PP_i$ also lies  in $\PP$.
%Each external path $P$ in $\PP_i$ is a subpath of $P_j$ for some $j\in\{1,\dots,k \}$ or is one of the paths among $Q'_1, Q'_2, Q'_3$.
%For each such path $P$, for every $x\in V(J)$, in bag $B_x$,
%replace each instance of the vertex of $H_i$ corresponding to $P$ by the vertex of $H$ corresponding to the path among $P_1,\ldots, P_k, Q'_1,\ldots,Q'_3$
%that contains $P$.
%This completes the description of $(B_x : x\in V(J))$.
%By construction, $|B_x|\leq 9$ for every $x\in V(J)$.


%We construct $\PP$ as before. Initialise $\PP:=\{P_1,\ldots,P_k,Y\}$. Then, for $i\in \{1,2,3\}$, each tripod in  $\PP_i$ is either fully contained in $F_i$ or it is \emph{internal} with none of its vertices in $F_i$. Add all these internal tripods in  $\PP_i$ to $\PP$. By construction, $\PP$ partitions $V(G)$ into tripods. The graph $H:= G/ \PP$ is planar since $G$ is planar and each tripod in $\PP$ induces a connected subgraph of $G$.

%Next we produce the tree-decomposition $(B_x:x\in V(J))$ of $H$ that satisfies the requirements of the lemma.  Let $J$ be the tree obtained from the disjoint union of $J_1$, $J_2$ and $J_3$ by adding one new node $u$ adjacent to $u_1$, $u_2$ and $u_3$. Let $B_u$ be the set of at most four vertices of $H$ corresponding to $Y,P_1,\dots,P_k$. For $i\in\{1,2,3\}$ and for each node $x\in V(J_i)$, initialise $B_x:= B^i_x$.

%As in  the proof of \cref{NearTriang},  the resulting structure,  $(B_x:x\in V(J))$, is not yet a tree-decomposition of $H$ since some bags may  contain vertices of $H_i$ that are not necessarily vertices of $H$. Note that unlike in  \cref{NearTriang} this does not only include elements of $\PP_i$  that are contained in $F$. In particular, $I_i$ is also not an element of $\PP$ and thus does not correspond to a vertex of $H$.   We remedy this  as follows. For $x\in V(J)$, in bag $B_x$, replace each instance of the vertex of $H_i$ corresponding to $I_i$ by the vertex of $H$ corresponding to $Y$. Similarly, by construction, $R_i^+$ is a subgraph of $P_{\alpha_i}$ for some $\alpha_i\in \{1,\dots,k\}$.  For $x\in V(J)$, in bag $B_x$, replace each instance of the vertex of $H_i$ corresponding to $R_i^+$  by the vertex of $H$ corresponding to $P_{\alpha_i}$. Finally, $R_{i+1}^-$ is a subgraph of $P_{\beta_i}$ for some $\beta_i\in \{1,\dots,k\}$. For $x\in V(J)$, in bag $B_x$, replace each instance of the vertex of $H_i$ corresponding to $R_{i+1}^-$  by the vertex of $H$ corresponding to $P_{\beta_i}$.

This shows that $(B_x:x\in V(J))$ is the desired tree-decomposition of $G/\PP$.
\end{proof}

\begin{proof}[Proof of \cref{CliqueFaceProductStructure}]
Let $G$ be a 2-connected plane multigraph, in which every facial cycle has length in $\{3,\dots,d\}$. Let $C$ be the outer-cycle of $G$. Let $G^+$ be obtained from $G$ by adding a new vertex $r$ adjacent to every vertex on the outerface of $G$. So $G^+$ is a 2-connected plane multigraph and $G=G^+\blah{C}$. Let $T$ be a BFS spanning tree of $G^+$ rooted at $r$. Let $\mathcal{L}=(L_0,L_1,\dots)$ be the corresponding BFS layering. By \cref{NewLemma}, $G$ has a partition $\PP$ into tripods with respect to $T$ such that $G / \PP$ has a tree-decomposition $\mathcal{T}$ of width at most 3 such that for every facial cycle $F$ of $G$ there is a bag $B$ of $\mathcal{T}$ for which every part of $\PP$ that intersects $F$ is in $B$. Since $G$ is a spanning subgraph of $\widehat{G}$, we may consider $\PP$ to be a partition of $\widehat{G}$, and $G/\PP \subseteq \widehat{G}/\PP$. Consider an edge $XY$ of $\widehat{G}/\PP$. By the definition of $\widehat{G}$, there is a facial cycle $F$ of $G$ such that both $X$ and $Y$ intersect $F$. By assumption, there is a bag $B$ of $\mathcal{T}$ such that every part of $\PP$ that intersects $F$ is in $B$. In particular, $X,Y\in B$. Hence $\TT$ is also a tree-decomposition of $\widehat{G}/\PP$. For each edge $vw$ of $\widehat{G}$, if $v\in L_i$ and $w\in L_j$ then $|i-j|\leq\floor{\frac{d}{2}}$. Grouping blocks of $\floor{\frac{d}{2}}$ consecutive layers in $\mathcal{L}$ gives a layering $\mathcal{L}'$ of $\widehat{G}$. Observe that each tripod in $\PP$ has at most $d+3\floor{\frac{d}{2}}-3$ vertices in each set of $\floor{\frac{d}{2}}$ layers of $\mathcal{L}$. Hence $\mathcal{L}'$ has layered width $d+3( \floor{\frac{d}{2}}-1)$ with respect to $\PP$. By \cref{PartitionProduct}, $\widehat{G}$ is contained in $H \boxtimes P \boxtimes K_{d+3\floor{d/2}-3}$ where $H:=\widehat{G}/\PP$, which has treewidth 3. \note{DW}{This proof needs to be changed to show that $G/\PP = \widehat{G}/\PP$, so $H$ is a minor of $G$ and is thus planar.}
\end{proof}

We now give two graph classes for which \cref{CliqueFaceProductStructure} is applicable.

\begin{lem}
	An embedded multigraph $G$ is edge-maximal 1-plane with no faces of length 2 if and only if there is a 2-connected plane multigraph $G_0$ such that:
	\begin{compactitem}
		\item each face of $G_0$ is bounded by a 3-cycle or a 4-cycle,
		\item no edge of $G_0$ is in the boundary of two facial 3-cycles,
		\item $G=\widehat{G_0}$.
	\end{compactitem}
\end{lem}

\begin{proof}
\end{proof}

\begin{cor}
	\label{New1Planar}
	Every 1-planar graph is contained in $H \boxtimes P \boxtimes K_7$ for some graph $H$ with $\tw(H) \leq 3$ and for some path $P$.
\end{cor}


\end{document}
