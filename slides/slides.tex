\documentclass[xcolor=dvipsnames]{beamer}
\usepackage{ods}
%\usepackage{ods-figs}
\usepackage[cm]{sfmath}
\usepackage[utf8]{inputenc}
\usepackage{array}
%\usepackage{enumitem}
%\usepackage{enumitem}
%\setitemize{itemsep=1.5ex}
%\setlength{\leftmargini}{0pt}
\usepackage{array}

\newcommand{\caressed}{$\kappa$aressed}
\newcommand{\caresses}{$\kappa$aresses}
\newcommand{\R}{\mathbb{R}}
\newcommand{\dual}[1]{#1^\star}
\newcommand{\Fary}{F\'ary}

\DeclareMathOperator{\tw}{tw}

\title{Graph Product Structure for Non-Minor-Closed Classes}
\author{Vida Dujmović \and Pat Morin \and David R. Wood}
\titlegraphic{\includegraphics[height=1em]{by}}

\begin{document}

\begin{frame}
  \titlepage
\end{frame}

\begin{frame}
    \frametitle{Outline}

    \begin{itemize}
        \item $k$-planar graphs
        \item $(k,d)$-shortcut systems
        \item statement of main theorem
        \item applications
        \item proof of main theorem
        \item open problem
    \end{itemize}
\end{frame}

\begin{frame}
    \frametitle{Motivation: $k$-Planar Graphs}
    \framesubtitle{$k$-planar: locally non-planar}

    \begin{itemize}
        \item $G$ is \emph{$k$-planar} if it has a drawing with $\le\!\! k$ crossings per edge
        \begin{center}
            \multiinclude[<+>][format=pdf,start=1]{figs/k-planar}%
        \end{center}
        \item<4-> Is $G\subseteq H\boxtimes P$ where $\tw(H)\le f(k)$?
    \end{itemize}
\end{frame}

\begin{frame}
    \frametitle{Abstraction: Shortcut Systems}

    \begin{itemize}
        \item \emph{$(k,d)$-shortcut system} for $G$:
        \begin{itemize}
            \item<2-> a set $\mathcal{P}$ of length-$\le\!\! k$ paths in $G$
            \item<2-> each $v\in V(G)$ appears interior to at most $d$ paths in $\mathcal{P}$
        \end{itemize}
        \begin{center}
            \multiinclude[<+>][format=pdf,start=1]{figs/shortcut-system}%
        \end{center}
        \item<3-> $G^{\mathcal{P}}:= G\cup\{vw:\mbox{$v$ and $w$ are endpoints of a path in $\mathcal{P}$}\}$
    \end{itemize}
\end{frame}


\begin{frame}
    \frametitle{Observation}
    \framesubtitle{Crossed edges are just shortcuts in a planar graph}

    If $G'$ is $k$-planar then $G'\subseteq G^{\mathcal{P}}$ where
    \begin{itemize}
        \item $G$ is planar
        \item $\mathcal{P}$ is a $(k+1,2)$-shortcut system
    \end{itemize}
    \begin{center}
        \multiinclude[<+>][format=pdf,start=1]{figs/kgp}%
    \end{center}
\end{frame}

\begin{frame}
  \frametitle{Main Theorem}
  \framesubtitle{If $G$ has product structure then so does $G^{\mathcal{P}}$}
  \textbf{Main Theorem:} \newline
  If $G\subseteq H\boxtimes P\boxtimes K_\ell$
  \begin{itemize}
    \item  $\tw(H)\le t$
    \item $\mathcal{P}$ is a $(k,d)$-shortcut system for $G$
  \end{itemize}
  then $G^\mathcal{P}\subseteq H'\boxtimes P\boxtimes K_{\ell'}$
  \begin{itemize}
    \item $\tw(H') \le t'=\binom{k+t}{t}-1$ and
    \item $\ell'=d\ell(k^3+3k)$
    % \item $K'$ is a clique of order $d\ell(k^3+3k)$
  \end{itemize}
  \vspace{1cm}
  \uncover<2->{\textbf{Example:} $G'$ is ${\color{red} k}$-planar
    \begin{itemize}
      \item $t=3$, $\ell=3$, $d=2$, $k={\color{red}k}+1$
      \item $t'=O(k^3)$, $\ell'=O(k^3)$ 
    \end{itemize}
  }
\end{frame}



\begin{frame}
    \frametitle{Applications}
    \framesubtitle{(Besides $k$-planar graphs)}

    \textbf{($\mathcal{G}$) Graphs:} any subgraph of $G^\mathcal{P}$ where $G$ has \emph{product structure}
    \begin{itemize}
        \item $(g,k)$-planar graphs
        \item $(g,\delta)$-string graphs
        \item $k$-nearest neighbour graphs of $2d$ point sets
        \item bounded-degree graphs from proper minor-closed families
        \item any $G^k$ where $G$ has bounded-degree and product structure
    \end{itemize}

    \textbf{($\mathcal{A}$) Applications:}
    \begin{itemize}
        \item queue layouts
        \item non-repetitive colouring
        \item $p$-centered colouring and $\ell$-vertex ranking
        \item universal graphs
    \end{itemize}

    Anything in $\mathcal{G}\times \mathcal{A}$
\end{frame}

\section{Proof of Main Theorem}

\begin{frame}
  \frametitle{Proof: The Picture}
  \begin{itemize}
    \item $G= H\boxtimes P\boxtimes K_{\ell}$
  \end{itemize}
\end{frame}

\begin{frame}
  \frametitle{Shortcuts over multiple layers}
  
  \begin{itemize}
    \item Shortcuts in $\mathcal{P}$ may jump $k$ layers
    \item Easy solution: Compress $k$ layers into one
    \begin{itemize}
      \item Length $k-1$ path $P'$ in $P$ compresses down to vertex
      \item $P'\boxtimes K_{\ell}\subseteq K_{\ell k}$
    \end{itemize}
    \item Increases $\ell$ by a factor of $k$
    \item $\mathcal{P}$ is a $(k,d)$-shortcut system over $H\boxtimes P\boxtimes K_{\ell k}$ with all shortcuts crossing $\le\!\! 1$ layers
  \end{itemize}
\end{frame}

\begin{frame}
  \frametitle{Projecting shortcuts onto $H\boxtimes P$}
    
  \begin{itemize}
    \item Contract each copy of $K_{k\ell}$, so we are left with $H\boxtimes P$
    \item Each shortcut in $\mathcal{P}$ projects onto a walk in $H\boxtimes P$
    \item Each vertex of $H\boxtimes P$ is used in at most $dk\ell$ shortcuts
    \item Studying a $(k,dk\ell)$-shortcut system $\mathcal{P'}$ over $G'=H\boxtimes P$
  \end{itemize}
\end{frame}


\begin{frame}
  \frametitle{Nice Tree Decompositions}
  
  
  \begin{itemize}
    \item A \emph{nice} rooted tree decomposition $\mathcal{T}=(B_x:x\in V(T))$ of $H$ has 
    \begin{itemize}
      \item $V(T) = V(H)$
      \item $x$ is lowest common ancestor of $\{x'\in V(T):x\in B_{x'}\}$
    \end{itemize}
    \begin{center}
      \includegraphics{figs/nice-td}
    \end{center}
    \item If $xy\in E(H)$ then $x\prec_T y$ or $y\prec_T x$
  \end{itemize}
\end{frame}


\begin{frame}
  \frametitle{Alternative View}
  
  \begin{itemize}
    \item<2-> Start with $B_x\gets\{x\}$ for each $x\in V(T)$
    \item<3-> If $xy\in E(H)$ and $x\prec_T y$ then drag $x$ down into $B_y$
  \end{itemize}
  \begin{center}
    \multiinclude[<+>][format=pdf,start=1]{figs/alternative-view}%
  \end{center}
\end{frame}


\begin{frame}
  \frametitle{A Useful Lemma}
  
  \textbf{Lemma (Pilipczuk--Siebertz):}\footnote{Yes, the same paper!} For any $z\in V(H)$ the number of $x\in V(H)$ such that $x\prec_T z\prec_T z'$ and $\mathop{dist}_H(x,z')\le k$ is at most $\binom{k+t}{t}$.
  \begin{center}
    \includegraphics{figs/pz}%
  \end{center}
\end{frame}




\begin{frame}
  \frametitle{Projecting shortcuts onto $T$}
  
  \begin{itemize}
    \item A shortcut in $I\in\mathcal{P'}$
    \begin{itemize}
      \item<2-> is a path $I$ in $G'=H\boxtimes P$
      \item<3-> $I$ projects onto a walk $W$ in $H$
      \item<4-> $W$ is a sequence $S_I$ of nodes in $T$
    \end{itemize}
  \end{itemize}
  \begin{center}
    \multiinclude[<+>][format=pdf,start=2,end=5]{figs/nice-td}%
  \end{center}
\end{frame}

\begin{frame}
  \frametitle{Lifting in $T$}
  \begin{itemize}
    \item For $v$ in $G$, $a(v)$ is the LCA of $\cup\{S_I:I\in\mathcal{P'},\, v\in V(I)\}$
    \item For $x\in V(T)$, $S_x:=\{v\in V(G): a(v)=x\}$
    \begin{center}
      \includegraphics{figs/nice-td-6}%
    \end{center}
    \item Note: $\mathcal{S}:=(S_x:x\in V(T))$ is a partition of $V(G)$
\end{itemize}
\end{frame}


\begin{frame}
  \frametitle{Now the Crazy Part}
  
    Let $H':=G/\mathcal{S}$
    
    \textbf{Claim:} $\tw(H')\le\binom{k+t}{t}-1$
    
    \textit{Proof:}
    \begin{itemize}
      \item Each $x\in V(H')$ is obtained by contracting $S_x$
      \item Use tree nice tree decomposition with the same tree $T$
      \item If $xz\in E(H')$ and $x\prec_T z$ then drag $x$ down into $B_z$
      \item Point: If $x\in B_{z'}$ then $x\prec_T z'\prec_T z$ and $\mathop{dist}_{H}(x,z)\le k$.  
      \item By Pilipczuk-Siebertz Lemma, $|B_{z'}|\le \binom{k+t}{t}$.
    \end{itemize}
\end{frame}




\begin{frame}
  \begin{itemize}
    \item each $S_x$ contains $O(1)$ vertices from each row of $H\boxtimes P$. (Easy, each vertex in $S_x$ from row $i$ comes from a shortcut through a vertex of $Y_x$ in row $i-1$, $i$, or $i+1$, so $3dk\ell(k+1)=3d\ell(k^2+k)$) 
  \end{itemize}
\end{frame}
  



\end{document}
