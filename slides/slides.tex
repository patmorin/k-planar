\documentclass[xcolor=dvipsnames]{beamer}
\usepackage{ods}
%\usepackage{ods-figs}
\usepackage[cm]{sfmath}
\usepackage[utf8]{inputenc}
\usepackage{array}
%\usepackage{enumitem}
%\usepackage{enumitem}
%\setitemize{itemsep=1.5ex}
%\setlength{\leftmargini}{0pt}
\usepackage{array}

\newcommand{\caressed}{$\kappa$aressed}
\newcommand{\caresses}{$\kappa$aresses}
\newcommand{\R}{\mathbb{R}}
\newcommand{\dual}[1]{#1^\star}
\newcommand{\Fary}{F\'ary}

\DeclareMathOperator{\tw}{tw}

\title{Graph Product Structure for Non-Minor-Closed Classes}
\author{Vida Dujmović \and Pat Morin \and David R. Wood}
\titlegraphic{\includegraphics[height=1em]{by}}

\begin{document}

\begin{frame}
  \titlepage
\end{frame}

\begin{frame}
    \frametitle{Outline}

    \begin{itemize}
        \item $k$-planar graphs
        \item $(k,d)$-shortcut systems
        \item statement of main theorem
        \item applications
        \item proof of main theorem
        \item open problem
    \end{itemize}
\end{frame}

\begin{frame}
    \frametitle{Motivation: $k$-Planar Graphs}
    \framesubtitle{$k$-planar: locally non-planar}

    \begin{itemize}
        \item $G$ is \emph{$k$-planar} if it has a drawing with $\le\!\! k$ crossings per edge
        \begin{center}
            \multiinclude[<+>][format=pdf,start=1]{figs/k-planar}%
        \end{center}
        \item<4-> Is $G\subseteq H\boxtimes P$ where $\tw(H)\le f(k)$?
    \end{itemize}
\end{frame}

\begin{frame}
    \frametitle{Abstraction: Shortcut Systems}

    \begin{itemize}
        \item \emph{$(k,d)$-shortcut system} for $G$:
        \begin{itemize}
            \item<2-> a set $\mathcal{P}$ of length-$\le\!\! k$ paths in $G$
            \item<2-> each $v\in V(G)$ appears interior to at most $d$ paths in $\mathcal{P}$
        \end{itemize}
        \begin{center}
            \multiinclude[<+>][format=pdf,start=1]{figs/shortcut-system}%
        \end{center}
        \item<3-> $G^{\mathcal{P}}:= G\cup\{vw:\mbox{$v$ and $w$ are endpoints of a path in $\mathcal{P}$}\}$
    \end{itemize}
\end{frame}


\begin{frame}
    \frametitle{Observation}
    \framesubtitle{Crossed edges are just shortcuts in a planar graph}

    If $G'$ is $k$-planar then $G'\subseteq G^{\mathcal{P}}$ where
    \begin{itemize}
        \item $G$ is planar
        \item $\mathcal{P}$ is a $(k+1,2)$-shortcut system
    \end{itemize}
    \begin{center}
        \multiinclude[<+>][format=pdf,start=1]{figs/kgp}%
    \end{center}
\end{frame}

\begin{frame}
    \frametitle{Main Theorem}
    \framesubtitle{If $G$ has product structure then so does $G^{\mathcal{P}}$}
    \textbf{Main Theorem:} If
    \begin{itemize}
        \item $G\subseteq H\boxtimes P\boxtimes K$
        \begin{itemize}
            \item $\tw(H)=t$
            \item $K$ is a clique of order $\ell$
        \end{itemize}
        \item $\mathcal{P}$ is a $(k,d)$-shortcut system
    \end{itemize}
    then $G^\mathcal{P}\subseteq H'\boxtimes P\boxtimes K'$
    \begin{itemize}
        \item $\tw(H') \le \binom{k+t}{t}-1$ and
        \item $K'$ is a clique of order $d\ell(k^3+3k)$
    \end{itemize}
\end{frame}

\begin{frame}
    \frametitle{Applications}

    \textbf{($\mathcal{G}$) Graphs:}
    \begin{itemize}
        \item Any subgraph of $G^\mathcal{P}$ where $G$ has \emph{product structure}
        \begin{itemize}
            \item $(g,k)$-planar graphs
            \item $(g,\delta)$-string graphs
            \item $k$-nearest neighbour graphs of points in $2d$
            \item bounded-degree graphs from proper minor-closed families
            \item any $G^k$ where $G$ has bounded-degree and product structure
        \end{itemize}
    \end{itemize}

    \textbf{($\mathcal{A}$) Applications:}
    \begin{itemize}
        \item queue layouts
        \item non-repetitive colouring
        \item $p$-centered colouring and $\ell$-vertex ranking
        \item universal graphs
    \end{itemize}

    Anything in $\mathcal{G}\times \mathcal{A}$
\end{frame}



\end{document}

\begin{frame}
  \frametitle{Proper Good Curves}

  \begin{itemize}
     \item A Jordan curve $C$ is
     \begin{itemize}
        \item \emph{good} if it contains a point \emph{in} the outer face of $G$
        \item \emph{proper} if the intersection of $C$ with each edge of $G$ is
         \begin{itemize}
            \item empty; or
            \item a single point (maybe a vertex); or
            \item the entire edge.
         \end{itemize}
     \end{itemize}
  \end{itemize}
  \begin{center}
    \includegraphics[width=.22\textwidth]{figs/pgc-2}
    \includegraphics[width=.22\textwidth]{figs/pgc-3}
    \includegraphics[width=.22\textwidth]{figs/pgc-4}
    \includegraphics[width=.22\textwidth]{figs/pgc-5} \\
    \includegraphics[width=.22\textwidth]{figs/pgc-6}
    \includegraphics[width=.22\textwidth]{figs/pgc-7}
    \includegraphics[width=.22\textwidth]{figs/pgc-8}
  \end{center}
\end{frame}

\begin{frame}
  \frametitle{Free Sets}

   \begin{itemize}[<+->]
     \item A set $F\subseteq V(G)$ is \emph{free} if for every
     $X\subset\R^2$ with $|X|=|F|$, $G$ has a \emph{non-crossing
     straight-line} drawing in which the vertices of $F$ are drawn on
     the points of $X$.

     \item Free sets have applications to
     \begin{itemize}
        \item untangling,
        \item column planarity,
        \item universal point subsets,
        \item partial simultaneous geometric drawings.
        \item \ldots
     \end{itemize}
   \end{itemize}
\end{frame}

\begin{frame}
   \frametitle{The Free-Set Theorem}

   \begin{itemize}
       \item[]\textbf{Theorem (Dujmovi\'c--Frati--Gon\c{c}alves--M.--Rote 2019):} A set $F\subseteq V(G)$ is a \emph{free set} if and only if there is a proper good curve that contains $F$.
   \end{itemize}
\end{frame}

\begin{frame}
  \frametitle{Proper Good Curves and Dual Cycles}

  \begin{itemize}
     \item Every proper-good curve containing $k$ vertices gives a dual
       cycle of length at least $k$.
       \begin{center}
         \only<1>{\includegraphics{figs/curve-cycle-1}}%
         \only<2>{\includegraphics{figs/curve-cycle-2}}%
         \only<3>{\includegraphics{figs/curve-cycle-3}}%
         \only<4->{\includegraphics{figs/curve-cycle-4}}
       \end{center}
     \item<5-> What about the other direction?
     \item<6-> Can we get a proper good curve containing many vertices from long dual cycle?
  \end{itemize}
\end{frame}

\begin{frame}
   \frametitle{Pause}

   \begin{itemize}
      \item From this point on $G$ is (wlog) a triangulation.
      \item $G$ has maximum degree $\Delta$.
      \item $\dual{G}$ is the dual of $G$, a 3-connected cubic planar
      graph whose largest face has size at most $\Delta$
   \end{itemize}
\end{frame}



\begin{frame}
   \frametitle{Dual Cycles---Circumference}

   \begin{itemize}[<+->]
      \item The longest cycle in $\dual{G}$ is called its \emph{circumference}
      \item Circumference of $n$-vertex 3-connected cubic (planar) $\dual{G}$
      \begin{itemize}
        \item Tait's Conjecture (1884): $\forall \dual{G}: c(\dual{G}) = n$
        \item Disproved by Tutte (1946): $\exists \dual{G}: c(\dual{G})< n$
        \item Gr\"unbaum Walther (1973): $\exists \dual{G}: c(\dual{G}) = O(n^{0.9859})$.
        \item Barnette (XXXX): $\forall \dual{G}: c(\dual{G})=\Omega(\log n)$.
        \item Bondy and Simonovits (XXXX): $\forall \dual{G}: c(\dual{G})=e^{\Omega(\sqrt{\log n})}$
        \item Jackson (XXXX): $\forall \dual{G}: c(\dual{G}) = \Omega(n^{0.6942})$
        \item Billinski et al. (2011): $\forall \dual{G}: c(\dual{G}) = \Omega(n^{0.7532})$
        \item Liu, Yu, Zhang (2019): $\forall \dual{G}: c(\dual{G}) = \Omega(n^{0.8})$
      \end{itemize}
   \end{itemize}
\end{frame}

\begin{frame}
   \frametitle{Dual Cycles and Proper Good Curves}

   \begin{itemize}[<+->]
      \item Every dual cycle defines a proper good curve
      \item But the curve contains no vertices!
      \item Can we ``bend'' the curve to pick up some vertices?
      \item Sometimes yes, sometimes no
      \begin{center}
        \only<1-4>{\includegraphics{figs/cycle-to-curve-1}}%
        \only<5>{\includegraphics{figs/cycle-to-curve-2}}%
        \only<6>{\includegraphics{figs/cycle-to-curve-3}}%
        \only<7>{\includegraphics{figs/cycle-to-curve-4}}%
        \only<8->{\includegraphics{figs/cycle-to-curve-5}}
      \end{center}
   \end{itemize}

\end{frame}

\begin{frame}
   \frametitle{Dual Cycles and Proper Good Curves}

   \begin{itemize}[<+->]
      \item A cycle $C$ in $\dual{G}$
      \begin{itemize}
         \item \emph{touches} $f$ if $f\cap C\neq\emptyset$
         \item \emph{caresses} $f$ if $f\cap C$ is a path
         \item \emph{pinches} $f$ if $f\cap C$ is a cycle or has more
               than one connected component.
      \end{itemize}
      \item $\tau$ouched, $\rho$inched, $\kappa$aressed
      \item $\tau = \rho + \kappa$
   \end{itemize}

   \begin{center}
      \includegraphics[height=.5\textheight]{figs/t0t1-2}
   \end{center}
\end{frame}

\begin{frame}
  \frametitle{Dual Cycles and Proper Good Curves}

  \begin{itemize}[<+->]
    \item[]\textbf{Theorem:} If $C$ \caresses\ $\kappa$ faces of $\dual{G}$ then there is a proper good curve $\tilde{C}$ that contains at least $\kappa/4$ vertices of $G$.
    \item[]\textit{Proof:} Take an independent set $\dual{S}$ of \caressed\ faces of $\dual{G}$ and ``bend'' $C$ so that it passes through each vertex of $S$.
  \end{itemize}
\end{frame}


\begin{frame}
  \frametitle{Summary So Far}


  \begin{center}
    \uncover<4->{{\color{blue}cycle in $G$ of length $\ell=c\kappa\Delta^4$ \\ $\Downarrow$} \\}
    \uncover<1->{cycle in $\dual{G}$ that \caresses\ $\kappa$ faces}\\
    \uncover<2->{$\Downarrow$ \\
    proper good curve that contains $\kappa/4$ vertices of $G$}\\
    \uncover<3->{$\Downarrow$ \\
    free set of size $\kappa/4$}
  \end{center}
\end{frame}

\begin{frame}
  \frametitle{The Main Theorem}

  \begin{itemize}
     \item[]\textbf{Theorem:} If $G$ has maximum degree $\Delta$ and $\dual{G}$ has a cycle of length $\ell$, then $\dual{G}$ has a cycle $C'$ that \caresses\ $\kappa=\Omega(\ell/\Delta^4)$ faces.
  \end{itemize}
\end{frame}

\begin{frame}
  \frametitle{Proof Ideas}

  \begin{itemize}[<+->]
     \item Want to show that $C$ $\kappa$aresses many faces
     \item $C$ has length $\ell$
     \item each face of $\dual{G}$ has most $\Delta$ edges
     \begin{itemize}[<+->]
       \item[$\therefore$] $C$ $\tau$ouches at least $\ell/\Delta$ faces
          ($\tau = \Omega(\ell/\Delta)$)
     \end{itemize}
     \item Recall: $\tau = \kappa + \rho$
     \item If $\kappa =\Omega(\ell/\Delta)$ we are done
     \item Otherwise $C$ $\rho$inches at least $\rho = \Omega(\ell/\Delta)$ faces
  \end{itemize}
\end{frame}


\begin{frame}
  \frametitle{A Bad Example}

  \begin{center}
     \includegraphics[width=.9\textwidth]{figs/two-caressed}
  \end{center}
  \begin{itemize}
    \item $\tau = \kappa + \rho$
    \item $\tau = 2k+3$, $\kappa = 4$, $\rho = 2k-1$
  \end{itemize}
\end{frame}


\begin{frame}
  \frametitle{Proof Idea}

  \begin{center}
     \only<1,2>{\includegraphics[height=.2\textheight]{figs/two-caressed-1}}%
     \only<3-8>{\includegraphics[height=.2\textheight]{figs/two-caressed-2}}%
     \only<9->{\includegraphics[height=.2\textheight]{figs/two-caressed-3}}
  \end{center}
  \begin{itemize}
    \item<2-> \alert<11>{Define two trees $T_0$ (inside $C$) and $T_1$ (outside $C$)}
    \item<4-> Show that:
    \begin{itemize}
      \item<5-> Each leaf of $T_i$ contains a \caressed\ face
      \item<6-> Each $T_i$ has $\Omega(\ell/\Delta)$ nodes
    \end{itemize}
    \item<7-> We are done if either $T_i$ has many leaves, so each $T_i$ must have few leaves
    \begin{itemize}
      \item<8->[$\therefore$] each $T_i$ has many \emph{bad nodes} \\ (degree-2 and containing no caressed faces)
    \end{itemize}
    \item<10-> With enough bad nodes, we can perform a \emph{surgery} that increases the number of leaves in $T_0$.
  \end{itemize}
\end{frame}


\begin{frame}
  \frametitle{The Trees $T_0$ and $T_1$}

  \begin{center}
   \begin{tabular}{cc}
      \only<1>{\includegraphics[width=.45\textwidth]{figs/t0t1-1}}%
      \only<2->{\includegraphics[width=.45\textwidth]{figs/t0t1-2}}%
        &
      \uncover<3->{\includegraphics[width=.45\textwidth]{figs/t0t1-3}} \\
      \uncover<4->{\includegraphics[width=.45\textwidth]{figs/t0t1-4}} &
      \uncover<5->{\includegraphics[width=.45\textwidth]{figs/t0t1-5}} \\
   \end{tabular}
  \end{center}
\end{frame}


%      \item For a node $u$ of $T_i$,
%        \[ \rho_u \le 2(\kappa_u + \delta_u) \Leftrightarrow
%           \tau_u \le 3\kappa_u + 2\delta_u \]
%    \end{itemize}
%    \item So if $\tau_u$ is big then either
%    \begin{itemize}
%       \item $\kappa_u$ is big; or
%       \item $\delta_u$ is big
%    \end{itemize}
%  \end{itemize}
%\end{frame}
%

\begin{frame}
  \frametitle{Proof Idea}

  \begin{itemize}
    \only<1>{\item Define two trees $T_0$ (inside $C$) and $T_1$ (outside $C$)}
    \only<2->{\item[$\checkmark$] Define two trees $T_0$ (inside $C$) and $T_1$ (outside $C$)}
    \item Show that:
    \begin{itemize}
      \item \alert<3->{Each leaf of $T_i$ contains a \caressed\ face}
      \item Each $T_i$ has $\Omega(\ell/\Delta)$ nodes
    \end{itemize}
    \item We are done if either $T_i$ has many leaves, so each $T_i$ must have few leaves
    \begin{itemize}
      \item[$\therefore$] each $T_i$ has many \emph{bad nodes} \\ (degree-2 and containing no caressed faces)
    \end{itemize}
    \item With enough bad nodes, we can perform a \emph{surgery} that increases the number of leaves in $T_0$.
  \end{itemize}
\end{frame}

\begin{frame}
  \frametitle{Every Leaf $\kappa$aresses a Face}

   \noindent\textbf{Lemma:}
   Let $P$ be a \emph{chord path} for $C$ and let $L$ and $R$ be the two faces
   of $P\cup C$ that each contain $P$ in their boundary. Then $R$
   contains at least one face of $\dual{T}$ that is caressed by $C$.

  \uncover<2->{\emph{Proof:} by induction on number of faces in $R$}
  \begin{center}
      \only<1,2>{\includegraphics[width=.45\textwidth]{figs/one-caressed-1}}%
      \only<3,4>{\includegraphics[width=.45\textwidth]{figs/one-caressed-2}}
      \only<5>{\includegraphics[width=.45\textwidth]{figs/one-caressed-3}}
  \end{center}

  \uncover<4->{
   \noindent\textbf{Corollary:} Every leaf of $T_i$ contains at least one
   caressed face.
  }

\end{frame}


\begin{frame}
  \frametitle{Proof Idea}

  \begin{itemize}
    \item[$\checkmark$] Define two trees $T_0$ (inside $C$) and $T_1$ (outside $C$)
    \item Show that:
    \begin{itemize}
      \only<1>{\item Each leaf of $T_i$ contains a \caressed\ face}
      \only<2->{\item[$\checkmark$] Each leaf of $T_i$ contains a \caressed\ face}
      \item Each $T_i$ has $\Omega(\ell/\Delta)$ nodes
    \end{itemize}
    \item We are done if either $T_i$ has many leaves, so each $T_i$ must have few leaves
    \begin{itemize}
      \item[$\therefore$] each $T_i$ has many \emph{bad nodes} \\ (degree-2 and containing no caressed faces)
    \end{itemize}
    \item \alert<3->{With enough bad nodes, we can perform a \emph{surgery} that increases the number of leaves in $T_0$.}
  \end{itemize}
\end{frame}


\begin{frame}
  \frametitle{Bad Nodes}

  \begin{itemize}[<+->]
    \item Recall: A node of $T_i$ is \emph{bad} if its has degree 2 and contains no caressed face.
    \item[]\textbf{Lemma:} Every bad node contains exactly one face of $\dual{G}$
    \item Recall: We have lots of bad nodes and very few leaves
  \end{itemize}
\end{frame}


\begin{frame}
  \frametitle{Surgery}

  \begin{center}
    \begin{tabular}{m{.3\textwidth}m{2em}m{.3\textwidth}}
    \only<1>{\includegraphics[height=.7\textheight]{figs/cases-1}}%
    \only<2->{\includegraphics[height=.7\textheight]{figs/cases-3}} &
    \uncover<3->{$\Rightarrow$} &
    \uncover<3->{\includegraphics[height=.7\textheight]{figs/cases-4}}
    \end{tabular}
  \end{center}
  \begin{itemize}
     \item<4-> increases number of leaves in $T_1$
        (see $a_0$)
     \item<5-> decreases length of $C$ by at most $\Delta^2$
  \end{itemize}
\end{frame}

\begin{frame}
  \frametitle{Surgery II}

  \begin{center}
    \begin{tabular}{m{.3\textwidth}m{2em}m{.3\textwidth}}
    \only<1->{\includegraphics[height=.7\textheight]{figs/surgery-1}} &
    \uncover<2->{$\Rightarrow$} &
    \uncover<2->{\includegraphics[height=.7\textheight]{figs/surgery-2}}
    \end{tabular}
  \end{center}
\end{frame}

\begin{frame}
  \frametitle{Proof Idea}

  \begin{itemize}
    \item[$\checkmark$] Define two trees $T_0$ (inside $C$) and $T_1$ (outside $C$)
    \item Show that:
    \begin{itemize}
      \item[$\checkmark$] Each leaf of $T_i$ contains a \caressed\ face
      \item \alert<3>{Each $T_i$ has $\Omega(\ell/\Delta)$ nodes}
    \end{itemize}
    \item We are done if either $T_i$ has many leaves, so each $T_i$ must have few leaves
    \begin{itemize}
      \item[$\therefore$] each $T_i$ has many \emph{bad nodes} \\ (degree-2 and containing no caressed faces)
    \end{itemize}
    \only<1>{\item With enough bad nodes, we can perform a \emph{surgery} that increases the number of leaves in $T_0$.}
    \only<2->{\item[$\checkmark$] With enough bad nodes, we can perform a \emph{surgery} that increases the number of leaves in $T_0$.}
  \end{itemize}
\end{frame}

\begin{frame}
  \frametitle{$T_i$ Has Many Nodes}

  Situation so far:
  \begin{itemize}[<+->]
    \item Need to show: $T_i$ has many nodes
    \item Danger: $T_i$ contains few nodes, each containing many $\tau$ouched faces
    \item Will show that for any node $u$,
      \[   \rho_u \le 2(\kappa_u + \delta_u)  \]
     ($\rho_u$inched faces (inside $u$), $\kappa_u$aressed faces (inside $u$),
        $\delta$egree of $u$)
    \item Recall: $\tau_u=\kappa_u+\rho_u$
    \item If $\tau_u$ is large then either:
       \begin{enumerate}[<+->]
          \item $\kappa_u$ is large (good, more $\kappa$aressed faces!); and/or
          \item $\delta_u$ is large (good, more leaves!).
       \end{enumerate}
  \end{itemize}
\end{frame}


\begin{frame}

\textbf{Lemma:}
Let $u$ be a node of $T_i$ {\color{gray}and let $\rho_u$, $\kappa_u$, and $\delta_u$ denote the number of pinched faces of $\dual{T}$ in $u$, the number of caressed faces of $\dual{T}$ in $u$, and the degree of $u$ in $T_i$, respectively}.  Then $\rho_u \le 2(\kappa_u+\delta_u)$.

\uncover<2->{\textit{Proof (by discharging):}}
\begin{itemize}
  \item<3-> Assign a charge of 2 to each $\rho$inched face
  \item<4-> Move charge around so that
  \begin{itemize}
    \item<5->No charge is left on any pinched face
    \item<6->Each caressed faces has a charge of at most 2
    \item<7->Each outgoing edge has a charge of at most 2
  \end{itemize}
\end{itemize}
\uncover<2->{
\begin{center}
                \begin{tabular}{cccc}
                        \includegraphics[height=.22\textheight]{figs/discharge-2} &
                        \includegraphics[height=.22\textheight]{figs/discharge-3} &
                        \includegraphics[height=.22\textheight]{figs/discharge-4} &
                        \includegraphics[height=.22\textheight]{figs/discharge-5}
                \end{tabular}
                \end{center}
}
\end{frame}

\begin{frame}
  \frametitle{Proof Idea}

  \begin{itemize}
    \item[$\checkmark$] Define two trees $T_0$ (inside $C$) and $T_1$ (outside $C$)
    \item Show that:
    \begin{itemize}
      \item[$\checkmark$] Each leaf of $T_i$ contains a \caressed\ face
      \only<1>{\item Each $T_i$ has $\Omega(\ell/\Delta)$ nodes}
      \only<2->{\item[$\checkmark$] Each $T_i$ has $\Omega(\ell/\Delta)$ nodes}
    \end{itemize}
    \item We are done if either $T_i$ has many leaves, so each $T_i$ must have few leaves
    \begin{itemize}
      \item[$\therefore$] each $T_i$ has many \emph{bad nodes} \\ (degree-2 and containing no caressed faces)
    \end{itemize}
    \item[$\checkmark$] With enough bad nodes, we can perform a \emph{surgery} that increases the number of leaves in $T_0$.
  \end{itemize}
\end{frame}

\begin{frame}
  \frametitle{Conclusion}

  \textbf{Theorem:} If $\dual{G}$ has a cycle of lenth $\ell$, then $\dual{G}$ has a cycle $C'$ that caresses $\Omega(\ell/\Delta^4)$ faces.

  \textbf{Consequence:} Every $n$-vertex planar graph of maximum degree $\Delta$ has a free set of size $\Omega(n^{0.8}/\Delta^4)$.

  \textbf{Open Problem:} Eliminate the dependence on $\Delta$.
\end{frame}


%\closing

\end{document}
