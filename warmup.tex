\documentclass{patmorin}
\usepackage[T1]{fontenc}
\usepackage[utf8]{inputenc}
\usepackage{amsmath}
\usepackage{amsfonts}
\usepackage{amsthm}
\usepackage{graphicx}
\usepackage{enumerate}
\usepackage{pat}
\usepackage{paralist}

\usepackage{tipa}
\usepackage{upgreek}
\usepackage{rotating}
\newcommand{\V}{\rotatebox[origin=c]{180}{$V$}}
\newcommand{\Y}{\rotatebox[origin=c]{180}{$Y$}}
%\usepackage{hyperref}

\setlength{\parskip}{1ex}

\title{\MakeUppercase{Notes on Layered $H$-Partitions of $k$-Planar Graphs}%
    \thanks{This work was partly funded by NSERC and MRI.}}

\author{Vida Dujmovi\'c%
        \thanks{School of Computer Science and Electrical Engineering, 
                University of Ottawa},\,\, 
        Pat Morin%
        \thanks{School of Computer Science, Carleton University},\,\, and
        David R. Wood%
        \thanks{School of Mathematical Sciences, Monash University}}

\newcommand{\dual}[1]{{#1}^\star}
\newcommand{\note}[2]{{\color{red}[#1:~#2]}}

\DeclareMathOperator{\dist}{dist}
\DeclareMathOperator{\depth}{depth}

\begin{document}
\maketitle


\begin{abstract}
  We prove that every $k$-planar graph has a layered $H$-partition of layered width $k(24k+4)$ in which $H$ has treewidth 3. This implies that $k$-planar graphs have a queue number upper-bounded by a polynomial in $k$ and have non-repetitive chromatic number upper-bounded by an exponential function of $k$.
\end{abstract}

\section{Introduction}

\section{$1$-Plane Graphs}

Let $G$ be an edge maximal 1-plane graph.  Here, edge-maximal should be taken to mean that any two vertices of $G$ on a common face have an edge joining them.  Assume also that $G$ is crossing minimal so that the embedding of $G$ minimizes the number of crossings over all possible embeddings of the (abstract) graph $G$.  From now on, we treat vertices of $G$ as points and edges as simple curves.

A \emph{kite} in $G$ is the subgraph $K=G[\{v,w,x,y\}]$ induced by the endpoints of a pair of crossing edges $vw,xy\in E(G)$.  It follows from edge-maximality that every kite is isomorphic to the complete graph $K_4$.
The edges $vw$ and $xy$ are called \emph{spars} of $K$.  The cycle $vxwy$ is called the \emph{sail} of $K$.  It follows from crossing minimality that none of the edges $vx$, $xw$, $wy$, or $yv$ are crossed by any other edges of $G$. Thus any edge that is a spar of a kite $K$ is not part of a sail of any kite $K'$. Observe that any spar of $K$ is incident on exactly four \emph{kite faces} of $G$, each of which has three edges and two vertices of $G$ on its boundary.

The 1-plane graph $G$ has a plane triangulation $G'$ as a subgraph that can be obtained by removing one spar from each kite in $G$.  Observe that, for any spar $vw\in E(G)\setminus E(G')$ that crosses $xy\in E(G')$, $G'$ contains the path $vxw$ (and $vyw$).  It follows that $\dist_{G'}(x,y)\le 2\dist_G(x,y)$ for any pair of vertices $x,y\in V(G)$.



\begin{lem}\lemlabel{induction} The setup:
  \begin{compactenum}
    \item Let $G$ and $G'$ be defined as above.
    \item Let $T$ be a BFS tree of $G'$ rooted at some vertex $r$.
    \item For each integer $i\ge 0$, let $V_i=\{v\in V(G):\dist_T(r,v)=i\}$. 
    \item Let $C$ be a cycle in $G'$ with $r$ in the exterior of $C$ and such that
    \begin{compactenum} 
      \item No edge of $C$ is crossed by any edge of $G$; and
      \item $V(C)$ can be partitioned into $P_1,\ldots,P_k$, $k\le 3$ such that for each $i\in\{1,\ldots,k\}$,
      \begin{compactenum}
        \item $C[P_i]$ is a path; and
        \item $|P_i\cap V_j| \le 15$ for all integers $j\ge 0$.
      \end{compactenum}
    \end{compactenum}
    \item Let $N$ and $N'$ be the subgraphs of $G$ and $G'$ consisting only of those edges and vertices contained in $C$ or the interior of $C$.
  \end{compactenum}
  Then $N$ has an $H$-partition $(B_x : x\in V(H))$ such that
  \begin{compactenum}
    \item $H$ is planar;
    \item for all integers $j\ge 0$, and all $x\in V(H)$, $|B_x\cap V_j|\le 15$; 
    \item for each $i\in\{1,\ldots,k\}$, there exists some $x_i\in V(H)$ such that $P_i=B_{x_i}$; and
    \item $H$ has a tree decomposition whose largest bag has size at most 4 and such that some bag contains $x_1,\ldots,x_k$.
  \end{compactenum}
\end{lem}

\begin{proof}
  The proof is by induction on the number of vertices of $N$.
  First note that $N'$ is a near-triangulation.  If $k=3$, set $R_i=P_i$ for each $i\in\{1,2,3\}$.  Otherwise, as before, split $P_1,\ldots,P_k$ to partition $V(C)$ into three sets $R_1$, $R_2$, and $R_3$ such that each $C[R_i]$ is a path and each $R_i$ contains vertices from at most one of $P_1,\ldots,P_k$. 
  
  For each $i\in\{1,2,3\}$ and $v\in R_i$, assign the colour $i$ to $v$.
  For each $v\in V(N')$, consider the path $P_v$ that consists of the subpath of the $v$ to $r$ path in $T$ that stops at the first vertex $v'\in V(C)$. Assign $v$ to have the same colour as $v'$.
  
  Sperner's Lemma ensures that $G'$ contains a triangular face $\tau=v_1v_2v_3$ whose three vertices are assigned different colours. For each $i\in\{1,2,3\}$, let $Q_i=P_{v_i}$ be the path in $T$ form $v_i$ to $v_i'$.  Let $Y$ denote the subgraph of $N'$ consisting of vertices and edges $Q_1$, $Q_2$, $Q_3$, and $\tau$.  Let $Y^+$ denote the subgraph of $N$ consisting of the vertices and edges of $Y$ plus the vertices and edges of every kite formed by a crossing between an edge of $G$ and an edge of $Y$.
  
  We claim that, for each integer $i\ge 0$, $|V(Y^+)\cap V_i|\le 15$.  To see this, first observe that $Y$ contains at most 3 vertices of $V_i$. If a vertex $x\in V(Y^+)\setminus V(Y)$ is contained in $V_i$, then this is because $Y$ contains an edge $vw$ with $v\in V_{i'}$ and $w\in\{V_{i'+1}$ for some $i'\in\{i-1,i\}$ and $G$ contains an edge $xy$ that crosses $vw$. (Recall that $vx,xw\in E(G')$, so $\dist_{G'}(w,r)-1\le\dist_{G'}(x,r)\le\dist_{G'}(v,r)+1$.)  The graph $Y$ contains at most 6 such edges, each of which accounts for at most 2 additional vertices of $V_i$.  Therefore, in total, $|Y^+\cap V_i|\le 15$.

  
  Finally, let $S$ and $S^+$ denote the subgraph of $G$ containing the edges and vertices of $Y$, respectively $Y^+$, and the edges and vertices of $C$.  The graph $S^+$ has some number of bounded faces, all contained in the interior of $C$. Some of the bounded faces of $S$ are kite faces. Call the non-kite bounded faces $F_1,\ldots,F_m$ and let $C_1,\ldots,C_m$ denote their boundaries.  We claim that, for each $i\in\{1,\ldots,m\}$, if some portion of $C_i$ is contained in an edge $vw\in E(G)$ then $vw$ is not crossed by any edge of $G$.  To see this, there are three cases to consider:
  \begin{enumerate}
    \item $vw\in E(C)$. By assumption, $vw$ is not crossed by any edge of $G$.
    \item $vw\in E(Y)$. In this case, the kite containing $vw$ is in $Y^+$, so $vw$ is only incident to kite faces.
    \item $vw\in E(Y^+)\setminus E(Y)$. In this case, either $vw$ is a sail edge in which case it is not crossed by definition, or $vw$ is a spar that was added to $Y^+$ because $vw$ crosses some edge $xy\in E(Y)$.  In this latter case, $vw$ is only incident to kite faces.
  \end{enumerate}
  Therefore each $C_i$ is a cycle in $G^+$ consisting entirely of uncrossed edges. The vertices of $C_i$ can be partitioned into at most three sets $P_1'$, $P_2'$, and $P_3'$ where $P_1'\subset V(Y^+)$, $P_2'\subseteq P_a$ and $P_3'\subseteq P_b$ for some $a,b\in\{1,2,3\}$. Furthermore $C_i[P_j']$ is a path for each $j\in\{1,\ldots,3\}$. Finally, the subgraph $N_i$ of $G$ consisting of the edges and vertices of $G$ contained in $C_i$ or its interior does not contain one of the three vertices of $\tau$. Therefore, we can apply induction using the cycle $C_i$ and the partition $P_1',P_2',P_3'$ of $V(C_i)$ to obtain the desired $H$-partition and tree decomposition of $N_i$.
  
  The remainder of the proof is as before. We use $V(Y^+)\setminus V(C)$ as an element of our $H$-partition.  The root bag in the tree decomposition contains the vertices obtained by contracting each of $P_1,\ldots,P_k$ and $Y^+$.
  
  Planarity comes from the fact that if two edges $vw$ and $xy$ cross, then they end up in the same bag of the $H$-partition.  This means that $H$ is actually obtained by contracting connected sets of vertices in the planar graph $G'$.
\end{proof}

\begin{thm}\thmlabel{1-plane}
  Every 1-plane graph $G$ has a layered $H$-partition of layered width at most 30 where $H$ is planar and has treewidth at most 3.
\end{thm}

\begin{proof}
  The same as before except that the layered width 15 from \lemref{induction} becomes 30 because \lemref{induction} uses a layering of $G'$ and distances in $G'$ can be a factor of 2 larger than in $G$.
\end{proof}


\section{$k$-Planar}


Let $G$ be a $k$-planar graph.  We will use of the \emph{plane augmentation} $G^+$ of $G$ obtained by adding a \emph{dummy vertex} at each crossing in $G$, then adding a \emph{star vertex} $v_f$ in each face $f$ of the resulting graph and making $v_f$ adjacent to each vertex of $f$.

We will make use of the fact that $\dist_{G^+}(x,y)\le k\cdot\dist_G(x,y)$ for any $x,y\in V(G)$.  This follows immediately from the fact that any edge $xy\in E(G)$ has a corresponding path $x-y$ in $G^+$ of length at most $k$.  We will also make use of the fact that every dummy vertex of $G^+$ has degree 8.  

\begin{lem}\lemlabel{k-induction} The setup:
  \begin{compactenum}
    \item Let $G$ be $k$-plane graph.
    \item Let $G^+$ be the plane augmentation of $G$.
    \item Let $T$ be a BFS spanning tree of $G^+$ rooted at some vertex $r$.
    \item For each integer $i\ge 0$, let $V^+_i=\{v\in V(G^+):\dist_{G^+}(r,v)=i\}$, and let $V_i=V(G)\cap V^+_i$.
    \item Let $C$ be a cycle in $G^+$ with $r$ in the exterior of $C$ and such that
    \begin{compactenum} 
      \item No edge of $C$ is crossed by any edge of $G$; and
      \item $V(C)$ can be partitioned into $P_1,\ldots,P_k$, $k\le 3$ such that for each $i\in\{1,\ldots,k\}$,
      \begin{compactenum}
        \item $C[P_i]$ is a path; and
        \item $|P_i\cap V_j| \le 24k+4$ for all integers $j\ge 0$.
      \end{compactenum}
    \end{compactenum}
    \item Let $N$ and $N^+$ be the subgraphs of $G$ and $G^+$ consisting only of those edges and vertices contained in $C$ or the interior of $C$.
  \end{compactenum}
  Then $N$ has an $H$-partition $(B_x : x\in V(H))$ such that
  \begin{compactenum}
    \item for every integer $j\ge 0$, and all $x\in V(H)$, $|B_x\cap V_j|\le 24k+4$; 
    \item for each $i\in\{1,\ldots,k\}$, there exists some $x_i\in V(H)$ such that $P_i=B_{x_i}$; and
    \item $H$ has a tree decomposition whose largest bag has size at most 4 and such that some bag contains $x_1,\ldots,x_k$.
  \end{compactenum}
\end{lem}


\begin{proof}
  We remark that $N^+$ is a near-triangulation.
  The proof is by induction on the number of vertices of $N^+$.

  If $k=3$, set $R_i=P_i$ for each $i\in\{1,2,3\}$.  Otherwise, as before, split $P_1,\ldots,P_k$ to partition $V(C)$ into three sets $R_1$, $R_2$, and $R_3$ such that each $C[R_i]$ is a path and each $R_i$ contains vertices from at most one of $P_1,\ldots,P_k$. 
  
  Next, we 3-colour $V(N^+)$ so that we can apply Sperner's Lemma.
  For each $i\in\{1,2,3\}$ and $v\in R_i$, assign the colour $i$ to $v$.
  For each $v\in V(N^+)$, consider the path $P_v$ that consists of the subpath of the $v$ to $r$ path in $T$ that stops at the first vertex $v'\in V(C)$. Assign $v$ to have the same colour as $v'$.
  
  Sperner's Lemma ensures that $N^+$ contains a triangular face $\tau=v_1v_2v_3$ whose three vertices are assigned different colours. For each $i\in\{1,2,3\}$, let $Q_i^+=P_{v_i}^+$ be the path in $T$ from $v_i$ to $v_i'$.  Let $Y^+$ denote the subgraph of $N^+$ consisting of vertices and edges of $Q_1^+$, $Q_2^+$, $Q_3^+$, and $\tau$.
  
  Define a vertex set $X\subset V(G)$ as follows:  The set $X$ contains $V(Y^+)\cap V(G)\setminus V(C)$.  Next, for each dummy vertex $v^+\in V(Y^+)\setminus V(G)$, we consider the two edges $vw,xy\in E(G_0)$ whose intersection is $v^+$.  These edges correspond to two paths $v\rightsquigarrow w$ and $x\rightsquigarrow y$.  We collect the set $A_{x^+}$ of all vertices of $G^+$ that are on any face that contains any dummy vertex in $v\rightsquigarrow w$ or $x\rightsquigarrow y$.  We add the set $A_{x^+}\cap V(N^+)\setminus V(C)$ to $X$.  
  
  Note that $A_{x^+}$ contains $\{v,w,x,y\}$ and that, for any vertex $z\in A_{x^+}$, $\dist_{G^+}(z,x^+) \le k$.  Careful counting also reveals that the size of $A_{x^+}$ is at most .
    
  We claim that, for each integer $i\ge 0$, $|V_i\cap X|\le 24k+4$.  To see why this is so, first note that $|V_i^+\cap V(Y^+)|\le 3$ since $V_i^+$ contains at most one vertex from each of $Q_1^+$, $Q_2^+$, and $Q_3^+$. For any vertex $v\in X$, there exists a vertex $v^+\in V(Y^+)$ such that $\dist_{G'}(v,v^+)\le k$.  Turning this around, we can assign each vertex $v\in X$ to a vertex $v^+\in V(Y^+)$ in such a way that each vertex of $Y^+$ is assigned at most 4 vertices of $X$ (each dummy vertex $v^+$ is assigned the four endpoints of the crossing pair of edges that generated $v^+$).  Since any vertex $v\in X\cap V_i$ is assigned to one of the at most $6(k+1)$ vertices in $X\cap (V(Y^+)\cap( V_{i-k}^+\cup\cdots\cup V_{i+k}^+))$, we have $|X\cap V_i|\le 24k+4$.

  Let $S^+$ denote the subgraph of $N^+$ consisting of the vertices and edges of $Y^+$ and the vertices and edges of $C$.  The graph $S^+$ is planar. It contains one triangular face $\tau$ and up to three additional bounded faces $F_1$, $F_2$, and $F_3$, where each $F_i$ does not have any vertices in common with $P_i$.
  
  For each $i\in\{1,2,3\}$, let $Z_i$ be the subset of $V(G)$ that is contained in the interior of $F_i$.  We claim that $N-X-V(C)$ does not contain any edge $vw\in E(G)$ with $v\in Z_i$ and $w\in Z_j$ for $i\neq j$.  Indeed, any such edge would contain a vertex of $V(Y^+)\setminus V(G)$ in its interior and therefore each of $v,w\in X\cup V(C)$.

  The set $X$ will become an element $B_x$ in our $H$-partition of $G$.  All that remains is to apply induction on three subgraphs.  For each $i\in\{1,\ldots,3\}$, let $N_i^+$ denote the subgraph of $G^+$ containing the vertices and edges of $G^+$ that are contained in the closure of $F_i$.  
  Let $G_i$ be the subgraph of $G$ obtained by removing the edges of $G$ that cross the boundary of $F_i$.
  Let $N_i$ denote the subgraph of $G_i$ containing the vertices and edges of $G_i$ that are contained in the closure of $F_i$.  Note that $G_i$ is a subgraph of $G_0$ and that the number of vertices of $N_i^+$ is less than the number of vertices of $N^+$ since $N_i^+$ does not contain one of the vertices of $\tau$.  Therefore, we can apply induction to obtain the desired $H$-partition of $N_i$ using the original graphs $G_0$ and $G^+$, the subgraph $G_i$ in place of $G$, the cycle $F_i$ in place of $C$, and partition of $V(F_i)$ into $V(F_i)\cap X$, $V(F_i)\cap P_{i-1}$, and $V(F_i)\cap P_{i+1}$.
  
  The remainder of the proof is as before. We use $X$ as an element of our $H$-partition. The root bag in the tree decomposition contains the vertices obtained by contracting each of $P_1,\ldots,P_k$ and $X$.
\end{proof}


\begin{thm}\thmlabel{1-plane}
  Every $k$-plane graph $G$ has a $(3, 24k^2+4k)$-partition.
\end{thm}

\begin{proof}
  The same as before except that the layered width $24k+4$ from \lemref{k-induction} becomes $24k^2+4k$ because \lemref{k-induction} uses a layering of $G^+$ and distances in $G^+$ can be a factor of k larger than in $G$.
\end{proof}











% Let $G$ be a $k$-planar graph with a distinguished vertex $r$. For each integer $i\ge 0$, let $V_i=\{v\in V(G):\dist_G(r,v)=i\}$ and let $V_{\le i}=\bigcup_{j=0}^i V_j$.  We will use distances to $r$ frequently, so for each $v\in V(G)$, we let $d_v=\dist_G(r,v)$ (so that $v\in V_{d_v}$). We say that a subgraph $P$ of $G$ has layered width at most $\ell$ if $|V_i\cap V(P)|\le \ell$ for each integer $i\ge 0$.
% 
% Let $P_1,\ldots,P_3$ be three vertex-disjoint subgraphs of $G$, each of which has layered width at most $\ell$ and none of which contains $r$. Let $B$ be the union of $P_1,\ldots,P_3$.  Let $N$ be a component of $G-V(B)$ that does not contain $r$.  Let $G'$ be the subgraph of $G$ consisting of edges and vertices of $N$, edges and vertices of $B$, and all edges $vw\in E(G)$ with $v\in V(N)$ and $w\in V(B)$.
% 
% \begin{lem}
%   There exists a vertex $v\in V(G')$ and three vertex-disjoint paths $Q_1$, $Q_2$, and $Q_3$ where each $Q_i$ begins at $v$, has all of its internal vertices in $V(N)$, ends at a vertex in $V(P_i)$ and has layered width at most 10. 
% \end{lem}
% 
% \begin{proof}
%   For simplicity, assume that every vertex of $P_i$ is adjacent to at least one vertex of $V(N)$ since we can ignore vertices of $P_i$ for which this is not the case.  
% 
%   For each integer $r\ge 0$, consider the induced subgraph $G_r'=G'[V(B)\cup V_{\le r}]$.  Let $r$ be the smallest integer such that $G_{r}'$ has at least one component $A$ that contains vertices from each of $P_1$, $P_2$, and $P_3$.  Select any vertex $w\in V(A)\setminus V_{<r}$. Suppose, for now that $w\not\in V(B)$.  Then, since no component of $V_{r-1}$ contains vertice from each of $P_1$, $P_2$, and $P_3$, it must be the case that $G_{r-1}'$ has components $A_1$, $A_2$, and $A_3$ where $w$ is adjacent to some vertex in each $A_i$, each $A_i$ contains a vertex of $P_i$, $A_1\not\in\{A_2,A_3\}$ and $A_1$ does not contain a vertex of $P_3$.  (Note that we may have $A_2=A_3$ and we may have $A_1$ containig vertices of $P_2$.) 
% 
%   If $A_2\neq A_3$, then we are done.  We set $v=w$ and, for each $i\in \{1,\ldots,3\}$ we take a path $Q_i$ from $v$ to $V(P_i)$ whose $j$th vertex is in $V_{r-j+1}\cap V(A_i)$.  These three paths are disjoint since $V(Q_i)\subseteq V(A_i)$ for each $i\in \{1,\ldots,3\}$.
% 
%   If $A_2=A_3$, there is still a little more work to do.  In this case we will still use the path $Q_1$ from $w$ to $V(P_1)$.  Now, $w$ is adjacent to a vertex $w_1$ in $A_2=A_3$ so we extend
% 
% 
% 
% 
% 
%    where (without loss of generality) $V(A_1)\cap V(P_1)\neq \emptyset$, $V(A_1)\cap (V(P_2)\cup V(P_3))=\emptyset$, $V(A_2)\cap V(P_2)\neq\emptyset$, $V(A_2)\cap V(P_1)=\emptyset$ and $w$ is adjacent to a vertex $w_i\in A_i$ for each $i\in\{1,2\}$.
% 
%   If 
% 
% 
% 
% 
% 
%    contains vertices of $P_1$, $A_2$ contains vertices of $P_2$ and (possibly )
% 
% 
% 
% 
%   We claim that $v$ has the desired property.  Indeed
% 
% \end{proof}
% 
% 
% 
% 
% 
% \section{A Metric Alternative To Sperner's Lemma}
% 
% I'm interested in having something more robust and less magical than Sperner's Lemma. Here's a first attempt.
% 
% Let $G$ be a planar triangulation with some distinguished vertex $r$. 
% 
% 
% 
% 
% A \emph{$t$-pod} (with respect to $r$) consists of vertex-disjoint paths $Q_1,\ldots,Q_t$, where each $Q_i$ a prefix of some $v_i\leadsto r$ path in $G$ and $v_1,\ldots,v_t$ form a clique in $G$.  We call $v_1,\ldots,v_t$ the \emph{central vertices} of the tripod and we call $Q_1,\ldots,Q_t$ the \emph{legs} of the tripod.  The final vertex in each path $Q_i$ is called \emph{foot} if the tripod.  
% 
% A $t$-pod $\Y$ with $t\le 3$, is called a \emph{tripod}.  A $t$-pod $\V$ with $t\le 2$ is called a \emph{bipod}.  A $t$-pod $I$ with $t\le 1$ is called a \emph{monopod}.
% 
% For a $t$-pod $Y$ with central vertices $v_1,\ldots,v_t$, we define the \emph{depth} of $Y$ as 
% \[
%     \depth(Y)=\max\{d_G(r,v):v\in V(T)\} 
%         = \max\{\dist_G(r, v_i): i\in\{1,\ldots,t\}\} \enspace .
% \]
% Observe that, since $v_1,\ldots,v_t$ form a clique in $G$, $\depth(Y)-1\le \dist_G(r, v_i) \le \depth(Y)$.
% 
% Let $C$ be a cycle in $G$ and $P_1,\ldots,P_a$, $a\le 3$ be a partition of $V(C)$ such that $C[P_i]$ is a path and $C[P_i]$ is also a bipod $\V_i$ in $G$. Let $Q_{i,1}$ and (possibly) $Q_{i,2}$ be the legs of $\V_i$ where we use the convention that, for each $i\in\{1,\ldots,a\}$, the last vertex of $Q_{i,2}$ and the first vertex of $Q_{i+1,1}$ are adjacent in $C$.  Let $v_{i,j}$ be first vertex of $Q_{i,j}$ and let $v_{i,j}'$ be the last vertex of $Q_{i,j}$.  With these definitions, $v_{i,j}$ is a central vertex of $\V_i$ and $v_{i,j}'$ is a foot of $\V_{i}$.
% 
% Note that, for each $i\in\{1,\ldots,a\}$, let $d_i = dist_{G}(r,\{v_{i,2},v_{i+1,1}'\})$.  
% 
% 
% 
% Let $N$ be the near-triangulation consisting of all edges and vertices of $G$ that are on or in the ineterior of $C$.  
% 
% 
% \section{Remarks}
% 
% \begin{enumerate}
%   \item The fact that $G'$ is a triangulation is not really important. We could arbitrarily triangulate $N'$ (compatible with $T$) and still find a Sperner triangle.
%   \item The fact that $T$ is plane seems important since it lets us use Sperner's Lemma.  But maybe Sperner's Lemma is overkill. $Y^+$ is just a crappy separator of small layered width that happens to break $N$ into components for which no component is adjacent to more than 2 of $Q_1,\ldots,Q_3$.
%   \item The fact that $C_i$ is a cycle of non-crossed edges in $G$ is not necessarily critical.  The non-crossed property isn't so important as long as no edge of $G-V(S^+)$ has endpoints in two different faces of $S^+$.
% 
%   More troublesome, again, is the use of Sperner's Lemma since it requires that $P_1,\ldots,P_k$ (and therefore also the boundary of $Y^+$) be not so wild that they interleave.  Again, maybe we can cook up something more robust than Sperner's Lemma.
% 
%   \item I think I can push through a proof for 2-plane graphs.
% \end{enumerate}

\bibliographystyle{plain}
\bibliography{warmup}

\end{document}
